\documentclass[10pt]{amsbook}

%%%%%%%%%%%%%%%%%%%%%%%%%%%%%%%%
%%                            %%
%%    LaTeX Customizations    %%
%%      Atanas Atanasov       %%
%%                            %%
%%%%%%%%%%%%%%%%%%%%%%%%%%%%%%%%

%% Customizations styles
% \newcommand{\boldsymbols}{}{}


%% Packages
\usepackage{amsmath, amsthm, amssymb}
%% \usepackage{mathabx} %% Use only if necessary (required for \widecheck)
\usepackage{array}
%\usepackage{fullpage}
\usepackage{ifthen}
\usepackage{ifpdf}
\ifpdf
  \usepackage[
    pdftex,
    letterpaper,
    includeheadfoot,
    centering=true,
    margin=1in,
    bindingoffset=0in
  ]{geometry}
  \usepackage[pdftex,final]{graphicx}
  \usepackage{psfragx}
\else
  \usepackage[
    dvips,
    letterpaper,
    includeheadfoot,
    centering=true,
    margin=1in,
    bindingoffset=0in
  ]{geometry}
  \usepackage[dvips,final]{graphicx}
  \usepackage{psfragx}
\fi
\usepackage{indentfirst}
\usepackage{suffix}
\usepackage{float}
\usepackage{enumerate}
% \usepackage{ifthen}
% \usepackage{listings}
\usepackage{hyperref}
\usepackage[all,2cell,knot]{xy} \UseAllTwocells \SilentMatrices
\xyoption{arc}
%\usepackage[mathscr]{eucal}
\usepackage{bm}
\usepackage{mathrsfs}
\usepackage{booktabs}
\usepackage{leftidx}
%\usepackage{youngtab}
%\newcommand{\ysp}{\textrm{}}

%% Environments

\newtheoremstyle{noindent}
  {6pt}      % Space above
  {6pt}      % Space below
  {}         % Body font
  {}         % Indent amount (empty = no indent, \parindent = para indent)
  {\scshape} % Thm head font
  {.}        % Punctuation after thm head
  {.5em}     % Space after thm head: " " = normal interword space;
             %   \newline = linebreak
  {}         % Thm head spec (can be left empty, meaning `normal')

\theoremstyle{noindent}
\numberwithin{equation}{subsection}
\newtheorem{theorem}[equation]{Theorem}
\newtheorem{proposition}[equation]{Proposition}
\newtheorem{lemma}[equation]{Lemma}
\newtheorem{corollary}[equation]{Corollary}
% \newtheorem{claim}[equation]{Claim}
\newtheorem{conjecture}[equation]{Conjecture}
\newtheorem{observation}[equation]{Observation}

\theoremstyle{noindent}
\newtheorem{definition}[equation]{Definition}
\newtheorem{exercise}[equation]{Exercise}
\newtheorem{question}[equation]{Question}
% \newtheorem{problem}[equation]{Problem}
\newtheorem{construction}[equation]{Construction}

\theoremstyle{remark}
\newtheorem{example}[equation]{Example}
%\newtheorem{hint}[equation]{Hint}
\newtheorem*{remark}{Remark}
%\newtheorem{remark}[equation]{Remark}
%\newtheorem{apology}[equation]{Apology}
%\newtheorem{warning}[equation]{Warning}

\newenvironment{claim}[1][Claim.]{\begin{trivlist}
\item[\hskip \labelsep {\bfseries #1}]}{\end{trivlist}}
\newenvironment{claim-mild}[1][Claim.]{{\bfseries #1} \hskip 0.2\labelsep}{}

% \newcounter{cnt_exercise}
% \setcounter{cnt_exercise}{1}
% \newenvironment{exercise}[1][Exercise]{\begin{trivlist}
% \item[\hskip \labelsep {\bfseries #1 \arabic{cnt_exercise}}]}{\addtocounter{cnt_exercise}{1} \end{trivlist}}
% \newenvironment{solution}[1][Solution.]{\begin{trivlist}
% \item[\hskip \labelsep {\emph{#1}}]}{\qed \end{trivlist}}
% \newenvironment{question}[1][Question.]{\begin{trivlist}
% \item[\hskip \labelsep {\bfseries #1}]}{\end{trivlist}}

\newcounter{cnt_problem}
\setcounter{cnt_problem}{1}
\newcommand\problem[1][Problem]{\vspace{0.15in} \noindent {\hskip \labelsep {\bfseries #1 \arabic{cnt_problem}}} \addtocounter{cnt_problem}{1}}
\WithSuffix\newcommand\problem*[1]{\vspace{0.15in} \noindent {\hskip \labelsep {\bfseries #1}}}

\floatstyle{plain}
\newfloat{figure}{thp}{}
\floatname{figure}{Figure}

\newenvironment{itemize_packed}{
\\[-20pt]
\begin{itemize}
  \setlength{\itemsep}{2pt}
  \setlength{\parskip}{0pt}
  \setlength{\parsep}{0pt}
}{\end{itemize}}

\newenvironment{enumerate_packed}[1][1.]{
\\[-20pt]
\begin{enumerate}[#1]
  \setlength{\itemsep}{2pt}
  \setlength{\parskip}{0pt}
  \setlength{\parsep}{0pt}
}{\end{enumerate}}

\newcommand{\topcell}[1]{\vtop{\null\hbox{#1}}}


%% Special font symbols
\ifthenelse{\isundefined{\boldsymbols}}{
\newcommand{\A}{\ensuremath{\mathbb{A}}}
\newcommand{\C}{\ensuremath{\mathbb{C}}}
\newcommand{\F}{\ensuremath{\mathbb{F}}}
\newcommand{\G}{\ensuremath{\mathbb{G}}}
\renewcommand{\H}{\ensuremath{\mathbb{H}}}
\newcommand{\N}{\ensuremath{\mathbb{N}}}
\renewcommand{\O}{\ensuremath{\mathbb{O}}}
\renewcommand{\P}{\ensuremath{\mathbb{P}}}
\newcommand{\Q}{\ensuremath{\mathbb{Q}}}
\newcommand{\R}{\ensuremath{\mathbb{R}}}
\newcommand{\T}{\ensuremath{\mathbb{T}}}
\newcommand{\V}{\ensuremath{\mathbb{V}}}
\newcommand{\Z}{\ensuremath{\mathbb{Z}}}
}{
\newcommand{\A}{\ensuremath{\mathbf{A}}}
\newcommand{\C}{\ensuremath{\mathbf{C}}}
\newcommand{\F}{\ensuremath{\mathbf{F}}}
\newcommand{\G}{\ensuremath{\mathbf{G}}}
\renewcommand{\H}{\ensuremath{\mathbf{H}}}
\newcommand{\N}{\ensuremath{\mathbf{N}}}
\renewcommand{\O}{\ensuremath{\mathbf{O}}}
\renewcommand{\P}{\ensuremath{\mathbf{P}}}
\newcommand{\Q}{\ensuremath{\mathbf{Q}}}
\newcommand{\R}{\ensuremath{\mathbf{R}}}
\newcommand{\T}{\ensuremath{\mathbf{T}}}
\newcommand{\V}{\ensuremath{\mathbf{V}}}
\newcommand{\Z}{\ensuremath{\mathbf{Z}}}
}
\newcommand{\Ac}{\ensuremath{\mathcal{A}}}
\newcommand{\Bc}{\ensuremath{\mathcal{B}}}
\newcommand{\Cc}{\ensuremath{\mathcal{C}}}
\newcommand{\Dc}{\ensuremath{\mathcal{D}}}
\newcommand{\Ec}{\ensuremath{\mathcal{E}}}
\newcommand{\Fc}{\ensuremath{\mathcal{F}}}
\newcommand{\Gc}{\ensuremath{\mathcal{G}}}
\newcommand{\Hc}{\ensuremath{\mathcal{H}}}
\newcommand{\Ic}{\ensuremath{\mathcal{I}}}
\newcommand{\Jc}{\ensuremath{\mathcal{J}}}
\newcommand{\Kc}{\ensuremath{\mathcal{K}}}
\newcommand{\Lc}{\ensuremath{\mathcal{L}}}
\renewcommand{\Mc}{\ensuremath{\mathcal{M}}}
\newcommand{\Nc}{\ensuremath{\mathcal{N}}}
\newcommand{\Oc}{\ensuremath{\mathcal{O}}}
\newcommand{\Pc}{\ensuremath{\mathcal{P}}}
\newcommand{\Rc}{\ensuremath{\mathcal{R}}}
\newcommand{\Sc}{\ensuremath{\mathcal{S}}}
\newcommand{\Tc}{\ensuremath{\mathcal{T}}}
\newcommand{\Uc}{\ensuremath{\mathcal{U}}}
\newcommand{\Vc}{\ensuremath{\mathcal{V}}}
\newcommand{\Wc}{\ensuremath{\mathcal{W}}}
\newcommand{\Xc}{\ensuremath{\mathcal{X}}}
\newcommand{\As}{\ensuremath{\mathscr{A}}}
\newcommand{\Bs}{\ensuremath{\mathscr{B}}}
\newcommand{\Cs}{\ensuremath{\mathscr{C}}}
\newcommand{\Ds}{\ensuremath{\mathscr{D}}}
\newcommand{\Fs}{\ensuremath{\mathscr{F}}}
\newcommand{\Gs}{\ensuremath{\mathscr{G}}}
\newcommand{\Hs}{\ensuremath{\mathscr{H}}}
\newcommand{\Ks}{\ensuremath{\mathscr{K}}}
\newcommand{\Ms}{\ensuremath{\mathscr{M}}}
\newcommand{\Ss}{\ensuremath{\mathscr{S}}}
\newcommand{\Ts}{\ensuremath{\mathscr{T}}}
\newcommand{\Ws}{\ensuremath{\mathscr{W}}}
% \newcommand{\Xs}{\ensuremath{\mathscr{X}}}
\newcommand{\Zs}{\ensuremath{\mathscr{Z}}}
\newcommand{\Cch}{\ensuremath{\check{C}}}
\newcommand{\Hch}{\ensuremath{\check{H}}}
\renewcommand{\a}{\ensuremath{\mathfrak{a}}}
\renewcommand{\b}{\ensuremath{\mathfrak{b}}}
\renewcommand{\c}{\ensuremath{\mathfrak{c}}}
\newcommand{\g}{\ensuremath{\mathfrak{g}}}
\newcommand{\h}{\ensuremath{\mathfrak{h}}}
\newcommand{\m}{\ensuremath{\mathfrak{m}}}
\newcommand{\n}{\ensuremath{\mathfrak{n}}}
\renewcommand{\o}{\ensuremath{\mathfrak{o}}}
\newcommand{\p}{\ensuremath{\mathfrak{p}}}
\newcommand{\q}{\ensuremath{\mathfrak{q}}}
\renewcommand{\r}{\ensuremath{\mathfrak{r}}}
\newcommand{\s}{\ensuremath{\mathfrak{s}}}
\renewcommand{\t}{\ensuremath{\mathfrak{t}}}
\renewcommand{\u}{\ensuremath{\mathfrak{u}}}
\renewcommand{\d}{\ensuremath{\partial}}


%% Operators
\renewcommand{\Im}{\mathop{\textrm{Im}}\nolimits}
\renewcommand{\Re}{\mathop{\textrm{Re}}\nolimits}
\DeclareMathOperator{\tr}{tr}
\DeclareMathOperator{\Tr}{Tr}
\DeclareMathOperator{\Hom}{Hom}
\DeclareMathOperator{\Aut}{Aut}
\DeclareMathOperator{\Inn}{Inn}
\DeclareMathOperator{\Out}{Out}
\DeclareMathOperator{\Ker}{Ker}
\DeclareMathOperator{\Coker}{Coker}
\DeclareMathOperator{\Obj}{Obj}
\DeclareMathOperator{\Mor}{Mor}
\DeclareMathOperator{\Kom}{Kom}
\DeclareMathOperator{\Gal}{Gal}
\DeclareMathOperator{\Spec}{Spec}
\DeclareMathOperator{\mSpec}{\mathfrak{m}Spec}
\DeclareMathOperator{\Proj}{Proj}
\DeclareMathOperator{\Tor}{Tor}
\DeclareMathOperator{\Ext}{Ext}
\DeclareMathOperator{\End}{End}
\DeclareMathOperator{\Rep}{Rep}
\DeclareMathOperator{\Ind}{Ind}
\DeclareMathOperator{\Res}{Res}
\DeclareMathOperator{\Ann}{Ann}
\DeclareMathOperator{\Ass}{Ass}
\DeclareMathOperator{\Frac}{Frac}
\DeclareMathOperator{\Homc}{\mathcal{H}om}
\DeclareMathOperator{\cl}{cl}
\DeclareMathOperator{\cp}{cp}
\DeclareMathOperator{\id}{id}
\DeclareMathOperator{\ev}{ev}
\DeclareMathOperator{\height}{ht}
\newcommand{\nil}{\textrm{nil}}
\DeclareMathOperator*{\colim}{colim}
\DeclareMathOperator{\codim}{codim}
\DeclareMathOperator{\depth}{depth}
\DeclareMathOperator{\ch}{ch}
\DeclareMathOperator{\characteristic}{char}
\DeclareMathOperator{\proj}{proj}
\DeclareMathOperator{\rad}{rad}
\DeclareMathOperator{\Conv}{Conv}
\DeclareMathOperator{\Hull}{Hull}
\DeclareMathOperator{\cone}{cone}
\DeclareMathOperator{\Supp}{Supp}
\DeclareMathOperator{\rank}{rank}
\DeclareMathOperator{\Rank}{Rank}
\DeclareMathOperator{\Span}{Span}
\DeclareMathOperator{\Map}{Map}
\DeclareMathOperator{\Mod}{Mod}
\DeclareMathOperator{\Cl}{Cl}
\DeclareMathOperator{\CaCl}{CaCl}
\DeclareMathOperator{\Pic}{Pic}
\DeclareMathOperator{\Fl}{Fl}
\DeclareMathOperator{\Gr}{Gr}
\DeclareMathOperator{\VF}{VF}
\DeclareMathOperator{\II}{II}
\DeclareMathOperator{\ad}{ad}
\DeclareMathOperator{\Ad}{Ad}
\DeclareMathOperator{\Adj}{Adj}
\DeclareMathOperator{\Pf}{Pf}
\DeclareMathOperator{\PD}{PD}
\DeclareMathOperator{\Vol}{Vol}
\DeclareMathOperator{\vol}{vol}
\DeclareMathOperator{\Ric}{Ric}
\DeclareMathOperator{\Lie}{Lie}
\DeclareMathOperator{\Sp}{Sp}
\newcommand{\op}{\textrm{op}}
\newcommand{\red}{\textrm{red}}
\DeclareMathOperator{\Sh}{Sh}
\DeclareMathOperator{\PSh}{PSh}
\DeclareMathOperator{\Ab}{\mathbf{Ab}}
\DeclareMathOperator{\Sets}{\mathbf{Sets}}
\DeclareMathOperator{\Sing}{Sing}
\DeclareMathOperator{\sign}{sign}
\DeclareMathOperator{\divisor}{div}
\DeclareMathOperator{\divergence}{div}
\DeclareMathOperator{\Hilb}{Hilb}
\DeclareMathOperator{\gl}{\mathfrak{gl}}
\DeclareMathOperator{\so}{\mathfrak{so}}
\DeclareMathOperator{\su}{\mathfrak{su}}
\renewcommand{\sl}{\mathop{\mathfrak{sl}}\nolimits}
\DeclareMathOperator{\ord}{ord}
\DeclareMathOperator{\lcm}{lcm}
\DeclareMathOperator{\simp}{simp}
\DeclareMathOperator{\cell}{cell}
\DeclareMathOperator{\ab}{ab}
\DeclareMathOperator{\diag}{diag}
\DeclareMathOperator{\res}{res}
\DeclareMathOperator{\interior}{int}
\DeclareMathOperator{\tors}{tors}
\DeclareMathOperator{\arccot}{arccot}
\DeclareMathOperator{\arcsec}{arcsec}
\DeclareMathOperator{\arccsc}{arccsc}
\DeclareMathOperator{\sech}{sech}
\DeclareMathOperator{\csch}{csch}
\DeclareMathOperator{\arcsinh}{arcsinh}
\DeclareMathOperator{\arccosh}{arccosh}
\DeclareMathOperator{\arctanh}{arctanh}
\DeclareMathOperator{\arccoth}{arccoth}
\DeclareMathOperator{\arcsech}{arcsech}
\DeclareMathOperator{\arccsch}{arccsch}
\newcommand{\extp}{\textstyle\bigwedge\nolimits}
% \newcommand{\dmod}{\textrm{-}\textbf{mod}}
\newcommand{\dmod}{\textrm{-mod}}


%% Notation
\newcommand{\set}[1]{\ensuremath{\{ {#1} \}}}
\newcommand{\dset}[1]{\ensuremath{\left\{ {#1} \right\}}}
\newcommand{\setdf}[2]{\ensuremath{\set{{#1} \,\mid\, {#2}}}}
\newcommand{\dsetdf}[2]{\ensuremath{\left\{ {#1} \; \middle| \; {#2} \right\}}}
\newcommand{\divides}[2]{\ensuremath{{#1} | {#2}}}
\newcommand{\ndivides}[2]{\ensuremath{{#1} \nmid {#2}}}
\renewcommand{\mod}[1]{\ensuremath{\textrm{mod } {#1}}}
\newcommand{\modsp}[1]{\ensuremath{\textrm{ mod } {#1}}}
\newcommand{\congruent}[3]{\ensuremath{{#1} \equiv {#2} \left( \mod{#3} \right)}}
\newcommand{\generate}[1]{\ensuremath{\langle {#1} \rangle}}
\newcommand{\card}[1]{\ensuremath{|{#1}|}}
\newcommand{\abs}[1]{\ensuremath{\left|{#1}\right|}}
\newcommand{\conj}[1]{\ensuremath{\overline{#1}}}
\newcommand{\dual}[1]{\ensuremath{\widehat{#1}}}
\newcommand{\bidual}[1]{\ensuremath{\widehat{\widehat{#1}}}}
\newcommand{\inprod}[1]{\ensuremath{\langle #1 \rangle}}


%% Shortcuts
\newcommand{\x}{\ensuremath{\times}}
\newcommand{\wtilde}[1]{\widetilde{#1}}
\newcommand{\what}[1]{\widehat{#1}}
\newcommand{\wcup}{\smallsmile}
\newcommand{\wcap}{\smallfrown}
\newcommand{\cn}{\colon}
\newcommand{\rarr}{\rightarrow}
\newcommand{\Rarr}{\Rightarrow}
\newcommand{\hrarr}{\hookrightarrow}
\newcommand{\xrarr}{\xrightarrow}
\newcommand{\larr}{\leftarrow}
\newcommand{\Larr}{\Leftarrow}
\newcommand{\hlarr}{\hookleftarrow}
\newcommand{\xlarr}{\xleftarrow}
\newcommand{\tl}{\triangleleft}
\newcommand{\tleq}{\trianglelefteq}

%% Tweaks

\renewcommand{\textrm}{\textnormal}
\renewcommand{\phi}{\varphi}
\renewcommand{\epsilon}{\varepsilon}

\setlength{\parindent}{0in}
%\setlength{\parskip}{0.2\baselineskip}
% \oddsidemargin -0.125in
% \evensidemargin -0.125in
% \textwidth 6.75in
% \topmargin -0.15in
% \textheight 8.95in

%%% Local Variables: 
%%% mode: latex
%%% TeX-master: "Study Guide"
%%% End: 


\begin{document}

\title[Study Guide, Harvard Mathematics Qualification Exam]{Study Guide \\ Harvard Mathematics Qualification Exam}
\author[A. Atanasov]{Atanas Atanasov}
\email{nasko@math.harvard.edu}
\author[C. Sia]{Charmaine Sia}
\email{sia@math.harvard.edu}
\date{}

\maketitle

\setcounter{tocdepth}{2}
\tableofcontents

Below is a set of guidelines which were used in the compilation of this document. They do not reflect any absolute practice in typesetting, but merely the combination of my own customs and some rational decisions which I took to standardize various parts of the document.
\begin{itemize}
\item When taking notes from a book, try to copy all definitions, propositions, theorems, and important remarks. Do not hesitate to add your own relevant observations as these could facilitate the learning process.
\item If there is a change of sentence structure which would simplify the statement of results, then apply that.
\item When the book in question is slightly old (e.g., Rudin, \emph{Real and Complex Analysis}), we should change mathematical notation in order to update the text.
\item Use as many of the predefined customizations as possible.
\item Avoid the use of ``one-to-one'' and ``onto''. Instead replace these with ``injective'' and ``surjective'' respectively.
\item Attempt to utilize any standard modern notation, e.g.\ $S^1$, Lie groups, etc.
\item If it is customary to use employ certain symbols for an object, then attempt to do so.
\item Use $\setminus$ for the difference of sets instead of $-$.
\item Use \texttt{\textbackslash cn} (without the customizations \texttt{\textbackslash colon}) for colons in functions, that is, visually $f \cn X \rarr Y$ looks better than $f : X \rarr Y$ (note the difference in spacing before the colon).
\item Be consistent with wording and spelling. For example, use holomorphic instead of analytic throughout. On a similar note, hyphenate ``non'' constructions such as ``non-constant'' and ``non-vanishing''.
\end{itemize}
\vspace{\baselineskip}

TODO:
\begin{itemize}
\item Add ``subsubsections'' which will not display in the table of contents. This is done to logically separate the various results in a chapter.
\item Check all files and convert adjectives ``nonX'' to ``non-X''.
\end{itemize}

% Chapter 1 -- Notes
\chapter{Notes}
\label{S:notes}

\section{Algebra}
\label{S:algebra}

Syllabus
\begin{description}
\item[Undergraduate] Dummit \& Foote, \emph{Abstract Algebra}, except chapters 15, 16 and 17. (math 122, 123)
\end{description}

\subsection{Dummut \& Foote, Preliminaries}

\begin{itemize}
\item Basics
  \begin{itemize}
  \item Sets
  \item Functions/maps and related properties -- bijective, surjective, left/right inverse, image/preimage, permutation, restriction/extension
  \item Relations -- reflexive, symmetric, transitive, equivalence relations, partitions, the equivalent of the last two
  \end{itemize}
\item Integers
  \begin{itemize}
  \item Ring structure
  \item Well ordering
  \item GCD \& LCM
  \item Division algorithm
  \item Euclidean algirithm
  \end{itemize}
\item Integers modulo $n > 1$
  \begin{itemize}
  \item Quotient of $\Z$
  \item Ring structure
  \end{itemize}
\end{itemize}

\subsection{Dummut \& Foote, Chapter 1: Introduction to Groups}

\begin{itemize}
\item Definition of a group, properties, and examples -- associativity, commutativity, inverses, order of an element (denoted by $|-|$)
\item Dihedral groups
  \begin{itemize}
  \item Notation $D_{2n}$, and $|D_{2n}| = 2n$
  \item $D_{2n} = \langle r,s \;|\; r^n = s^2 = 1, r s = s r^{-1} \rangle$
  \end{itemize}
\item Symmetric group -- presented as bijections, cycles, order as LCM, conjugacy classes and partitions
\item Matrix groups \\
  If $|F| = q < \infty$, then $|GL(n, F)| = (q^n-1)(q^n-q)\cdots(q^n-q^{n-1})$.
\item The Quaternion group \\
  $Q_8 = \{\pm 1, \pm i, \pm j, \pm k\}$ \\
  $i^2 = j^2 = k^2 = -1$ \\
  $i \cdot j = k$, $j \cdot k = i$, $k \cdot i = j$, and exchange introduces a sign
\item Homomorphisms and isomorphisms -- definition
\item Group actions -- definition
\end{itemize}

\subsection{Dummut \& Foote, Chapter 2: Subgroups}

\begin{itemize}
\item Subgroup criterion -- $H \subset G$ is a subgroup if $H \neq \emptyset$ and for all $x,y \in H$, we have $x y^{-1} \in H$
\item Centralizer -- for any $A \subset G$, its centralizer is $C_G(A) = \{ g \in G \;|\; g a = a g \textrm{ for all } a \in A \}$
\item Center -- $Z(G) = C_G(G)$
\item Normalizer -- for any $A \subset G$, its normalizer is $N_G(A) = \{ g \in G \;|\; g A g^{-1} = A \}$
  \begin{itemize}
  \item $N_G(A)$ is always a normal subgroup
  \item If $A$ is a subgroup, then $N_G(A)$ is the largest subgroup such that $A \tleq N_G(A)$.
  \item $C_G(A) \tleq N_G(A)$
  \end{itemize}
\item Cyclic groups and cyclic subgroups
  \begin{itemize}
  \item Any subgroup of a cyclic subgroup is cyclic.
  \item Consider a cyclic group of size $n$. For each $m|n$, there is a unique subgroup of size $m$.
  \item The number of generators of a cyclic group of size $n$ is $\phi(n)$.
  \end{itemize}
\item Subgroups generated by subsets of a group -- definition, well-defined
\item The lattice of a subgroups of a group
\end{itemize}

\subsection{Dummut \& Foote, Chapter 3: Quotient Groups and Homomorphisms}

\begin{itemize}
\item Lagrange's Theorem
\item First Isomorphism Theorem
\item Kernels are always normal, and each normal subgroup occurs as the kernel of a homomorphism (e.g.\ the quotient one).
\item Normality is not transitive.
\item The only group of prime order $p$ is $\Z/p\Z$.
\item Every subgroup of index 2 is normal.
\item If $H, K \leq G$ are finite, then $|H K| = |H| \, |K|/|H \cap K|$.
\item The set $H K$ is a group if and only if $H K = K H$.
\item If $H \leq N_G(K)$, then $H K$ is a subgroup of $G$. In particular, if $K \tleq G$, then $H K$ is a group for all $H \leq G$.
\item $|G/H| = |G|/|H|$
\item Second (Diamond) Isomorphism Theorem: Let $A, B \leq G$ and $A \leq N_G(B)$. Then $A \cap B \tleq A$ and
  \[
  A B/B \cong A/(A \cap B).
  \]
  In particular, the conclusion holds true if $B \tleq G$.
\item Third Isomorphism Theorem: If $H, K \tleq G$ and $H \leq K$, then $K/H \tleq G/H$ and
  \[
  (G/H)/(K/H) \cong G/K.
  \]
\item Fourth Isomorphism Theorem: for $N \tleq G$ there is a correspondence between the subgroups of $G/N$ and those of $G$ containing $N$ given by taking preimages under the quotient map $G \rarr G/N$; the correspondence presernes normality and inclusions, has numerous other expected properties.
\item A group $G$ is called \emph{simple} if it is nontrivial, and its only normal subgroups are $1$ and $G$.
\item A sequence of groups
  \[
  1 = N_0 \leq N_1 \leq \cdots \leq N_{k-1} \leq N_k = G
  \]
  is called a \emph{composition series} for $G$ if $N_{i-1} \tleq N_i$ and $N_i/N_{i-1}$ is simple for all $1 \leq i \leq k$.
\item Jordan-H\"older: Every finite group has a composition series, and its factors are unique up to permutation.
\item H\"older program: (1) classify finite simple groups, and (2) the ways of ``putting them together''
\item Examples of simple groups: $\Z/p\Z$, $SL(n, F)/Z(SL(n,F))$ for any finite field $F$ and $n \geq 2$ (except $SL(2,\F_2)$ and $SL(2,\F_3)$).
\item Every element of $S_n$ may be written as a product of transpositions.
\item The alternating group $A_n$ as the kernel of the homomorphism $\sign \cn S_n \rarr \{\pm 1\}$.
\end{itemize}

\subsection{Dummut \& Foote, Chapter 4: Group Actions}

\begin{itemize}
\item $|G| = |G_x| \, |G x|$
\item Let $H$ be a subgroup of the finite group $G$, and $G$ acts on the set of left cosets $A = G/H$. Then:
  \begin{enumerate}[(a)]
  \item $G$ acts transitively on $A$;
  \item the stabilizer of the point $1 H \in A$ is $H$;
    \item the kernel of this representation is $\bigcap_{x \in G} x H x^{-1}$ which is the largest normal subgroup of $G$ contained in $H$.
  \end{enumerate}
\item Cayley's Theorem: Every group $G$ of order $n$ is isomorphic to a subgroup of $S_n$.
\item If $G$ is a finite subgroup of order $n$ and $p$ is the smallest prime dividing $|G|$, then any subgroup of index $p$ is normal.
\item The number of conjugates of a subset $S \subset G$ is $[G:N_G(S)]$. In particular, for $s \in G$ we have $N_G(s) = C_G(s)$, so the number of conjugates of $s$ is $[G:N_G(s)]$.
\item The Class Equation: Let $G$ be a finite group and $g_1, \dots, g_r$ be representatives of the distinct conjugacy classes of $G$ not contained in the center $Z(G)$. Then
  \[
  |G| = |Z(G)| + \sum_{i=1}^r [G:C_G(g_i)].
  \]
\item Any group of prime power order has a nontrivial center.
\item If $G/Z(G)$ is cyclic, then $G$ is abelian.
\item Any group of order $p^2$ is abelian, hence isomorphic to either $\Z/p^2\Z$ or $(\Z/p\Z)^2$.
\item 
\end{itemize}

TODO

\subsection{Dummut \& Foote, Chapter 5: Direct and Semidirect Products of Abelian Groups}

TODO

\subsection{Dummut \& Foote, Chapter 6: Further Topics in Group Theory}

TODO

\subsection{Dummut \& Foote, Chapter 7: Introduction to Rings}

TODO

\subsection{Dummut \& Foote, Chapter 8: Euclidean Domains, Principal Ideal Domains and Unique Factorization Domains}

TODO

\subsection{Dummut \& Foote, Chapter 9: Polynomial Rings}

TODO

\subsection{Dummut \& Foote, Chapter 10: Introduction to Module Theory}

TODO

\subsection{Dummut \& Foote, Chapter 11: Vector Spaces}

TODO

\subsection{Dummut \& Foote, Chapter 12: Modules over Principal Ideal Domains}

TODO

\subsection{Dummut \& Foote, Chapter 13: Field Theory}

TODO

\subsection{Dummut \& Foote, Chapter 14: Galois Theory}

TODO

\subsection{Dummut \& Foote, Chapter 18: Representation Theory and Character Theory}

TODO

\subsection{Dummut \& Foote, Chapter 19: Examples and Applications of Character Theory}

TODO

%%% Local Variables: 
%%% mode: latex
%%% TeX-master: "Study_Guide"
%%% End: 

\section{Algebraic Geometry}
\label{S:algebraic-geometry}

Syllabus
\begin{description}
\item[Graduate] Harris, \emph{Algebraic geometry, a first course}, lectures 1-7, 11, 13, 14, 18. (math 137 and math 232a)
\end{description}

\subsection{Harris, Lecture 1: Affine and Projective Varieties}

\begin{itemize}
\item
  An inclusion-exclusion type formula holds for dimensions of linear spaces in projective space. Namely, if $\Lambda$ and $\Lambda'$ are linear spaces in a projective space $\P^n$ and $\overline{\Lambda,\Lambda'}$ is their span, then
\[
\dim(\overline{\Lambda,\Lambda'}) = \dim(\Lambda) + \dim(\Lambda') - \dim(\Lambda \cap \Lambda').
\]
Note that we are taking the dimension of the empty set as a linear space to be $-1$. A similar formula holds for linear (not affine) subspaces of affine space.

\item
  A set $\Gamma \subset \P^n$ of $d$ points may be described as the vanishing set of polynomials of degree $d$ and less. We say $\Gamma$ has \emph{degree} $d$. If $\Gamma$ is not contained in a line, then we may describe it by polynomials of degree $d-1$ and less.

\item
  We say that a set of points $p_i = [v_i] \in \P^n$ are \emph{independent} if so are the corresponding vectors $v_i$. The space $\P^n$ can contain at most $n+1$ independent points.

\item
  We say that a set of points $\Gamma \subset \P^n$ are in \emph{general position} if no $n+1$ or fewer of them are linearly dependent.

  \begin{theorem}
    For $k \geq 2$, any collection $\Gamma \subset \P^n$ of $d \leq k n$ points in general position may be described by polynomials of degree $k$ or less. This bound is sharp.
  \end{theorem}
  
  \begin{theorem}
    Any two ordered sets of $n+2$ points in general position in $\P^n$ are projectively equivalent.
  \end{theorem}

\item
  A hypersurface $X$ is a subvariety of $\P^n$ described as the zero locus of a single homogeneous polynomial $F$. We may always choose $F$ without repeated prime factors, and in this case, $F$ is unique up to multiplication by scalars (this requires the Nullstellensatz and holds only over algebraically closed fields). When this is done the degree of $F$ is called the \emph{degree} of the hypersurface $X$.

\item
  A \emph{complex analytic} variety is one which is the locus of (homogeneous) holomorphic functions on $\A_\C^n$ or $\P_\C^n$. All varieties in $\A_\C^n$ and $\P_\C^n$ are complex analytic since polynomials are holomorphic functions. The converse holds in projective but not in affine complex space.

  \begin{theorem}[Chow's Theorem]
    Any complex analytic subvariety $X \subset \P^n_\C$ is an algebraic variety.
  \end{theorem}
\end{itemize}

\subsection{Harris, Lecture 2: Regular Functions and Maps}

\begin{itemize}
\item
  For an affine variety $X \subset \A^n$, we define the ideal $I(X) \subset K[z_1, \dots, z_n]$ of polynomials which vanish at $X$. The \emph{coordinate ring} of $X$ is the quotient $A(X) = K[z_1,\dots,z_n]/I(X)$.
\item
  Consider an open $U \subset X$ of an affine variety, and $p \in U$ a point. We say a function $f$ on $U$ is regular at $p$ if in some neighbourhood $V$ of $p$ it is expressible as a quotient $g/h$ where $g,h \in K[z_1, \dots, z_n]$ are polynomials and $h(p) \neq 0$. We say $f$ is regular on $U$ if it is regular at all $p \in U$.

  \begin{lemma}
    The ring of regular functions regular at every point of an affine variety $X$ is the coordinate ring $A(X)$. More generally, if $U = U_f$ is a distinguished open, then the ring of regular function on $U$ is the localization $A(X)_f = A(X)[1/f]$.
  \end{lemma}

\item
  Similarly, for a projective variety $X \subset \P^n$, one can define an ideal $I(X)$ generated by all homogeneous polynomials vanishing at $X$. The \emph{homogeneous coordinate ring} of $X$ is defined by $S(X) = K[z_0,\dots,z_n]/I(X)$. Note that (the isomorphism class of) $S(X)$ is invariant under projective equivalence but not under general isomorphism of projective varieties.

\item
  If $G$ is a homogeneous polynomial, then the ring of regular functions on complement of the zero locus of $G$ is given by the $0$-th graded piece of the localization $S(X)_G = S(X)[1/G]$.

\item
  For any two positive integers $n$ and $d$, we define the \emph{Veronese map (embedding)} of degree $d$ as
  \begin{align*}
    \nu_d \cn \P^n &\rarr \P^N \\
    [X_0, \dots, X_n] &\mapsto [ \dots, X^I, \dots ]
  \end{align*}
  where $X^I$ ranges over all monomials of degree $d$ in $X_0, \dots, X_n$. It possesses the property that all hypersurfaces of degree $d$ in $\P^n$ are exactly the hyperplane sections of $\nu_d(\P^n) \subset \P^N$. The Veronese map carries any subvariety $X \subset \P^n$ isomorphically to its image $\nu_d(X)$. The dimension of the codomain can be deduced to be $N = \binom{n+d}{d}-1$. The image of a Veronese map is called a \emph{Veronese variety}.

\item
  For any positive integers $m$ and $n$, we define the associated \emph{Segre map}
  \begin{align*}
    \sigma \cn \P^n \x \P^m &\rarr \P^{(n+1)(m+1)-1} \\
    ([X_0,\dots,X_m],[Y_0,\dots,Y_n]) &\mapsto [\dots,X_i Y_j,\dots],
  \end{align*}
  where the coordinates in the target space range over all pairwise products of coordinates $X_i$ and $Y_j$. The image of $\sigma$ is called a \emph{Segre variety} and is denoted $\Sigma_{m,n}$.
\end{itemize}

\subsection{Harris, Lecture 3: Cones, Projections, and More About Products}

\begin{itemize}
\item
  Every quadric hypersurface $Q \subset \P^n$ can be put in the form $X_0^2 + \cdots + X_k^2 = 0$ for some integer $k \geq 0$. We call $k+1$ the \emph{rank} of $Q$.
  
\item
  Consider two polynomials $f$ and $g$ of degrees $m$ and $n$ respectively with coefficients in a field $K$. We would like to investigate when $f$ and $g$ have a common factor. This happens if and only if there exists a polynomial $h$ of degree $m+n-1$ divisible by both. Equivalently, this means that the polynomials $f$, $z \cdot f$, \dots, $z^{n-1} \cdot f$, $g$, $z \cdot g$, \dots, $z^{m-1} \cdot g$ fail to be linearly independent. In turn this is equivalent to the statement that the determinant
  \[
  R(f,g) =
  \left|
    \begin{array}{ccccccccccc}
      a_0 & a_1 & \cdot & \cdot & a_m & 0   & 0 & \cdot & \cdot & \cdot & 0 \\
      0   & a_0 & a_1 & \cdot & \cdot & a_m & 0 & \cdot & \cdot & \cdot & 0 \\
      \vdots \\
      0   & \cdot & \cdot & 0 & a_0 & a_1 & \cdot & \cdot & \cdot & \cdot & a_m \\
      b_0 & b_1 & \cdot & \cdot & \cdot & b_n & 0 & \cdot & \cdot & \cdot & 0 \\
      0 & b_0 & b_1 & \cdot & \cdot & \cdot & b_n & 0 & \cdot & \cdot & 0 \\
      \vdots \\
      0 & \cdot & \cdot & 0 & b_0 & b_1 & \cdot & \cdot & \cdot & \cdot & b_n
    \end{array}
  \right|
  \]
  is zero. This determinant $R(f,g)$ is often called the \emph{resultant} of $f$ and $g$. In fact, the definition above makes sense for polynomials with coefficients in any ring.

  \begin{theorem}
    Two polynomials $f$ and $g$ in one variable over a field $K$ will have a common factor if and only if $R(f,g) = 0$.
  \end{theorem}

  The resultant is useful in the proof of the following result.
  
  \begin{theorem}
    The projection $\overline{X}$ of $X \subset \P^n$ from $p \notin X$ to $\P^{n-1}$ is a projective variety.
  \end{theorem}

\item
  The image of a closed set is not always closed. In special cases, this could however be derived.
  
  \begin{theorem}
    Let $Y$ be any variety and $\pi \cn Y \x \P^n \rarr Y$ be the projection on the first factor. Then the image $\pi(X)$ of any closed subset $X \subset Y \x \P^n$ is a closed subset of $Y$.
  \end{theorem}

  This leads to the following.

  \begin{theorem}
    If $X \subset \P^n$ is any projective variety and $\phi \cn X \rarr \P^m$ any regular map, then the image of $\phi$ is a projective subvariety of $\P^m$.
  \end{theorem}

  \begin{corollary}
    If $X \subset \P^n$ is any connected variety and $f$ any regular function on $X$, then $f$ is constant.
  \end{corollary}

  \begin{corollary}
    If $X \subset \P^n$ is any connected variety other than a point and $Y \subset \P^n$ is any hypersurface then $X \cap Y = \emptyset$.
  \end{corollary}

\item
  We call a set \emph{constructible} if it is expressible as the union of locally closed subsets. Equivalently, a subset $Z \subset \P^n$ is constructible if there exists a nested sequence $X_1 \supset X_2 \supset \cdots \supset X_n$ of closed subsets of $\P^n$ such that
  \[
  Z = X_1 \setminus (X_2 \setminus (X_3 \setminus \cdots \setminus X_n)).
  \]

  \begin{theorem}[Chevalley]
    Let $X \subset \P^m$ be a quasi-projective variety, $f \cn X \rarr \P^n$ a regular map, and $U \subset X$ any constructible set. Then $f(U)$ is a constructible subset of $\P^n$.
  \end{theorem}
\end{itemize}

\subsection{Harris, Lecture 4: Families and Parameter Spaces}

\begin{itemize}
\item
  A \emph{family of projective varieties} in $\P^n$ with base $B$ is simply a closed subvariety $\Vc \subset B \x P^n$. The fibers $V_b = (\pi_1)^{-1}(b)$ of $\Vc$ over points of $b$ are referred to as \emph{members of the family}.

\item
  Let $X \subset \P^n$ be any projective variety and $\{ V_b \}$ any family of projective varieties in $\P^n$ with base $B$. Then the set
  \[
  \{ b \in B \;|\; X \cap V_b \neq \emptyset \}
  \]
  is closed in $B$. More generally, if $\{ W_b \}$ is another family of projective varieties in $\P^n$ with base $B$, then
  \[
  \{ b \in B \;|\; V_b \cap W_b \neq \emptyset \}
  \]
  is a closed subvariety of $B$.

\item
  Even more generally, if $\{ W_c \}$ is a family of projective varieties in $\P^n$ with a possibly different base $C$, then the set
  \[
  \{ (b,c) \in B \x C \;|\; V_b \cap W_c \neq \emptyset \}
  \]
  is a closed subvariety of $B \x C$.

\item
  In a similar vein, for any $X \subset \P^n$ and family $\{ V_b \}$ the set
  \[
  \{ b \in B \;|\; V_b \subset X \}
  \]
  is constructible and the set
  \[
  \{ b \in B \;|\; X \subset V_b \}
  \]
  is closed in $B$.

\item
  Recall that points in ${\P^n}^\ast$ can be thought of as hyperplanes in $\P^n$. Under this identification, the set
  \[
  \Gamma = \{ (H, p) \in {\P^n}^\ast \x \P^n \;|\; p \in H \}
  \]
  is called the \emph{universal hyperplane}. The fiber over $H \in {\P^n}^\ast$ is precisely the hyperplane $H \subset \P^n$. We call it universal since for any flat family of hyperplanes $\Vc \subset B \x \P^n$ there exists a unique regular map $B \rarr {\P^n}^\ast$ such that the diagram
  \[\xymatrix{
    \Vc \ar[r] \ar[d] & \Gamma \ar[d] \\
    B \ar[r] & {\P^n}^\ast
  }\]
  is cartesian.

\item
  For any $X \subset \P^n$, the \emph{universal hyperplane section} is
  \[
  \Omega_X = \{ (H, p) \in {\P^n}^\ast \x \P^n \;|\; p \in H \cap X \} = (\pi_2)^{-1}(X),
  \]
  where $\pi_2 \cn \Gamma \rarr \P^n$ is the projection on the second factor of t he universal hyperplane $\Gamma$.

\item
  A \emph{section} of a family of projective varieties $\Vc \subset B \x \P^n$ is a map $\sigma \cn B \rarr \Vc$ such that $\pi_1 \circ \sigma = \id_B$. A \emph{rational section} is a section defined on some non-empty open subset $U \subset B$.
\end{itemize}

\subsection{Harris, Lecture 5: Ideals of Varieties, Irreducible Decomposition, and the Nullstellensatz}

\begin{itemize}
\item
  The following result enables us to establish a correspondence between geometry and algebra. Furthermore, it implies several very useful corollaries.

  \begin{theorem}[Nullstellensatz]
    For any ideal $I \subset K[z_1, \dots, z_n]$, the ideal of functions vanishing on the common zero locus of $I$ is the radical of $I$, i.e.,
    \[
    I(V(I)) = \rad(I).
    \]
    Thus, there is a bijective correspondence between subvarieties $X \subset \A^n$ and radical ideals $I \subset K[z_1, \dots, z_n]$.
  \end{theorem}

  \begin{theorem}[Weak Nullstellensatz]
    Any ideal $I \subset K[x_1, \dots, x_n]$ with no common zeros is the unit ideal.
  \end{theorem}

  \begin{theorem}
    Every prime ideal in $K[x_1, \dots, x_n]$ is the intersection of the ideals of the form $(x_1 - a_1, \dots, x_n - a_n)$ containing it.
  \end{theorem}

\item
  We say an ideal $I$ cuts out $X \subset \A^n$ \emph{set-theoretically} if $V(I) = X$; we say $I$ cuts out $X$ \emph{scheme-theoretically} if $I = I(X)$.

\item
  Consider the case of projective varieties. There is almost a bijection between closed varieties $X \subset \P^n$ and radical homogeneous ideals $I \subset K[Z_0, \dots, Z_n]$. Define the \emph{saturation} of $I$ to be
  \[
  \overline{I} = \{ F \in K[Z_0, \dots, Z_n] \;|\; (Z_0, \dots, Z_n)^k \cdot F \subset I \textrm{ fo rsome } k \}.
  \]

  \begin{proposition}
    The following conditions for a pair of homogeneous ideals $I, J \subset K[Z_0, \dots, Z_n]$ are equivalent:
    \begin{enumerate}[(i)]
    \item $I$ and $J$ have the same saturation;
    \item $I_m = J_m$ for all $m \gg 0$;
    \item $I$ and $J$ agree locally, that is, they generate the same ideal in each localization $K[Z_0, \dots, Z_n, Z_i^{-1}]$ of $K[Z_0, \dots, Z_n]$.
    \end{enumerate}
  \end{proposition}

  We say that an ideal $I$ cuts out $X \subset \P^n$ \emph{scheme-theoretically} if $\overline{I} = I(X)$.

\item
  A variety is called \emph{irreducible} if for any pair of closed subvarieties $Y, Z \subset X$ such that $Y \cup Z = X$, either $Y = X$ or $Z = X$.

\item
  An affine variety $X \subset \A^n$ is irreducible if and only if $I(X)$ is prime.

\item
  If a projective variety $X \subset \P^n$ is irreducible, then so is every non-empty open affine $U = X \cap \A^n$. The converse is also true if we consider all hyperplane compliments of $X$, but not if we restrict our attention only to the standard opens.

\item
  A variety is irreducible if and only if every Zariski open subset is dense, i.e., every two Zariski non-empty open subsets intersect.

\item
  The image of an irreducible variety under a regular map is irreducible.

\item
  The following are two potentially useful criteria for irreducibility.

  \begin{theorem}
    Let $X \subset \P^n$ be an irreducible variety, and let $\Omega_X \subset {\P^n}^\ast \x X$ be its universal hyperplane section. Then $\Omega_X$ is irreducible.
  \end{theorem}

  \begin{theorem}
    Let $f \cn Z \rarr Y$ is a regular map. Assume that
    \begin{enumerate}[(i)]
    \item $f$ is open;
    \item $Y$ is irreducible;
    \item for a dense open set of points $p \in Y$, the fiber $f^{-1}(p)$ is irreducible.
    \end{enumerate}
    Then $Z$ is irreducible.
  \end{theorem}

\item
  The following is useful in discussing fibers.

  \begin{proposition}
    Let $\pi \cn X \rarr Y$ be any regular map with $Y$ irreducible, and let $Z \subset X$ be any locally closed subset. Then for a general point $p \in Y$ the closure of the fiber $Z_p = Z \cap \pi^{-1}(p)$ is the intersection of the closure $\overline{Z}$ of $Z$ with the fiber $X_p = \pi_{-1}(p)$.
  \end{proposition}
\end{itemize}

\subsection{Harris, Lecture 6: Grassmannians and Related Varieties}

\begin{itemize}
\item
  TODO
\end{itemize}

TODO

\subsection{Harris, Lecture 7: Rational Functions and Rational Maps}

TODO

\subsection{Harris, Lecture 11: Definitions of Dimension and Elementary Examples}

TODO

\subsection{Harris, Lecture 13: Hilbert Polynomials}

TODO

\subsection{Harris, Lecture 14: Smoothness and Tangent Spaces}

TODO

\subsection{Harris, Lecture 18: Degree}

TODO

%%% Local Variables: 
%%% mode: latex
%%% TeX-master: "Study_Guide"
%%% End: 

\section{Complex Analysis}
\label{S:complex-analysis}

Syllabus
\begin{description}
\item[Undergraduate] Ahlfors, \emph{Complex Analysis} (2nd ed), chapters 1-4 and section 5.1. (math 113)
\item[Graduate] Ahlfors, \emph{Complex Analysis} (2nd ed), chapter 5, section 6.1, and 6.2. (math 213a)
\end{description}

\subsection{Ahlfors, Chapter 1: Complex Numbers}

\begin{proposition}[Lagrange's identity]
  The equality
  \[
  \left| \sum_{i=1}^n a_i b_i \right|^2 = \sum_{i=1}^n |a_i|^2 \sum_{i=1}^n |b_i|^2 - \sum_{1 \leq i < j \leq n} \left| a_i \overline{b_j} - a_j \overline{b_i} \right|^2.
  \]
  holds for all $a_1, b_1, \dots, a_n, b_n \in \C$.
\end{proposition}

The following inequalities hold true for all complex values.
\begin{gather*}
-|a| \leq \Re a \leq |a| \\
-|a| \leq \Im a \leq |a| \\
|a+b| \leq |a|+|b| \\
|a-b| \geq ||a|-|b|| \\
|a_1 b_1 + \cdots + a_n b_n|^2 \leq (|a_1|^2 + \cdots + |a_n|^2)(|b_1|^2 + \cdots + |b_n|^2)
\end{gather*}

\subsection{Ahlfors, Chapter 2: Complex Functions}

\subsubsection{Introduction to the Concept of Holomorphic Function}

\begin{definition}
  A function $u(x,y)$ is called \emph{harmonic} if it satisfies the differential equation
  \[
  \Delta u = \frac{\d^2 u}{\d x^2} + \frac{\d^2 u}{\d y^2} = 0.
  \]
\end{definition}

The real and imaginary parts of a holomorphic function are harmonic.

\begin{proposition}
  If $u(x,y)$ and $v(x,y)$ have continuous first-order partial derivatives which satisfy the \emph{Cauchy-Riemann differential equations}
  \begin{align*}
    \frac{\d u}{\d x} &=  \frac{\d v}{\d y}, &
    \frac{\d u}{\d y} &= -\frac{\d v}{\d x}
  \end{align*}
  then $f(z) = u(z) + i v(z)$ is holomorphic with homomorphic derivative $f'(z)$, and conversely. In complex form, these two equations become
  \[
  \frac{\d f}{\d x} = -i \frac{\d f}{\d y}.
  \]
\end{proposition}

We will use the informal notation
\begin{align*}
  \frac{\d f}{\d z} &= \frac{1}{2} \left( \frac{\d f}{\d x} - i \frac{\d f}{\d y} \right), &
  \frac{\d f}{\d \overline{z}} &= \frac{1}{2} \left( \frac{\d f}{\d x} + i \frac{\d f}{\d y} \right).
\end{align*}

\begin{theorem}[Gauss-Lucas]
  For any polynomial $P(z) \in \C[z]$, the roots of $P'(z)$ lie in the convex hull of the roots of $P(z)$.
\end{theorem}

\begin{definition}
  The \emph{order} of a rational function $R(z) = P(z)/Q(z)$ for $P,Q \in \C[z]$ is $\max\{\deg P, \deg Q\}$.
\end{definition}

\begin{proposition}
  A rational function $R$ of order $p$ has $p$ zeros and $p$ poles, and the equation $R(z) = a$ has exactly $p$ roots.
\end{proposition}

When counting poles, zeros, and solutions in the result above, we considered $R$ as a function on the Riemann sphere $\C\P^1 = \C \cup \{\infty\}$.

\begin{proposition}
  If $Q$ is a polynomial with distinct roots $\alpha_1, \dots, \alpha_n$, and if $P$ is a polynomial of degree $< n$, then
  \[
   \frac{P(z)}{Q(z)} = \sum_{k=1}^n \frac{P(\alpha_k)}{Q'(\alpha_k)(z-\alpha_k)}.
  \]
\end{proposition}

\subsubsection{Elementary Theory of Power Series}

\begin{corollary}[Lagrange's interpolation problem]
  There exists a unique polynomial of degree $< n$ with given values $c_k$ at the points $\alpha_k$.
\end{corollary}

The inequalities
\begin{gather*}
  \liminf \alpha_n + \liminf \beta_n \leq \liminf (\alpha_n + \beta_n) \leq \liminf \alpha_n + \limsup \beta_n, \\
  \liminf \alpha_n + \limsup \beta_n \leq \limsup (\alpha_n + \beta_n) \leq \limsup \alpha_n + \limsup \beta_n
\end{gather*}
hold for any two real sequences $\{\alpha_n\}$ and $\{\beta_n\}$.

\begin{proposition}
  The limit of a sequence of a uniformly converging continuous functions is continuous.
\end{proposition}

\begin{definition}
  Let $\{f_n\}$ be a sequence of functions. We say that a positive sequence $\{a_n\}$ is \emph{majorant} for $\{f_n\}$ if there exists a constant $M > 0$ such that $|f_n| \leq M a_n$ for all sufficiently large $n$. Conversely $\{f_n\}$ is called the \emph{minorant} of $\{a_n\}$.
\end{definition}

\begin{proposition}[Weierstrass $M$-test]
  If $\sum a_n$ converges, then $\sum f_n$ converges uniformly.
\end{proposition}

\begin{theorem}
  For each power series $\sum_n a_n z^n$ there exists a number
  \[
  R = \frac{1}{\limsup \sqrt[n]{|a_n|}}\in [0,\infty],
  \]
  called the \emph{radius of convergence}, with the following properties:
  \begin{enumerate}[(i)]
  \item the series converges absolutely for all $|z| < R$;
  \item if $0 \leq \rho < R$, then the convergence is uniform for $|z| \leq \rho$;
  \item if $|z| > R$, then the terms of the series are unbounded, hence the series does not converge;
  \item in $|z| < R$ the sum of the series is a holomorphic function; its derivative can be obtained via termwise derivation, and has identical radius of convergence.
  \end{enumerate}
\end{theorem}

\begin{proposition}
  If $\sum a_n z^n$ and $\sum b_n z^n$ have radii of convergence $R_1$ and $R_2$, then $\sum a_n b_n z^n$ has radius of convergence at least $R_1 R_2$.
\end{proposition}

\begin{proposition}
  If $\lim_n |a_n|/|a_{n+1}| = R$, then $\sum a_n z^n$ has radius of convergence $R$.
\end{proposition}

The following result can be used to analyze when a power series converges at a point of its circle of convergence. Without loss of generality, we assume $R = 1$ and $z = 1$.

\begin{theorem}[Abel's Limit Theorem]
  If $\sum a_n$ converges, then $f(z) = \sum a_n z^n$ tends to $f(1)$ as $z$ approaches $1$ in such a way that $|1-z|/(1-|z|)$ remains bounded. Geometrically this means that $z$ stays in an angle $<180^\circ$ with vertex $1$, symmetrically to the part $(-\infty,1)$ of the real axis (a \emph{Stolz angle}).
\end{theorem}

\subsection{Ahlfors, Chapter 3: Holomorphic Functions as Mappings}

\subsubsection{Elementary Point Set Topology}

\begin{definition}
  A metric space is called \emph{totally bounded} if, for every $\epsilon > 0$, $X$ can be covered by finitely map balls of radius $\epsilon$.
\end{definition}

\begin{theorem}
  A metric space is compact if and only if it is complete and totally bounded.
\end{theorem}

\begin{corollary}[Heine-Borel]
  A subset of $\R^n$ is compact if and only if it is closed and bounded.
\end{corollary}

\begin{theorem}
  A metric space is compact if and only if every infinite sequence has a limit point.
\end{theorem}

\begin{corollary}[Bolzano-Weierstrass]
  A metric space is compact if every bounded sequence has a convergent subsequence.
\end{corollary}

\begin{proposition}[Cantor's Lemma]
  If $E_1 \supset E_2 \supset \cdots$ is a decreasing sequence of non-empty compact sets, then $\bigcap_n E_n$ is not empty.
\end{proposition}

\begin{proposition}
  The following properties hold for any continuous map:
  \begin{enumerate}[(i)]
  \item the image of any compact sets is compact;
  \item the image of any connected set is connected.
  \end{enumerate}
\end{proposition}

\begin{corollary}
  A continuous bijective map with compact domain is a homeomorphism.
\end{corollary}

\begin{proposition}
  On a compact metric space, every continuous function is uniformly continuous.
\end{proposition}

\begin{definition}
  A connected open set $\Omega \subset \C$ is called a \emph{region}.
\end{definition}

\begin{theorem}
  A holomorphic function in a region $\Omega$ whose derivative vanishes identically is constant. The same is true if either the real part, the imaginary part, the modulus, or the argument is constant.
\end{theorem}

\subsubsection{Conformality}

\begin{definition}
  A holomorphic function $f \cn U \rarr \C$ is called \emph{conformal} if $f'$ does not vanish in $U$.
\end{definition}

Conformal maps are interesting for the following two properties.
\begin{proposition}
  Let $f$ be a conformal map.
  \begin{enumerate}[(i)]
  \item The angle between the image of a tangent vector at a point $z$ and the original vector is always $\arg f'(z)$.
  \item Infinitesimal segments near a point $z$ are scaled by $|f'(z)|$ independently of their direction.
  \end{enumerate}
\end{proposition}

\begin{corollary}
  Conformal maps preserve angles between curves.
\end{corollary}

Consider a $\Cc^1$ path $\gamma$ and a region $E$. It is customary to write
\begin{align*}
  L(\gamma) &= \int_\gamma |\gamma'(t)| d t = \int_\gamma |d z|, &
  A(E) &= \iint_E d x d y
\end{align*}
for the length of $\gamma$ and the area of $E$. The image path $f \circ \gamma$ and image region $f(E)$ under a conformal map $f$ satisfy
\begin{align*}
  L(f \circ \gamma) &= \int_\gamma |f'(\gamma(t))| |\gamma'(t)| d t = \int_\gamma |f'(z)| |d z|, &
  A(f(E)) &= \iint_E |f'(z)|^2 d x d y.
\end{align*}
In the case of a region, the image area is computed with multiplicity.

\subsubsection{Linear Transformations}

\begin{definition}
  Any $a,b,c,d \in \C$ satisfying $a d - b c \neq 0$ determine a biholomorphic map $\C\P^1 \rarr \C\P^1$ given by
  \[
  z \mapsto \frac{a z + b}{c z + d}.
  \]
  Such maps are called \emph{linear transformations}.  The \emph{group of linear transformations} is isomorphic to $PGL(2,\C)$ via
  \[
  \begin{pmatrix} a & b \\ c & d \end{pmatrix} \mapsto
  \left( z \mapsto \frac{a z + b}{c z + d} \right).
  \]
  A linear transformation is called \emph{normalized} if some corresponding matrix has determinant $1$. The \emph{group of normalized linear transformations} is isomorphic to $SL(2,\C)/\{\pm I\}$.
\end{definition}

\begin{proposition}
  For any distinct $z_2, z_3, z_4 \in \C\P^1$, there exists a unique linear transformation $T$ satisfying $T z_2 = 1$, $T z_3 = 0$, $T z_4 = \infty$.
\end{proposition}

\begin{definition}
  The \emph{cross ratio} $(z_1, z_2, z_3, z_4)$ is the image of $z_1$ under the linear transformation which carries $z_2$, $z_3$, $z_4$ into $1$, $0$, $\infty$.
\end{definition}

\begin{proposition}
  \mbox{}
  \begin{enumerate}[(i)]
  \item Any distinct $z_1, z_2, z_3, z_4 \in \C\P^1$ and any linear transformation $T$ satisfy
    \[
    (T z_1, T z_2, T z_3, T z_4) = (z_1, z_2, z_3, z_4).
    \]
  \item A cross ratio $(z_1, z_2, z_3, z_4)$ is real if and only if the four points lie on a circle.
    \item A linear transformation carries circles into circles.
  \end{enumerate}
\end{proposition}

Note that in (ii) and (iii) above, a circle in $\C\P^1$ refers to either a circle in $\C$ or a straight line in $\C$ together with $\infty$.

\begin{proposition}[Ptolemy's Theorem]
  If the consecutive vertices $z_1, z_2, z_3, z_4$ of a quadrilateral lie on a circle, then
  \[
  |z_1 - z_3| \cdot |z_2 - z_4| = |z_1 - z_2| \cdot |z_3 - z_4| + |z_1 - z_4| \cdot |z_2 - z_3|.
  \]
\end{proposition}

\begin{definition}
  Consider a linear transformation $T$ which carries the real axis to a circle $C$. For any $w$, we say the points $z = T w$ and $z^\ast = T \overline{w}$ are \emph{symmetric with respect to $C$}. Equivalently, if $C$ passes through $z_1, z_2, z_3$ then any pair of symmetric points $z, z^\ast$ satisfy $(z^\ast, z_1, z_2, z_3) = \overline{(z, z_1, z_2, z_3)}$.
\end{definition}

Note that the definition of symmetry is independent of the chosen linear transformation $T$ or the three points $z_1, z_2, z_3$ on $C$.

\begin{proposition}
  \mbox{}
  \begin{enumerate}[(i)]
  \item If $C$ is a line, then the symmetry operation is reflection in $C$.
  \item If $C$ is a circle, then the symmetry operation is what we call inversion in plane geometry.
  \end{enumerate}
\end{proposition}

\begin{proposition}[Symmetry principle]
  Let $T$ be a linear transformation which carries the circle $C_1$ into the circle $C_2$. If $z$ and $z^\ast$ are symmetric about $C_1$, then $T z$ and $T z^\ast$ are symmetric about $C_2$.
\end{proposition}

\begin{definition}
  An \emph{orientation} of a circle $C$ is determined by an oriented triple of distinct points $z_1, z_2, z_3 \in C$. We say that a point $z \notin C$ is on the \emph{right} of $C$ if $\Im(z, z_1, z_2, z_3) > 0$, and on the \emph{left} of $C$ if $\Im(z, z_1, z_2, z_3) < 0$.
\end{definition}

Each circle $C$ in the complex plane $\C$ can be endowed with the counterclockwise orientation. With respect to it, the interior of $C$ is on its left, and it exterior on its right.

\begin{definition}
  Consider a linear transformation
  \[
  T(z) = k \cdot \frac{z-a}{z-b}.
  \]
  The preimages under $T$ of straight lines through the origin are circles through $a$ and $b$ (denoted by $C_1$). The preimages of concentric circles about the origin are circles with the equation
  \[
  \left| \frac{z-a}{z-b} \right| = \rho/|k|,
  \]
  where $\rho$ is the radius of the original circle. These are called \emph{circles of Apollonius} with limit points $a$ and $b$ (denoted by $C_2$). The configuration formed by all the circles $C_1$ and $C_2$ is referred to as the \emph{circular net} or the \emph{Steiner circles} determined by $a$ and $b$.
\end{definition}

\begin{proposition}
  \mbox{}
  \begin{enumerate}[(i)]
  \item There is exactly one $C_1$ and one $C_2$ through each point in the plane with the exception of $a$ and $b$.
  \item Every $C_1$ meets every $C_2$ at right angles.
  \item Reflection in a $C_1$ transforms every $C_2$ into itself and every $C_1$ into another $C_1$. Reflection in a $C_2$ transforms every $C_1$ into itself and every $C_2$ into another $C_2$.
    \item The limits points are symmetric with respect to each $C_2$, but not with respect to any other circle.
  \end{enumerate}
\end{proposition}

The following two maps are often useful in constructing conformal isomorphisms.

\begin{proposition}
  Let $\Hc = \{ z \in \C \;|\; \Im z > 0 \}$ be the upper half-plane and $\Delta = \{ z \in \C \;|\; |z| < 1 \}$ the unit disk. The maps $F \cn \Hc \rarr \Delta$ and $G \cn \Delta \rarr \Hc$ given by
  \begin{align*}
    F(z) &= \frac{i-z}{i+z}, &
    G(w) = i \frac{1-w}{1+w}
  \end{align*}
  are conformal isomorphisms and inverses to each other.
\end{proposition}

\begin{theorem}
  If $f \cn \Delta \rarr \Delta$ is a conformal automorphism of the unit disk $\Delta$, then there exist $\theta \in \R$ and $\alpha \in \Delta$ such that
  \[
  f(z) = e^{i\theta} \frac{\alpha-z}{1-\overline{\alpha}z}.
  \]
\end{theorem}

\begin{corollary}
  Each conformal automorphism of the upper half-plane $\Hc$ is given by the action of a matrix in $SL(2,\R)$
\end{corollary}

\subsection{Ahlfors, Chapter 4: Complex Integration}

\subsubsection{Fundamental Theorems}

A definite integral of any complex-valued function $f$ satisfies
\[
\left| \int_a^b f(t) d t \right| \leq \int_a^b |f(t)| d t.
\]
The \emph{line (path) integral} of $f$ along a path $\gamma \cn [a,b] \rarr \C$ is defined as
\[
\int_\gamma f d z = \int_a^b f(\gamma(t)) \gamma'(t) d t,
\]
and it is independent of the parametrization of $\gamma$. This has the properties
\begin{align*}
  \int_{-\gamma} f d z &= -\int_\gamma f d z, &
  \int_{\gamma_1 + \cdots + \gamma_n} f d z &= \int_{\gamma_1} f d z + \cdots + \int_{\gamma_n} f d z.
\end{align*}
We can further define
\begin{align*}
  \int_\gamma f \overline{d z} &= \overline{\int_\gamma \overline{f} d z}, &
  \int_\gamma f d x &= \frac{1}{2} \left( \int_\gamma f d z + \int_\gamma f \overline{d z} \right), &
  \int_\gamma f d y &= \frac{1}{2i} \left( \int_\gamma f d z - \int_\gamma f \overline{d z} \right).
\end{align*}
The integral with respect to \emph{arc length} is defined as
\[
\int_\gamma f d s = \int_\gamma f |d z| = \int_\gamma f(\gamma(t)) |\gamma'(t)| d t,
\]
and it is again independent of parametrization. Among others, it satisfies the following two properties:
\begin{align*}
  \int_{-\gamma} f |d z| &= -\int_\gamma f |d z|, &
  \left| \int_\gamma f d z \right| &\leq \int_\gamma |f| \cdot |d z|.
\end{align*}
The \emph{length} of a path $\gamma$ is defined as
\[
\ell(\gamma) = \int_\gamma |d z|.
\]

\begin{theorem}
  The line integral $\int_\gamma p d x + q d y$, defined in $U \subset \C$, depends only on the endpoints of $\gamma$ if and only if there exists a function $F(x,y)$ on $U$ with partial derivatives $\d F/\d x = p$, $\d F/\d y = q$.
\end{theorem}

\begin{corollary}
  The integral $\int_\gamma f d z$, with continuous $f$, depends only on the endpoints of $\gamma$ if and only if $f$ is the derivative of a holomorphic function on $U$.
\end{corollary}

\subsubsection{Cauchy's Integral Formula}

\begin{definition}
  Let $\gamma$ be a piecewise differentiable closed curve in a region $\Omega$, and $a$ a point in $\Omega \setminus \gamma$. The \emph{index} (also called \emph{winding number}) of a point $a$ with respect to a curve $\gamma$ is given by the equation
  \[
  n(\gamma,a) = \frac{1}{2 \pi i} \int_\gamma \frac{d z}{z-a}.
  \]
\end{definition}

\begin{proposition}
  \mbox{}
  \begin{enumerate}[(i)]
  \item The index $n(\gamma,a)$ is an integer.
  \item $n(-\gamma,a) = -n(\gamma,a)$
  \item If $\gamma$ lies inside a circle, then $n(\gamma,a) = 0$ for all points outside of the same circle.
  \item As a function of $a$ the index $n(\gamma,a)$ is constant in each of the regions determined by $\gamma$, and zero in the unbounded region.
  \end{enumerate}
\end{proposition}

\begin{theorem}
  Suppose $f$ is holomorphic in an open disk $\Delta$, and let $\gamma$ be a closed path in $\Delta$. For any point $z \in \Delta \setminus \gamma$,
  \[
  n(\gamma,z) \cdot f(z) = \frac{1}{2 \pi i} \int_\gamma \frac{f(\zeta)}{\zeta-z} d \zeta.
  \]
\end{theorem}

When $\gamma$ is a circle centered at $a$, then $n(\gamma,a) = 1$. The resulting equality
\[
f(z) = \frac{1}{2 \pi i} \int_\gamma \frac{f(\zeta)}{\zeta-z} d \zeta
\]
is called \emph{Cauchy's integral formula}. The following is a generalization which allows us to compute higher derivatives in this fashion.

\begin{proposition}
  For any $n \geq 0$,
  \[
  f^{(n)}(z) = \frac{n!}{2 \pi i} \int_\gamma \frac{f(\zeta) \, d \zeta}{(\zeta - z)^{n+1}}.
  \]
\end{proposition}

The following two results follow.

\begin{theorem}[Morera's Theorem]
  If $f$ is defined and continuous in a region $\Omega$, and if $\int_\gamma f d z = 0$ for all closed curves $\gamma$ in $\Omega$, then $f$ is holomorphic in $\Omega$.
\end{theorem}

\begin{theorem}[Liouville's Theorem]
  A bounded entire function is constant.
\end{theorem}

\subsubsection{Local Properties of Holomorphic Functions}

\begin{theorem}
  Suppose that $f$ is holomorphic in $\Omega' = \Omega \setminus \{a\}$ where $\Omega$ is region. A necessary and sufficient condition that there exist a holomorphic function in $\Omega$ which coincides with $f$ in $\Omega'$ is that $\lim_{z \rarr a} (z-a)f(z) = 0$. The extended function is uniquely determined.
\end{theorem}

A recursive application of the previous result yields the following.

\begin{theorem}
  If $f$ is holomorphic in a region $\Omega$, containing $a$, it is possible to write
  \[
  f(z) = f(a) + \frac{f'(a)}{1!}(z-a) + \frac{f''(a)}{2!}(z-a)^2 + \cdots + \frac{f^{(n-1)}(a)}{(n-1)!}(z-a)^{n-1} + f_n(z) (z-a)^n
  \]
  where $f_n$ is holomorphic in $\Omega$. Furthermore, inside a circle $\gamma$ the function $f_n$ can be computed as
  \[
  f_n(z) = \frac{1}{2 \pi i} \int_\gamma \frac{f(\zeta) d \zeta}{(\zeta-a)^n(\zeta-z)}.
  \]
\end{theorem}

\begin{proposition}
  Let $f$ be a holomorphic function on a region $\Omega$ and $a \in \Omega$. If $f^{(n)}(a) = 0$ for all $n \geq 0$, then $f$ vanishes identically on $\Omega$.
\end{proposition}

It follows that if a holomorphic function $f$ is not identically zero, then at least one of its derivatives does not vanish. Let $n \geq 0$ be the smallest integer such that $f^{(n)}(a) \neq 0$. It follows that we can write $f(z) = (z-a)^n f_n(z)$ where $f_n$ is holomorphic and $f_n(a) \neq 0$. We say that $f$ has a \emph{zero of order $n$ at $a$}. An immediate consequence is that zeros are isolated points. This can be reformulated as follows.

\begin{proposition}[Analytic continuation]
  Let $f$ and $g$ be two holomorphic functions in a region $\Omega$. If $f$ and $g$ agree on a set with an accumulation point, then $f = g$ on all of $\Omega$.
\end{proposition}

\begin{corollary}
  A holomorphic function is uniquely determined by its values on any set with an accumulation point in the region of holomorphicity.
\end{corollary}

\begin{definition}
  If a function $f$ is holomorphic in a neighbourhood of a point $a$, except perhaps at $a$, then we say $f$ has an \emph{isolated singularity} at $a$. If $\lim_{z \rarr a} f(z) = \infty$, then we say $f$ has a \emph{pole} at $a$.
\end{definition}

If $f$ has a pole at $a$, then the function $g = 1/f$ is holomorphic in a neighbourhood of $a$. The \emph{order of the pole} of $f$ at $a$ is defined as the order of the zero of $g$ at $a$. In other words, the order of the pole is the unique integer such that $h(z) = (z-a)^n f(z)$ is holomorphic at $a$ and $h(a) \neq 0$. Even even more detail we could consider the conditions
\begin{enumerate}[(1)]
\item $\lim_{z \rarr a} |z-a|^\alpha |f(z)| = 0$,
\item $\lim_{z \rarr a} |z-a|^\alpha |f(z)| = \infty$
\end{enumerate}
for real values $\alpha$. If (1) holds for some $\alpha$, then it holds for all $\alpha' \geq \alpha$. Similarly, if (2) holds for some $\alpha$, then it holds for all $\alpha' \leq \alpha$. There are three possibilities:
\begin{enumerate}[(i)]
\item condition (1) holds for all $\alpha$, hence $f$ vanishes identically;
\item there exists some $\alpha_0$ such that (1) holds for $\alpha > \alpha_0$ and (2) for $\alpha < \alpha_0$; then $f$ has either a zero or a pole at $a$;
\item neither (1) nor (2) holds; we say $f$ has an \emph{essential singularity} at $a$.
\end{enumerate}

\begin{theorem}
  A holomorphic function comes arbitrarily close to any complex value in every neighbourhood of an essential singularity.
\end{theorem}

\begin{definition}
  A function which is holomorphic in a region $\Omega$ with the exception of finitely many poles is called \emph{meromorphic}.
\end{definition}

\begin{proposition}
  Let $z_j$ be the zeros of a function $f$ which is holomorphic in a disk $\Delta$ and does not vanish identically, each zero being counted as many times as its order indicates. For every closed curve $\gamma$ in $\Delta$ which does not pass through a zero
  \[
  \sum_j n(\gamma,z_j) = \frac{1}{2 \pi i} \int_\gamma \frac{f'}{f} d z,
  \]
  where the sum has only finitely many terms $\neq 0$.
\end{proposition}

\begin{theorem}
  Suppose that $f$ is holomorphic at $z_0$, $f(z_0) = w_0$, and that $f(z) - w_0$ has a zero of order $n$ at $z_0$. If $\epsilon > 0$ is sufficiently small, there exists a corresponding $\delta > 0$ such that for all $a$ with $|a-w_0| < \delta$ the equation $f(z) = a$ has exactly $n$ roots in the disk $|z-z_0| < \epsilon$.
\end{theorem}

\begin{corollary}
  A non-constant holomorphic map is open.
\end{corollary}

\begin{corollary}
  If $f$ is holomorphic at $z_0$ with $f'(z_0) \neq 0$, then it maps a neighbourhood of $z_0$ conformally and homeomorphically onto a connected open.
\end{corollary}

\begin{theorem}[The maximum principle]
  If $f$ is holomorphic and non-constant in a region $\Omega$, then $|f|$ has no maximum in $\Omega$.
\end{theorem}

Alternatively, one can restate this as follows.

\begin{corollary}
  If $|f|$ is defined and continuous on a closed bounded set $E$ and holomorphic on the interior of $E$, then the maximum of $|f|$ on $E$ is attained somewhere in $\d E$.
\end{corollary}

\begin{theorem}[Schwartz's Lemma]
  If $f(z)$ is holomorphic for $|z| < 1$ and satisfies the conditions $|f(z)| \leq 1$, $f(0) = 0$, then $|f(z)| \leq |z|$ and $|f'(0)| \leq 1$. If $|f(z)| = |z|$ for some $z \neq 0$, or if $|f'(0)| = 1$, then $f(z) = c z$ where $c$ is a constant satisfying $|c| = 1$.
\end{theorem}

\begin{corollary}
  Every bijective conformal mapping of a disk onto another (or a half plane) is given by a linear transformation.
\end{corollary}

\subsubsection{The General Form of Cauchy's Theorem}

\begin{theorem}[Cauchy's Theorem]
  If $f$ is holomorphic in a region $\Omega$, then
  \[
  \int_\gamma f d z = 0
  \]
  for every null-homologous cycle $\gamma$ in $\Omega$.
\end{theorem}

\begin{corollary}
  If $\Omega$ is simply-connected, then $\int_\gamma f d z = 0$ for all cycles $\gamma$.
\end{corollary}

\begin{corollary}
  In a simply-connected region, every holomorphic function has an antiderivative.
\end{corollary}

\begin{corollary}
  If $f$ is holomorphic and non-vanishing in a simply-connected region, then it is possible to define single-valued holomorphic branches of $\log f(z)$ and $\sqrt[n]{f(z)}$ in that region.
\end{corollary}

\subsubsection{The Calculus of Residues}

\begin{definition}
  The \emph{residue} of $f$ at an isolated singularity $a$ is the unique complex number $R$ which makes $f(z) - R/(z-a)$ the derivative of a single-valued holomorphic function in an annulus $0 < |z-a| < \delta$. We denote it by $R = \res_{z=a} f(z) = \res_a f$.
\end{definition}

\begin{theorem}
  If $f$ has a pole of order $n$ at $z_0$, then
  \[
  \res_{z_0} f = \lim_{z \rarr z_0} \frac{1}{(n-1)!} \left( \frac{d}{dz} \right)^{n-1} (z-z_0)^n f(z).
  \]
\end{theorem}

\begin{theorem}[Cauchy Residue Theorem]
  Let $f$ be holomorphic except for isolated singularities $a_j$ in a region $\Omega$. Then
  \[
  \frac{1}{2 \pi i} \int_\gamma f d z = \sum_j n(\gamma,a_j) \res_{a_j} f
  \]
  for any null-homologous cycle $\gamma$ in $\Omega$ which does not pass through any of the points $a_j$.
\end{theorem}

\begin{corollary}[Argument principle]
  If $f$ is meromorphic in a region $\Omega$ with zeros $a_j$ and poles $b_k$, then
  \[
  \frac{1}{2 \pi i} \int_\gamma \frac{f'}{f} d z = \sum_j n(\gamma,a_j) - \sum_k n(\gamma, b_k)
  \]
  for every null-homologous cycle $\gamma$ in $\Omega$ which does not pass through any of the zeros or poles.
\end{corollary}

\begin{corollary}[Rouch\'e's Theorem]
  Let $\gamma$ be a null-homologous cycle in a region $\Omega$ such that $n(\gamma,z)$ is either $0$ or $1$ for any point $z \in U \setminus \gamma$. Suppose that $f$ and $g$ are holomorphic in $\Omega$ and satisfy $|g| < |f|$ on $\gamma$. Then $f$ and $f+g$ have the same number of zeros enclosed by $\gamma$.
\end{corollary}

The following few paragraph summarize a few applications of Cauchy Residue Theorem to the evaluation of definite integrals.
\begin{enumerate}[1.]
\item
  Integrals of the form
  \[
  I = \int_0^{2\pi} R(\cos\theta,\sin\theta) d\theta
  \]
  where $R$ is a rational function may be evaluated by the substitution $z = e^{i\theta}$. Then
  \[
  I = -i \int_{|z|=1} R\left( \frac{1}{2}\left( z + \frac{1}{z} \right), \frac{1}{2i}\left( z - \frac{1}{z} \right) \right) \frac{d z}{z}.
  \]
  Provided there are no poles of $R$ on the unit circle, an application of Cauchy Residue Theorem finishes the job.
\item
  Integrals of the form
  \[
  I = \int_{-\infty}^\infty R(x) d x
  \]
  for rational $R$ converge if and only if in the rational function $R$ the degree of the denominator is at least two units greater than the degree of the numerator, and if no poles lie on the real axis. Integrating along a semicircular contour whose diameter is along the real axis yields
  \[
  I = 2 \pi i \sum_{\Im z > 0} \res_z R.
  \]
\item
  The integrals
  \begin{align*}
    \int_{-\infty}^\infty R(x) \cos x d x, && \int_{-\infty}^\infty R(x) \sin x d x
  \end{align*}\
  can be evaluated as the real and imaginary parts of
  \[
  I = \int_{-\infty}^\infty R(x) e^{i x} d x.
  \]
  Provided $R$ has a zero of at least order two at infinity, an analogous semicircle computation aided by the Cauchy Residue Theorem yields
  \[
  I = 2 \pi i \sum_{\Im z > 0} \res_z R(z)e^{i z}.
  \]
  The statement holds even under the weaker hypothesis that $R(\infty) = 0$, not necessarily of order two or higher (a rectangular contour is better suited for this case).
\end{enumerate}

\subsubsection{Harmonic Functions}

\begin{lemma}
  All linear functions $a x + b y$ are harmonic.
\end{lemma}

\emph{Laplace's equation}
\[
\Delta u = \frac{\d^2 u}{\d x^2} + \frac{\d^2 u}{\d y^2} = 0
\]
in polar coordinates becomes
\[
r \frac{\d}{\d r} \left( r \frac{\d u}{\d r} \right) + \frac{\d^2 u}{\d \theta^2} = 0.
\]

\begin{corollary}
  Any harmonic function which depends only on $r$ must be of the form $a \log r + b$.
\end{corollary}

\begin{lemma}
  If $u$ is a harmonic function on a region $\Omega$, then
  \[
  f(z) = \frac{\d u}{\d x} - i \frac{\d u}{\d y}
  \]
  is holomorphic on $\Omega$.
\end{lemma}

\begin{definition}
  The \emph{conjugate differential} to
  \[
  d u = \frac{\d u}{\d x} d x + \frac{\d u}{\d y} d y
  \]
  is
  \[
  \leftidx{^\ast}{d u} = -\frac{\d u}{\d y} d x + \frac{\d u}{\d x} d y.
  \]
\end{definition}

In general however, there is no single-valued function $v$ such that $d v = \leftidx{^\ast}{d u}$. It is easy to see that
\[
f d z = d u + \leftidx{^\ast}{d u}.
\]
Since by Cauchy's Theorem the integral of $f d z$ vanishes along any null-homologous cycle, and the exact differential $d u$ vanishes along any cycle, it follows that
\[
\int_\gamma \leftidx{^\ast}{d u} = \int_\gamma - \frac{\d u}{\d y} d x + \frac{\d u}{\d x} d y
\]
is zero for all null-homologous cycles $\gamma$. If the domain over which we are working in is simply-connected, then $\int_\gamma \leftidx{^\ast}{d u} = 0$ for all cycles $\gamma$, hence there is a well-defined (up to additive constant) single-valued function $v$ such that $d v = \leftidx{^\ast}{d u}$. The following is an important generalization.

\begin{theorem}
  If $u_1$ and $u_2$ are harmonic in a region $\Omega$, then
  \[
  \int_\gamma u_1 \leftidx{^\ast}{d u_2} - u_2 \leftidx{^\ast}{d u_1} = 0
  \]
  for all null-homologous cycles $\gamma$.  
\end{theorem}

Applying the previous result to $u_1 = \log r$, $u_2 = u$, and a cycle $\gamma$ made up of two counterclockwise oriented circles, we obtain the following.

\begin{theorem}
  The arithmetic mean of a harmonic function over concentric circles $|z| = r$ is a linear function of $\log r$,
  \[
  \frac{1}{2\pi} \int_{|z|=r} u d\theta = \alpha \log r + \beta,
  \]
  and if $u$ is harmonic in a disk then $\alpha = 0$ and the arithmetic mean is constant.
\end{theorem}

The following is a useful consequence.

\begin{theorem}[Maximum principle for harmonic functions]
  A non-constant harmonic function has neither a maximum nor a minimum in its region of definition. Consequently, the maximum and minimum on a closed bounded set $E$ are taken on the boundary of $E$.
\end{theorem}

Note that if $f$ is non-vanishing and holomorphic, then $\log|f|$ is a well-defined harmonic function. The maximum principle for holomorphic functions can be derived as a consequence of the previous result by applying it to $\log|f|$.

\begin{corollary}
  A function $u$ continuous on a closed bounded set $E$ and harmonic on its interior is uniquely determined by its values on $\d E$.
\end{corollary}

\begin{theorem}[Poisson's formula]
  Suppose that $u$ is harmonic for $|z| < R$, continuous for $|z| \leq R$. Then
  \[
  u(a) = \frac{1}{2\pi} \int_{|z|=R} \frac{R^2-|a|^2}{|z-a|^2} u(z) d\theta
  \]
  for all $|a| < R$.
\end{theorem}

An interesting consequence is that we can express $u(z)$ as the real part of
\[
f(z) = \frac{1}{2 \pi i} \int_{|\zeta|=R} \frac{\zeta+z}{\zeta-z} u(\zeta) \frac{d\zeta}{\zeta} + i C.
\]
Taking the imaginary part yields an explicit formula for the conjugate harmonic function.

\begin{definition}
  For any piecewise continuous function $U(\theta)$ in $0 \leq \theta \leq 2\pi$, define the \emph{Poisson integral} of $U$ as
  \[
  P_U(z) = \frac{1}{2\pi} \int_0^{2\pi} \Re \frac{e^{i\theta}+z}{e^{i\theta}-z} U(\theta) d\theta,
  \]
  a function of $|z| \leq 1$.
\end{definition}

It is not hard to see that
\begin{align*}
  P_{U+V} &= P_U + P_V, &
  P_{c U} &= c P_U, &
  P_c &= c,
\end{align*}
and $U \geq 0$ implies $P_U \geq 0$. These properties can be summed by stating $P$ is a positive linear functional. Furthermore, $a \leq U \leq b$ implies $a \leq P_U \leq b$.

\begin{theorem}[Schwarz's Theorem]
  The function $P_U(z)$ is harmonic for $|z| < 1$, and
  \[
  \lim_{z \rarr e^{i\theta_0}} P_U(z) = U(\theta_0)
  \]
  provided that $U$ is continuous at $\theta_0$.
\end{theorem}

The following is a restatement of Poisson's integral and Schwarz's Theorem for the half-plane.

\begin{theorem}
  If $U(\xi)$ is piecewise continuous and bounded for all real $\xi$, then
  \[
  P_U(z) = \frac{1}{\pi} \int_{-\infty}^\infty \frac{y}{(x-\xi)^2 + y^2} U(\xi) d\xi
  \]
  represents a harmonic function in the upper half-plane with boundary values $U(\xi)$ at points of continuity.
\end{theorem}

\begin{theorem}
  Let $\Omega^+$ be the part in the upper half-plane of a symmetric region $\Omega$ (that is, satisfying $\overline{\Omega} = \Omega$), and let $\sigma$ be the real axis in $\Omega$. Suppose $v$ is continuous in $\Omega^+ \cup \sigma$, harmonic in $\Omega^+$, and zero on $\sigma$. Then $v$ has a harmonic extension to $\Omega$ which satisfies the symmetry relation $v(\overline{z}) = -v(z)$. In the same situation, if $v$ is the imaginary part of a holomorphic function $f$ in $\Omega^+$ (that is, the function $f$ is real on the real axis), then $f$ has a holomorphic extension which satisfies $f(z) = \overline{f(\overline{z})}$.
\end{theorem}

\subsection{Ahlfors, Chapter 5: Series and Product Developments}

\subsubsection{Power Series Expansions}

\begin{theorem}
  Suppose that $f_n$ is holomorphic in the region $\Omega_n$, and that the sequence $\{f_n\}$ converges to a limit function $f$ in a region $\Omega$, uniformly on every compact subset of $\Omega$. Then $f$ is holomorphic in $\Omega$. Moreover, $f_n'$ converges uniformly to $f'$ on every compact subset of $\Omega$.
\end{theorem}

\begin{corollary}
  If a series of holomorphic terms $f = \sum f_n$ converges uniformly on every compact subset of a region $\Omega$, then the sum $f$ is holomorphic in $\Omega$, and the series can be differentiated term by term.
\end{corollary}

Proving uniform convergence can be facilitated by the maximum principle. For example, if $f_n$ are holomorphic in the disk $|z| < 1$, and if it can be shown that the sequence converges uniformly on each circle $|z| = r_m$ for $\lim_{m \rarr \infty} r_m = 1$, then the uniform convergence hypothesis follows.

\begin{theorem}
  If the functions $f_n$ are holomorphic and non-vanishing in a region $\Omega$, and if $f_n$ converges to $f$, uniformly on every compact subset of $\Omega$, then $f$ is either identically zero or non-vanishing in $\Omega$.
\end{theorem}

\begin{remark}
  The previous result holds if we replace ``non-vanishing'' with ``has at most $m$ zeros'' where $m$ is a non-negative integer.
\end{remark}

\begin{theorem}
  If $f$ is holomorphic in the region $\Omega$, containing $z_0$, then the representation
  \[
  f(z) = \sum_{n=0}^\infty \frac{f^{(n)}(z_0)}{n!} (z-z_0)^n
  \]
  is valid in the largest open disk of center $z_0$ contained in $\Omega$.
\end{theorem}

\subsubsection{Partial Fractions and Factorization}

\begin{theorem}
  Let $\{b_r\}$ be a sequence of complex numbers with $\lim_{r \rarr \infty} b_r = \infty$, and let $P_r(\zeta)$ be polynomials without constant term. Then there are functions which are meromorphic in the whole plane with poles at the points $b_r$, and the corresponding singular parts are $P_r(1/(z-b_r))$. Moreover, the most general meromorphic function of this kind can be written in the form
  \[
  f(z) = \sum_r \left( P_r\left( \frac{1}{z-b_r} \right) - p_r(z) \right) + g(z),
  \]
  where the $p_r(z)$ are suitably chosen polynomials and $g(z)$ is holomorphic in the whole plane.
\end{theorem}

Using the previous theorem, the following series can be derived.
\begin{align*}
  \frac{\pi^2}{\sin^2 \pi z} &= \sum_{n=-\infty}^\infty \frac{1}{(z-n)^2} \\
  \pi \cot \pi z &= \frac{1}{z} + \sum_{n \neq 0} \left( \frac{1}{z-n} + \frac{1}{n} \right) = \frac{1}{z} + \sum_{n=1}^\infty \frac{2 z}{z^2 - n^2} \\
  \frac{\pi}{\sin \pi z} &= \lim_{m \rarr \infty} \sum_{n=-m}^m \frac{(-1)^n}{z-n}
\end{align*}

\begin{definition}
  An infinite product $\prod_{n=1}^\infty p_n$ is said to converge if only at most a finite number of the factors are zero, and if the partial products formed by the non-vanishing factors tend to a finite limit which is different from zero.
\end{definition}

If $\prod p_n$ converges, then it is clear that $\lim_{n \rarr \infty} p_n = 1$. Therefore, it is customary to write an infinite product as
\[
\prod_{n=1}^\infty (1+a_n).
\]
A necessary condition for convergence is then $\lim_{n \rarr \infty} a_n = 0$.

\begin{theorem}
  The infinite product $\prod (1+a_n)$ with $1+a_n \neq 0$ converges if and only if so does the series
  \[
  \sum_{n=1}^\infty \log(1+a_n).
  \]
  whose terms represent the values of the principal branch of the logarithm.
\end{theorem}

\begin{definition}
  We say the product $\prod (1+a_n)$ \emph{converges absolutely} if so does the series $\sum \log(1+a_n)$.
\end{definition}

\begin{theorem}
  A necessary and sufficient condition for the absolute convergence of $\prod (1+a_n)$ is the absolute convergence of $\sum |a_n|$.
\end{theorem}

The previous result has a counterpart for uniform convergence if we allow to replace $a_n$ by functions $f_n$. This requires a slightly more technical statement if we allow zeros of these functions. More precisely, we consider only sets on which only finitely many of the factors can vanish. If these factors are omitted, then it is sufficient to study the uniform convergence of the remaining product.

\begin{proposition}
  The value of an absolutely convergent product does not change if the factors are reordered.
\end{proposition}

\begin{definition}
  A function which is holomorphic in the entire complex plane is called \emph{entire}.
\end{definition}

\begin{proposition}
  Every entire function can be expressed as the derivative of some entire function.
\end{proposition}

\begin{proposition}
  Every non-vanishing entire function is of the form $e^f$ where $f$ is an entire function.
\end{proposition}

It follows that every entire function $f$ with finitely many zeros can be expressed in the form
\[
f(z) = z^m e^{g(z)} \prod_{n=1}^N \left( 1 - \frac{z}{a_n} \right).
\]
If there are infinitely many zeros, then the obvious generalization is
\[
f(z) = z^m e^{g(z)} \prod_{n=1}^\infty \left( 1 - \frac{z}{a_n} \right).
\]
This is a valid expression only if the infinite product converges uniformly on every compact set. If this is so, then the product represents an entire function with zeros at the same points, and with the same multiplicities as $f(z)$.

\begin{proposition}
  The product $\prod (1 - z/a_n)$ converges absolutely if and only if $\sum 1/|a_n|$ is convergent, and in this case the convergence is uniform in every compact set.
\end{proposition}

In general, convergence may not be attained. To meet the needs, convergence-producing factors must be introduced.

\begin{theorem}
  There exists an entire function with arbitrarily prescribed zeros $a_n$ provided that, in the case of infinitely many zeros, $a_n \rarr \infty$. Every entire function with these and no other zeros can be written in the form
  \[
  f(z) = z^m e^{g(z)} \prod_{n=1}^\infty \left( 1 - \frac{z}{a_n} \right) \exp\left( \frac{z}{a_n} + \frac{1}{2}\left(\frac{z}{a_n}\right)^2 + \cdots + \frac{1}{m_n}\left(\frac{z}{a_n}\right)^{m_n} \right)
  \]
  where the product is taken over all $a_n \neq 0$, then $m_n$ are certain integers, and $g(z)$ is an entire function. Furthermore, we can choose all $m_n$ to be equal to some fixed integer $h$. Pick $h$ as the smallest integer for which $\sum 1/|a_n|^{h+1}$ converges. Then the infinite product above is called the \emph{canonical product} associated with the sequence $\{a_n\}$, and the integer $h$ is called the \emph{genus} of the canonical product.
\end{theorem}

Whenever possible, we attempt to work with canonical products. If in this presentation $g$ is a polynomial, then $f$ is said to have \emph{finite genus} $\max\{\deg g, h\}$.

\begin{corollary}
  Every function which is meromorphic in the whole plane is the fraction of two entire functions.
\end{corollary}

Using the previous theorem, the following infinite product can be derived.
\[
\sin \pi z = \pi z \prod_{n \neq 0} \left( 1 - \frac{z}{n} \right) e^{z/n} = \pi z \prod_{n=1}^\infty \left( 1 - \frac{z^2}{n^2} \right)
\]

Consider the function
\[
G(z) = \prod_{n=1}^\infty \left( 1 + \frac{z}{n} \right) e^{-z/n}
\]
which has zeros at all negative integers. It is evident $G(-z)$ has the positive integers for zeros. Combining these facts with the infinite product expansion of $\sin \pi z$ given above, we obtain
\[
z G(z) G(-z)= \frac{\sin \pi z}{\pi}.
\]
By comparing $G(z)$ and $G(z-1)$, it can be shown that
\[
G(z-1) = z e^\gamma G(z)
\]
where
\[
\gamma = \lim_{n \rarr \infty} \left( 1 + \frac{1}{2} + \frac{1}{3} + \cdots + \frac{1}{n} - \log n \right)
\]
is called \emph{Euler's constant}. The function
\[
\Gamma(z) = \frac{e^{-\gamma z}}{z G(z)}
\]
is called \emph{Euler's gamma function} and satisfies
\[
\Gamma(z+1) = z \Gamma(z).
\]
A more explicit representation is
\[
\Gamma(z) = \frac{e^{-\gamma z}}{z} \prod_{n=1}^\infty \left( 1 + \frac{z}{n} \right)^{-1} e^{z/n},
\]
and one of the previous formulas takes the form
\[
\Gamma(z) \Gamma(1-z) = \frac{\pi}{\sin \pi z}.
\]

\begin{proposition}
  The function $\Gamma$ is meromorphic with simple poles at $0, -1, -2, \dots$ and no zeros. For any $n \geq 0$,
  \[
  \res_{-n}\Gamma = \frac{(-1)^n}{n!}.
  \]
\end{proposition}

\begin{proposition}
  For all $z$ with $\Re z > 0$,
  \[
  \Gamma(z) = \int_0^\infty t^{z-1} e^{-t} d t.
  \]
\end{proposition}

A few notable values are
\begin{align*}
\Gamma(1) = \Gamma(2) = 1, && \Gamma(1/2) = \sqrt{\pi}.  
\end{align*}
Combining the first equality with the functional equation we deduce the following.

\begin{proposition}
  For all positive integers $n$,
  \[
  \Gamma(n) = (n-1)!.
  \]
\end{proposition}

\begin{theorem}[Legendre's duplication formula]
  The equality
  \[
  \sqrt{\pi} \: \Gamma(2 z) = 2^{2 z - 1} \Gamma(z) \Gamma\left( z + \frac{1}{2} \right)
  \]
  holds for all $z \in \C$.
\end{theorem}

\begin{theorem}[Stirling's formula]
  There exists an entire function $J$ which tends to $0$ as $z \rarr \infty$ in a half plane $\Re z \geq c > 0$ satisfying
  \[
  \Gamma(z) = \sqrt{\frac{2\pi}{z}} \left(\frac{z}{e}\right)^z e^{J(z)}.
  \]
  Therefore
  \[
  \lim_{n \rarr \infty} \frac{n!}{\sqrt{2 \pi n} \left( \frac{n}{e} \right)^n} = 1.
  \]
\end{theorem}

\subsubsection{Entire Functions}

\begin{theorem}[Jensen's formula]
  Let $f$ be an entire function not vanishing at the origin. Pick some $\rho > 0$ and let $a_1, \dots, a_n$ be the zeros of $f$ in the closed disk $|z| \leq \rho$, multiple zeros being repeated. Then
  \[
  \log|f(0)| = -\sum_{i=1}^n \log\left(\frac{\rho}{|a_i|}\right) + \frac{1}{2\pi} \int_0^{2\pi} \log|f(\rho e^{i\theta})| d\theta.
  \]
\end{theorem}

\begin{definition}
  Let $f$ be an entire function with zeros $\{a_n\}$ satisfying $f(0) \neq 0$. Let $M(r)$ be the maximum of $|f(z)|$ on $|z| = r$. The \emph{order} of the entire function $f$ is defined as
  \[
  \lambda = \limsup_{r \rarr \infty} \frac{\log\log M(r)}{\log r}.
  \]
\end{definition}

According to the definition $\lambda$ is the smallest number such that
\[
M(r) \leq e^{r^{\lambda+\epsilon}}
\]
for any given $\epsilon > 0$ as soon as $r$ is sufficiently large.

\begin{theorem}[Hadamard's Theorem]
  The genus and the order of an entire function satisfy the inequality $h \leq \lambda \leq h+1$.
\end{theorem}

\begin{corollary}
  An entire function of non-integer order assumes every finite value infinitely many times.
\end{corollary}

\subsubsection{The Riemann Zeta Function}

Since the series $\sum n^{-\sigma}$ converges  uniformly for all real $\sigma \geq \sigma_0$ for any fixed $\sigma_0 > 0$, it follows that the \emph{Riemann zeta function}
\[
\zeta(s) = \sum_{n=1}^\infty \frac{1}{n^s}
\]
is holomorphic in the half-plane $\Re s > 1$.

\begin{theorem}
  If $\Re s > 1$, then
  \[
  \zeta(s) = \prod_p \frac{1}{1 - p^{-s}}.
  \]
\end{theorem}

The following allows us to extend $\zeta$ to the entire complex plane.

\begin{theorem}
  If $\Re s > 1$, then
  \[
  \zeta(s) = -\frac{\Gamma(1-s)}{2 \pi i} \int_C \frac{(-z)^{s-1}}{e^z-1} d z
  \]
  where $(-z)^{s-1}$ is defined on the complement of the positive real axis as $e^{(s-1)\log(-z)}$ with $-\pi < \Im\log(-z) < \pi$, and $C$ is a path traversing above and below the positive real axis.
\end{theorem}

\begin{corollary}
  The function $\zeta$ can be extended to a meromorphic function in the whole complex plane whose only pole is a simple at $s = 1$ with residue $1$.
\end{corollary}

\begin{theorem}
  If $n$ is a non-negative integer, then
  \[
  \zeta(-n) = \frac{(-1)^n n!}{2 \pi i} \int_C \frac{z^{-n-1}}{e^z - 1} d z.
  \]
  In particular, $\zeta$ vanishes at all even negative integers, at the odd ones it is given by
  \[
  \zeta(-2 m + 1) = \frac{(-1)^m B_m}{2 m},
  \]
  and
  \[
  \zeta(0) = -\frac{1}{2}.
  \]
\end{theorem}

\begin{theorem}[Functional equation for the Riemann zeta function]
  \[
  \zeta(s) = 2^s \pi^{s-1} \sin\frac{\pi s}{2} \Gamma(1-s) \zeta(1-s).
  \]
\end{theorem}

\begin{corollary}
  The function
  \[
  \xi(s) = \frac{1}{2} s (1-s) \pi^{-s/2} \Gamma(s/2) \zeta(s)
  \]
  is entire and satisfies $\xi(s) = \xi(1-s)$.
\end{corollary}

\begin{proposition}
  The order of $\xi$ is $1$.
\end{proposition}

\subsubsection{Normal Families}

In what follows, a \emph{family} of functions $\Fc$ would refer to a set of functions defined in a fixed region $\Omega \subset \C$, and with values in a metric space $(S,d)$.

\begin{definition}
  The functions in a family $\Fc$ are said to be \emph{equicontinuous} on a set $E \subset \Omega$ if for each $\epsilon > 0$, there exists $\delta > 0$, such that for all $f \in \Fc$, $|z-z_0| < \delta$ implies $d(f(z),f(z_0)) < \epsilon$ for all $z_0,z \in E$.
\end{definition}

Note that each function $f$ in an equicontinuous family $\Fc$ on $E$ is itself uniformly continuous on $E$.

\begin{definition}
  A family $\Fc$ is said to be \emph{normal} in $\Omega$ if every sequence of functions $\{f_n\} \subset \Fc$ contains a subsequence which converges uniformly on every compact subset of $\Omega$.
\end{definition}

Note that the above definition does not require the limit functions of convergent subsequences to lie in $\Fc$.

\begin{definition}
  An \emph{exhaustion} of a region $\Omega$ is an increasing sequence of compact subspaces $E_k \subset \Omega$ such that $\bigcup_k E_k = \Omega$ satisfying the following condition: for every compact subset $E \subset \Omega$ there exists some $k$ such that $E \subset E_k$.
\end{definition}

It can be shown that exhaustions always exists for regions in the complex plane. Our aim is to define a metric on the space of functions $\Omega \rarr S$ such that convergence in this metric would be unanimous with uniform convergence on all compact subsets $E \subset \Omega$. We start by defining a new metric $\delta$ on $S$ given by
\[
\delta(a,b) = \frac{d(a,b)}{1+d(a,b)}.
\]
Then for each two functions $f, g \cn \Omega \rarr S$ we define
\[
\delta_k(f,g) = \sup_{x \in E_k} \delta(f(x),g(x))
\]
which may be regarded as the distance between $f$ and $g$ on $E_k$. Finally, we set
\[
\rho(f,g) = \sum_{k=1}^\infty \delta_k(f,g) 2^{-k}.
\]

\begin{proposition}
  The so defined function $\rho$ is a metric on the space of functions $\Omega \rarr S$. Furthermore, if $S$ is a complete metric space, then so is the space of functions endowed with the metric $\rho$.
\end{proposition}

\begin{proposition}
   A sequence of functions $\{f_n\}$ converges with respect to $\rho$ if and only if it converges uniformly on every compact subset $E \subset \Omega$.
\end{proposition}
 
An application of the Bolzano-Weierstrass Theorem to the metric space $\Fc$ endowed with the restriction of $\rho$ yields the following.

\begin{theorem}
  A family $\Fc$ is normal if and only if its closure $\overline{\Fc}$ with respect to $\rho$ is compact.
\end{theorem}

If $\overline{\Fc}$ is compact, it is customary to say $\Fc$ is \emph{relatively compact}.

\begin{theorem}
  If $S$ is complete, then $\Fc$ is normal if and only if it is totally bounded.
\end{theorem}

The following is a restatement using the original metric $d$.

\begin{theorem}
  A family $\Fc$ is totally bounded if and only if to every compact set $E \subset \Omega$ and every $\epsilon > 0$ it is possible to find $f_1, \dots, f_n \in \Fc$ such that every $f \in \Fc$ satisfies $d(f,f_i) < \epsilon$ on $E$ for some $f_i$.
\end{theorem}

\begin{theorem}[Arzela's Theorem]
  A family $\Fc$ of continuous functions with values in a metric space $S$ is normal in the region $\Omega \subset \C$ if and only if:
  \begin{enumerate}[(i)]
  \item $\Fc$ is equicontinuous on every compact set $E \subset \Omega$;
  \item for any $z \in \Omega$ the set $\{ f(z) \;|\; f \in \Fc \}$ lies in a compact subset of $S$.
  \end{enumerate}
\end{theorem}

\begin{theorem}
  A family of holomorphic functions is normal with respect to $\C$ if and only if the functions in $\Fc$ are uniformly bounded on every compact set.
\end{theorem}

If a family of holomorphic functions satisfies the conditions of the previous result, we say it is \emph{locally bounded}.

\begin{theorem}
  A locally bounded family of holomorphic functions has locally bounded derivatives.
\end{theorem}

\begin{lemma}
  If a sequence of meromorphic functions converges in the sense of $\C\P^1$, uniformly on every compact set, then the limit functions is meromorphic or identically equal to $\infty$. If a sequence of holomorphic functions converges in the same sense, then the limit function is either holomorphic or identically equal to $\infty$.
\end{lemma}

\begin{theorem}
  A family of holomorphic or meromorphic functions $\{f\}$ is normal in the spherical sense if and only if the expressions
  \[
  \rho(f) = \frac{2|f'(z)|}{1+|f(z)|^2}
  \]
  are locally bounded.
\end{theorem}

\subsection{Ahlfors, Chapter 6: Conformal Mapping. Dirichlet's Problem}

\subsubsection{The Riemann Mapping Theorem}

\begin{theorem}[Riemann Mapping Theorem]
  Given any simply-connected  region $\Omega$ which is not the whole plane, and a point $z_0 \in \Omega$, there exists a unique holomorphic function $f : \Omega \rarr \C$, normalized by the conditions $f(z_0) = 0$, $f'(z_0) > 0$, such that $f$ is bijective from $\Omega$ onto the unit disk $|w| < 1$.  
\end{theorem}

\begin{theorem}
  Let $f \cn \Omega \rarr \Omega'$ be a homeomorphism between two regions. If $\{z_n\}$ or $z(t)$ tends to the boundary of $\Omega$, then $\{f(z_n)\}$ or $f \circ z$ tends to the boundary of $\Omega'$.
\end{theorem}

\begin{theorem}
  Suppose that the boundary of a simply-connected region $\Omega$ contains an analytic arc $\gamma$ as a one-sided free boundary arc. Then the function $f$ which maps $\Omega$ onto the unit disk can be extended to a function which is holomorphic and bijective onto $\Omega \cup \gamma$. The image of $\gamma$ is an arc $\gamma'$ on the unit circle.
\end{theorem}

\subsubsection{Conformal Mappings of Polygons}

\begin{theorem}
  The functions $F$ which map $|w| < 1$ conformally onto polygons with angles $\alpha_k \pi$, $k = 1, \dots, n$, are of the form
  \[
  F(w) = C \int_0^w \prod_{k=1}^n (w-w_k)^{-\beta_k} d w + C'
  \]
  where $\beta_k = 1-\alpha_k$, the $w_k$ are points on the unit circle, and $C$,$C'$ are complex constants.
\end{theorem}

TODO: Read about infinite products in Stein \& Shakarchi

%%% Local Variables: 
%%% mode: latex
%%% TeX-master: "Study_Guide"
%%% End: 

\section{Algebraic Topology}
\label{S:algebraic-topology}

Syllabus
\begin{description}
\item[Undergraduate] Hatcher, \emph{Algebraic Topology}, chapter 1 (but not the additional topics). (math 131) 
\item[Graduate] Hatcher, \emph{Algebraic Topology}, chapter 2 (including additional topics) and chapter 3 (without additional topics). (math 231a)
\end{description}

\subsection{Hatcher, Chapter 0: Some Underlying Geometric Notions}

\begin{definition}
  The \emph{join} of two topological spaces $X$ and $Y$, denoted $X \ast Y$, is the quotient of $X \x Y \x I$ under the identifications $(x,y_1,0) \sim (x,y_2,0)$ and $(x_1,y,1) \sim (x_2,y,1)$. In other words, this amount to collapsing $X \x Y \x \{0\}$ to $X$ and $X \x Y \x \{1\}$ to $Y$. On can also think of the join as the set of formal linear combinations $t_1 x + t_2 y$ for $x \in X$, $y \in Y$, $t_1, t_2 \in \R$ satisfying $t_1 + t_2 = 1$. Alternatively, one may think of it as the collection of all line segments joining points in $X$ with points in $Y$.
\end{definition}

The join is an associative operation. Two useful examples are:
\begin{enumerate}[(i)]
\item the $n$-fold join of the one point space is the $(n-1)$-simplex $\Delta^{n-1}$;
\item the $n$-fold join of the two point space $S^0$ is the $(n-1)$-sphere $S^{n-1}$.
\end{enumerate}

\begin{proposition}
  If $(X,A)$ is a CW pair and $A$ is a contractible subcomplex, then the quotient map $X \rarr X/A$ is a homotopy equivalence.
\end{proposition}

Compare this with the following.

\begin{proposition}
  If $(X,A)$ is a CW pair and $A$ is a contractible in $X$, that is, the inclusion $A \hrarr X$ is homotopic to the constant map, then $X/A \simeq X \vee S A$.
\end{proposition}

\begin{proposition}
  If $(X_1,A)$ is a CW pair and the two attaching maps $f,g \cn A \rarr X_0$ are homotopic, then $X_0 \sqcup_f X_1 \simeq X_0 \sqcup_g X_1$.
\end{proposition}

A pair of topological spaces $(X,A)$ is said to have the \emph{homotopy extension property} if given a map $f_0 \cn X \rarr Y$ and a homotopy $g_t \cn A \rarr Y$ for $0 \leq t \leq 1$ satisfying $f_0|_A = g_0$, then there exists a homotopy $f_t \cn X \rarr A$ for $0 \leq t \leq 1$ satisfying $f_t|_A = g_t$ for all $t$.

\begin{proposition}
  A pair $(X,A)$ has the homotopy extension property if and only if $X \x \{0\} \cup A \x I$ is a deformation retract of $X \x I$.
\end{proposition}

\begin{proposition}
  All CW pairs $(X,A)$ have the homotopy extension property.
\end{proposition}

\begin{proposition}
  Suppose the pairs $(X,A)$ and $(Y,A)$ satisfy the homotopy extension property, and $f \cn X \rarr Y$ is a homotopy equivalence with $f|_A = \id_A$. Then $f$ is a homotopy equivalence rel $A$.
\end{proposition}

\begin{corollary}
  If $(X,A)$ satisfies the homotopy extension property and the inclusion $A \hrarr X$ is a homotopy equivalence, then $A$ is a deformation retract of $X$.
\end{corollary}

\begin{corollary}
  A map $f \cn X \rarr Y$ is a homotopy equivalence if and only if $X$ is a deformation retract of the mapping cylinder $M_f$. Hence, two spaces $X$ and $Y$ are homotopy equivalent if and only if there is a third space containing both $X$ and $Y$ as deformation retracts.
\end{corollary}

\subsection{Hatcher, Chapter 1: The Fundamental Group}

\subsubsection{Basic Constructions}

\begin{proposition}
  Let $X$ be a topological space, and $h$ a path from $x_0$ to $x_1$ in $X$. Then the map $\beta_h \cn \pi_1(X,x_1) \rarr \pi_1(X,x_0)$ given by $\beta_h[f] = [h \cdot f \cdot \overline{h}]$ is an isomorphism.
\end{proposition}

\begin{theorem}[Brouwer Fixed Point Theorem]
  Every continuous map $h \cn D^2 \rarr D^2$ has a fixed point.
\end{theorem}

\begin{theorem}[Borsuk-Ulam]
  For every continuous map $f \cn S^2 \rarr \R^2$ there exists a pair of antipodal points $x$ and $-x$ with $f(x) = f(-x)$.
\end{theorem}

\begin{corollary}
  There is no injective continuous map $S^2 \rarr \R^2$, and hence $S^2$ is not homeomorphic to any subset of $\R^2$.
\end{corollary}

\begin{corollary}
  Whenever $S^2$ is expressed as the union of three closed sets $A_1$, $A_2$, and $A_3$, then at least one of these sets must contain a pair of antipodal points $\{x,-x\}$.
\end{corollary}

\begin{proposition}
  $\pi_1(X \x Y, (x_0, y_0)) \cong \pi_1(X,x_0) \x \pi_1(Y,y_0)$
\end{proposition}

\begin{proposition}
  If $n \geq 2$, then $\pi_1(S^n) = 0$.
\end{proposition}

\begin{proposition}
  If a space $X$ retracts onto a space $A$, then the homomorphism $i_\ast \cn \pi_1(A,x_0) \rarr \pi_1(X,x_0)$ induced by the inclusion $i \cn A \rarr X$ is injective. If $A$ is a deformation retract of $X$, then $i_\ast$ is an isomorphism.
\end{proposition}

\begin{proposition}
  If $\phi \cn X \rarr Y$ is a homotopy equivalence, then the induced homomorphism $\phi_\ast \cn \pi_1(X,x_0) \rarr \pi_1(Y, \phi(x_0))$ is an isomorphism for all $x_0 \in X$.
\end{proposition}

\subsubsection{Van Kampen's Theorem}

\begin{theorem}[Seifert--van Kampen]
  If $X$ is the union of path-connected open sets $A_\alpha$ each containing the basepoint $x_0 \in X$ and if each intersection $A_\alpha \cap A_\beta$ is path-connected, then the homomorphism $\Phi \cn \coprod_\alpha \pi_1(A_\alpha) \rarr \pi_1(X)$ is surjective. If in addition each intersection $A_\alpha \cap A_\beta \cap A_\gamma$ is path-connected, then the kernel of $\Phi$ is the normal subgroup $N$ generated by all elements of the form $i_{\alpha\beta}(\omega)i_{\beta\alpha}(\omega)^{-1}$ for $\omega \in \pi_1(A_\alpha \cap A_\beta)$, and hence $\Phi$ induces an isomorphism $\pi_1(X) \cong \coprod_\alpha \pi_1(A_\alpha)/N$.
\end{theorem}

\begin{proposition}
  Suppose we attach a collection of $2$-cells $e_\alpha^2$ to a path-connected space $X$ via maps $\phi_\alpha \cn S^1 \rarr X$, producing a space $Y$. Fix basepoints $s_0 \in S^1$ and $x_0 \in X$. Choose a path $\gamma_\alpha$ from $x_0$ to $\phi(s_0)$ for each $\alpha$ so that $\gamma_\alpha \phi_\alpha \overline{\gamma_\alpha}$ is a loop in $X$ based at $x_0$. Then the inclusion $X \hrarr Y$ induces a surjection $\pi_1(X,x_0) \rarr \pi_1(Y,x_0)$ whose kernel is generated by the loops $\gamma_\alpha \phi_\alpha \overline{\gamma_\alpha}$.
\end{proposition}

\begin{corollary}
  For every group $G$ there is a $2$-dimensional CW complex $X_G$ with $\pi_1(X_G) \cong G$.
\end{corollary}

For $\Sigma_g = (T^2)^{\# g}$ and $N_g = (\R\P^2)^{\# g}$, we have
\begin{align*}
  \pi_1(\Sigma_g) &= \langle a_1, b_1, \dots, a_g, b_g \;|\; a_1 b_1 a_1^{-1} b_1^{-1} \cdots a_g b_g a_g^{-1} b_g^{-1} \rangle, \\
  \pi_1(N_g) &= \langle a_1, \dots, a_g \;|\; a_1^2 \cdots a_g^2 \rangle.
\end{align*}

\begin{corollary}
  The surfaces $\Sigma_g$ and $\Sigma_h$ are homotopy equivalent if and only if $g = h$.
\end{corollary}

\subsubsection{Covering Spaces}

\begin{proposition}[Homotopy lifting property]
  Given a covering space $p \cn \wtilde{X} \rarr X$, a homotopy $f_t \cn Y \rarr X$, and a map $\wtilde{f_0} \cn Y \rarr \wtilde{X}$ lifting $f_0$, then there exists a unique homotopy $\wtilde{f_t} \cn Y \rarr \wtilde{X}$ of $\wtilde{f_0}$ that lifts $f_t$.
\end{proposition}

When $Y$ is a point, the previous results is also known as the \emph{path lifting property}. Concretely, for each path $f \cn I \rarr X$ and each lift $\wtilde{x_0}$ of the starting point $x_0 = f(0)$, there is a unique path $\wtilde{f} \cn I \rarr \wtilde{X}$ lifting $f$ and starting at $\wtilde{x_0}$.

\begin{proposition}
  The map $p_\ast \cn \pi_1(\wtilde{X}, \wtilde{x_0}) \rarr \pi_1(X,x_0)$ induced by a covering $p \cn (\wtilde{X}, \wtilde{x_0}) \rarr (X,x_0)$ is injective. The image subgroup $p_\ast(\pi_1(\wtilde{X}, \wtilde{x_0}))$ in $\pi_1(X,x_0)$ consists of the homotopy classes of loops in $X$ based at $x_0$ whose lifts to $\wtilde{X}$ starting at $\wtilde{x_0}$ are loops.
\end{proposition}

\begin{proposition}
  The number of sheets of a covering space $p \cn (\wtilde{X}, \wtilde{x_0}) \rarr (X,x_0)$ with $X$ and $\wtilde{X}$ path-connected equals the index of $p_\ast(\pi_1(\wtilde{X},\wtilde{x_0}))$ in $\pi_1(X,x_0)$.
\end{proposition}

\begin{proposition}[Lifting criterion]
  Let $p \cn (\wtilde{X}, \wtilde{x_0}) \rarr (X,x_0)$ be a covering space and $f \cn (Y,y_0) \rarr (X,x_0)$ a map with $Y$ path-connected and locally path-connected. Then a lift $\wtilde{f} \cn (Y, y_0) \rarr (\wtilde{X},\wtilde{x_0})$ of $f$ exists if and only if $f_\ast(\pi_1(Y,y_0)) \subset p_\ast(\pi_1(\wtilde{X},\wtilde{x_0}))$.
\end{proposition}

\begin{proposition}[Unique lifting property]
  Given a covering space $p \cn \wtilde{X} \rarr X$ and a map $f \cn Y \rarr X$, if two lifts $\wtilde{f_1}, \wtilde{f_2} \cn Y \rarr \wtilde{X}$ of $f$ agree at one point of $Y$ and $Y$ is connected, then $\wtilde{f_1} = \wtilde{f_2}$.
\end{proposition}

\begin{proposition}
  Suppose $X$ is a path-connected, locally path-connected, and semilocally simply-connected. Then for every subgroup $H \subset \pi_1(X,x_0)$ there is a covering space $p \cn X_H \rarr X$ such that $p_\ast(\pi_1(X_H,\wtilde{x_0})) = H$ for a suitably chosen basepoint $\wtilde{x_0} \in X_H$.
\end{proposition}

\begin{corollary}
  Any topological space which is path-connected, locally-path connected, and semilocally simply-connected admits a universal cover.
\end{corollary}

\begin{proposition}
  If $X$ is path-connected and locally path-connected, then two path-connected covering spaces $p_1 \cn \wtilde{X_1} \rarr X$ and $p_2 \cn \wtilde{X_2} \rarr X$ are isomorphic via an isomorphism $f \cn \wtilde{X_1} \rarr \wtilde{X_2}$ taking a basepoint $\wtilde{x_1} \in p_1^{-1}(x_0)$ to a basepoint $\wtilde{x_2} \in p_2^{-1}(x_0)$ if and only if $(p_1)_\ast(\pi_1(\wtilde{X_1}, \wtilde{x_1})) = (p_2)_\ast(\pi_1(\wtilde{X_2}, \wtilde{x_2}))$.
\end{proposition}

\begin{theorem}
  Let $X$ be path-connected, locally path-connected, and semilocally path-connected. Then there is a bijection between the set of basepoint-preserving isomorphism classes of path-connected covering spaces $p \cn (\wtilde{X},\wtilde{x_0}) \rarr (X,x_0)$ and the set of subgroups of $\pi_1(X,x_0)$, obtained by associating the subgroup $p_\ast(\pi_1(\wtilde{X},\wtilde{x_0}))$ to the covering space $(\wtilde{X},\wtilde{x_0})$. If basepoints are ignored, this correspondence gives a bijection between the isomorphism classes of path-connected covering spaces $p \cn \wtilde{X} \rarr X$ and conjugacy classes of subgroups of $\pi_1(X,x_0)$.
\end{theorem}

\begin{theorem}
  Let $X$ be a topological space satisfying the hypothesis of the previous result. The set of $n$-sheeted covering spaces (not necessarily connected) of $X$ endowed with basepoints is in bijection with the set of homomorphisms $\pi_1(X,x_0) \rarr S_n$, where $S_n$ stands for the symmetric group on $n$ symbols. If we drop basepoints then there is a bijection with the set of homomorphisms up to conjugation, that is, two homomorphisms $\phi_1, \phi_2 \cn \pi_1(X,x_0) \rarr S_n$ correspond to equivalent covers if and only if $\phi_1 \circ \phi_2^{-1}$ is an inner automorphism of $S_n$.
\end{theorem}

\begin{definition}
  For a covering space $p \cn \wtilde{X} \rarr X$, an automorphism of covers $\wtilde{X} \rarr \wtilde{X}$ is called a \emph{deck transformation}. These form a group denoted $\Aut(\wtilde{X}/X)$.
\end{definition}

\begin{definition}
  A covering space $p \cn \wtilde{X} \rarr X$ is called \emph{normal} if for each $x_1, x_2 \in \wtilde{X}$ satisfying $p(x_1) = p(x_2)$ there exists a deck transformation $\phi \in \Aut(\wtilde{X}/X)$ such that $\phi(x_1) = x_2$.
\end{definition}

\begin{proposition}
  Let $p \cn (\wtilde{X},\wtilde{x_0}) \rarr (X,x_0)$ be a path-connected covering space of a path-connected, locally path-connected space $X$, and let $H$ be the subgroup $p_\ast(\pi_1(\wtilde{X},\wtilde{x_0})) \subset \pi_1(X,x_0)$. Then:
  \begin{enumerate}[(a)]
  \item this covering space is normal if and only if $H$ is a normal subgroup of $\pi_1(X,x_0)$;
  \item $\Aut(\wtilde{X}/X)$ is isomorphic to the quotient $N(H)/H$ where $N(H)$ is the normalizer of $H$ in $\pi_1(X,x_0)$.
  \end{enumerate}
  In particular, $\Aut(\wtilde{X}/X)$ is isomorphic to $\pi_1(X,x_0)/H$ if $\wtilde{X}$ is a normal covering. Hence for the universal cover $\wtilde{X} \rarr X$ we have $\Aut(\wtilde{X}/X) \cong \pi_1(X)$.
\end{proposition}

\begin{definition}
  A group action of a group $G$ on a topological space $Y$ is called a \emph{covering space action} if the following condition holds: each $y \in Y$ has a neighbourhood $U$ such that all the images $g(U)$ for varying $g \in G$ are disjoint. In other words, $g_1(U) \cap g_2(U) \neq \emptyset$ implies $g_1 = g_2$.
\end{definition}

Note that for each covering $\wtilde{X} \rarr X$ the action of $\Aut(\wtilde{X}/X)$ on $\wtilde{X}$ is a covering space action.

\begin{proposition}
  Each covering action of a group $G$ on a space $Y$ satisfies the following:
  \begin{enumerate}[(a)]
  \item the quotient map $p \cn Y \rarr Y/G$, $p(y) = G y$, is a normal covering space;
  \item if $Y$ is path-connected, then $G$ is the group of deck transformations of this covering space $Y \rarr Y/G$;
    \item if $Y$ is path-connected and locally path-connected, then $G$ is isomorphic to $\pi_1(Y/G)/p_\ast(\pi_1(Y))$.
  \end{enumerate}
\end{proposition}

For each $m, n \geq 1$ there is a covering space $\Sigma_{m n + 1} \rarr \Sigma_{m+1}$. Conversely, if there is a covering $\Sigma_g \rarr \Sigma_h$ then $g = m n + 1$ and $h = m + 1$ for some $m, n \geq 1$.

\begin{proposition}
  Consider maps $X \rarr Y \rarr Z$ such that both $Y \rarr Z$ and the composition $X \rarr Z$ are covering spaces. If $Z$ is locally path-connected, then $X \rarr Y$ is a covering space. Furthermore, if $X \rarr Z$ is normal, then so is $X \rarr Y$.
\end{proposition}

\begin{proposition}
  Consider a covering action of a group $G$ on a path-connected, locally path-connected space $X$. Then any subgroup $H \subset G$ determines a composition of covering spaces $X \rarr X/H \rarr X/G$. Furthermore, the following properties hold.
  \begin{enumerate}[(a)]
  \item Every path-connected covering space between $X$ and $X/G$ is isomorphic to $X/H$ for some subgroup $H \subset G$.
  \item Two such covering spaces $X/H_1$ and $X/H_2$ of $X/G$ are isomorphic if and only if $H_1$ and $H_2$ are conjugate subgroups of $G$.
    \item The covering space $X/H \rarr X/G$ is normal if and only if $H$ is a normal subgroup of $G$, in which case the group of deck transformations of this cover is $G/H$.
  \end{enumerate}
\end{proposition}

\subsection{Hatcher, Chapter 2: Homology}

\subsubsection{Simplicial and Singular Homology}

\begin{proposition}
  Consider the decomposition of a topological space $X$ into its path-components $X = \bigsqcup_\alpha X_\alpha$. Then $H_\bullet(X) \cong \bigoplus_\alpha H_\bullet(X_\alpha)$.
\end{proposition}

\begin{theorem}[Exact sequence of a pair]
  For every pair $(X,A)$, we have a long exact sequence
  \[\xymatrix{
    \cdots \ar[r] & H_n(A) \ar[r] & H_n(X) \ar[r] & H_n(X,A) \ar[r]^-\d & H_{n-1}(A) \ar[r] & \cdots.
  }\]
  The connecting homomorphism $\d \cn H_n(X,A) \rarr H_{n-1}(A)$ has a simple description: if a class $[\alpha] \in H_n(X,A)$ is represented by a relative cycle $\alpha$, then $\d[\alpha] = [\d \alpha] \in H_{n-1}(A)$.
\end{theorem}

The following is a mild generalization.

\begin{theorem}[Exact sequence of a triple]
  For every triple $(X,A,B)$, we have a long exact sequence
  \[\xymatrix{
    \cdots \ar[r] & H_n(A,B) \ar[r] & H_n(X,B) \ar[r] & H_n(X,A) \ar[r]^-\d & H_{n-1}(A,B) \ar[r] & \cdots.
  }\]
\end{theorem}

\begin{theorem}[Excision]
  Given subspaces $Z \subset A \subset X$ such that the closure of $Z$ is contained in the interior of $A$, then the inclusion $(X \setminus Z, A \setminus Z) \hrarr (X,A)$ induces isomorphisms $H_n(X \setminus Z, A \setminus Z) \rarr H_n(X,A)$ for all $n$. Equivalently, for subspaces $A, B \subset X$ whose interiors cover $X$, the inclusion $(B, A \cap B) \hrarr (X,A)$ induces isomorphisms $H_n(B,A \cap B) \rarr H_n(X,A)$ for all $n$.
\end{theorem}

We call a pair $(X,A)$ \emph{good} if $A$ is a nonempty closed subspace and it is the deformation retract of a neighbourhood.

\begin{proposition}
  For good pairs $(X,A)$, the quotient map $q \cn (X,A) \rarr (X/A,A/A)$ induces isomorphisms $q_\ast \cn H_n(X,A) \rarr H_n(X/A,A/A) \cong \wtilde{H}_n(X/A)$ for all $n$.
\end{proposition}

\begin{theorem}
  If $(X,A)$ is a good pair, then  there is an exact sequence
  \[\xymatrix{
    \cdots \ar[r] & \wtilde{H}_n(A) \ar[r] & \wtilde{H}_n(X) \ar[r] & \wtilde{H}_n(X/A) \ar[r]^-\d & \wtilde{H}_{n-1}(A) \ar[r] & \cdots.
  }\]
\end{theorem}

\begin{corollary}
  If the CW complex $X$ is the union of subcomplexes $A$ and $B$, then the inclusion $(B,A \cap B) \hrarr (X,A)$ induces isomorphisms $H_n(B,A \cap B) \rarr H_n(X,A)$ for all $n$.
\end{corollary}

\begin{corollary}
  For a wedge sum $\bigvee_\alpha X_\alpha$, the inclusions $i_\alpha \cn X_\alpha \rarr \bigvee_\alpha X_\alpha$ induce an isomorphism
  \[
  \bigoplus_\alpha (i_\alpha)_\ast \cn \bigoplus_\alpha \wtilde{H}_\bullet(X_\alpha) \rarr \wtilde{H}_\bullet\left( \bigvee_\alpha X_\alpha \right),
  \]
  provided that the wedge sum is formed at basepoints $x_\alpha \in X_\alpha$ such that the pairs $(X_\alpha,x_\alpha)$ are good.
\end{corollary}

\begin{theorem}
  If nonempty subsets $U \subset \R^m$ and $V \subset \R^n$ are homeomorphic, then $m = n$.
\end{theorem}

\begin{proposition}
  If $A$ is a retract of $X$, then the maps $H_n(A) \rarr H_n(X)$ induced by the inclusion $A \hrarr X$ are injective.
\end{proposition}

\begin{corollary}
  There exists no retraction $D^n \rarr S^n$.
\end{corollary}

\begin{corollary}[Brouwer Fixed Point Theorem]
    Every continuous map $h \cn D^n \rarr D^n$ has a fixed point.
\end{corollary}

\begin{proposition}
  For all $n$, there are isomorphisms $\wtilde{H}_n(X) \cong \wtilde{H}_{n+1}(S X)$.
\end{proposition}

\begin{proposition}
  Let $X$ be a finite-dimensional CW complex.
  \begin{enumerate}[(a)]
  \item If $X$ has dimension $n$, then $H_i(X) = 0$ for $i > n$ and $H_n(X)$ is free.
  \item If there are no cells of dimension $n-1$ or $n+1$, then $H_n(X)$ is free with basis in bijective correspondence with the $n$-cells.
  \item If $X$ has $k$ $n$-cells, then $H_n(X)$ is generated by at most $k$ elements.
  \end{enumerate}
\end{proposition}

\subsubsection{Computations and Applications}

\begin{definition}
  Each map $f \cn S^n \rarr S^n$ induces a homomorphism $f_\ast \cn H_n(S^n) \rarr H_n(S^n)$ which is multiplication by an integer $d$ called the \emph{degree} of $f$ denoted $\deg f$.
\end{definition}

\begin{proposition}
  \mbox{}
  \begin{enumerate}[(a)]
  \item $\deg \id_{S^n} = 1$.
  \item If $f$ is not surjective, then $\deg f = 0$.
  \item If $f \simeq g$, then $\deg f = \deg g$.
  \item $\deg (f \circ g) = \deg f \deg g$.
  \item If $f$ is the reflection in $S^{n-1}$ of $S^n$, then $\deg f = -1$.
  \item The antipodal map has degree $(-1)^{n+1}$.
  \item If $f \cn S^n \rarr S^n$ has no fixed points, then $\deg f = (-1)^{n+1}$.
  \item If $S f \cn S^{n+1} \rarr S^{n+1}$ denotes the suspension map of $f \cn S^n \rarr S^n$, then $\deg S f = \deg f$.
  \end{enumerate}
\end{proposition}

\begin{theorem}
  A continuous nonvanishing vector field on $S^n$ exists if and only if $n$ is odd.
\end{theorem}

\begin{proposition}
  If $n$ is even, then $\Z/2$ is the only nontrivial group that can act freely on $S^n$.
\end{proposition}

\begin{proposition}
  Let $f \cn S^n \rarr S^n$ be a continuous map, and $y \in S^n$ be a point whose preimage is finite, say $f^{-1}(y) = \{ x_1, \dots, x_m \}$. Let $U_1, \dots, U_m$ be disjoint neighbourhoods of the $x_i$ mapped homeomorphically to a neighbourhood $V$ of $y$. The \emph{local degree} of $f$ at $x_i$, denoted $\deg f|_{x_i}$, is an integer $d$ such that $f_\ast \cn H_n(U_i, U_i \setminus \{x_i\}) \rarr H_n(V, V \setminus \{y\})$ is multiplication by $d$. Then $\deg f = \sum_i \deg f|_{x_i}$.
\end{proposition}

\begin{proposition}
  Let $X$ be a CW complex.
  \begin{enumerate}[(a)]
  \item $H_\bullet(X^n,X^{n-1}) \cong \Z^\ell_{(n)}$ where $\ell$ is the number of $n$-cells of $X$ (potentially infinite).
  \item $H_k(X^n) = 0$ if $k > n$. In particular, if $X$ is finite dimensional, then $H_k(X) = 0$ for $k > \dim X$.
  \item The inclusion $i \cn X^n \hrarr X$ induces isomorphisms $i_\ast \cn H_k(X^n) \rarr H_k(X)$ for $k < n$.
  \end{enumerate}
\end{proposition}

There is an alternative formulation of homology which makes it easy to compute. For any CW complex $X$ there is a chain complex
\[\xymatrix{
  \cdots \ar[r] & H_{n+1}(X^{n+1}, X^n) \ar[r]^-{d_{n+1}} & H_n(X^n, X^{n-1}) \ar[r]^-{d_n} & H_{n-1}(X^{n-1}, X^{n-2}) \ar[r] & \cdots.
}\]
The homology of this complex, denoted $H^{CW}_n(X)$, is called \emph{cellular homology}. Note that $H_n(X^n, X^{n-1})$ is a free abelian group generated by the $n$-cells of $X$. The differentials $d_n$ can be computed by $d_n(e_\alpha^n) = \sum_\beta d_{\alpha\beta} e_\beta^{n-1}$ where $d_{\alpha\beta}$ is the degree of the map $S_\alpha^{n-1} \rarr X^{n-1} \rarr S_\beta^{n-1}$ that is the composition of the attaching map of $e_\alpha^n$ with the quotient map collapsing $X^{n-1} \setminus e_\beta^{n-1}$ to a point.

The following computations follow.
\begin{align*}
  H_\bullet(\Sigma_g) &\cong \Z_{(0)} \oplus \Z^{2g}_{(1)} \oplus \Z \\
  H_\bullet(N_g) &\cong \Z_{(0)} \oplus (\Z^{g-1} \oplus \Z/2)_{(1)} \\
  H_k(\R\P^n) &\cong
  \begin{cases}
    \Z & \textrm{if $k = 0$ and if $k = n$ is odd}, \\
    \Z/2 & \textrm{if $k$ is odd and $0 < k < n$}, \\
    0 & \textrm{otherwise}.
  \end{cases} \\
  H_k(L_m(\ell_1, \dots, \ell_n)) &\cong
  \begin{cases}
    \Z & \textrm{if $k = 0$ or $2n-1$}, \\
    \Z/m & \textrm{if $k$ is odd and $0 < k < 2n-1$}, \\
    0 & \textrm{otherwise}.
  \end{cases}
\end{align*}

\begin{theorem}
  $H^{CW}_\bullet(X) \cong H_\bullet(X)$.
\end{theorem}

For an abelian group $G$ and an integer $n \geq 1$, we can construct a CW complex $X$ satisfying $\wtilde{H}_\bullet(X) \cong G_{(n)}$. Furthermore, it can be shown that the homotopy type of $X$ is uniquely determined by the previous condition (provided that for $n > 1$, we require $X$ to be simply-connected), hence we will refer to $X$ as a \emph{Moore space} and denote it by $M(G,n)$.

\begin{theorem}
  For finite CW complexes $X$, the Euler characteristic is
  \[
  \chi(X) = \sum_n (-1)^n \rank H_n(X^n, X^{n-1}) = \sum_n (-1)^n \rank H_n(X).
  \]
\end{theorem}

For example,
\begin{align*}
  \chi(\Sigma_g) &= 2 - 2 g, &
  \chi(N_g) &= 2 - g.
\end{align*}

Suppose $r \cn X \rarr A$ is a retraction and $i \cn A \rarr X$ the associated inclusion. We have already shown that $i_\ast \cn H_\bullet(A) \rarr H_\bullet(X)$ is injective, hence the long exact sequence of the pair $(X,A)$ splits into short exact sequences
\[\xymatrix{
  0 \ar[r] & H_n(A) \ar[r] & H_n(X) \ar[r] & H_n(X,A) \ar[r] & 0.
}\]
Furthermore, the relation $r_\ast i_\ast = \id_{H_\bullet(A)}$ implies the above short exact sequences are split.

\begin{proposition}[Split short exact sequences]
  For a short exact sequence
  \[\xymatrix{
    0 \ar[r] & A \ar[r]^-i & B \ar[r]^-j & C \ar[r] & 0
  }\]
  of abelian groups the following statements are equivalent:
  \begin{enumerate}[(a)]
  \item there is a homomorphism $p \cn B \rarr A$ such that $p \circ i = \id_A$;
  \item there is a homomorphism $s \cn C \rarr B$ such that $j \circ s = \id_C$;
  \item there is an isomorphism $B \cong A \oplus C$ making the diagram
    \[\xymatrix@R=0.1in{
      && B \ar[dr]^-j \ar[dd]^-\cong \\
      0 \ar[r] & A \ar[ur]^-i \ar[dr] && C \ar[r] & 0 \\
      && A \oplus C \ar[ur]
    }\]
    commute, where the maps in the lower row are the obvious projections.
  \end{enumerate}
\end{proposition}

\begin{theorem}[Mayer-Vietoris sequence]
  Let $X$ be a topological space and $A, B \subset X$ two subspaces such that their interiors cover $X$. Then there is a long exact sequence as follows.
  \[\xymatrix{
    \cdots \ar[r] & H_n(A \cap B) \ar[r] & H_n(A) \oplus H_n(B) \ar[r] & H_n(X) \ar[r]^-\d & H_{n-1}(A \cap B) \ar[r] & \cdots 
  }\]
  The connecting homomorphism $\d \cn H_n(X) \rarr H_{n-1}(A \cap B)$ has the following explicit form. Consider a class $\alpha \in H_n(X)$ represented by a cycle $z$. By further subdivision we can ensure that $z = x + y$ such that $x$ lies in $A$ and $y$ in $B$. Note that $x$ and $y$ need not be cycles themselves, but $\d x = -\d y$. Then $\d \alpha \in H_{n-1}(A \cap B)$ is represented by a the cycle $\d x = -\d y$.
\end{theorem}

There is also a similar sequence in reduced homology. Even further, we can take $A$ and $B$ to be deformation retracts of neighbourhoods $U$ and $V$ respectively satisfying $X = U \cap V$. This is particularly useful when $X$ is a CW complex and $A$ and $B$ subcomplexes since $U$ and $V$ as described always exist. The following can often be viewed as a generalization of the Mayer-Vietoris sequence.

\begin{theorem}
  Consider two maps $f, g \cn X \rarr Y$ and form the space $Z = (X \x I \sqcup Y)/\!\!\sim$ by identifying $(x,0) \sim f(x)$ and $(x,1) \sim g(x)$ for all $x \in X$. Less formally, we can describe $Z$ as $X \x I$ glued to $Y$ at one end via $f$ and at the other via $g$. Let $i \cn Y \hrarr Z$ be the evident inclusion. Then there is a long exact sequence as follows.
  \[\xymatrix@C=0.4in{
    \cdots \ar[r] & H_n(X) \ar[r]^-{f_\ast - g_\ast} & H_n(Y) \ar[r]^-{i_\ast} & H_n(Z) \ar[r]^-\d & H_{n-1}(X) \ar[r] & \cdots
  }\]
\end{theorem}

\begin{theorem}[Relative Mayer-Vietoris sequence]
  Let $(X,Y) = (A \cup B, C \cup D)$ such that: (1) $C \subset A$, (2) $D \subset B$, (3) $X$ is the union of the interiors of $A$ and $B$, and (4) $Y$ the union of the interiors of $C$ and $D$. Then there is a long exact sequence in relative homology as follows.
  \[\xymatrix@C=0.2in{
    \cdots \ar[r] & H_n(A \cap B, C \cap D) \ar[r] & H_n(A,C) \oplus H_n(B,D) \ar[r] & H_n(X,Y) \ar[r] & H_{n-1}(A \cap B, C \cap D) \ar[r] & \cdots
  }\]
\end{theorem}

All variants of the Mayer-Vietoris sequence hold for reduced homology too. Furthermore, all preceding results in this section generalize to homology with coefficients.

\subsubsection{The Formal Viewpoint}

\begin{definition}
  A \emph{(reduced) homology theory} is a sequence is covariant functors $\wtilde{h}_n$ from the category of CW complexes to the category of abelian groups which satisfy the following axioms.
  \begin{enumerate}[(1)]
  \item If $f \simeq g$, then $f_\ast = g_\ast \cn \wtilde{h}_n(X) \rarr \wtilde{h}_n(Y)$.
  \item There are boundary homomorphisms $\d \cn \wtilde{h}_n(X/A) \rarr \wtilde{h}_{n-1}(A)$ defined for each CW pair $(X,A)$, fitting into an exact sequence
    \[\xymatrix{
      \cdots \ar[r]^-\d & \wtilde{h}_n(A) \ar[r]^-{i_\ast} & \wtilde{h}_n(X) \ar[r]^-{q_\ast} & \wtilde{h}_n(X/A) \ar[r]^-{\d} & \wtilde{h}_{n-1}(A) \ar[r]^-{i_\ast} & \cdots,
    }\]
    where $i \cn A \rarr X$ and $q \cn X \rarr X/A$ are respectively the evident inclusion and quotient maps. Furthermore, the boundary mas are natural: for $f \cn (X,A) \rarr (Y,B)$ inducing a quotient map $\overline{f} \cn X/A \rarr Y/B$, the diagrams
    \[\xymatrix{
      \wtilde{h}_n(X/A) \ar[r]^-\d \ar[d]_{\overline{f}_\ast} & \wtilde{h}_{n-1}(A) \ar[d]^-{f_\ast} \\
      \wtilde{h}_n(Y/B) \ar[r]^-\d & \wtilde{h}_{n-1}(B)
    }\]
    commute.
  \item For a wedge sum $X = \bigvee_\alpha X_\alpha$ with inclusions $i_\alpha \cn X_\alpha \hrarr X$, the direct sum map
    \[
    \bigoplus_\alpha (i_\alpha)_\ast \cn \bigoplus_\alpha \wtilde{h}_n(X_\alpha) \rarr \wtilde{h}_n(X)
    \]
    is an isomorphism for all $n$.
  \end{enumerate}
\end{definition}

\subsubsection{Homology and Fundamental Group}

\begin{theorem}[Hurewicz Theorem]
  By regarding loops as singular $1$-cycles, we obtain a homomorphism $h \cn \pi_1(X, x_0) \rarr H_1(X)$. If $X$ is path-connected, then $h$ is surjective and has kernel the commutator subgroup of $\pi_1(X)$, so $h$ induces an isomorphism between the abelianization of $\pi_1(X)$ onto $H_1(X)$. If $\gamma \cn S^1 \rarr X$ is a loop, then the map $h$ may alternatively be described as $h[\gamma] = \gamma_\ast(\alpha)$ where $\alpha$ is the (oriented) generator of $H_1(S^1)$.
\end{theorem}

\subsubsection{Classical Applications}

\begin{proposition}
  \mbox{}
  \begin{enumerate}[(a)]
  \item For an embedding $h \cn D^k \rarr S^n$, we have $\wtilde{H}_\bullet(S^n \setminus h(D^k)) = 0$.
  \item For an embedding $h \cn S^k \rarr S^n$ with $k < n$, we have $\wtilde{H}_\bullet(S^n \setminus h(S^k)) \cong \Z_{(n-k-1)}$.
  \end{enumerate}
\end{proposition}

\begin{theorem}[Invariance of Domain]
  If $U$ is an open set in $\R^n$, then for any embedding $h \cn U \rarr \R^n$ the image $h(U)$ must be an open set in $\R^n$. The statement holds if we replace $\R^n$ with $S^n$ throughout.
\end{theorem}

\begin{corollary}
  If $M$ is a compact $n$-manifold and $N$ a connected $n$-manifold, then an embedding $h \cn M \rarr N$ must be surjective, hence a homeomorphism.
\end{corollary}

\begin{theorem}[Hopf]
  The only finite-dimensional division algebras over $\R$ which are commutative and have an identity are $\R$ and $\C$.
\end{theorem}

\begin{proposition}
  An odd map $f \cn S^n \rarr S^n$, satisfying $f(-x) = -f(x)$ for all $x \in S^n$, must have odd degree. The claim also holds if we replace odd with even throughout.
\end{proposition}

\begin{proposition}
  If $p \cn \wtilde{X} \rarr X$ is a two-sheeted cover, then there is a long exact sequence as follows.
  \[\xymatrix{
    \cdots \ar[r] & H_n(X; \Z/2) \ar[r]^-{\tau_\ast} & H_n(\wtilde{X}; \Z/2) \ar[r]^-{p_\ast} & H_n(X; \Z/2) \ar[r] & H_{n-1}(X; \Z/2) \ar[r] & \cdots
  }\]
  The map $\tau_\ast$, called the \emph{transfer homomorphism}, is given by summing up the two lifts of a given chain.
\end{proposition}

\begin{corollary}[Borsuk-Ulam]
  For every map $g \cn S^n \rarr \R^n$ there exists a point $x \in S^n$ such that $g(x) = g(-x)$.
\end{corollary}

\subsubsection{Simplicial Approximation}

\begin{theorem}[Simplicial approximation]
  If $K$ is a finite simplicial complex and $L$ an arbitrary simplicial complex, then any map $f \cn K \rarr L$ is homotopic to a map that is simplicial with respect to some iterated barycentric subdivision of $K$.
\end{theorem}

For a map $\phi \cn \Z^n \rarr \Z^n$ we define its \emph{trace} $\tr \phi$ in the usual sense. More generally, consider a finitely generated abelian group $A$ with torsion part $A_T$. For any $\phi \cn A \rarr A$, we define $\tr \phi = \tr\overline{\phi}$ where $\overline{\phi} \cn A/A_T \rarr A/A_T$ is the induced map mod torsion.

\begin{definition}
  Let $X$ be a finite CW complex and $f \cn X \rarr X$ a continuous map. The \emph{Lefschetz number} of $f$ is
  \[
  \tau(f) = \sum_i (-1)^n \tr(f_\ast \cn H_n(X) \rarr H_n(X)).
  \]
\end{definition}

Note that $\tau(\id_X) = \chi(X)$.

\begin{theorem}[Lefschetz Fixed Point Theorem]
  If $X$ is a finite simplicial complex, or more generally a retract of a finite simplicial complex, and $f \cn X \rarr X$ a map with $\tau(f) \neq 0$, then $f$ has a fixed point.
\end{theorem}

It is the case that every compact, locally contractible space that can be embedded in $\R^n$ for some $n$ is a retract of a finite simplicial complex. In particular, this includes compact manifolds and finite CW complexes.

\begin{theorem}[Simplicial Approximation to CW Complexes]
  Every CW complex $X$ is homotopy equivalent to a simplicial complex, which can be chosen to be of the same dimension as $X$, finite if $X$ is finite, and countable if $X$ is countable.
\end{theorem}

\subsection{Hatcher, Chapter 3: Cohomology}

\subsubsection{Cohomology Groups}

The homology of a space $X$ is customarily constructed in two steps: (1) form a chain complex $C_\bullet(X)$ (simplicial, singular, cellular, etc.), and then (2) take its homology. To make the transition to cohomology, one needs to dualize after step (1). In other words, assuming we are going to work with coefficients over an abelian group $G$, we first form $C^\bullet(X; G) = \Hom(C_\bullet; G)$. The homology of this complex is then denoted $H^\bullet(X; G)$. We proceed to investigate the relation between $H^\bullet(X; G)$ and $\Hom(H_\bullet(X), G)$.

\begin{theorem}
  If a chain complex $C_\bullet$ of free abelian groups has homology groups $H_\bullet(C)$, then the cohomology groups $H^\bullet(C; G)$ of the cochain complex $\Hom(C_\bullet, G)$ are determined by the split exact sequences
  \[\xymatrix{
    0 \ar[r] & \Ext(H_{n-1}(C), G) \ar[r] & H^n(C; G) \ar[r]^-{h} & \Hom(H_n(C), G) \ar[r] & 0.
  }\]
\end{theorem}

The $\Ext(H,G)$ groups are defined in the following fashion. Every abelian group has a free resolution $0 \rarr F_1 \rarr F_0 \rarr H \rarr 0$. We apply the functor $\Hom(-,G)$ to it and take homology. It turns out that the cohomology is nontrivial only in degree $1$ (this is specific to the category of abelian groups $\Z\dmod$), and we define $\Ext(H, G) = H^1(H; G) = H_1(\Hom(F_\bullet, G))$. For computational purposes, the following properties are useful.
\begin{enumerate}[(a)]
\item $\Ext(H \oplus H', G) \cong \Ext(H, G) \oplus \Ext(H', G)$.
\item $\Ext(H,G) = 0$ if $H$ is free.
\item $\Ext(\Z/n, G) \cong G/n G$.
\end{enumerate}
The following result summarizes the above facts when $G = \Z$.

\begin{corollary}
  If the homology groups $H_n$ and $H_{n-1}$ of a chain complex $C$ of free abelian groups are finitely generated, with torsion subgroups $T_n \subset H_n$ and $T_{n-1} \subset H_n$, then
  \[
  H^n(C; Z) \cong (H_n/T_n) \oplus T_{n-1}.
  \]
\end{corollary}

\begin{corollary}
  If a chain map between chain complexes of free abelian groups induces an isomorphism on homology groups, then it induces an isomorphism on cohomology groups with any coefficient group $G$.
\end{corollary}

A large number of the results from the previous chapter hold for cohomology -- one only needs to reverse the direction of sequences.

\subsubsection{Cup Product}

Unlike the general theories of homology and cohomology in which coefficients could be taken in an arbitrary abelian group, we are required to work over a commutative ring $R$ in order to define cup products. For cochains $\phi \in C^k(X; R)$ and $\psi \in C^\ell(X; R)$, the cup product $\phi \wcup \psi \in C^{k+\ell}(X;R)$ is the cochain whose value on a singular simplex $\sigma \cn \Delta^{k+\ell} \rarr X$ is given by the formula
\[
(\phi \wcup \psi)(\sigma) = \phi(\sigma|_{[v_0, \dots, v_k]}) \cdot \psi(\sigma|_{[v_k, \dots, v_{k+\ell}]}).
\]

\begin{proposition}
  For $\phi \in C^k(X;R)$ and $\psi \in C^\ell(X; R)$, we have
  \[
  \delta(\phi \wcup \psi) = \delta\phi \wcup \psi + (-1)^k \phi \wcup \delta\psi.
  \]
\end{proposition}

The previous result implies that the cup product of two cocycles is again a cocycle, and the cup product of a cocycle and a coboundary (in either order) is again a coboundary. Therefore, the cup product on cochains induces a cup product on cohomology, namely
\[\xymatrix{
H^k(X;R) \x H^\ell(X;R) \ar[r]^-\wcup & H^{k+\ell}(X;R).
}\]
Associativity and distributivity on the level of cochains implies these properties for the product on cohomology too. If $R$ is unital, then there is an identity for the cup product -- the class $1 \in H^0(X;R)$ defined by the $0$-cocycle taking the value $1$ on each singular $0$-simplex. There is a relative version of the cup product which takes the form
\[\xymatrix{
  H^k(X,A;R) \x H^\ell(X,B;R) \ar[r]^-\wcup & H^{k+\ell}(X, A \cup B; R),
}\]
and this specializes to
\[\xymatrix@R=0in{
  H^k(X,A;R) \x H^\ell(X;  R) \ar[r]^-\wcup & H^{k+\ell}(X, A; R), \\
  H^k(X;  R) \x H^\ell(X,A;R) \ar[r]^-\wcup & H^{k+\ell}(X, A; R), \\
  H^k(X,A;R) \x H^\ell(X,A;R) \ar[r]^-\wcup & H^{k+\ell}(X, A; R).
}\]

\begin{proposition}
  For a map $f \cn X \rarr Y$, the induced maps $f^\ast \cn H^n(Y;R) \rarr H^n(X;R)$ satisfy $f^\ast(\alpha \wcup \beta) = f^\ast(\alpha) \wcup f^\ast(\beta)$, and similarly in the relative case.
\end{proposition}

This prompts us to note that we can regard $H^\bullet(-;R)$ as a functor from the category of topological spaces to the category of graded $R$-algebras. We proceed summarize a few common cohomology rings.
\begin{align*}
  H^\bullet(\R\P^n;\Z/2) &\cong (\Z/2)[\alpha]/(\alpha^{n+1}), \quad |\alpha| = 1\\
  H^\bullet(\R\P^\infty;\Z/2) &\cong (\Z/2)[\alpha], \quad |\alpha| = 1 \\[5pt]
  H^\bullet(\C\P^n;\Z) &\cong \Z[\alpha]/(\alpha^{n+1}), \quad |\alpha| = 2 \\
  H^\bullet(\C\P^\infty;\Z) &\cong \Z[\alpha], \quad |\alpha| = 2 \\[5pt]
  H^\bullet(\H\P^n;\Z) &\cong \Z[\alpha]/(\alpha^{n+1}), \quad |\alpha| = 4 \\
  H^\bullet(\H\P^\infty;\Z) &\cong \Z[\alpha], \quad |\alpha| = 4 \\
  \intertext{The cohomology of real projective spaces over $\Z$ are slightly more delicate.}
  H^\bullet(\R\P^{2n};\Z) &\cong \Z[\alpha]/(2\alpha,\alpha^{n+1}), \quad |\alpha| = 2 \\
  H^\bullet(\R\P^{2n+1};\Z) &\cong \Z[\alpha,\beta]/(2\alpha,\alpha^{n+1},\beta^2,\alpha\beta), \quad |\alpha| = 2, |\beta| = 2n+1 \\
  H^\bullet(\R\P^\infty;\Z) &\cong \Z[\alpha]/(2\alpha), \quad |\alpha| = 2
\end{align*}

\begin{proposition}
  The isomorphisms
  \[\xymatrix@R=0in{
    H^\bullet(\bigsqcup_\alpha X_\alpha;R) \ar[r] & \prod_\alpha H^\bullet(X_\alpha;R) \\
    \wtilde{H}^\bullet(\bigvee_\alpha X_\alpha;R) \ar[r] & \prod_\alpha \wtilde{H}^\bullet(X_\alpha;R)
  }\]
  whose coordinates are induced by the inclusions $i_\alpha \cn X_\alpha \hrarr \bigsqcup_\alpha X_\alpha$ and $i_\alpha \cn X_\alpha \hrarr \bigvee_\alpha X_\alpha$ respectively are ring isomorphisms.
\end{proposition}

The second isomorphism above provides us with a tool to reject spaces as being homotopy equivalent to wedge products.

\begin{theorem}
  When $R$ is commutative, the rings $H^\bullet(X,A;R)$ are graded commutative, that is, the identity
  \[
  \alpha \wcup \beta = (-1)^{k\ell} \beta \wcup \alpha
  \]
  holds for all $\alpha \in H^k(X,A;R)$ and $\beta \in H^\ell(X,A;R)$.
\end{theorem}

The \emph{cross product}, also known as the \emph{external cup product}, is a map
\[\xymatrix{
  H^k(X;R) \x H^\ell(Y;R) \ar[r]^-\x & H^{k+\ell}(X \x Y; R)
}\]
given by
\[
a \x b = p_1^\ast(a) \wcup p_2^\ast(b),
\]
where $p_1 \cn X \x Y \rarr X$ and $p_2 \cn X \x Y \rarr Y$ are the evident projections. It is not hard to see this map is bilinear, hence induces a linear map $H^k(X;R) \otimes H^\ell(Y;R) \rarr H^{k+1}(X \x Y; R)$.

\begin{theorem}[K\"unneth formula]
  The cross product $H^\bullet(X;R) \otimes_R H^\bullet(Y;R) \rarr H^\bullet(X \x Y;R)$ is an isomorphism of rings if $X$ and $Y$ are CW complexes and $H^k(Y;R)$ is a finitely generated free $R$-module for all $k$.
\end{theorem}

It turns out the hypothesis $X$ and $Y$ are CW complexes is unnecessary. The result also hold in a relative setting.

\begin{theorem}[Relative K\"unneth formula]
  For CW pairs $(X,A)$ and $(Y,B)$ the cross product homomorphism $H^\bullet(X,A;R) \otimes_R H^\bullet(Y,B;R) \rarr H^\bullet(X \x Y, A \x Y \cup X \x B; R)$ is an isomorphism of rings if $H^k(Y,B;R)$ is a finitely generated free $R$-module for each $k$.
\end{theorem}

The relative version yields a reduced one which involves the smash product $X \wedge Y$.

\begin{theorem}
  If $\R^n$ has the structure of a division algebra over $\R$, then $n$ must be a power of $2$.
\end{theorem}

\subsubsection{Poincar\'e Duality}

Fix a coefficient ring $R$. For any space $X$ and any subset $A \subset X$, the \emph{local homology} of $X$ at $A$ is $H_\bullet(X|A;R) = H_\bullet(X, X \setminus A; R)$ and similarly for cohomology. As usual, omitting $R$ means we are working over $\Z$.

\begin{definition}
  A compact manifold without boundary is called \emph{closed}.
\end{definition}

\begin{theorem}
  The homology groups of a closed manifold are finitely generated.
\end{theorem}

From now on, consider a manifold $M$. It is easy to see that $H_\bullet(M|x;R) \cong R_{(n)}$ for all $x \in M$. An \emph{$R$-orientation} of $M$ at $x$ is a choice of a generator $\mu_x \in H_n(M|x;R)$. An $R$-orientation at a point $x$ determines $R$-orientations for all $y$ in a neighbourhood $B$ (open ball of finite radius) of $x$ via the canonical isomorphisms
\[
H_n(M|y;R) \cong H_n(M|B;R) \cong H_n(M|x;R).
\]
An \emph{$R$-orientation} of $M$ is a consistent choice of local $R$-orientations at all $x \in M$. This is better handled via the space
\[
M_R = \{ \mu_x \in H_n(M|x;R) \;|\; x \in M \}
\]
topologized in an appropriate manner. There is a canonical projection $M_R \rarr M$ which turns this into a covering space. Since $H_n(M|x;R) \cong H_n(M|x) \otimes R$, each $r \in R$ determines a subcovering space
\[
M_r = \{ \pm \mu_x \otimes r \in H_n(M|x;R) \;|\; x \in M, \mu_x \textrm{ is a generator of } H_n(M|x) \} \subset M_R.
\]
The space
\[
\wtilde{M} = \{ \mu_x \in H_n(M|x) \;|\; \mu_x \textrm{ is a generator of } H_n(M|x) \}
\]
is also of great interest. For example $M_\Z = \bigsqcup_{k \geq 0} M_k$ where $M_0 \cong M$ and $M_k \cong \wtilde{M}$ for all $k \geq 1$. Similarly, if $r \in R^\x$ has order $2$, then $M_r \cong M$, and otherwise $M_r \cong \wtilde{M}$. Note that $\wtilde{M}$ is a two-sheeted cover of $M$ (potentially not connected), and a ($\Z$-)orientation of $M$ is nothing but a section $M \rarr \wtilde{M}$ of this cover.

\begin{proposition}
  If $M$ is connected, then $M$ is orientable if and only if $\wtilde{M}$ has two components.
\end{proposition}

\begin{corollary}
  If $M$ is simply-connected, or more generally if $\pi_1(M)$ has no subgroup of index two, then $M$ is orientable.
\end{corollary}

An orientable manifold is $R$-orientable for all $R$, while a non-orientable manifold is $R$-orientable if and only if $R$ contains a unit of (additive) order $2$, which is equivalent to having $2 = 0$ in $R$. It follows that every manifold is $\Z/2$-orientable. In practice, the important cases are $R = \Z$ and $R = \Z/2$.

\begin{theorem}
  Let $M$ be a closed connected $n$-manifold.
  \begin{enumerate}[(a)]
  \item If $M$ is $R$-orientable, then the natural map $H_n(M;R) \rarr H_n(M|x;R) \cong R$ is an isomorphism for all $x \in M$.
  \item If $M$ is not $R$-orientable, then the natural map $H_n(M;R) \rarr H_n(M|x;R) \cong R$ is injective with image $\{ r \in R \;|\; 2r = 0 \}$ for all $x \in M$.
  \item $H_i(M;R) = 0$ for $i > n$.
  \end{enumerate}
\end{theorem}

\begin{corollary}
  Let $M$ be an $n$-manifold. If $M$ is orientable, then $H_n(M) \cong \Z$, and if not, then $H_n(M) = 0$. In either case $H_n(M;\Z/2) \cong \Z/2$.
\end{corollary}

\begin{definition}
  A \emph{fundamental (orientation) class} for $M$ with coefficients in $R$ is an element of $H_n(M;R)$ whose image in $H_n(M|x;R)$ is a generator for all $x \in M$.
\end{definition}

\begin{corollary}
  If $M$ is a closed connected $n$-manifold, the torsion subgroup of $H_{n-1}(M)$ is trivial if $M$ is orientable and $\Z/2$ if $M$ is non-orientable.
\end{corollary}

\begin{proposition}
  If $M$ is a connected non-compact $n$-manifold, then $H_i(M;R) = 0$ for $i \geq n$.
\end{proposition}

For an arbitrary space $X$ and a coefficient ring $R$, define an $R$-bilinear \emph{cap product}
\[
\wcap \cn C_k(X;R) \x C^\ell(X;R) \rarr C_{k-\ell}(X;R)
\]
for $k \geq \ell$ by setting
\[
\sigma \wcap \phi = \phi(\sigma|_{[v_0, \dots, v_\ell]}) \sigma|_{[v_\ell,\dots,v_k]}.
\]
The formula
\[
\d(\sigma \wcap \phi) = (-1)^\ell(\d\sigma \wcap \phi - \sigma \wcap \delta\phi)
\]
implies this induces a map on homology and cohomology
\[\xymatrix{
  H_k(X;R) \x H^\ell(X;R) \ar[r]^-\wcap & H_{k-\ell}(X;R).
}\]
There are relative forms
\[\xymatrix@R=0in{
  H_k(X,A;R) \x H^\ell(X  ;R) \ar[r]^-\wcap & H_{k-\ell}(X,A;R) \\
  H_k(X,A;R) \x H^\ell(X,A;R) \ar[r]^-\wcap & H_{k-\ell}(X  ;R),
}\]
and more generally
\[\xymatrix{
  H_k(X, A \cup B; R) \x H^\ell(X,A;R) \ar[r]^-\wcap & H_{k-\ell}(X,B;R)
}\]
defined for open $A, B \subset X$. The naturality of the cap product is expressed via the following diagram.
\[\xymatrix{
  H_k(X) \x H^\ell(X) \ar@<-1.5pc>[d]_-{f_\ast} \ar[r]^-\wcap & H_{k-\ell}(X) \ar[d]^-{f_\ast} \\
  H_k(Y) \x H^\ell(Y) \ar@<-1.5pc>[u]_-{f^\ast} \ar[r]^-\wcap & H_{k-\ell}(Y)
}\]
More precisely, we have
\[
f_\ast(\alpha) \wcap \phi = f_\ast(\alpha \wcap f^\ast(\phi)).
\]
We are now ready to state our main result.

\begin{theorem}[Poincar\'e Duality]
  Let $M$ be a closed $R$-orientable $n$-manifold with fundamental class $[M] \in H_n(M;R)$. Then the map $D \cn H^k(M;R) \rarr H_{n-k}(M;R)$ defined by $D(\alpha) = [M] \wcap \alpha$ is an isomorphism for all $k$.
\end{theorem}

One can define \emph{cohomology with compact support}, denoted $H_c^\bullet(X)$, as the cohomology of the cochain complex which is formed by all compactly supported cochains (that is, vanishing on chains outside a compact set). It is clear that $H^\bullet(X) \cong H^\bullet_c(X)$ for all compact spaces $X$.

\begin{proposition}
  If a space $X$ is the union of a directed set of subspaces $X_\alpha$ with the property that each compact set in $X$ is contained in some $X_\alpha$, then the natural map $\varinjlim H_i(X_\alpha;G) \rarr H_i(X;G)$ is an isomorphism for all $i$ and $G$.
\end{proposition}

For any space $X$, the compact sets $K \subset X$ form a directed system since the union of any two compact sets is compact. If $K \subset L$ is an inclusion if compact sets, then there is a natural map $H^\bullet(X, X \setminus K; G) \rarr H^\bullet(X, X \setminus L; G)$. It is possible to check that the resulting limit $\varinjlim H^\bullet(X, X \setminus K; G)$ equals $H^\bullet_c(X; G)$. This could be a useful property, for example, one can compute that $H^\bullet_c(\R^n) \cong \Z_{(n)}$. Poincar\'e Duality then generalizes in the following way.

\begin{theorem}
  The duality map $D_M \cn H_c^k(M;R) \rarr H_{n-k}(M;R)$ is an isomorphism for all $k$ whenever $M$ is an $R$-oriented $n$-manifold.
\end{theorem}

\begin{corollary}
  A closed manifold of odd dimension has Euler characteristic zero.
\end{corollary}

The cup and cap product are related by the formula
\[
\psi(\alpha \wcap \phi) = (\phi \wcup \psi)(\alpha)
\]
for $\alpha \in C_{k+\ell}(X;R)$, $\phi \in C^k(X;R)$, and $\psi \in C^\ell(X;R)$. For a closed $R$-orientable $n$-manifold $M$, consider the \emph{cup product pairing}
\[
H^k(M;R) \x H^{n-k}(M;R) \rarr R, \qquad
(\phi,\psi) \mapsto (\phi \wcap \psi)[M].
\]

\begin{proposition}
  The cup product pairing is non-singular for closed $R$-orientable manifolds when $R$ is a field, or when $R = \Z$ and torsion in $H^\bullet(X;R)$ is factored out.
\end{proposition}

\begin{corollary}
  If $M$ is a closed connected orientable $n$-manifold, then for each element $\alpha \in H^k(M)$ of infinite (additive) order that is not a proper multiple of another element, there exists an element $\beta \in H^{n-k}(M)$ such that $\alpha \wcup \beta$ is a generator of $H^n(M)$. With coefficients in a field the same conclusion holds for any $\alpha \neq 0$.
\end{corollary}

Let $H^k_{\textrm{free}}(M)$ denote $H^k(M)$ modulo torsion. If $M$ is closed orientable manifold of dimension $2n$, then the middle-dimensional cup product pairing $H^n_{\textrm{free}}(M) \x H^n_{\textrm{free}}(M) \rarr \Z$ is a nonsingular bilinear form on $H^n_{\textrm{free}}(M)$. This form is symmetric when $n$ is even, and skew-symmetric when $n$ is odd. In the latter case, it is always possible to chose a basis so the bilinear form is given by a matrix formed by $2 \x 2$ blocks $\left(\begin{smallmatrix} 0 & -1 \\ 1 & \phantom{-}0 \end{smallmatrix}\right)$ along the diagonal and $0$ everywhere else. In particular, this implies that if $n$ is odd, then the rank of $H^n(M)$ is even. If $n$ is even, classifying symmetric bilinear forms is an interesting algebraic question. For any given $n$, there are finitely many such but their number grows quickly with $n$.

\begin{theorem}[J.H.C. Whitehead]
  The homotopy type of a simply-connected closed $4$-manifold is uniquely determined by its cup product structure.
\end{theorem}

A compact manifold $M$ with boundary is defined to be $R$-orientable if $M \setminus \d M$ is $R$-orientable as a manifold without boundary. Orientability in the case of boundary implies there exists a fundamental class $[M]$ in $H_n(M, \d M; R)$ restricting to a given orientation at each point of $M \setminus \d M$. The following is a generalization of Poincar\'e duality.

\begin{theorem}
  Suppose $M$ is a compact $R$-orientable $n$-manifold whose boundary $\d M$ is decomposed as the union of two compact $(n-1)$-dimensional manifolds $A$ and $B$ with common boundary $\d A = \d B = A \cap B$. Then cap product with a fundamental class $[M] \in H_n(M, \d M; R)$ gives isomorphisms $D_M \cn H^k(M,A;R) \rarr H_{n-k}(M,B;R)$ for all $k$.
\end{theorem}

The possibility that $A$, $B$, or $A \cap B$ is empty is not excluded. The cases $A = \emptyset$ and $B = \emptyset$ are sometimes called \emph{Lefschetz duality}.

\begin{theorem}[Alexander Duality]
  If $K$ is a compact, locally contractible, non-empty, proper subspace of $S^n$, then $\wtilde{H}_i(S^n \setminus K) \cong \wtilde{H}^{n-i-1}(K)$ for all $i$.
\end{theorem}

\begin{corollary}
  If $X \setminus \R^n$ is compact and locally contractible then $H_i(X)$ is $0$ for $i \geq n$ and torsion-free for $i = n-1$ and $n-2$.
\end{corollary}

\begin{proposition}
  If $K$ is a compact, locally contractible subspace of an orientable $n$-manifold $M$, then there are isomorphisms $H_i(M, M \setminus K) \cong H^{n-i}(K)$ for all $i$.
\end{proposition}

The condition of local contractibility can be removed if one uses \v{C}ech instead of singular cohomology.

\begin{definition}
  Let $M$ and $N$ be connected closed orientable $n$-manifolds with fundamental classes $[M] \in H_n(M)$ and $[N] \in H_n(N)$ respectively. The \emph{degree} of a map $f \cn M \rarr N$, denoted $d = \deg f$, is such an integer that $f_\ast[M] = d [N]$.
\end{definition}

\begin{proposition}
  For any closed orientable $n$-manifold $M$ there is a degree $1$ map $M \rarr S^n$.
\end{proposition}

\begin{proposition}
  Let $f \cn M \rarr N$ be a map between connected closed orientable $n$-manifolds. Suppose $B \subset N$ is a ball such that $f^{-1}(B)$ is the disjoint union of balls $B_i$ each mapped homeomorphically by $f$ onto $B$. Then the degree of $f$ is $\sum_i \epsilon_i$ where $\epsilon_i$ is $+1$ or $-1$ according to whether $f|_{B_i} \cn B_i \rarr B$ preserves or reverses local orientations induced from the given fundamental classes $[M]$ and $[N]$.
\end{proposition}

\begin{proposition}
  A $p$-sheeted covering $M \rarr N$ of connected closed orientable manifolds has degree $\pm p$.
\end{proposition}

\subsubsection{Universal Coefficients for Homology}

\begin{theorem}[Universal Coefficients for Homology]
  For each pair of spaces $(X,A)$ there are split exact sequences
  \[\xymatrix{
    0 \ar[r] & H_n(X,A) \otimes G \ar[r] & H_n(X,A;G) \ar[r] & \Tor(H_{n-1}(X,A),G) \ar[r] & 0
  }\]
  for all $n$, and these sequences are natural with respect to maps $(X,A) \rarr (Y,B)$.
\end{theorem}

The following result enables us to compute the $\Tor$ groups.

\begin{proposition}
  \mbox{}
  \begin{enumerate}[(a)]
  \item $\Tor(A,B) \cong \Tor(B,A)$.
  \item $\Tor(\bigoplus_i A_i, B) \cong \bigoplus_i \Tor(A_i,B)$.
  \item $\Tor(A,B) = 0$ if $A$ or $B$ is free, or more generally torsion-free.
  \item $\Tor(A,B) \cong \Tor(T(A),B)$ where $T(A)$ is the torsion subgroup of $A$.
  \item $\Tor(\Z/n,A) \cong \Ker(A \xrarr{n \cdot} A)$.
  \item For each short exact sequence $0 \rarr B \rarr C \rarr D \rarr 0$ there is a natural associated exact sequence
    \[\xymatrix@C=0.2in{
      0 \ar[r] & \Tor(A,B) \ar[r] & \Tor(A,C) \ar[r] & \Tor(A,D) \ar[r] & A \otimes B \ar[r] & A \otimes C \ar[r] & A \otimes D \ar[r] & 0.
    }\]
  \end{enumerate}
\end{proposition}

\begin{corollary}
  \mbox{}
  \begin{enumerate}[(a)]
  \item $H_n(X;\Q) \cong H_n(X;\Z) \otimes \Q$, so when $H_n(X;\Z)$ is finitely generated, the dimension of $H_n(X;\Q)$ as a $\Q$-vector space equals the rank of of $H_n(X;\Z)$.
  \item If $H_n(X;\Z)$ and $H_{n-1}(X;\Z)$ are finitely generated, then for $p$ prime, $H_n(X;\Z/p)$ consists of
    \begin{enumerate}[(i)]
    \item a $\Z/p$ summand for each $\Z$ summand of $H_n(X;\Z)$,
    \item a $\Z/p$ summand for each $\Z/p^k$ summand in $H_n(X;\Z)$, $k \geq 1$,
    \item a $\Z/p$ summand for each $\Z/p^k$ summand in $H_{n-1}(X;\Z)$, $k \geq 1$.
    \end{enumerate}
  \end{enumerate}
\end{corollary}

\begin{corollary}
  \mbox{}
  \begin{enumerate}[(a)]
  \item $\wtilde{H}_\bullet(X;\Z) = 0$ if and only if $\wtilde{H}_\bullet(X;\Q) = 0$ and $\wtilde{H}_\bullet(X;\Z/p) = 0$ for all primes $p$.
  \item A map $f \cn X \rarr Y$ induces isomorphisms on homology with $\Z$ coefficients if and only if it induces isomorphisms on homology with $\Q$ and $\Z/p$ coefficients for all primes $p$.
  \end{enumerate}
\end{corollary}

\subsubsection{The General K\"unneth Formula}

\begin{theorem}[K\"unneth formula for PID]
  If $X$ and $Y$ are CW complexes and $R$ is a principal ideal domain, then there are split short exact sequences
  \[\xymatrix@C=0.2in{
    0 \ar[r] & \bigoplus_i H_i(X;R) \otimes_R H_{n-i}(Y;R) \ar[r] & H_n(X \x Y; R) \ar[r] & \bigoplus_i \Tor_R(H_i(X;R),H_{n-i-1}(Y;R)) \ar[r] & 0
  }\]
  natural in $X$ and $Y$.
\end{theorem}

\begin{corollary}
  If $F$ is a field and $X$ and $Y$ are CW complexes, then the cross product map
  \[
  h \cn \bigoplus_i H_i(X;F) \otimes_F H_{n-i}(Y;F) \rarr H_n(X \x Y; F)
  \]
  is an isomorphism for all $n$.
\end{corollary}

There is a relative version of the K\"unneth formula which reads
\[\xymatrix@C=0.2in@R=0in{
  0 \ar[r] & \bigoplus_i H_i(X,A;R) \otimes_R H_{n-i}(Y,B;R) \ar[r] & H_n(X \x Y, A \x Y \cup X \x B; R) \ar[r] & \hspace{0.14in} \\
  & \hspace{2.03in} \ar[r] & \bigoplus_i \Tor_R(H_i(X,A;R),H_{n-i-1}(Y,B;R)) \ar[r] & 0.
}\]
In the relative case, this reduces to
\[\xymatrix@C=0.2in{
  0 \ar[r] & \bigoplus_i \wtilde{H}_i(X;R) \otimes_R \wtilde{H}_{n-i}(Y;R) \ar[r] & \wtilde{H}_n(X \wedge Y; R) \ar[r] & \bigoplus_i \Tor_R(\wtilde{H}_i(X;R),\wtilde{H}_{n-i-1}(Y;R)) \ar[r] & 0,
}\]
where $X \wedge Y$ stands for the \emph{smash product} of $X$ and $Y$. It is possible to combine the K\"unneth formula to the more concise form
\[
H_n(X \x Y; R) \cong \bigoplus_i H_i(X; H_{n-i}(Y;R)),
\]
and similarly for relative and reduced homology. There is a version for cohomology which reads
\[
H^n(X \x Y; R) \cong \bigoplus_i H^i(X; H^{n-i}(Y;R)).
\]
Both of these hold if we replace $R$ with an arbitrary coefficient group $G$.

%%% Local Variables: 
%%% mode: latex
%%% TeX-master: "Study_Guide"
%%% End: 

\section{Differential Geometry}
\label{S:differential-geometry}

Syllabus
\begin{description}
\item[Undergraduate] Boothby, \emph{An introduction to differentiable manifolds and Riemannian geometry}, sections VII.1 , VIII.1 and VIII.2. (math 136)
\item[Graduate] Boothby, \emph{An introduction to differentiable manifolds and Riemannian geometry}, chapters I-V and VII. (math 132 and 230a)
\end{description}

TODO

%%% Local Variables: 
%%% mode: latex
%%% TeX-master: "Study_Guide"
%%% End: 

\section{Real Analysis}
\label{S:real-analysis}

Syllabus
\begin{description}
\item[Undergraduate] Royden, \emph{Real Analysis} (3rd ed), chapters 1-10. (math 114).
\item[Graduate] Rudin, \emph{Real and Complex Analysis}, chapters 1-9. (math 212a).
\item[Additional] Stein and Shakarchi, \emph{Real Analysis: Measure Theory, Integration and Hilbert Spaces} may also be a good source for some of this material.
\end{description}

TODO: Royden

\subsection{Rudin, Prologue}

We define the complex function
\[
\exp(z) = \sum_{n=0}^\infty \frac{z^n}{n!},
\]
and often write $e^z = \exp(z)$. The given series converges absolutely for every $z$, and uniformly on every bounded set in the complex plane. It satisfies $\exp(a+b) = \exp(a)\exp(b)$ for all $a,b \in \C$.

\begin{theorem}
  \mbox{}
  \begin{enumerate}[(a)]
  \item For every $z \in \C$, we have $e^z \neq 0$.
  \item $\exp$ is its own derivative.
  \item The restriction of $\exp$ to the real axis is a monotonically increasing positive function, and
    \begin{align*}
      \lim_{x \rarr \infty} e^x &= \infty, &
      \lim_{x \rarr -\infty} e^x &= 0.
    \end{align*}
  \item There exists a positive number $\pi$ such that $e^{\pi i/2} = i$, and such that $e^z = 1$ if and only if $z/(2\pi i)$ is an integer.
  \item $\exp$ is a periodic function with period $2 \pi i$.
  \item The mapping $t \mapsto e^{i t}$ maps the real axis onto the unit circle.
  \item If $w$ is a nonzero complex number, then $w = e^z$ for some $z \in \C$.
  \end{enumerate}
\end{theorem}

\subsection{Rudin, Chapter 1: Abstract Integration}

\begin{definition}
  A collection $\Tc$ of subsets of a set $X$ is said to be a \emph{topology} on $X$ if $\Tc$ has the following three properties:
  \begin{enumerate}[(i)]
  \item $\emptyset \in \Tc$ and $X \in \Tc$;
  \item if $V_i \in \Tc$ for $i = 1, \dots, n$, then $V_1 \cap \cdots \cap V_n \in \Tc$;
  \item if $\{V_\alpha\}$ is an arbitrary collection of members of $\Tc$, then $\bigcup_\alpha V_\alpha \in \Tc$.
  \end{enumerate}
  The elements of $\Tc$ are called \emph{open sets} in $X$, and $X$ is called a \emph{topological space}.
\end{definition}

\begin{definition}
  A map $f \cn X \rarr Y$ between two topological spaces is called \emph{continuous} if $f^{-1}(U)$ is open in $X$ for all open $U \subset Y$.

  A map $f \cn X \rarr Y$ between two topological spaces is called \emph{continuous at $x \in X$} if for every open $V \subset Y$ around $f(x)$, there exists an open $U \subset X$ such that $f(U) \subset V$.
\end{definition}

\begin{proposition}
  A map $f \cn X \rarr Y$ between two topological spaces is continuous if and only if it is continuous around all $x \in X$.
\end{proposition}

\begin{definition}
  A collection $\Mc$ of subsets of a set $X$ is said to be a $\sigma$-algebra on $X$ if $\Mc$ has the following three properties:
  \begin{enumerate}[(i)]
  \item $X \in \Mc$;
  \item if $A \in \Mc$, then $A^c = X \setminus A \in \Mc$;
  \item if $A = \bigcup_{n=1}^\infty A_n$ for some $A_n \in \Mc$, then $A \in \Mc$.
  \end{enumerate}
  We call the elements of $\Mc$ \emph{measurable sets} in $X$, and $X$ a \emph{measurable space}.
\end{definition}

\begin{definition}
  Let $X$ be a measurable space, and $Y$ a topological space. A map $f \cn X \rarr Y$ is called \emph{measurable} if $f^{-1}(U)$ is measurable in $X$ for every open $U \subset Y$. 
\end{definition}

\begin{remark}
  Rudin could have alternatively defined a measurable map between two measurable spaces. What is the motivation behind the more restrictive definition above?
\end{remark}

\begin{proposition}[Further properties of metric spaces]
  Let $X$ be a measurable space with $\sigma$-algebra $\Mc$.
  \begin{enumerate}[(a)]
  \item $\emptyset \in \Mc$.
  \item If $A_i \in \Mc$ for $1 \leq i \leq n$, then $\bigcup_{i=1}^n A_n \in \Mc$.
  \item If $A_i \in \Mc$ for $i \geq 1$, then $\bigcap_{n=1}^\infty A_n \in \Mc$.
  \item If $A, B \in \Mc$, then $A \setminus B \in \Mc$.
  \end{enumerate}  
\end{proposition}

\begin{theorem}
  Let $g \cn Y \rarr Z$ be a continuous function between topological spaces.
  \begin{enumerate}[(a)]
  \item If $f \cn X \rarr Y$ is a continuous function for a topological space $X$, then $h = g \circ f \cn X \rarr Z$ is continuous.
  \item Let $f \cn X \rarr Y$ is a measurable function for a measurable space $X$, then $h = g \circ f \cn X \rarr Z$ is measurable.
  \end{enumerate}
\end{theorem}

\begin{theorem}
  Let $u$ and $v$ be real measurable functions on a measurable space $X$, and $\Phi \cn \R^2 \rarr Y$ a continuous map for a topological space $Y$. Then $h \cn X \rarr Y$ defined by $h(x) = \Phi(u(x),v(x))$ is measurable.
\end{theorem}

\begin{corollary}
  Let $X$ be a measurable space.
  \begin{enumerate}[(a)]
  \item If $f = u + i v$, where $u$ and $v$ are real measurable functions on $X$, then $f$ is a complex measurable function on $X$.
  \item If $f = u + i v$ is a complex measurable function on $X$, then $u$, $v$, and $|f|$ are real measurable functions on $X$.
  \item If $f$ and $g$ are complex measurable functions on $X$, then so are $f + g$ and $f g$.
  \item If $E \subset X$ is a measurable set, then the function $\chi_E \cn X \rarr \C$ given by
    \[
    \chi_E(x) =
    \begin{cases}
      1 & \textrm{if } x \in E, \\
      0 & \textrm{if } x \notin E
    \end{cases}
    \]
    is measurable.
  \item If $f$ is a complex measurable function on $X$, there is a complex measurable function $\alpha$ on $X$ such that $|\alpha| = 1$ and $f = \alpha |f|$.
  \end{enumerate}
\end{corollary}

\begin{theorem}
  If $\Fc$ is any collection of subsets of $X$, then there exists a smallest $\sigma$-algebra $\Mc$ in $X$ such that $\Fc \subset \Mc$. We say $\Mc$ is \emph{generated by $\Fc$}.
\end{theorem}

\begin{definition}
  Let $X$ be a topological space. The \emph{Borel measure} $\Bc$ on $X$ is generated by the opens in $X$. The elements of $\Bc$ are called \emph{Borel sets}.
\end{definition}

\begin{proposition}
  Every continuous map is \emph{Borel measurable}, that is, measurable when the domain is endowed with the Borel measure.
\end{proposition}

\begin{theorem}
  Let $X$ be a measurable space with $\sigma$-algebra $\Mc$, $Y$ a topological space, and $f \cn X \rarr Y$ a map.
  \begin{enumerate}[(a)]
  \item If $\Omega$ is the collection of all sets $E \subset Y$ such that $f^{-1}(E) \in \Mc$, then $\Omega$ is a $\sigma$-algebra on $Y$. In short, pushforwards of $\sigma$-algebras are $\sigma$-algebras.
  \item If $f$ is measurable and $E$ is a Borel set in $Y$, then $f^{-1}(E) \in \Mc$.
  \item If $Y = [-\infty,\infty]$ and $f^{-1}((\alpha,\infty]) \in \Mc$ for all $\alpha \in \R$, then $f$ is measurable.
  \end{enumerate}
\end{theorem}

\begin{definition}
  Let $\{a_n\}$ be a sequence in $[-\infty,\infty]$, and put
  \begin{align*}
    \limsup_{n \rarr \infty} a_n &= \inf\{ \sup\{ a_k, a_{k+1}, \dots \} \;|\; k \geq 1 \}, \\
    \liminf_{n \rarr \infty} a_n &= \sup\{ \inf\{ a_k, a_{k+1}, \dots \} \;|\; k \geq 1 \}.
  \end{align*}
\end{definition}

Then
\[
\liminf_n a_n = -\limsup_n(-a_n).
\]
If $\{a_n\}$ converges, then
\[
\lim_n a_n = \limsup_n a_n = \liminf_n a_n.
\]

Suppose $\{f_n\}$ is a sequence of extended-real function on a set $X$. Then $\sup_n f_n$ and $\limsup_n f_n$ are functions defined on $X$ by
\begin{align*}
  \left( \sup_n f_n \right)(x) &= \sup_n(f_n(x)), \\
  \left( \limsup_n f_n \right)(x) &= \limsup_n(f_n(x)).
\end{align*}
Similarly for $\inf_n f_n$ and $\liminf_n f_n$. If $f(x) = \lim_n f_n(x)$ is well-defined for all $x \in X$, then we call $f$ the \emph{pointwise limit} of the sequence $\{f_n\}$.

\begin{theorem}
  If $f_n \cn X \rarr [-\infty,\infty]$ is measurable for $n \geq 1$, then so are $\sup_n f_n$ and $\limsup_n f_n$.
\end{theorem}

\begin{corollary}
  \mbox{}
  \begin{enumerate}[(a)]
  \item The limit of every pointwise convergent sequence of complex measurable functions is measurable.
  \item If $f$ and $g$ are measurable (with range $[-\infty,\infty]$), then so are $\max\{f,g\}$ and $\min\{f,g\}$. In particular, this is true of the functions $f^+ = \max\{f,0\}$ and $f^- = -\min\{f,0\}$. These are respectively called the \emph{positive} and \emph{negative parts of $f$}. We have $f = f^+ - f^-$ and $|f| = f^+ + f^-$.
  \end{enumerate}
\end{corollary}

\begin{proposition}
  If $f = g-h$, $g \geq 0$, and $f \geq 0$, then $f^+ \leq g$ and $f^- \leq h$.
\end{proposition}

\begin{definition}
  A function $s$ on a measurable space $X$ whose range consists of only finitely many points in $[0,\infty)$ is called \emph{simple}.
\end{definition}
If $\alpha_1, \dots, \alpha_n$ are the distinct values assumed by $s$, and $A_i = f^{-1}(\alpha_i)$, then $s = \sum_{i=1}^n \alpha_i \chi_{A_i}$. The function $s$ is measurable if and only if all $A_i$ are measurable.

\begin{theorem}
  Let $f \cn X \rarr [0,\infty]$ be a measurable function. There exists simple measurable functions $s_n$ on $X$ such that:
  \begin{enumerate}[(a)]
  \item $0 \leq s_1 \leq s_2 \leq \cdots \leq f$;
  \item $s_n \rarr f$ pointwise.
  \end{enumerate}
\end{theorem}

\begin{remark}
  It can be shown using the construction in the proof of the theorem above, that if $f$ is bounded, then $s_n$ is a uniformly convergent sequence.
\end{remark}

\begin{definition}
  \mbox{}
  \begin{enumerate}[(a)]
  \item A \emph{(positive) measure} is a function $\mu$, defined on a $\sigma$-algebra $\Mc$, whose range is in $[0,\infty]$ and which is \emph{countably additive}. This means that is $\{A_i\}$ is a disjoint countable collection of elements of $\Mc$, then
    \[
    \mu\left( \bigcup_{i=1}^\infty A_i \right) = \sum_{i=1}^\infty \mu(A_i).
    \]
    To avoid trivialities, we shall also assume that $\mu(A) < \infty$ for at least one $A \in \Mc$.
  \item A \emph{measure space} is a measurable space which has a positive measure defined on the $\sigma$-algebra of its measurable sets.
  \item A \emph{complex measure} is a complex-valued countable additive function defined on a $\sigma$-algebra.
  \end{enumerate}
\end{definition}

\begin{theorem}
  For any positive measure $\mu$ on a $\sigma$-algebra $\Mc$, the following holds:
  \begin{enumerate}[(a)]
  \item $\mu(\emptyset) = 0$;
  \item if $A_1, \dots, A_n$ are pairwise disjoint elements of $\Mc$, then $\mu(A_1 \cup \cdots \cup A_n) = \mu(A_1) + \cdots + \mu(A_n)$;
  \item if $A, B \in \Mc$ and $A \subset B$, then $\mu(A) \leq \mu(B)$;
  \item if $A = \bigcup_{n=1}^\infty A_n$ for $A_1 \subset A_2 \subset \cdots$ where all $A_n$ are elements of $\Mc$, then $\mu(A_n) \rarr \mu(A)$ as $n \rarr \infty$;
  \item if $A = \bigcap_{n=1}^\infty A_n$ for $A_1 \supset A_2 \supset \cdots$ where all $A_n$ are elements of $\Mc$, and $\mu(A_1)$ is finite, then $\mu(A_n) \rarr \mu(A)$ as $n \rarr \infty$.
  \end{enumerate}
\end{theorem}

\begin{example}
  Let $X$ be any set.
  \begin{enumerate}[(a)]
  \item For any $E \subset X$ define $\mu(E) = \infty$ if $E$ is infinite, and $\mu(E) = |E|$ if $E$ is finite. This $\mu$ is called the \emph{counting measure on $X$}.
  \item For $x_0 \in X$, define $\mu(E) = 1$ is $x_0 \in E$, and $\mu(E) = 0$ otherwise. This $\mu$ is called the \emph{unit mass concentrated at $x_0 \in X$}.
  \end{enumerate}
\end{example}

To make sense of arithmetic on $[0,\infty]$, we define additionally $a + \infty = \infty + a = \infty$ if $0 \leq a \leq \infty$, and
\[
a \cdot \infty = \infty \cdot a =
\begin{cases}
  \infty & \textrm{if } 0 < a \leq \infty, \\
  0 & \textrm{if } a = 0.
\end{cases}
\]
This ensures the commutative, associative, and distributive laws hold in $[0,\infty]$ without any restriction. It is important to note that cancellation laws have to be treated with some care: $a + b = a + c$ implies $b=c$ only when $a < \infty$, and $ac = ac$ implies $b=c$ only when $0 < a < \infty$.

\begin{proposition}
  If $0 \leq a_1 \leq a_2 \leq \cdots$, $0 \leq b_1 \leq b_2 \leq \cdots$, $a_n \rarr a$, and $b_n \rarr b$, then $a_n b_n \rarr a b$.
\end{proposition}

\begin{corollary}
  Sums and products of measurable functions into $[0,\infty]$ are measurable.
\end{corollary}

\begin{definition}
  If $s \cn X \rarr [0,\infty)$ is a measurable simple function of the form $s = \sum_{i=1}^n \alpha_i \chi_{A_i}$, where $\alpha_1, \dots, \alpha_n$ are distinct values of $s$, and if $E \in \Mc$, we define
  \[
  \int_E s d\mu = \sum_{i=1}^n \alpha_i \mu(A_i \cap E).
  \]
  If $f \cn X \rarr [0, \infty]$ is measurable, and $E \in \Mc$, we define
  \[
  \int_E f d\mu = \sup \int_E s d\mu,
  \]
  where the supremum is taken over all simple measurable functions $s$ such that $0 \leq s \leq f$.
\end{definition}
This is called the \emph{Lebesgue integral of $f$ over $E$, with respect to the measure $\mu$}.

\begin{proposition}
  \mbox{}
  \begin{enumerate}[(a)]
  \item If $0 \leq f \leq g$, then $\int_E f d\mu \leq \int_E g d\mu$.
  \item If $A \subset B$ and $f \geq 0$, then $\int_A f d\mu \leq \int_B f d\mu$.
  \item If $f \geq 0$ and $c \in [0,\infty)$ is a constant, then $\int_E c f d\mu = c \int_E f d\mu$.
  \item If $f(x) = 0$ for all $x \in E$, then $\int_E f d\mu = 0$, even if $\mu(E) = \infty$.
  \item If $\mu(E) = 0$, then $\int_E f d\mu = 0$, even if $f(x) = \infty$ for all $x \in E$.
  \item If $f \geq 0$, then $\int_E f d\mu = \int_X \chi_E f d\mu$.
  \end{enumerate}
\end{proposition}

\begin{proposition}
  Let $s$ and $t$ be nonnegative measurable simple functions on $X$. For $E \in \Mc$, define $\phi(E) = \int_E s d\mu$. Then $\phi$ is a measure on $\Mc$. Also $\int_X (s+t) d\mu = \int_X s d\mu + \int_E t d\mu$.
\end{proposition}

\begin{theorem}[Lebesgue's Monotone Convergence Theorem]
  Let $\{f_n\}$ be a sequence of measurable functions on $X$, and suppose that
  \begin{enumerate}[(a)]
  \item $0 \leq f_1 \leq f_2 \leq \cdots$,
  \item $f_n \rarr f$ pointwise on $X$.
  \end{enumerate}
  Then $f$ is measurable, and
  \[
  \lim_n \int_X f_n d\mu = \int_X f d\mu.
  \]
\end{theorem}

\begin{theorem}
  If $f_n \cn X \rarr [0,\infty]$ is measurable for $n \geq 1$, and $f = \sum_{n=1}^\infty f_n$, then $\int_X f d\mu = \sum_{n=1}^\infty \int_X f_n d\mu$.
\end{theorem}

The following follows by employing the counting measure on $\Z^+$.

\begin{corollary}
  If $a_{i j} \geq 0$ for all $i,j \geq 0$, then
  \[
  \sum_{i=1}^\infty \sum_{j=1}^\infty a_{i j} = \sum_{j=1}^\infty \sum_{i=1}^\infty a_{i j}.
  \]
\end{corollary}

\begin{proposition}[Fatou's Lemma]
  If $f_n \cn X \rarr [0,\infty]$ is measurable for all $n \geq 1$, then
  \[
  \int_X \left( \liminf_n f_n \right) d\mu \leq \liminf_n \int_X f_n d\mu.
  \]
\end{proposition}

\begin{theorem}
  Suppose $f \cn X \rarr [0,\infty]$ is measurable, and $\phi(E) = \int_E f d\mu$ for $E \in \Mc$. Then $\phi$ is a measure on $\Mc$, and
  \[
  \int_X f d\phi = \int_X d f d\mu.
  \]
\end{theorem}

\begin{definition}
  Let $L^1(\mu)$ be the collection of all complex measurable functions $f$ on $X$ for which $\int_X |f| d\mu < \infty$. The elements of $L^1(\mu)$ are called \emph{Lebesgue measurable functions with respect to $\mu$}.
\end{definition}

\begin{definition}
  If $f = u + i v$ for $u$ and $v$ real measurable functions on $X$, and if $f \in L^1(\mu)$, then we define
  \[
  \int_E f d\mu = \int_E u^+ d\mu - \int_E u^- d\mu + i \int_E v^+ d\mu - i \int_E v^- d\mu
  \]
  for every $E \in \Mc$. In some circumstances it is desirable to extend this to the case when $f$ is a measurable function with range in $[-\infty,\infty]$. Then, we set $\int_E f d\mu = \int_E f^+ d\mu - \int_E f^- d\mu$, provided that at least one of the integrals on the right is finite.
\end{definition}

\begin{theorem}
  If $f, g \in L^1(\mu)$ and $\alpha,\beta \in \C$, then $\alpha f + \beta g \in L^1(\mu)$, and
  \[
  \int_X (\alpha f + \beta g) d\mu = \alpha \int_X f d\mu + \beta \int_X f d\mu.
  \]
\end{theorem}

\begin{theorem}
  If $f \in L^1(\mu)$, then $\left| \int_X f d\mu \right| \leq \int_X |f| d\mu$.
\end{theorem}

\begin{theorem}[Lebesgue's Dominated Convergence Theorem]
  Suppose $\{f_n\}$ is a sequence of complex measurable functions on $X$ such that $f = \lim_n f_n$ is defined pointwise on all $X$. If there is a function $g \in L^1(\mu)$ such that $|f_n| \leq g$ for all $n \geq 1$, then $f \in L^1(\mu)$,
  \[
  \lim_n \int_X |f_n - f| d\mu = 0,
  \]
  and
  \[
  \lim_n \int_X f_n d\mu = \int_X f d\mu.
  \]
\end{theorem}

\begin{definition}
  Let $P$ be a property which a point $x \in X$ may or may not have. If $\mu$ is a measure on a $\sigma$-algebra $\Mc$ and if $E \in \Mc$, the statement ``$P$ holds almost everywhere on $E$'' (abbreviated ``$P$ holds a.e.\ on $E$'') means that there exists an $N \in \Mc$ such that $\mu(N) = 0$, $N \subset E$, and $P$ holds at every point of $E \setminus N$. Note that the concept depends strongly on the used measure $\mu$.
\end{definition}

\begin{definition}
  If $f$ and $g$ are measurable functions and if $\mu(\{ x \;|\; f(x) \neq g(x) \}) = 0$, we say that $f = g$ a.e.\ on $X$, and write $f \sim g$. Note that $\sim$ is an equivalence relation.
\end{definition}

\begin{proposition}
  If $f \sim g$, then, for every $E \in \Mc$, we have $\int_E f d\mu = \int_E g d\mu$.
\end{proposition}

\begin{theorem}
  Let $(X, \Mc, \mu)$ be a measure space, and $\Mc^\ast$ be the collection of all $E \subset X$ for which there exist $A, B \in \Mc$ with $A \subset E \subset B$ and $\mu(B \setminus A) = 0$, and define $\mu(E) = \mu(A)$ in this situation. Then $\Mc^\ast$ is a $\sigma$-algebra, and $\mu$ is a measure on $\Mc^\ast$.
\end{theorem}

\begin{definition}
  The extended measure $\mu$ is called \emph{complete}, since all subsets of sets of measure $0$ are now measurable; the $\sigma$-algebra $\Mc^\ast$ is called the $\mu$-completion of $\Mc$.
\end{definition}

In light of the previous theorem and definition, we may only require functions to be well-defined and measurable on the complement of a zero measure set. This is sometimes a very useful trick.

\begin{theorem}
  Suppose $\{f_n\}$ is a sequence of complex measurable functions defined a.e.\ on $X$ such that
  \[
  \sum_{n=1}^\infty \int_X |f_n| d\mu < \infty.
  \]
  Then the series $f(x) = \sum_{n=1}^\infty f_n(x)$ converges for almost all $x$, $f \in L^1(\mu)$, and
  \[
  \int_X f d\mu = \sum_{n=1}^\infty \int_X f_n d\mu.
  \]
\end{theorem}

\begin{theorem}
  \mbox{}
  \begin{enumerate}[(a)]
  \item If $f \cn X \rarr [0,\infty]$ is measurable, $E \in \Mc$, and $\int_E f d\mu = 0$, then $f = 0$ a.e.\ on $E$.
  \item If $f \in L^1(\mu)$ and $\int_E f d\mu = 0$ for every $E \in \Mc$, then $f = 0$ a.e.\ on $X$.
  \item If $f \in L^1(\mu)$ and $\left| \int_X f d\mu \right| = \int_X |f| d\mu$, then there is a constant $\alpha$ such that $\alpha f = |f|$ a.e.\ on $X$.
  \end{enumerate}
\end{theorem}

\begin{theorem}
  Suppose $\mu(X) < \infty$, $f \in L^1(\mu)$, $S$ is a closed set in the complex plane, and the averages
  \[
  A_E(f) = \frac{1}{\mu(E)} \int_E f d\mu
  \]
  lie in $S$ for all $E \in \Mc$ with $\mu(E) > 0$. Then $f(x) \in S$ for almost all $x \in X$.
\end{theorem}

\begin{theorem}
  Let $\{E_k\}$ be a sequence of measurable sets in $X$, such that $\sum_{k=1}^\infty \mu(E_k) < \infty$. Then almost all $x \in X$ lie in at most finitely many of the sets $E_k$.
\end{theorem}

\subsection{Rudin, Chapter 2: Positive Borel Measures}

\begin{definition}
  Let $X$ be a topological space.
  \begin{enumerate}[(a)]
  \item A set $E \subset X$ is called \emph{closed} if its complement $E^c$ is open.
  \item The \emph{closure} $\overline{E}$ of $E \subset X$ is the smallest closed set in $X$ which contains $E$.
  \item A topological space is called \emph{compact} if every open cover has a finite subcover.
  \item A neighbourhood of a point $p \in X$ is an open set which contains $p$.
  \item $X$ is \emph{Hausdorff} if for every two points can be separated by opens.
  \item $X$ is \emph{locally compact} if every point of $X$ has a neighbourhood whose closure is compact (alternatively, for every $p \in X$, there is a neighbourhood $U$ of $p$, and compact $K \subset X$ satisfying $U \subset K$).
  \end{enumerate}
\end{definition}

\begin{theorem}[Heine-Borel Theorem]
  A subset of an Euclidean space is compact if and only it is closed and bounded.
\end{theorem}

\begin{proposition}
  Every metric space is Hausdorff.
\end{proposition}

\begin{theorem}
  A closed subset of a compact space is also compact.
\end{theorem}

\begin{corollary}
  If $A \subset B \subset X$ and $B$ has a compact closure, then so does $A$.
\end{corollary}

\begin{theorem}
  Suppose $X$ is Hausdorff, $K \subset X$ is compact, and $p \in K^c$. Then there exist open sets $U$ and $V$ such that $p \in U$, $K \subset V$, and $U \cap V = \emptyset$.
\end{theorem}

\begin{corollary}
  \mbox{}
  \begin{enumerate}[(a)]
  \item Compact subsets of Hausdorff spaces are closed.
  \item In a Hausdorff space, the intersections of a closed and a compact subset is compact.
  \end{enumerate}
\end{corollary}

\begin{theorem}
  If $\{K_\alpha\}$ is a collection of compact subsets of a Hausdorff space and their intersection is empty, then there exists a finite subcollection with empty intersection.
\end{theorem}

\begin{theorem}
  Suppose $X$ is a locally compact Hausdorff space, $U \subset X$, and $K \subset X$ is compact. Then there exists an open $V \subset X$ with compact closure which satisfies
  \[
  K \subset V \subset \overline{V} \subset U.
  \]
\end{theorem}

\begin{definition}
  Let $X$ be a topological space, and consider a function $f \cn X \rarr \R$.
  \begin{enumerate}[(i)]
  \item We call $f$ \emph{lower semicontinuous} if $f^{-1}((\alpha,\infty))$ is open for all $\alpha \in \R$.
  \item We call $f$ \emph{upper semicontinuous} if $f^{-1}((-\infty,\alpha))$ is open for all $\alpha \in \R$.
  \end{enumerate}
  An alternative definition could be made by replacing $\R$ with $[-\infty,\infty]$.
\end{definition}

\begin{proposition}
  Let $X$ be a topological space. A function $f \cn X \rarr \R$ is continuous if and only if it is both upper and lower semicontinuous.
\end{proposition}

\begin{example}
  \mbox{}
  \begin{enumerate}[(a)]
  \item Characteristic functions of open sets are lower semicontinuous.
  \item Characteristic functions of closed sets are upper semicontinuous.
  \item The supremum of any collection of lower semicontinuous function is lower semicontinuous.
  \item The infimum of any collection of upper semicontinuous function is upper semicontinuous.
  \end{enumerate}
\end{example}

\begin{definition}
  Let $X$ be a topological space, and $f \cn X \rarr \C$ a function. The \emph{support of $f$} is
  \[
  \Supp f = \overline{f^{-1}(\C \setminus \{0\})}.
  \]
  The collection of all continuous complex functions on $X$ with compact support is denoted $\Cc_c(X)$.
\end{definition}

Note that $\Cc_c(X)$ is a vector space over $\C$. We could have also replaced ``continuous'' above with ``measurable'' to yield another interesting object of study.

\begin{theorem}
  The image of a compact set under a continuous map is compact.
\end{theorem}

\begin{corollary}
  The image of any $f \in \Cc_c(X)$ is compact in $\C$.
\end{corollary}

\begin{definition}
  For any compact $K \subset X$, the notation
  \[
  K \prec f
  \]
  will mean that $f \in \Cc_c(X)$, $0 \leq f \leq 1$, and $f|_K = 1$. Similarly, for any open $U \subset X$, the notation
  \[
  f \prec U
  \]
  will mean that $f \in \Cc_c(X)$, $0 \leq f \leq 1$, and $\Supp f \subset U$. Then, $K \prec f \prec U$ will simultaneously mean $K \prec f$ and $f \prec U$.
\end{definition}

\begin{proposition}[Urysohn's Lemma]
  Let $X$ be a locally compact Hausdorff space, $U \subset X$ an open, and $K \subset X$ a compact. Then there exists $f \in \Cc_c(X)$ such that $K \prec f \prec U$.
\end{proposition}

\begin{theorem}
  Compact Hausdorff spaces admit \emph{partitions of unity}. That is, for any compact space $K$, and a (finite) open cover $U_1, \dots, U_n$, there exist $f_i \in \Cc(X)$ satisfying $f_i \prec U_i$, and $\sum_i f_i = 1$. The statement also holds true for compact subspaces of locally compact Hausdorff spaces.
\end{theorem}

\begin{theorem}[Riesz Representation Theorem]
  Let $X$ be a locally compact Hausdorff space, and $\Lambda \cn \Cc_c(X) \rarr \C$ be a \emph{positive linear functional}, that is, $\Lambda$ is a morphism of vector spaces over $\C$, and for every $f \geq 0$, we have $\Lambda(f) \geq 0$. Then there exists a $\sigma$-algebra $\Mc$ on $X$ containing all Borel sets of $X$, and a unique positive measure $\mu$ on $\Mc$ which represents $\Lambda$ in the sense that:
  \begin{enumerate}[(a)]
  \item $\Lambda(f) = \int_X f d\mu$ for every $f \in \Cc_c(X)$;
  \item $\mu(K) < \infty$ for every compact $K \subset X$;
  \item for every $E \in \Mc$, we have
    \[
    \mu(E) = \inf\{ \mu(U) \;|\; E \subset U, U \textrm{ is open} \};
    \]
  \item the relation
    \[
    \mu(E) = \sup\{ \mu(K) \;|\; K \subset E, K \textrm{ is compact} \}
    \]
    holds for every open $E \subset X$, and for every $E \in \Mc$ with $\mu(E) < \infty$;
  \item $\Mc$ is complete, i.e., if $E \in \Mc$, $A \subset E$, and $\mu(E) = 0$, then $A \in \Mc$.
  \end{enumerate}
\end{theorem}

\begin{definition}
  Let $X$ be a locally compact Hausdorff space $X$. A measure $\mu$ defined on the $\sigma$-algebra of Borel sets is called a \emph{Borel measure}.
\end{definition}

\begin{definition}
  Assume $\mu$ is a positive Borel measure on $X$. We call it \emph{outer (resp.\ inner) regular} if every Borel set $E \subset X$ has property (c) (resp.\ property (d)) from the Riesz Representation Theorem. If a measure is both inner and outer regular, then we call it \emph{regular}.
\end{definition}

\begin{definition}
  A topological space is called \emph{$\sigma$-compact} if it is the union of finitely many compacts, or alternatively, every open cover has a countable subcover. We call a topological space \emph{$\sigma$-locally compact} if it is both $\sigma$- and locally compact.

  A set $E$ in a measure space is said to have \emph{$\sigma$-finite measure} if $R$ is a countable union of measurable sets $E_i$ with $\mu(E_i) < \infty$.
\end{definition}

\begin{theorem}
  Let $X$ be a $\sigma$-locally compact Hausdorff space. If $\Mc$ and $\mu$ as in the Riesz Representation Theorem, then they have the following properties:
  \begin{enumerate}[(a)]
  \item for any $E \in \Mc$ and $\epsilon > 0$, there is a closed $F \subset X$ and an open $U \subset X$ satisfying $F \subset E \subset U$ and $\mu(U \setminus F) < \epsilon$;
  \item $\mu$ is a regular Borel measure on $X$;
  \item for any $E \in \Mc$, there are sets $A$ and $B$ such that $A$ is a countable union of closed sets, $B$ is a countable intersection of open sets, $A \subset E \subset B$, and $\mu(B \setminus A) = 0$.
  \end{enumerate}
\end{theorem}

\begin{theorem}
  Let $X$ be a locally compact Hausdorff space in which every open is $\sigma$-compact. If $\mu$ is a positive Borel measure on $X$ such that $\mu(K) < \infty$ for all compact $K \subset X$, then $\mu$ is regular.
\end{theorem}

\begin{definition}
  Let $W = \prod_{i=1}^k I_i \subset \R^k$ where $I_i \subset \R$ are intervals (open, closed, or any combination thereof) with endpoints $\alpha_i$ and $\beta_i$. We call such a set a \emph{$k$-cell}. The \emph{volume of $W$} is defined to be $\vol(W) = \prod_{i=1}^k (\beta_i - \alpha_i)$.

  If $a = (a_1, \dots, a_k) \in \R^k$ and $\delta > 0$, we call the set
  \[
  Q(a, \delta) = \prod_{i=1}^k [a_i, a_i+\delta)
  \]
  a \emph{$\delta$-box with corner at $a$}.

  For an integer $n \geq 1$, let $\Omega_n$ denote the set of $2^{-n}$-boxes with corners at points of $P_n = 2^{-n} \Z^k$.
\end{definition}

\begin{proposition}
  \mbox{}
  \begin{enumerate}[(a)]
  \item For any fixed $n \geq 1$, the set $\Omega_n$ is a partition of $\R^k$.
  \item If $Q \in \Omega_n$, $Q' \in \Omega_r$, and $r < n$, then either $Q \subset Q'$ or $Q \cap Q' = \emptyset$.
  \item If $Q \in \Omega_r$, then $\vol(Q) = 2^{-r k}$; and if $r < n$, then $|P_n \cap Q| = 2^{(n-r)k}$.
  \item Every nonempty open in $\R^k$ is a countable union of disjoint boxes in $\bigcup_{n \geq 1} \Omega_n$.
  \end{enumerate}
\end{proposition}

\begin{theorem}
  There exists a positive complete measure $m$ defined on a $\sigma$-algebra $\Mc$ on $\R^k$ with the following properties:
  \begin{enumerate}[(a)]
  \item $m(W) = \vol(W)$ for every $k$-cell $W$;
  \item $\Mc$ contains all Borel sets of $\R^k$; more precisely, $E \in \Mc$ if and only if there exists $A,B \subset \R^k$ such that $A \subset E \subset B$, $A$ is a countable union of closed sets, $B$ is a countable union of opens, and $m(B \setminus A) = 0$; also, $m$ is regular;
  \item $m$ is translation-invariant, i.e., for every $E \in \Mc$ and $x \in \R^k$, we have $m(E+x) = m(E)$;
  \item if $\mu$ is any positive translation-invariant Borel measure on $\R^k$ such that $\mu(K) < \infty$ for every compact $K \subset \R^k$, then there is a constant $c > 0$ such that $\mu(E) = c \, m(E)$ for all Borel sets $E \subset \R^k$;
  \item for every linear transformation $T \cn \R^k \rarr \R^k$, we have $m(T(E)) = \det T \, m(E)$ for all $E \in \Mc$; in particular, $m(T(E)) = m(E)$ when $T$ is a rotation.
  \end{enumerate}
\end{theorem}

\begin{definition}
  The members of $\Mc$ are called \emph{Lebesgue measurable sets in $\R^k$}; $m$ is the \emph{Lebesgue measure on $\R^k$}. When clarity requires it, we shall write $m_k$ in place of $m$.

  For any $E \subset \R^k$, the measure $m$ induces a restricted measure $m_E$ on $E$ (and also a $\sigma$-algebra $\Mc_E$). We will write $L^1(E) = L^1(m_E)$, and, in particular, $L^1(\R^k) = L^1(m)$.

  If $k = 1$, $I$ is an interval (open, closed, or any combination thereof) with endpoints $a$ and $b$, and $f \in L^1(I)$, it is customary to denote
  \[
  \int_a^b f(x) d x = \int_I f d m.
  \]
\end{definition}

\begin{proposition}
  For any continuous complex function $f$ on $[a,b]$, the Riemann integral of $f$ and the Lebesgue integral of $f$ over $[a,b]$ agree.
\end{proposition}

\begin{remark}
  A natural question to ask is whether all elements of $\Mc$ are Borel. Another such is whether all subsets of $\R^k$ are measurable. The answer of both of these is negative (for the latter one see the following result).
\end{remark}

\begin{theorem}
  If $A \subset \R$ and every subset of $A$ is Lebesgue measurable, then $m(A) = 0$.
\end{theorem}

\begin{corollary}
  Every set of positive measure has a nonmeasurable subset.
\end{corollary}

\begin{theorem}[Lusin's Theorem]
  Suppose $f$ is a complex measurable function on $X$, $\mu(A) < \infty$, $f|_{X \setminus A} = 0$, and $\epsilon > 0$. Then there exists a $f \in \Cc_c(X)$ such that
  \[
  \mu(\{ x \;|\; f(x) \neq g(x)\}) < \epsilon.
  \]
  Furthermore, we may arrange so that
  \[
  \sup_{x \in X} |g(x)| \leq \sup_{x \in X} |f(x)|.
  \]
\end{theorem}

\begin{corollary}
  Assume the hypotheses of Lusin's Theorem are satisfied and that $|f| \leq 1$. Then there is a sequence $\{g_n\}$ such that $g_n \in \Cc_c(X)$, $|g_n| \leq 1$, and
  \[
  f(x) = \lim_n g_n(x) \qquad a.e.
  \]
\end{corollary}

\begin{theorem}[Vitali-Carath\'eodory Theorem]
  Suppose $f \in L^1(\mu)$, $f$ is real-valued, and $\epsilon > 0$. Then there exist functions $u, v \cn X \rarr \R$ such that $u \leq f \leq v$, $u$ is upper semicontinuous and bounded above, $v$ is lower semicontinuous and bounded below, and $\int_X (v - u) d\mu < \epsilon$.
\end{theorem}

\subsection{Rudin, Chapter 3: $L^p$-spaces}

\begin{definition}
  Consider an interval $(a,b) \subset [-\infty,\infty]$. A function $\phi \cn (a,b) \rarr \R$ is called \emph{convex} if the inequality
  \[
  \phi((1-\lambda) x + \lambda y) \leq (1-\lambda) \phi(x) + \lambda \phi(y)
  \]
  holds for all $a < x,y < b$, and $0 \leq \lambda \leq 1$.

  Graphically, the above means for any $x,y \in (a,b)$ the graph of $\phi$ between these two points is below the straight line connecting $(x,\phi(x))$ and $(y,\phi(y))$. Alternatively, convexity is equivalent to the inequality
  \[
  \frac{\phi(t)-\phi(s)}{t-s} \leq \frac{\phi(u)-\phi(t)}{u-t}
  \]
  for all $a < s < t < u < b$.
\end{definition}

The following is a consequence of the mean value theorem for differentiation and the alternative formulation of convexity above.

\begin{proposition}
  A differentiable $\phi \cn (a,b) \rarr \R$ is convex if and only if $\phi'$ is monotonically increasing.
\end{proposition}

\begin{theorem}
  If $\phi \cn (a,b) \rarr \R$ is convex, then it is also continuous.
\end{theorem}

\begin{remark}
  The statement above would not hold had the domain not been an open interval.
\end{remark}

\begin{theorem}[Jensen's Inequality]
  Let $\mu$ be a positive measure on a $\sigma$-algebra $\Mc$ on a set $X$ with $\mu(X) = 1$. If $f \in L^1(\mu)$, $a < f < b$, and $\phi$ is convex on $(a,b)$, then
  \[
  \phi\left( \int_X f d\mu \right) \leq \int_X (\phi \circ f) d\mu.
  \]
\end{theorem}

\begin{remark}
  The cases $a = -\infty$ and $b = \infty$ are not excluded. In either of these, it may occur that $\phi \circ f$ is not an element of $L^1(\mu)$. Then its integral exists in the extended sense and equals $+\infty$, in other words, the equality is vacuous.
\end{remark}

\begin{example}
  \mbox{}
  \begin{enumerate}[(a)]
  \item Taking $\phi(x) = e^x$, we obtain
    \[
    \exp\left( \int_X f d\mu \right) \leq \int_X e^f d\mu.
    \]
  \item Taking $X = \{p_1, \dots, p_n\}$, $\mu(\{p_i\}) = 1/n$ for all $1 \leq i \leq n$, and $f(p_i) = x_i$, we obtain
    \[
    \exp\frac{x_1 + \cdots + x_n}{n} \leq \frac{e^{x_1} + \cdots + e^{x_n}}{n}.
    \]
    Taking $y_i = e^{x_i}$, this turns into the AM-GM inequality
    \[
    (y_1 \cdots y_n)^{1/n} \leq \frac{y_1 + \cdots + y_n}{n}.
    \]
  \item If we modify the previous example only by setting $\mu(\{p_i\}) = \alpha_i > 0$ with $\sum_{i=1}^n \alpha_i = 1$, we obtain
    \[
    y_1^{\alpha_1} \cdots y_n^{\alpha_n} \leq \alpha_1 y_1 + \cdots + \alpha_n y_n.
    \]
  \end{enumerate}
\end{example}

\begin{definition}
  If two positive real numbers $p$ and $q$ satisfy $p + q = p q$, or alternatively $1/p + 1/q = 1$, we say they are \emph{conjugate exponents}.
\end{definition}

\begin{remark}
  Note that $1/p + 1/q = 1$ implies $1 < p,q < \infty$.
\end{remark}

\begin{theorem}
  Let $p$ and $q$ be conjugate exponents, $X$ a measure space with measure $\mu$, and $f,g \cn X \rarr [0,\infty]$ measurable functions. Then
  \[
  \tag{H\"older's inequality}
  \int_X f g d\mu \leq \left( \int_X f d\mu \right)^{1/p} \left( \int_X g d\mu \right)^{1/q},
  \]
  and
  \[
  \tag{Minkowski's inequality}
  \left( \int_X (f+g)^p d\mu \right)^{1/p} \leq \left( \int_X f^p d\mu \right)^{1/p} + \left( \int_X g^p d\mu \right)^{1/p}.
  \]
\end{theorem}

\begin{remark}
  When $p = q = 2$, H\"older's inequality becomes the well-known (Cauchy-)Schwartz inequality.
\end{remark}

\begin{remark}
  Assuming $\int_X f^p d\mu < \infty$ and $\int_X g^q d\mu < \infty$, equality holds in H\"older's inequality if and only if there exists constants $\alpha$ and $\beta$, not both $0$, such that $\alpha f^p = \beta g^q$ a.e. An analogous statement could be made regarding Minkowski's inequality.
\end{remark}

\begin{definition}
  Let $X$ be a measure space with a positive measure $\mu$. For any $0 < p < \infty$ and $f \cn X \rarr \C$ measurable, define its $L^p$-norm
  \[
  \|f\|_p = \left( \int_X |f|^p d\mu \right)^{1/p}.
  \]
  Let $L^p(\mu)$ consist of all such functions with finite $L^p$-norm.

  As before, we write $L^p(\R^k) = L^p(\mu)$ if $\mu$ is the Lebesgue measure on $\R^k$, and similarly for measurable subsets of $\R^k$.

  If $\mu$ is the counting measure on a set $A$, then we denote $\ell^p(A) = L^p(A)$. If $A$ is countably infinite, we may even write $\ell^p$. In this case, we may imagine the elements of $\ell^p$ as complex sequences $x = \{x_n\}$ satisfying
  \[
  \|x\|_p = \left( \sum_{n=1}^\infty |x_n|^p \right)^{1/p} < \infty.
  \]
\end{definition}

\begin{definition}
  Suppose $f \cn X \rarr [0,\infty]$ is measurable. Let $S$ be the set of all $\alpha \in \R$ satisfying
  \[
  \mu(f^{-1}((\alpha,\infty])) = 0.
  \]
  If $S = \emptyset$, then put $\beta = 0$. If $S \neq \emptyset$, then put $\beta = \inf S$. Since
  \[
  g^{-1}((\beta,\infty]) = \bigcup_{n=1}^n g^{-1}\left(\left(\beta+\frac{1}{n}, \infty \right]\right),
  \]
  and since the union of a countable collection of measure $0$ sets has measure $0$, it follows that $\beta \in S$. We call $\beta$ the \emph{essential supremum} of $f$.

  For any measurable $f \cn X \rarr \C$, we define its $L^\infty$-norm $\|f\|_\infty$ to be the essential supremum of $|f|$. Let $L^\infty(\mu)$ consist of all such functions with finite $L^\infty$-norm. We call these functions \emph{essentially bounded on $X$}.
\end{definition}

\begin{remark}
  It follows from the definition that $|f| \leq \lambda$ a.e.\ if and only if $\|f\|_\infty \leq \lambda$.
\end{remark}

\begin{definition}
  As before, we will write $L^\infty(\R^k) = L^\infty(\mu)$ if $\mu$ is the Lebesgue measure on $\R^k$. Similarly, we write $\ell^\infty(A) = L^\infty(\mu)$ if $\mu$ is the counting measure on a set $A$. We may write $\ell^\infty$ alone if $A$ is countable.
\end{definition}

\begin{theorem}
  Let $p$ and $q$ be conjugate exponents. If $f \in L^p(\mu)$, $g \in L^q(\mu)$, then
  \[
  \|f g\|_1 \leq \|f\|_p \|g\|_q,
  \]
  hence $f g \in L^1(\mu)$.
\end{theorem}

\begin{theorem}
  Let $1 \leq p \leq \infty$. If $f, g \in L^p(\mu)$, then
  \[
  \|f+g\|_p \leq \|f\|_p + \|g\|_p,
  \]
  hence $f+g \in L^p(\mu)$.
\end{theorem}

\begin{remark}
  If $f \in L^p(\mu)$, and $\alpha \in \C$, then $\|\alpha f\|_p = |\alpha| \, \|f\|_p$. This remark combined with the previous result implies  that $L^p(\mu)$ is a vector space over $\C$.
\end{remark}

\begin{remark}
  Let $d \cn L^p(\mu) \x L^p(\mu) \rarr [0,\infty)$ be given by $d(f,g) = \|f-g\|_p$. It is not hard to check that $d(f,g) = 0$ if and only if $f \sim g$, that is, $f = g$ a.e. In other words, $d$ descends to a metric on $L^p(\mu)/\!\!\sim$. It can also be checked that $\sim$ respects linearity, hence $L^p(\mu)/\!\!\sim$ is a vector space over $\C$.
\end{remark}

\begin{remark}
  To avoid cumbersome details, we may sometimes treat $L^p(\mu)$ as a metric space and actually refer to $L^p(\mu)/\!\!\sim$. In this case the elements of $L^p(\mu)$ are not functions, but equivalence classes of functions.
\end{remark}

\begin{remark}
  It is elementary though essential to define \emph{convergence} and being \emph{Cauchy} in $L^p(\mu)$.
\end{remark}

\begin{theorem}
  For every $1 \leq p \leq \infty$, and every positive measure $\mu$, the metric space $L^p(\mu)$ is complete.
\end{theorem}

\begin{theorem}
  If $1 \leq p \leq \infty$ and $\{f_n\}$ is a Cauchy sequence in $L^p(\mu)$ with limit $f$, then $\{f_n\}$ has a subsequence which converges pointwise a.e.\ to $f$.
\end{theorem}

\begin{theorem}
  Let $S$ be the class of all complex, measurable, simple functions on $X$ such that $\mu(s^{-1}(\C \setminus \{0\})) < \infty$. If $1 \leq p < \infty$, then $S$ is dense in $L^p(\mu)$.
\end{theorem}

\begin{theorem}
  For $1 \leq p < \infty$, $\Cc_c(X)$ is dense in $L^p(\mu)$.
\end{theorem}

\begin{remark}
  Fix $k \geq 1$, and consider the spaces $\Cc_c(\R^k) \subset L^p(\R^k)$. For every $1 \leq p \leq \infty$, the $L^p$-norm induces a genuine metric on $\Cc_c(\R^k)$ (unlike in $L^p(\R^k)$, we do not need equivalence classes). If $1 \leq p < \infty$, then $\Cc_c(\R^k)$ is dense in $L_p(\R^k)$ and the latter is complete, hence we may treat $L^p(\R^k)$ as the completion of $\Cc_c(\R^k)$ with respect to the metric induced by the $L^p$-norm. In the case $p = \infty$, the  completion of $\Cc_c(\R^k)$ is not $L^p(\R^k)$ but $\Cc_0(\R^k)$, the space of all continuous functions on $\R^k$ which ``vanish at infinity'' (see below). It is also interesting to note that for $f \in \Cc^c(\R^k)$, we have
  \[
  \|f\|_\infty = \sup_{x \in \R^k} |f(x)|.
  \]
\end{remark}

\begin{definition}
  Let $X$ be a locally compact Hausdorff space. A continuous function $f \cn X \rarr \C$ is said to \emph{vanish at infinity} if for every $\epsilon > 0$ there exists a compact $K \subset X$ such that $|f|_{X \setminus K}| < \epsilon$. The set of all functions is denoted $\Cc_0(X)$.
\end{definition}

\begin{remark}
  It is clear that $\Cc_c(X) \subset \Cc_0(X)$, and equality is attained if $X$ is compact. In that case, we may write $\Cc(X)$ for either of them.
\end{remark}

\begin{theorem}
  Let $X$ be a locally compact Hausdorff space. The completion of $\Cc_c(X)$ with respect to the supremum norm
  \[
  \|f\| = \sup_{x \in X} |f(x)|
  \]
  is $\Cc_0(X)$.
\end{theorem}

\subsection{Rudin, Chapter 4: Elementary Hilbert Space Theory}

\begin{definition}
  An \emph{inner product} for a complex vector space $H$ is a map $(-,-) \cn H \x H \rarr \C$ which satisfies the following properties:
  \begin{enumerate}[(a)]
  \item $(y,x) = \overline{(x,y)}$ for all $x,y \in H$;
  \item $(x+y,z) = (x,z) + (y,z)$ for all $x,y,z \in H$;
  \item $(\alpha x,y) = \alpha(x,y)$ for all $\alpha \in \C$, $x,y \in H$;
  \item $(x,x) \geq 0$ for all $x \in H$;
  \item $(x,x) = 0$ only if $x = 0$.
  \end{enumerate}
  A complex vector space endowed with an inner product is called an \emph{inner product space}.
\end{definition}

\begin{definition}
  The \emph{norm} of $x \in H$ is $\|x\| = \sqrt{(x,x)}$, where $\sqrt{-}$ denotes the non-negative square root.
\end{definition}

The following are immediate consequences of the given properties:
\begin{enumerate}[(a)]
\item $(0,x) = 0$ for all $x \in H$;
\item for every $y \in H$, the map $H \rarr \C$ given by $x \mapsto (x,y)$ is linear;
\item $(x,\alpha y) = \overline{\alpha}(x,y)$ for all $\alpha \in \C$, $x,y \in H$;
\item $\|x\|^2 = (x,x)$ for all $x \in H$.
\end{enumerate}

\begin{proposition}[(Cauchy-)Schwartz Inequality]
  For all $x,y \in H$, we have
  \[
  |(x,y)| \leq \|x\| \, \|y\|.
  \]
\end{proposition}

\begin{proposition}[Triangle Inequality]
  For all $x,y \in H$, we have
  \[
  \|x+y\| \leq \|x\| + \|y\|.
  \]
\end{proposition}

\begin{remark}
  Any inner product space $H$ possesses a natural metric structure in which the distance between $x,y \in H$ is given by $\|x-y\|$. This can be verified using the properties and inequalities listed above.
\end{remark}

\begin{definition}
  A \emph{Hilbert space} is an inner product space which is complete in its natural metric structure.
\end{definition}

Let $H$ denote a Hilbert space for the remainder of this chapter.

\begin{example}
  \mbox{}
  \begin{enumerate}[(a)]
  \item The vector space $\C^n$ endowed with $(x,y) = \sum_{i=1}^n x_i \overline{y_i}$ is a Hilbert space.
  \item If $\mu$ is any positive measure on a set $X$, then $L^2(\mu)$ is a Hilbert space with inner product
    \[
    (f,g) = \int_X f \overline{g} d\mu.
    \]
    Note that in this case the induced norm agrees with the $L^2$-norm.
  \item The vector space of all continuous functions $[0,1] \rarr \C$ is an inner product space when endowed with
    \[
    (f,g) = \int_0^1 f(t) \overline{g(t)} d t.
    \]
    This however is not a Hilbert space since completeness fails to hold.
  \end{enumerate}
\end{example}

\begin{theorem}
  For any fixed $y \in H$, the maps $H \rarr \C$, $H \rarr \C$, and $H \rarr \R$ respectively given by
  \[
  x \mapsto (x,y), \qquad
  x \mapsto (y,x), \qquad
  x \mapsto \|x\|
  \]
  are all continuous.
\end{theorem}

\begin{definition}
  A \emph{closed subspace} of a Hilbert space is a subspace closed in the natural metric structure.
\end{definition}

\begin{remark}
  If $M \subset H$ is a subspace, then so is its closure $\overline{M}$.
\end{remark}

\begin{definition}
  Let $V$ be a vector space. A subset $E \subset V$ is called \emph{convex} if the straight segment joining any two points of $E$ is also contained in $E$. Formally this means that if $x,y \in E$ and $0 < t < 1$, then $(1-t) x + t y \in E$.
\end{definition}

\begin{remark}
  The following properties follow immediately from the definition of convexity:
  \begin{enumerate}[(a)]
  \item every subspace of a vector space is convex;
  \item any translate of a convex set is also convex.
  \end{enumerate}
\end{remark}

\begin{definition}
  Let $V$ be an inner product space. We call two vectors $x,y \in V$ \emph{orthogonal}, and write $x \perp y$, if $(x,y) = 0$. For $x \in V$, let $x^\perp$ denote the set all vectors in $V$ perpendicular to it. Similarly, for any subspace $M \subset V$, let $M^\perp$ denote the set of all vectors in $V$ perpendicular to everything in $M$.
\end{definition}

\begin{remark}
  \mbox{}
  \begin{enumerate}[(a)]
  \item The relation $\perp$ on $V$ is symmetric.
  \item Both $x^\perp$ and $M^\perp$ are subspaces.
  \item For any $x \in H$, the space $x^\perp \subset H$ is closed. Similarly, so is $M^\perp = \bigcap_{x \in M} x^\perp$.
  \end{enumerate}
\end{remark}

\begin{theorem}
  Every nonempty, closed, convex set in a Hilbert space contains a unique element of smallest norm.
\end{theorem}

\begin{theorem}
  Let $M$ be a closed subspace of a Hilbert space $H$.
  \begin{enumerate}[(a)]
  \item Every $x \in H$ has a unique decomposition $x = P(x) + Q(x)$, where $P(x) \in M$ and $Q(x) \in M^\perp$.
  \item $P(x)$ and $Q(x)$ are the nearest points to $x$ in $M$ and $M^\perp$ respectively.
  \item The maps $P \cn H \rarr M$ and $Q \cn H \rarr M^\perp$ are both linear.
  \item Every $x \in H$ satisfies $\|x\|^2 = \|P(x)\|^2 + \|Q(x)\|^2$.
  \end{enumerate}
  The maps $P$ and $Q$ are called the \emph{orthogonal projections} of $H$ onto $M$ and $M^\perp$.
\end{theorem}

\begin{corollary}
  If $M \neq H$, then there exists $y \in H \setminus \{0\}$ such that $y \perp M$.
\end{corollary}

\begin{theorem}
  If $L \cn H \rarr \C$ is continuous and linear, then there exists a unique $y \in H$ such that $L(x) = (x,y)$ for all $x \in H$.
\end{theorem}

\begin{definition}
  Let $V$ be a vector space. For any subset $S \subset V$, let $\langle S \rangle$ denote the set of finite linear combinations of elements of $S$, also called its \emph{span}.
\end{definition}

\begin{definition}
  A collection of vectors $\{ u_\alpha \;|\; \alpha \in A \}$ in a Hilbert space $H$ is called \emph{orthonormal} if $(u_\alpha,u_\beta) = \delta_{\alpha\beta}$ for all $\alpha,\beta \in A$. To each $x \in H$, we associate a function $\what{x} \cn H \rarr \C$ given by $\alpha \mapsto (x,u_\alpha)$. The numbers $\what{x}(\alpha)$ are called the \emph{Fourier coefficients} of $x$ relative to $\{u_\alpha\}$.
\end{definition}

\begin{theorem}
  Suppose $\{ u_\alpha \;|\; \alpha \in A \}$ is an orthonormal set in a Hilbert space $H$ and $F \subset A$ is a finite subset. Let $M_F$ be the span of $\{ u_\alpha \;|\; \alpha \in F \}$.
  \begin{enumerate}[(a)]
  \item For any $\phi \cn A \rarr \C$ satisfying $\phi|_{A \setminus F} = 0$, the vector
    \[
    y = \sum_{\alpha \in F} \phi(\alpha) u_\alpha
    \]
    satisfies $\what{y}(\alpha) = \phi(\alpha)$ for all $\alpha \in A$, and
    \[
    \|y\|^2 = \sum_{\alpha \in F} |\phi(\alpha)|^2.
    \]
  \item Define $s_F \cn H \rarr H$ by
    \[
    s_F(x) = \sum_{\alpha \in F} \what{x}(\alpha) u_\alpha.
    \]
    Then for all $x \in H$, we have
    \[
    \| x - s_F(x) \| < \| x - s \|
    \]
    for all $s \ M_F$, except for $s = s_F(x)$, and
    \[
    \sum_{\alpha \in F} |\what{x}(\alpha)|^2 \leq \|x\|^2.
    \]
  \end{enumerate}
\end{theorem}


Before delving into the next result, let us clarify the meaning of $\sum_{\alpha \in A} \phi(\alpha)$ for infinite $A$. For this, we assume $0 \leq \phi \leq \infty$.  Then $\sum_{\alpha \in A} \phi(\alpha)$ will denote the supremum of finite sums $\phi(\alpha_1) + \cdots + \phi(\alpha_n)$, where $\alpha_1, \dots, \alpha_n \in A$ are distinct. Note that this agrees with $\int_A \phi d\mu$ if $\mu$ is the counting measure on $A$. We can then refer to $\ell^2(A)$ which is a Hilbert space with inner product
\[
(\phi,\psi) = \sum_{\alpha \in A} \phi(\alpha) \overline{\psi(\alpha)}.
\]
We know that the functions $A \rarr \C$ which vanish at all but finitely many points form a dense subset of $\ell^2(A)$. Furthermore, for all $\phi \in \ell^2(A)$, the set $\phi^{-1}(\C \setminus \{0\})$ is at most countable.

\begin{lemma}
  Assume the following:
  \begin{enumerate}[(a)]
  \item $X$ and $Y$ are metric spaces, and $X$ is complete;
  \item $f \cn X \rarr Y$ is continuous;
  \item $X$ has a dense subset $X_0$ on which $f$ is an isometry;
  \item $f(X_0)$ is dense in $Y$.
  \end{enumerate}
  Then $f \cn X \rarr Y$ is an isometry, and in particular, a bijection.
\end{lemma}

\begin{theorem}[Riesz-Fischer Theorem]
  Let $\{ u_\alpha \;|\; \alpha \in A \}$ be an orthonormal set in the Hilbert space $H$, and $P$ the span of $\{u_\alpha\}$. The inequality
  \[
  \sum_{\alpha \in A} |\what{x}(\alpha)|^2 \leq \|x\|^2
  \]
  holds for all $x \in H$, and $x \mapsto \what{x}$ is a continuous map $H \rarr \ell^2(A)$ whose restriction to $\overline{P}$ is an isometry.
\end{theorem}

\begin{theorem}
  Let $\{ u_\alpha \;|\; \alpha \in A \}$ be an orthonormal set in the Hilbert space $H$, and $P$ the span of $\{u_\alpha\}$. The following are equivalent:
  \begin{enumerate}[(a)]
  \item $\{u_\alpha\}$ is a maximal orthonormal set in $H$;
  \item the span of $\{u_\alpha\}$ is dense in $H$;
  \item the equality
    \[
    \sum_{\alpha \in A} |\what{x}(\alpha)|^2 = \|x\|^2
    \]
    holds for all $x \in H$;
  \item the equality
    \[
    \tag{Parseval's identity}
    \sum_{\alpha \in A} \what{x}(\alpha) \overline{\what{y}(\alpha)} = (x,y)
    \]
    holds for all $x,y \in H$.
  \end{enumerate}
\end{theorem}

\begin{definition}
  Maximal orthonormal sets are called \emph{complete orthonormal set} or \emph{orthonormal base}.
\end{definition}

\begin{remark}
  Note that an orthonormal base in a Hilbert space need not be a basis for the underlying vector space. Actually, it will only be one in the finite-dimensional case.
\end{remark}

\begin{corollary}
  If $\{ u_\alpha \;|\; \alpha \in A \}$ is a maximal orthogonal set in a Hilbert space $H$, then the map $H \rarr \ell^2(A)$ given by $x \mapsto \what{x}$ is an isomorphism of Hilbert spaces.
\end{corollary}

\begin{theorem}[Hausdorff Maximality]
  Every nonempty partially ordered set contains a maximal totally ordered subset.  
\end{theorem}

\begin{theorem}
  Every orthonormal set $B$ in a Hilbert space $H$ is contained in a maximal orthonormal set in $H$.
\end{theorem}

\begin{corollary}
  Every Hilbert space admits a maximal orthonormal set.
\end{corollary}

In what follows, we will identify maps $S^1 \rarr X$ with periodic maps $\R \rarr X$ with period $2\pi$. More concretely, we shall sometimes write $f(t)$ rather than $f(e^{i t})$, even if we think of $f$ as defined on $S^1$.

Consider the class of all Lebesgue measurable, $2\pi$-periodic functions on $\R$ for which the norm
\[
\|f\|_p = \left( \frac{1}{2\pi} \int_{-\pi}^\pi |f(t)|^p d t \right)^{1/p}
\]
is finite. Under the identification above, the set of these functions is $L^p(S^1)$. Note the normalizing factor $1/2\pi$ inserted for the sake of convenience. Similarly for $L^\infty(S^1)$ and $\Cc(S^1)$.

\begin{definition}
  A \emph{trigonometric polynomial} is a finite sum of the form
  \[
  f(t) = a_0 + \sum_{n=1}^N (a_n \cos n t + b_n \sin n t)
  \]
  for defined for $t \in \R$ where $a_i, b_j \in \C$. These coincide with finite sums of the form
  \[
  f(t)= \sum_{n=-N}^N c_n e^{i n t}
  \]
  where $c_n \in \C$ which are often more convenient.
\end{definition}

It is clear that all trigonometric polynomials have period $2\pi$. For $n \in \Z$, we set $u_n(t) = e^{i n t}$. If we define the inner product in $L^2(S^1)$ as $(f,g) = 1/2\pi \int_{-\pi}^\pi f(t)\overline{g(t)} d t$, then $(u_m,u_n) = \delta_{m n}$. The orthonormal set $\{ u_n \;|\; n \in \Z \}$ in $L^2(S^1)$ is called the \emph{trigonometric system}.

\begin{theorem}
  For any $f \in \Cc(S^1)$ and $\epsilon > 0$, there exists a trigonometric polynomial $P$ such that $|f-P| < \epsilon$.
\end{theorem}

Combining the above with the fact $\Cc(S^1)$ is dense in $L^2(S^1)$, we obtain the following.

\begin{corollary}
  The trigonometric system is a maximal orthonormal set in $L^2(S^1)$.
\end{corollary}

\begin{definition}
  For any $f \in L^1(S^1)$, we define the \emph{Fourier coefficients of $f$} by the formula
  \[
  \what{f}(n) = (f, u_n) = \frac{1}{2\pi} \int_{-\pi}^{\pi} f(t) e^{-i n t} d t.
  \]
  The \emph{Fourier series} of $f$ is $\sum_{n=-\infty}^{\infty} \what{f}(n) e^{i n t}$, and its partial sums are $s_N(t) = \sum_{n=-N}^N \what{f}(n) e^{i n t}$.
\end{definition}

The following are consequences of previous results for $L^p$ spaces.

\begin{corollary}
  \mbox{}
  \begin{enumerate}[(a)]
  \item For any $f,g \in L^2(S^1)$, we have
    \[
    \sum_{n = -\infty}^\infty \what{f}(n) \overline{\what{g}(n)} = \frac{1}{2\pi} \int_{-\pi}^{\pi} f(t)\overline{g(t)} d t,
    \]
    and the series on the left converges absolutely.
  \item For any $f \in L^2(S^1)$, we have
    \[
    \lim_{N \rarr \infty} \|f-s_N\|_2 = 0,
    \]
    since a special case of (a) implies
\[
\|f-s_N\|^2_2 = \sum_{|n| > N} \left| \what{f}(n) \right|^2.
\]
  \end{enumerate}
\end{corollary}

\subsection{Rudin, Chapter 5: Examples of Banach Space Techniques}

\begin{definition}
  A vector space $X$ is called \emph{normed} if it is endowed with a \emph{norm}, that is, a map $\|-\| \cn X \rarr [0,\infty)$ satisfying the following conditions:
  \begin{enumerate}[(a)]
  \item $\|x+y\| \leq \|x\| + \|y\|$ for all $x,y \in X$;
  \item $\|\alpha x\| = |\alpha| \, \|x\|$ for all $\alpha \in \C$, $x \in X$;
  \item $\|x\| = 0$ implies $x = 0$ for all $x \in X$.
  \end{enumerate}
\end{definition}

\begin{remark}
  Any normed vector space $X$ possesses a natural metric structure in which the distance between $x,y \in X$ is given by $\|x-y\|$. This can be verified using the properties listed above.
\end{remark}

\begin{definition}
  A \emph{Banach space} is a normed vector space which is complete with respect to the metric induced by its norm.
\end{definition}

\begin{remark}
  Every Hilbert space is a Banach space in a natural way.
\end{remark}

\begin{definition}
  Let $X$ and $Y$ be normed vector spaces. The \emph{norm} of a linear map $\Lambda \cn X \rarr Y$ is given by
  \[
  \|\Lambda\| = \sup\{ \|\Lambda(x)\| \;|\; x \in X, \|x\| \leq 1 \}.
  \]
Geometrically, $\Lambda$ maps the closed unit ball in $X$ to the closed ball of radius $\|\Lambda\|$ around $0$ in $Y$.
\end{definition}

\begin{definition}
  If $\|\Lambda\| < \infty$, then $\Lambda$ is called \emph{bounded}.
\end{definition}

\begin{theorem}
  Let $X$ and $Y$ be normed vector spaces. For a linear transformation $\Lambda \cn X \rarr Y$, the following are equivalent:
  \begin{enumerate}[(a)]
  \item $\Lambda$ is bounded;
  \item $\Lambda$ is continuous;
  \item $\Lambda$ is continuous at $0 \in X$.
  \end{enumerate}
\end{theorem}

\begin{theorem}[Baire's Theorem]
  In a complete metric space $X$, the intersection of every countable collection of dense opene subsets is dense. In particular (except in the trivial case $X = \emptyset$), the intersection is not empty.
\end{theorem}

\begin{corollary}
  In a complete metric space, the intersection of any countable collection of dense countable intersections of opens is again a dense countable intersection of opens.
\end{corollary}

The following describes why Baire's Theorem is sometimes called the Category Theorem.

\begin{definition}
  Let $X$ be a metric space. We call a set $E \subset X$ \emph{nowhere dense} if $\overline{E}$ contains no nonempty opens.
\end{definition}

\begin{definition}[Baire's terminology]
  Any countable union of nowhere dense sets is called to be of the \emph{first category}. All other subsets are called to be of the \emph{second categoty}.
\end{definition}

Using this language Baire's Theorem becomes.

\begin{corollary}
  No complete metric space is of the first category.
\end{corollary}

\begin{theorem}[Banach-Steinhauss Theorem]
  Let $X$ be a Banach space, $Y$ a normed vector space, and $\{ \Lambda_\alpha \cn X \rarr Y \;|\; \alpha \in A \}$ a collection of bounded linear transformations. Then there exists an $M < \infty$ such that
  \[
  \|\Lambda_\alpha\| \leq M
  \]
  for all $\alpha \in A$, or
  \[
  \sup_{\alpha \in A} \|\Lambda_\alpha(x)\| = \infty
  \]
  for all $x$ belonging to some countable intersection of opens in $X$ which is also dense.
\end{theorem}

\begin{remark}
  The Banach-Steinhauss Theorem is sometimes referred to as the \emph{uniform boundedness principle}.
\end{remark}

\begin{theorem}[Open Mapping Theorem]
  Let $Y$ and $V$ be the open units of two Banach spaces $X$ and $Y$ respectively. To every surjective bounded linear transformation $\Lambda \cn X \rarr Y$ there corresponds a $\delta > 0$ satisfying $\Lambda(U) \supset \delta V$.
\end{theorem}

\begin{theorem}
  If $X$ and $Y$ are Banach spaces and $\Lambda \cn X \rarr Y$ is a bijective bounded linear transformation, then there is a $\delta > 0$ such that $\|\Lambda(x)\| \geq \delta\|x\|$. In other words, $\Lambda^{-1} \cn Y \rarr X$ is a bounded linear transformation too.
\end{theorem}

TODO here

\subsection{Rudin, Chapter 6: Complex Measures}

TODO

\subsection{Rudin, Chapter 7: Differentiation}

TODO

\subsection{Rudin, Chapter 8: Integration on Product Spaces}

TODO

\subsection{Rudin, Chapter 9: Fourier Transforms}

TODO

%%% Local Variables: 
%%% mode: latex
%%% TeX-master: "Study_Guide"
%%% End: 


%%% Local Variables: 
%%% mode: latex
%%% TeX-master: "Study_Guide"
%%% End: 


% Chapter 2 -- Past Exams
\chapter{Past Exams}
\label{S:past-exams}

\newcommand{\done}{$\bullet$}
\newcommand{\some}{$\circ$}
\newcommand{\todo}{$\cdot$}
\newcommand{\none}{}
\begin{center}
  \begin{tabular}{l
      c@{\hspace{2pt}}c@{\hspace{2pt}}c c@{\hspace{2pt}}c@{\hspace{2pt}}c
      c@{\hspace{2pt}}c@{\hspace{2pt}}c c@{\hspace{2pt}}c@{\hspace{2pt}}c
      c@{\hspace{2pt}}c@{\hspace{2pt}}c c@{\hspace{2pt}}c@{\hspace{2pt}}c}
    \toprule
    \multicolumn{1}{l}{}
    & \multicolumn{3}{c}{\hyperref[S:spring-2010]{S 2010}}
    & \multicolumn{3}{c}{\hyperref[S:fall-2009]  {F 2009}}
    & \multicolumn{3}{c}{\hyperref[S:spring-2009]{S 2009}}
    & \multicolumn{3}{c}{\hyperref[S:fall-2008]  {F 2008}}
    & \multicolumn{3}{c}{\hyperref[S:spring-2008]{S 2008}}
    & \multicolumn{3}{c}{\hyperref[S:fall-2007]  {F 2007}} \\
    & \hyperref[S:spring-2010-1]{D1} & \hyperref[S:spring-2010-2]{D2} & \hyperref[S:spring-2010-3]{D3}
    & \hyperref[S:fall-2009-1]  {D1} & \hyperref[S:fall-2009-2]  {D2} & \hyperref[S:fall-2009-3]  {D3}
    & \hyperref[S:spring-2009-1]{D1} & \hyperref[S:spring-2009-2]{D2} & \hyperref[S:spring-2009-3]{D3}
    & \hyperref[S:fall-2008-1]  {D1} & \hyperref[S:fall-2008-2]  {D2} & \hyperref[S:fall-2008-3]  {D3}
    & \hyperref[S:spring-2008-1]{D1} & \hyperref[S:spring-2008-2]{D2} & \hyperref[S:spring-2008-3]{D3}
    & \hyperref[S:fall-2007-1]  {D1} & \hyperref[S:fall-2007-2]  {D2} & \hyperref[S:fall-2007-3]  {D3} \\
    \midrule
    Algebra
    & \none & \done & \none & \none & \none & \none
    & \none & \none & \none & \none & \none & \none
    & \none & \none & \done & \none & \none & \none \\
    Algebraic Geometry
    & \none & \none & \none & \none & \none & \none
    & \none & \none & \none & \none & \none & \none
    & \none & \none & \none & \none & \none & \none \\
    Complex Analysis
    & \done & \done & \done & \done & \done & \done
    & \todo & \done & \todo & \done & \done & \todo
    & \done & \todo & \done & \done & \done & \todo \\
    Algebraic Topology
    & \done & \done & \done & \some & \done & \todo
    & \done & \done & \done & \todo & \done & \done
    & \done & \some & \some & \done & \done & \done \\
    Differential Geometry
    & \none & \none & \none & \none & \none & \none
    & \none & \none & \none & \none & \none & \none
    & \none & \none & \none & \none & \none & \none \\
    Real Analysis
    & \none & \none & \none & \none & \none & \none
    & \none & \none & \none & \none & \none & \none
    & \none & \none & \none & \none & \none & \none \\
    \bottomrule
  \end{tabular}
  
  \vspace\baselineskip
  Legend: \done\, done, \some\, partly done, \todo\, to be done
\end{center}
\vspace\baselineskip

\section{Spring 2010}
\label{S:spring-2010}

\subsection{Day 1}
\label{S:spring-2010-1}
\mbox{}

\problem*{Problem 1}

TODO

\problem*{Problem 2}

TODO

\problem*{Problem 3}

Let $\lambda$ be a real number greater than 1. Show that the equation $z e^{\lambda-z} = 1$ has exactly one solution with $|z| < 1$, and that this solution $z$ is real.

\begin{proof}[Solution]
  We first show that $z e^{\lambda-z} - 1$ has one solution. Consider the entire functions $f(z) = z e^{\lambda-z}$ and $g(z) = -1$, so that we are interested in the roots of $f+g$ in the unit disk $\Delta$. First of all $z$ has one root in $\Delta$, and $e^{\lambda-z}$ does not vanish, hence $f$ has exactly one root in $\Delta$. To apply Rouch\'e's Theorem, it suffices to observe that
  \[
  |f| =
  |z| |e^{\lambda-z}| =
  e^{\lambda-\Re z} >
  1 =
  |g| \leq
  |f|
  \]
  on the unit circle $\d\Delta$. This completes the first part of the problem. To conclude the one root is real, consider the restriction of $f+g$ to the line segment $[-1,1]$, namely
  \begin{align*}
    (f+g)(-1) &= -e^{\lambda+1} < 0, &
    (f+g)(1) &= e^{\lambda-1} > 0.
  \end{align*}
  Since $(f+g)|_\R$ is real-values, the Intermediate Value Theorem shows $f+g$ vanishes somewhere on along $(-1,1)$, and since there is a unique root in $\Delta$, this completes our claim.
\end{proof}

\problem*{Problem 4}

TODO

\problem*{Problem 5}

TODO

\problem*{Problem 6}

Let $X$ be a topological space. We say that two covering spaces $f \cn Y \rarr X$ and $g \cn Z \rarr X$ are isomorphic if there exists a homeomorphism $h \cn Y \rarr Z$ such that $g \circ h = f$. If $X$ is compact oriented surface of genus $g$, how many connected $2$-sheeted covering spaces does $X$ have, up to isomorphism?

\begin{proof}[Solution]
  It is a classical theorem that covering spaces of a connected (and sufficiently nice, such as a manifold) space $X$ are in a bijective correspondence with subgroups of $\pi_1(X)$ up to conjugation. To a subgroup $H \subset \pi_1(X)$ corresponds to covering space with $[\pi_1(X):H]$ sheets. Therefore we are interested in counting the index $2$ subgroups up to conjugation in $\pi_1(\Sigma_g)$ where $\Sigma_g$ is the $g$-holed torus for $g \geq 0$. Recall that all index $2$ subgroups are normal, hence we may remove ``up to conjugation'' in the previous sentence. Such subgroups induce natural quotient homomorphism $\pi_1(\Sigma_g) \rarr \Z/2$, and conversely every such surjective homomorphism induces an index 2 subgroup, namely, its kernel. But $\Z/2$ is abelian, hence a homomorphism $\phi \cn \pi_1(\Sigma_g) \rarr \Z/2$ factors through
  \[
  \pi_1(\Sigma_g)/[\pi_1(\Sigma_g),\pi_1(\Sigma_g)] \cong H_1(\Sigma_g) \cong \Z^{2g}.
  \]
  In the first of the two isomorphisms above, we applied the Hurewicz Theorem. Our problem now is reduced to counting the surjective homomorphisms $\Z^{2g} \rarr \Z/2$. Note that $\Z^{2g}$ is the free abelian group of $2g$ generators, say $e_1, \dots, e_{2g}$. Therefore, homomorphisms $\wtilde{\phi} \cn \Z^{2g} \rarr \Z/2$ are uniquely determined by $\wtilde{\phi}(e_i)$, and conversely, every such choice yields a homomorphism. The only non-surjective homomorphism corresponds to the assignment of $[0] \in \Z/2$ to every generator $e_i$. In conclusion, $\Sigma_g$ has exactly
  \[
  2^{2g}-1
  \]
  $2$-sheeted covering spaces up to isomorphism.
\end{proof}

\subsection{Day 2}
\label{S:spring-2010-2}
\mbox{}

\problem*{Problem 1}

TODO

\problem*{Problem 2}

Let $a$ be an arbitrary real number and $b$ a positive real number. Evaluate the integral
\[
\int_0^\infty \frac{\cos(a x)}{\cosh(b x)} d x.
\]

\begin{proof}[Solution]
  Start by noting that
  \[
  I =
  \int_0^\infty \frac{\cos(a x)}{\cosh(b x)} d x =
  \frac{1}{2} \int_{-\infty}^\infty \frac{\cos(a x)}{\cosh(b x)} d x =
  \frac{1}{2} \int_{-\infty}^\infty \frac{e^{i a x}}{\cosh(b x)} d x.
  \]
  Shifting the line of integration up by $\pi i/b$ we get
  \[
  \int_{-\infty}^\infty \frac{e^{i a (x + \pi i/b)}}{\cosh(b(x + \pi i/b))} d x =
  - e^{-\pi a/b} \int_{-\infty}^\infty \frac{e^{i a x}}{\cosh(b x)} d x =
  - 2 I e^{-\pi a/b}.
  \]
  In doing so, we picked only one pole at $\pi i/2 b$ with residue
  \begin{align*}
    \res_{\pi i/2 b} \frac{e^{i a x}}{\cosh(b x)}
    &=
    e^{i a \pi i/2 b} \res_{\pi i/2 b} \frac{1}{\cosh(b x)} =
    e^{-a \pi/2 b} \lim_{z \rarr \pi i/2 b} \frac{z - \frac{\pi i}{2 b}}{\cosh(b x)} \\
    &=
    e^{-a \pi/2 b} \lim_{z \rarr \pi i/2 b} \frac{1}{b \sinh(b x)} =
    \frac{1}{i b} e^{-a \pi/2 b}.
  \end{align*}
  We used two statements worth mentioning: (1) if $f$ has a simple pole at $z_0$ and $g$ is holomorphic at $z_0$, then $\res_{z_0} f g = g(z_0) \res_{z_0} f$; (2) a complex version of L'H\^opital's rule. Putting the collected information together, we obtain
  \[
  2 I - \left( - 2 I e^{a \pi/ b} \right) = 2 \pi i \frac{1}{i b} e^{-a \pi/ 2 b}.
  \]
  Rearranging, we conclude
  \[
  I = \frac{\pi e^{- a \pi/2 b}}{b(1 + e^{- a \pi/b})}. \qedhere
  \]
\end{proof}

\problem*{Problem 3}

Let $\Lambda_1$ and $\Lambda_2 \subset \R^4$ be complementary $2$-planes, and let $X = \R^4 \setminus (\Lambda_1 \cup \Lambda_2)$ be the complement of their union. Find the homology and cohomology groups of $X$ with integer coefficients.
j
\begin{proof}[Solution]
  Since $\Lambda_1$ and $\Lambda_2$ are complimentary, there exists an isomorphism $T \cn \R^4 \rarr \R^4$ such that $R(\Lambda_1) = \R^2 \x \{0\}^2$ and $T(\Lambda_2) = \{0\}^2 \x \R^2$. Such $\R$-linear isomorphisms are homeomorphisms, hence we may assume that $\Lambda_i$ by its image under $T$. Then
  \[
  X = \R^4 \setminus (\Lambda_1 \cup \Lambda_2) = (\R^2 \setminus \{0\})^2.
  \]
  The space $\R^2 \setminus \{0\}$ deformation retracts to $S^1$, so
  \[
  H_\bullet(\R^2 \setminus \{0\}) = \Z_{(0)} \oplus \Z_{(1)}.
  \]
  The homology in all degrees is finitely generated and free, hence the K\"unneth formula implies
  \[
  H_\bullet(X) \cong
  H_\bullet(\R^2 \setminus \{0\})^{\otimes 2} \cong
  \Z_{(0)} \oplus \Z_{(1)}^2 \oplus \Z_{(2)}.
  \]
  All groups are free, hence any $\Ext(-,\Z)$ group involving them would vanish. The Universal Coefficients Theorem enables us to compute the cohomology
  \[
  H_\bullet(X) \cong \Z_{(0)} \oplus \Z_{(1)}^2 \oplus \Z_{(2)}. \qedhere
  \]
\end{proof}

\problem*{Problem 4}

TODO

\problem*{Problem 5}

TODO

\problem*{Problem 6}

Let $p$ be a prime, and let $G$ be the group $\Z/p^2\Z \oplus \Z/p^2\Z$.
\begin{enumerate}[(a)]
\item How many subgroups of order $p$ does $G$ have?
\item How many subgroups of order $p^2$ does $G$ have? How many of these are cyclic?
\end{enumerate}

\begin{proof}[Solution]
  \problem*{(a)}
  Every subgroup of order $p$ is generated by an element of order $p$, and each subgroup possesses $\phi(p) = p-1$ such generators. Therefore, it suffices to count the number of elements of order $p$ and divide by $p-1$. Let $|-|$ denote the order of an element. The order of an element $(a,b) \in G$ is given by $|(a,b)| = \lcm(|a|,|b|)$. If $|(a,b)| = p$, then the possibilities are $(|a|,|b|) = (1,p),(p,1),(p,p)$. Since $\Z/p^2\Z$ has exactly $p$ elements of order dividing $p$, and exactly one of oder $1$, it follows it has $p-1$ elements of order $p$. Putting all information together, the number of subgroups of order $p$ is
  \[
  \frac{1 \cdot (p-1) + (p-1) \cdot 1 + (p-1) \cdot (p-1)}{p-1} =
  (1 + 1 + p-1) = p+1.
  \]
  
  \problem*{(b)}
  Let us start by counting the cyclic subgroups of order $p^2$. Each such corresponds to $\phi(p^2) = p(p-1)$ generators. The group $\Z/p^2\Z$ has $\phi(p) = p(p-1)$ generators, hence $p(p-1)$ elements of order $p^2$. If $|(a,b)| = p^2$, then $(|a|,|b|) = (1,p^2), (p^2,1), (p,p^2), (p^2,p), (p^2,p^2)$. The number of cyclic subgroups of order $p^2$ is
  \[
  \frac{ 1 \cdot p(p-1) + p(p-1) \cdot 1 + (p-1) \cdot p(p-1) + p(p-1) \cdot (p-1) + p(p-1) \cdot p(p-1)}{p(p-1)} =
  p(p+1).
  \]
  Next, let us consider the non-cyclic subgroups of order $p^2$. It is easy to see each of these is isomorphic to $\Z/p\Z \oplus \Z/p\Z$. There are exactly $p^2$ elements of order dividing $p$ in $G$, and also $p^2$ such elements in $\Z/p\Z \oplus \Z/p\Z$, hence there is exactly one non-cyclic subgroup of order $p^2$. In total, there are
  \[
  p(p+1) + 1 = p^2 + p + 1
  \]
  subgroups of order $p^2$ in $G$.
\end{proof}

\subsection{Day 3}
\label{S:spring-2010-3}
\mbox{}

\problem*{Problem 1}

TODO

\problem*{Problem 2}

Let $f$ be holomorphic on a domain containing the closed disk $\{ z \;|\; |z| \leq 3 \}$, and suppose that
\[
f(1) = f(i) = f(-1) = f(-i) = 0.
\]
Show that
\[
|f(0)| \leq \frac{1}{80} \max_{|z|=3}|f(z)|.
\]
and find all such functions for which equality holds.

\begin{proof}[Solution]
  Consider the holomorphic functions
  \[
  h(z) = (z-1)(z+1)(z-i)(z+i) = z^4-1
  \]
  and
  \[
  g(z) =
  \begin{cases}
    f(z)/h(z) & \textrm{if } z \neq \pm 1, \pm i, \\
    f'(1)/4 & \textrm{if } z = 1, \\
    i f'(i)/4 & \textrm{if } z = i, \\
    -f'(-1)/4 & \textrm{if } z = -1, \\
    -i f'(-i)/4 & \textrm{if } z = -i
  \end{cases}
  \]
  defined on the same domain as $f$. Let us compute the minimum of $h$ on the circle $C = \{ z \;|\; |z| = 3 \}$. We have
  \[
  |h(3 e^{i\theta})|^2 =
  \left(3^4 e^{4 i \theta} - 1\right)\left(3^4 e^{-4 i \theta} - 1\right) =
  (3^8-1) - 2 \cdot 3^4 \cos 4\theta,
  \]
  hence the required minimum of $|h|$ is
  \[
  \sqrt{3^8 - 1 - 2 \cdot 3^4} = 3^4 - 1 = 80.
  \]
  Setting $c = \max_{|z|=3}|f(z)|$, it follows that
  \[
  |g| \leq c/80
  \]
  on $C$, hence on $|z| \leq 3$ by the maximum modulus principle. But then $|f| \leq c|z^4-1|/80$, so
  \[
  |f(0)| \leq \frac{c |0^4 - 1|}{80} = \frac{c}{80}.
  \]
  If equality is attained, then
  \[
  |g(0)| = \frac{|f(0)|}{|0^4-1|} = |f(0)| = \frac{c}{80}.
  \]
  Combined with the fact $|g| \leq c/80$ on $C$ and the maximum modulus principle, we conclude $g = c c'/80$ is constant, where $c'$ is a constant of absolute value $1$. It follows that $f = c c' h/80$.
\end{proof}

\problem*{Problem 3}

TODO

\problem*{Problem 4}

TODO

\problem*{Problem 5}

Let $X = \R\P^2 \x \R\P^4$.
\begin{enumerate}[(a)]
\item Find the homology groups $H_\bullet(X, \Z/2)$.
\item Find the homology groups $H_\bullet(X, \Z)$.
\item Find the cohomology groups $H^\bullet(X,\Z)$.
\end{enumerate}

\begin{proof}[Solution]
  \problem*{(a)}
  Recall that
  \[
  H_\bullet(\R\P^n;\Z/2) \cong \bigoplus_{0 \leq i \leq n} \Z/2_{(i)}.
  \]
  Note that $\Z/2$ is a field, and apply the K\"unneth formula to compute
  \begin{align*}
    H_\bullet(X;\Z/2)
    &\cong
    H_\bullet(\R\P^2;\Z/2) \otimes H_\bullet(\R\P^4;\Z/2) \\
    &\cong
    \left( \Z/2_{(0)} \oplus \Z/2_{(1)} \oplus \Z/2_{(2)} \right) \otimes \left( \Z/2_{(0)} \oplus \Z/2_{(1)} \oplus \Z/2_{(2)} \oplus \Z/2_{(3)} \oplus \Z/2_{(4)} \right) \\
    &\cong
    \Z/2_{(0)} \oplus (\Z/2)_{(1)}^2 \oplus (\Z/2)_{(2)}^3 \oplus (\Z/2)_{(3)}^3 \oplus (\Z/2)_{(4)}^3 \oplus (\Z/2)_{(5)}^2 \oplus \Z/2_{(6)}.
  \end{align*}
  
  \problem*{(b)}
  We have
  \begin{align*}
    H_\bullet(\R\P^2) &\cong \Z_{(0)} \oplus \Z/2_{(1)}, &
    H_\bullet(\R\P^4) &\cong \Z_{(0)} \oplus \Z/2_{(1)} \oplus \Z/2_{(3)}.
  \end{align*}
  The general K\"unneth formula tells us we have split short exact sequences
  \[\xymatrix@C=0.2in{
    0 \ar[r] & \bigoplus_i H_i(\R\P^2) \otimes H_{n-i}(\R\P^4) \ar[r] & H_n(X) \ar[r] & \bigoplus_i \Tor(H_i(\R\P^2),H_{n-i-1}(\R\P^4)) \ar[r] & 0.
  }\]
  Using this, we compute
  \begin{align*}
    H_0(X) &\cong \Z, &
    H_1(X) &\cong (\Z/2)^2, &
    H_2(X) &\cong \Z/2, &
    H_3(X) &\cong (\Z/2)^2, \\
    H_4(X) &\cong \Z/2, &
    H_5(X) &\cong \Z/2, &
    H_6(X) &\cong 0.
  \end{align*}
  
  \problem*{(c)}
  The universal coefficients for cohomology implies we have split short exact sequences
  \[\xymatrix{
    0 \ar[r] & \Ext(H_{n-1}(X),\Z) \ar[r] & H^n(X) \ar[r] & \Hom(H_n(X),\Z) \ar[r] & 0
  }\]
  for all $n$. We use these to compute
  \begin{align*}
    H_0(X) &\cong \Z, &
    H_1(X) &\cong 0, &
    H_2(X) &\cong (\Z/2)^2, &
    H_3(X) &\cong \Z/2, \\
    H_4(X) &\cong (\Z/2)^2, &
    H_5(X) &\cong \Z/2, &
    H_6(X) &\cong \Z/2. && \qedhere
  \end{align*}
\end{proof}

\problem*{Problem 6}

TODO

\section{Fall 2009}
\label{S:fall-2009}

\subsection{Day 1}
\label{S:fall-2009-1}
\mbox{}

\problem*{Problem 1}

TODO

\problem*{Problem 2}

Let $\C\P^n$ be complex projective $n$-space.
\begin{enumerate}[(a)]
\item Describe the cohomology ring $H^\bullet(\C\P^n, \Z)$ and, using the K\"unneth formula, the cohomology ring $H^\bullet(\C\P^n \x \C\P^n, \Z)$.
\item Let $\Delta \subset \C\P^n \x \C\P^n$ be the diagonal, and $\delta = i_\ast[\Delta] \in H_{2n}(\C\P^n \x \C\P^n, \Z)$ the image of the fundamental class of $\Delta$ under the inclusion $i \cn \Delta \hrarr \C\P^n \x \C\P^n$. In terms of your description of $H^\bullet(\C\P^n \x \C\P^n, \Z)$ above, find the Poincar\'e dual $\delta^\ast \in H^{2n}(\C\P^n \x \C\P^n, \Z)$ of $\delta$.
\end{enumerate}

\begin{proof}[Solution]
  \problem*{(a)}
  We know that $H^\bullet(\C\P^n) \cong \Z[\alpha]/(\alpha^{n+1})$ with $|\alpha| = 2$, so the K\"unneth formula implies
  \[
  H^\bullet(\C\P^n \x \C\P^n) \cong
  \Z[\alpha]/(\alpha^{n+1}) \otimes_\Z \Z[\beta]/(\beta^{n+1}) \cong
  \Z[\alpha,\beta]/(\alpha^{n+1},\beta^{n+1}).
  \]
  
  \problem*{(b)}
  TODO
\end{proof}

\problem*{Problem 3}

TODO

\problem*{Problem 4}

Let $\Omega \subset \C$ the the open set
\[
\Omega = \{ z \;|\; |z| < 2 \textrm{ and } |z-1| > 1 \}.
\]
Give a conformal isomorphism between $\Omega$ and the unit disc $\Delta = \{ z \;|\; |z| < 1 \}$.

\begin{proof}[Solution]
  Set $\Omega_1 = \Omega$. We proceed to give a sequence of conformal isomorphisms $f_i \cn \Omega_i \rarr \Omega_{i+1}$.
  \begin{align*}
    f_1(z) &= z-2 &
    \Omega_2 &= \{ z \;|\; |z+2| < 2 \textrm{ and } |z+1| > 1 \} \\
    f_2(z) &= \frac{1}{z} &
    \Omega_3 &= \{ z \;|\; -1/2 < \Re z < -1/4 \} \\
    f_3(z) &= -4 \pi i \left( z + \frac{1}{4} \right) &
    \Omega_4 &= \{ z \;|\; 0 < \Im z < \pi \} \\
    f_4(z) &= e^z &
    \Omega_5 &= \{ z \;|\; \Im z > 0 \} \\
    f_5(z) &= \frac{i-z}{i+z} &
    \Omega_6 &= \{ z \;|\; |z| < 1 \}
  \end{align*}
  The composition $f = f_5 \circ \cdots \circ f_1$ is given by
  \[
  f(z) = \frac{i - \displaystyle\exp\left( - \frac{\pi i (6-z)}{z-2} \right)}{i + \displaystyle\exp\left( - \frac{\pi i (6-z)}{z-2} \right)}. \qedhere
  \]
\end{proof}

\problem*{Problem 5}

TODO

\problem*{Problem 6}

TODO

\subsection{Day 2}
\label{S:fall-2009-2}
\mbox{}

\problem*{Problem 1}

Let $\Delta = \{ z \;|\; |z| < 1 \}$ be the unit disk, and $\Delta^\ast = \Delta \setminus \{0\}$ the punctured disk. A holomorphic function $f$ on $\Delta^\ast$ is said to have an \emph{essential singularity} at $0$ if $z^n f(z)$ does not extend to a holomorphic function on $\Delta$ for any $n$.

Show that if $f$ has an essential singularity at $0$, then $f$ assumes values arbitrarily close to every complex number in any neighbourhood of $0$ -- that is, for any $w \in \C$, $\epsilon > 0$, and $\delta > 0$, there exists $z \in \Delta^\ast$ with
\[
|z| < \delta
\qquad\textrm{and}\qquad
|f(z) - w| < \epsilon.
\]

\begin{proof}[Solution]
  For contradiction assume the opposite -- there exists some $w \in \C$, $\epsilon > 0$, and $\delta > 0$, such that for all $z \in B(0,\delta) \setminus \{0\}$, we have $|f(z) - w| \geq \epsilon$. Consider the function $g(z) = 1/(f(z) - w)$ which is well-defined on $B(0,\delta) \setminus \{0\}$ and bounded by $1/\epsilon$. Riemann's Theorem on removable singularities implies that $g$ extends holomorphically over $0$. By abusing notation, we call this extension $g$ again. Suppose that $g$ has a zero of order $n$ at $0$, in other words, there exists a holomorphic function $h$ such that $h(z) = g(z)/z^n$ away from $0$. But then
  \[
  f(z) = \frac{1}{g(z)} + w = \frac{h(z)}{z^n} + w
  \]
  has a pole of order $n$ at $0$, contradicting the hypothesis there is an essential singularity there. This completes our claim.
\end{proof}

\begin{remark}
  This statement is known as the Casorati-Weierstrass Theorem (see Ahlfors, page 129 or Stein \& Shakarchi, page 86).
\end{remark}

\problem*{Problem 2}

Let $X = S^1 \vee S^1$ be a figure $8$, $p \in X$ the point of attachment, and let $\alpha$ and $\beta \cn [0,1] \rarr X$ be loops with base point $p$ (that is, such that $\alpha(0) = \alpha(1) = \beta(0) = \beta(1) = p$) tracing out the two halves of $X$. Let $Y$ be the CW complex formed by attaching two $2$-disks to $X$, with attaching maps homotopic to
\[
\alpha^2 \beta \qquad\textrm{and}\qquad \alpha\beta^2.
\]
\begin{enumerate}[(a)]
\item Find the homology groups $H_i(Y,Z)$.
\item Find the homology groups $H_i(Y,\Z/3)$.
\end{enumerate}

\begin{proof}[Solution]
  \problem*{(a)}
  We associate a cellular chain complex
  \[\xymatrix@C=0.5in{
    \cdots \ar[r] & 0 \ar[r] & \Z^2_{(2)} \ar[r]^-{\left(\begin{smallmatrix} 2 & 1 \\ 1 & 2 \end{smallmatrix}\right)} & \Z^2_{(1)} \ar[r]^-{\left(\begin{smallmatrix} 0 & 0 \end{smallmatrix}\right)} & \Z_{(1)} \ar[r] & 0 \ar[r] & \cdots,
  }\]
  whose homology is
  \[
  H_\bullet(Y,Z) \cong \Z_{(0)} \oplus \Z/3_{(1)}.
  \]
  
  \problem*{(b)}
  The universal coefficients for homology yields split short exact sequences
  \[\xymatrix{
    0 \ar[r] & H_n(Y) \otimes \Z/3 \ar[r] & H_n(Y,\Z/3) \ar[r] & \Tor(H_{n-1}(Y),\Z/3) \ar[r] & 0.
  }\]
  We obtain
  \[
  H_\bullet(Y,\Z/3) \cong \Z/3_{(0)} \oplus \Z/3_{(1)} \oplus \Z/3_{(2)}. \qedhere
  \]
\end{proof}

\problem*{Problem 3}

TODO

\problem*{Problem 4}

TODO

\problem*{Problem 5}

TODO

\problem*{Problem 6}

TODO

\subsection{Day 3}
\label{S:fall-2009-3}
\mbox{}

\problem*{Problem 1}

TODO

\problem*{Problem 2}

Let $\tau_1$ and $\tau_2 \in \C$ be a pair of complex numbers, independent over $\R$, and $\Lambda = \Z\langle \tau_1, \tau_2 \rangle \subset \C$ the lattice of integral linear combinations of $\tau_1$ and $\tau_2$. An entire meromorphic function is said to be \emph{doubly periodic} with respect to $\Lambda$ if
\[
f(z + \tau_1) = f(z + \tau_2) = f(z) \qquad \forall z \in \C.
\]
\begin{enumerate}[(a)]
\item Show that an entire holomorphic function doubly periodic with respect to $\Lambda$ is constant.
\item Suppose now that $f$ is an entire meromorphic function doubly periodic with respect to $\Lambda$, and that $f$ is either holomorphic or has one simple pole in the closed parallelogram
  \[
  \{ a \tau_1 + b \tau_2 \;|\; a,b \in [0,1] \subset \R \}.
  \]
  Show that $f$ is constant.
\end{enumerate}

\begin{proof}[Solution]
  \problem*{(a)}
  Let $P$ denote the closed parallelogram defined above. It is clear that $P$ is compact, its image under $f$ is also compact, hence closed and bounded. On the other hand, $f(\C) = f(P)$ since $f$ is doubly periodic, so $f$ is bounded. Liouville's Theorem implies $f$ is constant as required.

  \problem*{(b)}
  Without loss of generality (by shifting $P$ slightly), we may assume $P$ has no poles on $\d P$. The Cauchy Residue Theorem implies combined with the double periodicity yields
  \[
  \sum_z \res_z f = \frac{1}{2 \pi i} \int_{\d P} f = 0.
  \]
  The sum above is taken over all poles of $f$ occurring inside $P$. If $f$ is holomorphic then $f$ is constant by part (a). If $f$ had a simple pole, then the above equality is contradiction.
\end{proof}

\problem*{Problem 3}

TODO

\problem*{Problem 4}

TODO

\problem*{Problem 5}

Find the fundamental groups of the following spaces:
\begin{enumerate}[(a)]
\item $SL(2,\R)$
\item $SL(2,\C)$
\item $SO(3,\R)$
\end{enumerate}

\begin{proof}[Solution]
  \problem*{(a)}
  TODO
  
  \problem*{(b)}
  TODO
  
  \problem*{(c)}
  TODO
\end{proof}

\problem*{Problem 6}

TODO

\section{Spring 2009}
\label{S:spring-2009}

\subsection{Day 1}
\label{S:spring-2009-1}
\mbox{}

\problem*{Problem 1}

TODO

\problem*{Problem 2}

TODO

\problem*{Problem 3}

TODO

\problem*{Problem 4}

Let $X = S^1 \vee S^1$ be a figure $8$.
\begin{enumerate}[(a)]
\item Exhibit two three-sheeted covering spaces $f \cn Y \rarr X$ and $g \cn Z \rarr X$ such that $Y$ and $Z$ are not homeomorphic.
\item Exhibit two three-sheeted covering spaces $f \cn Y \rarr X$ and $g \cn Z \rarr X$ such that $Y$ and $Z$ are homeomorphic, but not as covering spaces of $X$.
\item Exhibit a normal three-sheeted covering space of $X$.
\item Exhibit a non-normal three-sheeted covering space of $X$.
\item Which of the above would still be possible if we consider two-sheeted covering spaces instead of thee-sheeted ones?
\end{enumerate}

\begin{proof}[Solution]
  We start by treating $\pi_1(X) = \langle a, b \rangle$ where we use the wedge-point as a basepoint.
  
  \problem*{(a)}
  Consider the two covering spaces corresponding the subgroups
  \begin{align*}
    \langle a^3, b^3, a b^{-1}, a^{-1} b \rangle, &&
    \langle a^3, b, a b a^{-1}, a^{-1} b a \rangle.
  \end{align*}
  The former has the property that removing any single point from the covering space leaves it connected, while the latter does not. Hence the two covers are not homeomorphic.
  
  \problem*{(b)}
  Consider the two covering spaces corresponding to subgroups
    \begin{align*}
    \langle a, b^3, b a b^{-1}, b^{-1} a b \rangle, &&
    \langle a^3, b, a b a^{-1}, a^{-1} b a \rangle.
  \end{align*}
  Drawing the respective spaces it is clear they are homeomorphic. To see they are not isomorphic as covers consider the preimage of one of the circles in $X$, say the one given by the image of $a$. If the covers are isomorphic, then these preimages would be homeomorphic. In the former case, the preimage is three disjoint circles and in the latter case a triangle. Counting connected components, we conclude these two spaces cannot be homeomorphic, hence the two covers are not isomorphic.

  \problem*{(c)}
  Both covers we used in part (b) were normal. To be specific we consider the first one, namely corresponding to the subgroup $H = \langle a, b^3, b a b^{-1}, b^{-1} a b \rangle$. It suffices to show $H \subset \pi_1(X)$ is normal. To do this, it suffices to show $H$ is closed under conjugation. Since $\pi_1(X)$ is generated by $a$ and $b$, it suffices to show $H$ is closed under conjugation by $a, a^{-1}, b, b^{-1}$. Since $a \in H$, we are left to consider $b$ and $b^{-1}$ only which is a routine exercise.

  \problem*{(d)}
  The cover corresponding to the subgroup
  \[
  \langle a, b^3, b a b, b a^{-1} b \rangle
  \]
  is not normal.

  \problem*{(e)}
  Considering two-sheeted covering spaces, we could have answered in the following manner.
  \begin{enumerate}[(a)]
  \item
    Such covers exist for the same reason -- for example, consider
    \begin{align*}
      \langle a^2, b^2, a b \rangle, &&
      \langle a, b^2, b a b \rangle.
    \end{align*}
    
  \item
    The answer is also in the affirmative -- consider
    \begin{align*}
      \langle a, b^2, b a b \rangle, &&
      \langle b, a^2, a b a \rangle.
    \end{align*}
    
  \item
    All two-sheeted covers are normal since all index 2 subgroups are normal. Any of the above examples would suffice.
    
  \item
    As explained in part (c) above, no non-normal covers exist. \qedhere
  \end{enumerate}
\end{proof}

\problem*{Problem 5}

TODO

\problem*{Problem 6}

TODO

\subsection{Day 2}
\label{S:spring-2009-2}
\mbox{}

\problem*{Problem 1}

TODO

\problem*{Problem 2}

Show that the function defined by
\[
f(z) = \sum_{n=0}^\infty z^{2^n}
\]
is analytic in the open disk $|z| < 1$, but has no analytic continuation to any large domain.

\begin{proof}[Solution]
  For the first part, it suffices to compute the radius of convergence of the series defining $f$. Set $a_k = 1$ if $k = 2^n$ for an integer $n \geq 0$, and $a_k = 0$ otherwise. Then the radius of convergence is given by
  \[
  R =
  \left( \limsup_{k \rarr \infty} \sqrt[k]{|a_k|} \right)^{-1}
  \left( \limsup_{k \rarr \infty}
    \begin{cases}
      1 & \textrm{if $k$ is a power of $2$}, \\
      0 & \textrm{otherwise}
    \end{cases}
  \right)^{-1} =
  1.
  \]
  For the second part, we first claim that it suffices to show $\lim_{r \rarr 1} f(r) = \infty$. Assume for contradiction that $f$ extends to a region $\Omega$ strictly larger than the unit disk. It follows that $\Omega$ contains some point on the unit circle and a small disk around it. Since points of the form $\exp(2\pi i k/2^\ell)$ for $k,\ell \geq 0$ are dense in the unit circle, it follows that $\Omega$ contains such a point, say $z_0 = \exp(2 \pi i k/2^\ell)$. Since $f$ extends holomorphically over $z_0$, it follows that $\lim_{r \rarr 1} f(r z_0)$ is bounded. Then
  \[
  |f(r z_0)| =
  \left| \sum_{n=0}^\infty r^{2^n} \exp\left( 2 \pi i \frac{k}{2^\ell} 2^n \right) \right| \geq
  C + \left| \sum_{n=\ell}^\infty r^{2^n} \exp\left( 2 \pi i k 2^{n-\ell} \right) \right| =
  C + \left| \sum_{n=\ell}^\infty r^{2^n} \right|,
  \]
  where $C$ can be chosen to be a constant independent of $r$. If the above expression were bounded, this would contradict the unboundedness of $\lim_{r \rarr 1} f(r)$. It suffices to provide an argument for our last point, namely
  \[
  f(r) =
  \sum_{n=0}^\infty r^{2^n} \geq
  \sum_{n=0}^N r^{2^n} \geq
  N r^{2^N}.
  \]
  Above, we could have chosen $N$ to be any positive integer. Given such a choice of $N$, consider $r = N^{-1/2^{N+1}}$, so that
  \[
  f(r) \geq \sqrt{N}.
  \]
  Since $N$ could have been chosen arbitrarily large, this furnished the necessary claim.
\end{proof}

Nasko: Is there a way to solve the second part of the given problem with techniques from complex analysis rather than brute force?

\problem*{Problem 3}

TODO

\problem*{Problem 4}

TODO

\problem*{Problem 5}

TODO

\problem*{Problem 6}

Let $X = S^2 \x \R\P^3$ and $Y = S^3 \x \R\P^2$.
\begin{enumerate}
\item Find the homology groups $H_n(X, \Z)$ and $H_n(Y, \Z)$ for all $n$.
\item Find the homology groups $H_n(X, \Z/2)$ and $H_n(Y, \Z/2)$ for all $n$.
\item Find the homotopy groups $\pi_1(X)$ and $\pi_1(Y)$.
\end{enumerate}

\begin{proof}[Solution]
  \problem*{(a)}
  Recall that
  \begin{align*}
    H_\bullet(S^2)    &= \Z_{(0)} \oplus \Z_{(2)}, &
    H_\bullet(\R\P^3) &= \Z_{(0)} \oplus \Z/2_{(1)} \oplus \Z_{(3)}, \\
    H_\bullet(S^3)    &= \Z_{(0)} \oplus \Z_{(3)}, &
    H_\bullet(\R\P^2) &= \Z_{(0)} \oplus \Z/2_{(1)}.
  \end{align*}
  Since the homology of $S^n$ is free and finitely generated in all degrees, the K\"unneth formula in its simplest form implies
  \begin{align*}
    H_\bullet(X)
    &\cong
    H_\bullet(S^2) \otimes H_\bullet(\R\P^3) \cong
    \Z_{(0)} \oplus \Z/2_{(1)} \oplus \Z_{(2)} \oplus (\Z \oplus \Z/2)_{(3)} \oplus \Z_{(5)}, \\
    H_\bullet(Y)
    &\cong
    H_\bullet(S^3) \otimes H_\bullet(\R\P^2) \cong
    \Z_{(0)} \oplus \Z/2_{(1)} \oplus \Z_{(3)} \oplus \Z/2_{(4)}.
  \end{align*}
    
  \problem*{(b)}
  Universal coefficient for homology implies
  \begin{align*}
    H_0(X,\Z/2) &\cong \Z/2, &
    H_1(X,\Z/2) &\cong \Z/2, &
    H_2(X,\Z/2) &\cong (\Z/2)^2, \\
    H_3(X,\Z/2) &\cong (\Z/2)^2, &
    H_4(X,\Z/2) &\cong \Z/2, &
    H_5(X,\Z/2) &\cong \Z/2, \\[5pt]
    H_0(Y,\Z/2) &\cong \Z/2, &
    H_1(Y,\Z/2) &\cong \Z/2, &
    H_2(Y,\Z/2) &\cong \Z/2, \\
    H_3(Y,\Z/2) &\cong \Z/2, &
    H_4(Y,\Z/2) &\cong \Z/2, &
    H_5(Y,\Z/2) &\cong \Z/2.
  \end{align*}
  
  \problem*{(c)}
  We compute
  \begin{align*}
    \pi_1(X) &\cong
    \pi_1(S^2) \x \pi_1(\R\P^3) \cong
    1 \x \Z/2 \cong
    \Z/2, \\
    \pi_1(Y) &\cong
    \pi_1(S^3) \x \pi_1(\R\P^2) \cong
    1 \x \Z/2 \cong
    \Z/2. \qedhere
  \end{align*}
\end{proof}

\subsection{Day 3}
\label{S:spring-2009-3}
\mbox{}

\problem*{Problem 1}

TODO

\problem*{Problem 2}

TODO

\problem*{Problem 3}

TODO

\problem*{Problem 4}

TODO

\problem*{Problem 5}

TODO

\problem*{Problem 6}

Let $X$ and $Y$ be two CW complexes.
\begin{enumerate}[(a)]
\item Show that $\chi(X \x Y) = \chi(X) \chi(Y)$.
\item Let $A$ and $B$ be two subcomplexes of $X$ such that $X = A \cup B$. Show that $\chi(X) = \chi(A) + \chi(B) - \chi(A \cap B)$.
\end{enumerate}

\begin{proof}[Solution]
  In order to define $\chi(X)$ and $\chi(Y)$, we need to assume they are finite.
  
  \problem*{(a)}
  For all $n \in \Z$, let the number of $n$-cells of $X$ and $Y$ be $x_n$ and $y_n$ respectively. It is a trivial observation that
  \begin{align*}
    x_n &= \rank H_n(X^n,X^{n-1}), &
    y_n &= \rank H_n(Y^n,Y^{n-1}).
  \end{align*}
  It is an easy algebraic exercise to show that the Euler characteristic of a finite chain complex does not change when we take its homology. Therefore cellular homology implies
  \[
  \chi(X) =
  \sum_n (-1)^n \rank H_n(X) =
  \sum_n (-1)^n \rank H_n(X^n,X^{n-1}) =
  \sum_n (-1)^n x_n
  \]
  and similarly for $Y$. On the other hand, the CW structures on $X$ and $Y$ induce such a structure on $X \x Y$ in which the number of $n$-cells is $\sum_i x_i y_{n-i}$. Therefore,
  \[
  \chi(X \x Y) =
  \sum_n (-1)^n \sum_i x_i y_{n-i} =
  \sum_{i,j} (-1)^{i+j} x_i y_j =
  \left( \sum_i (-1)^i x_i \right) \left( \sum_j (-1)^j y_j \right) =
  \chi(X) \chi(Y).
  \]
  
  \problem*{(b)}
  Since $A$ and $B$ are subcomplexes of $X$, there exists neighbourhoods $U_A$ and $U_B$ of $A$ and $B$ respectively which deformation retract for the respective subcomplexes. It follows that there exists a Mayer-Vietoris long exact sequence relative $H_\bullet(A \cap B)$, $H_\bullet(A) \oplus H_\bullet(B)$, and $H_\bullet(X)$. The required equality follows directly by counting ranks in this sequence.
\end{proof}

\section{Fall 2008}
\label{S:fall-2008}

\subsection{Day 1}
\label{S:fall-2008-1}
\mbox{}

\problem*{Problem 1}

TODO

\problem*{Problem 2}

Evaluate the integral
\[
\int_0^\infty \frac{\sqrt{t}}{(1+t)^2} d t.
\]

\begin{proof}[Solution]
  We start with a change of variable $t = x^2$ or $x = \sqrt{t}$ (the square root is taken to be positive here), which yields
  \[
  I =
  \int_0^\infty \frac{\sqrt{t}}{(1+t)^2} d t =
  2 \int_0^\infty \frac{x^2}{(1+x^2)^2} d x =
  \int_{-\infty}^\infty \frac{x^2}{(1+x^2)^2} d x.
  \]
  We are lead to consider the meromorphic function $f(z) = z^2/(1+z^2)^2$ which has two double poles at $i$ and $-i$ respectively. The residue at $i$ is
  \[
  \res_i f =
  \lim_{z \rarr i} \frac{1}{1!} \frac{d}{d z} (z-i)^2 \frac{z^2}{(z-i)^2(z+i)^2} =
  2 i \lim_{z \rarr i} \frac{z}{(z+i)^3} = \frac{1}{4 i}.
  \]
  Consider a semicircular contour in the upper half-plane spanning the interval $[-R,R]$ for $R > 0$. Since $f$ is rational and the degree of the numerator is 2 less than the degree of the denominator, it follows that the integral over the semicircle converges to $0$ as $R \rarr \infty$. Therefore
  \[
  I = 2 \pi i \res_i f = \frac{\pi}{2}. \qedhere
  \]
\end{proof}

TODO

\problem*{Problem 3}

TODO

\problem*{Problem 4}

Let $X = (S^1 \x S^1) \setminus \{p\}$ be a once-punctured torus.
\begin{enumerate}[(a)]
\item How many connected, 3-sheeted covering spaces $f \cn Y \rarr X$ are there?
\item Show that for any of these covering spaces, $Y$ is either a 3-times punctured torus or a once-punctured surface of genus 2.
\end{enumerate}

\begin{proof}[Solution]
  \problem*{(a)}
  TODO
  
  \problem*{(b)}
  TODO
\end{proof}

\problem*{Problem 5}

TODO

\problem*{Problem 6}

TODO

\subsection{Day 2}
\label{S:fall-2008-2}
\mbox{}

\problem*{Problem 1}

TODO

\problem*{Problem 2}

Let $\Omega \subset \C$ be the open region
\[
\Omega = \{ z \;|\; |z-1| < 1 \textrm{ and } |z-i| < 1 \}.
\]
Find a conformal map $f \cn \Omega \rarr \Delta$ of $\Omega$ onto the unit disk $\Delta = \{ z \;|\; |z| < 1 \}$.

\begin{proof}[Solution]
  Set $\Omega_1 = \Omega$. We proceed to give a sequence of conformal isomorphisms $f_i \cn \Omega_i \rarr \Omega_{i+1}$.
  \begin{align*}
    f_1(z) &= \frac{1}{z} &
    \Omega_2 &= \{ z \;|\; \Re z > 1/2 \textrm{ and } \Im z < -1/2 \} \\
    f_2(z) &= i \left( z + \frac{-1 + i}{2} \right) &
    \Omega_3 &= \{ z \;|\; \Re z > 0 \textrm{ and } \Im z > 0 \} \\
    f_3(z) &= z^2 &
    \Omega_4 &= \{ z \;|\; \Im z > 0 \} \\
    f_4(z) &= \frac{i-z}{i+z} &
    \Omega_5 &= \{ z \;|\; |z| < 1 \}
  \end{align*}
  The composition $f = f_4 \circ \cdots \circ f_1$ is given by
  \[
  f(z) = \frac{z^2 - 2 i (i-1) z - 2 i}{3 z^2 + 2 i (i-1) z + 2 i}. \qedhere
  \]
\end{proof}

\problem*{Problem 3}

TODO

\problem*{Problem 4}

TODO

\problem*{Problem 5}

TODO

\problem*{Problem 6}

Let $X = S^2 \wedge \R\P^2$ be the wedge of the $2$-sphere and the real projective plane. (This is the space obtained from the disjoint union of the $2$-sphere and the real projective plane by the equivalence relation that identifies a given point in $S^2$ with a given point in $\R\P^2$, with the quotient topology.)
\begin{enumerate}[(a)]
\item Find the homology groups $H_n(X,\Z)$ for all $n$.
\item Describe the universal covering space of $X$.
\item Find the fundamental group $\pi_1(X)$.
\end{enumerate}

\begin{proof}[Solution]
  \problem*{(a)}
  Recall that $\wtilde{H}_\bullet(X) \cong \wtilde{H}_\bullet(S^2) \oplus \wtilde{H}_\bullet(\R\P^2)$ and
  \begin{align*}
    \wtilde{H}_\bullet(S^2)    &\cong \Z_{(2)}, &
    \wtilde{H}_\bullet(\R\P^2) &\cong \Z/2_{(1)},
  \end{align*}
  so
  \[
  \wtilde{H}_\bullet(X) \cong \Z/2_{(1)} \oplus \Z_{(2)}.
  \]
  Since $X$ is connected, we have
  \[
  H_\bullet(X) \cong
  \wtilde{H}_\bullet(X) \oplus \Z_{(0)} \cong
  \Z_{(0)} \oplus \Z/2_{(1)} \oplus \Z_{(2)}.
  \]
  
  \problem*{(b)}
  The universal covering space $\wtilde{X}$ of $X$ can be described as the union of three $2$-spheres of radius $1$ centered at $(-2,0,0)$, $(0,0,0)$, and $(2,0,0)$ sitting inside $\R^3$ (the first two touch at $(-1,0,0)$ and the latter two at $(1,0,0)$). The covering map $q \cn \wtilde{X} \rarr X$ is the quotient under the identification $x \sim -x$.
  
  \problem*{(c)}
  Observing that the covering space described above is two-sheeted, we obtain $\pi_1(X) \cong \Z/2$. Alternatively, one could also derive this from the van Kampen Theorem which implies that
  \[
  \pi_1(X) \cong
  \pi_1(S^2) \ast \pi_1(\R\P^2 \cong
  1 \ast (\Z/2) \cong
  \Z/2. \qedhere
  \]
\end{proof}

\subsection{Day 3}
\label{S:fall-2008-3}
\mbox{}

\problem*{Problem 1}

TODO

\problem*{Problem 2}

TODO

\problem*{Problem 3}

Let $X$ and $Y$ be compact, connected, oriented $3$-manifolds, with
\begin{align*}
  \pi_1(X) &= (\Z/3) \oplus \Z \oplus \Z, &
  \pi_1(Y) &= (\Z/6) \oplus \Z \oplus \Z \oplus \Z.
\end{align*}
\begin{enumerate}[(a)]
\item Find $H_n(X,\Z)$ and $H_n(Y,\Z)$ for all $n$.
\item Find $H_n(X \x Y, \Q)$ for all $n$.
\end{enumerate}

\begin{proof}[Solution]
  \problem*{(a)}
  Let us start with $X$ and we will later handle $Y$ analogously. We expect non-trivial homology only in degrees $0$ to $3$ since $X$ is a $3$-manifold. Connectedness implies $H_0(X) \cong H^0(X) \cong \Z$. Orientedness implies $H_3(X) \cong H^3(X) \cong \Z$. The Hurewicz Theorem implies
  \[
  H_1(X) \cong \pi_1(X)/[\pi_1(X),\pi_1(X)] \cong (\Z/3) \oplus \Z^2.
  \]
  Finally, we are left to determine $H_2(X)$. Poincar\'e duality implies $H^2(X) \cong H_1(X)$. Since $X$ is a closed manifold, there is a finite CW complex structure. Since the double dual of a free abelian group is naturally isomorphic to itself, and the groups in the cellular complex are such, a version of universal coefficients from cohomology from homology holds. We compute
  \[
  H_2(X) \cong
  \Hom(H^2(X),\Z) \oplus \Ext(H^3(X),\Z) \cong
  \Z^2.
  \]
  Similarly one can compute the homology of $Y$. We summarize
  \begin{align*}
    H_\bullet(X) &\cong \Z_{(0)} \oplus (\Z/3 \oplus \Z^2)_{(1)} \oplus \Z^2_{(2)} \oplus \Z_{(3)}, \\
    H_\bullet(Y) &\cong \Z_{(0)} \oplus (\Z/6 \oplus \Z^3)_{(1)} \oplus \Z^3_{(2)} \oplus \Z_{(3)}.
  \end{align*}
  
  \problem*{(b)}
  Using the fact all $\Tor(-,\Q)$ groups vanish, then universal coefficients implies
  \begin{align*}
    H_\bullet(X;\Q)
    &\cong
    H_\bullet(X;\Z) \otimes \Q \cong
    \Q_{(0)} \oplus \Q^2_{(2)} \oplus \Q^2_{(2)} \oplus \Q_{(3)}, \\
    H_\bullet(Y;\Q)
    &\cong
    H_\bullet(Y;\Z) \otimes \Q \cong
    \Q_{(0)} \oplus \Q^3_{(2)} \oplus \Q^3_{(2)} \oplus \Q_{(3)}.
  \end{align*}
  Since $\Q$ is a field, a simple version of the K\"unneth formula implies
  \[
  H_\bullet(X \x Y; \Q) \cong
  H_\bullet(X; \Q) \otimes H_\bullet(Y; \Q) \cong
  \Q_{(0)} \oplus \Q^5_{(1)} \oplus \Q^{11}_{(2)} \oplus \Q^{14}_{(3)} \oplus \Q^{11}_{(4)} \oplus \Q^5_{(5)} \oplus \Q_{(6)}. \qedhere
  \]
\end{proof}

\problem*{Problem 4}

TODO

\problem*{Problem 5}

TODO

\problem*{Problem 6}

TODO

\section{Spring 2008}
\label{S:spring-2008}

\subsection{Day 1}
\label{S:spring-2008-1}
\mbox{}

\problem*{Problem 1}

Let $K = \C(x)$ be the field of rational functions in one variable over $\C$, and consider the polynomial
	\[
	f(y) = y^4 + x \cdot y^2 + x \in K[y].
	\]
\begin{enumerate}[(a)]
\item Show that $f$ is irreducible in $K[y]$.
\item Let $L = K[y]/(f)$. Is $L$ a Galois extension of $K$?
\item Let $L'$ be the splitting field of $f$ over $K$.  Find the Galois group of $L'/K$.
\end{enumerate}

\begin{proof}[Solution]
  \problem*{(a)} This follows from applying the following generalized version of Eisenstein's criterion to the UFD $K = \C(x)$ and the prime ideal $(x) \subseteq K$:
  
  \textbf{(Generalized Eisenstein's Criterion)} Let $D$ be a UFD and let $f(x) = a_nx^n + a_{n-1}x^{n-1} + \cdots + a_0 \in D[x]$ be a polynomial.  Suppose there exists a prime ideal $P$ such that
  \begin{enumerate}
  \item $a_n \not \in P$,
  \item $a_{n-1}, \ldots, a_0 \in P$,
  \item $a_0 \not \in P^2$.
  \end{enumerate}
  Then $f$ is irreducible over $F[x]$, where $F$ is the field of fractions of $D$.  If $f$ is primitive, then it is also irreducible in $D[x]$.
  
  \problem*{(b)}
  TO DO: From the wording of part (c) it seems that $f$ doesn't split completely in $L$ but I can't figure out how to show it.  Part (c) is done assuming that part (b) has been done.
  
  \problem*{(c)}
  Let $\alpha$ be a root of $f(y) = 0$, so that $L = K(\alpha)$.  Then $L' = L(\beta)$, where $\beta$ is another root of $f(y) = 0$ such that $\beta \not \in L$.  Note that $-\alpha$ is also a root of $f(y) = 0$, so that the irreducible polynomial for $\beta$ over $L$ has degree 2.  Hence $[L' : K] = [L' : L][L : K] = 2 \cdot 4 = 8$, and the Galois group of $L'/K$ has order 8.  Now, the Galois group associated to the splitting field of an irreducible polynomial of degree 4 must be a transitive subgroup of $S_4$.  The transitive subgroups of $S_4$ are $S_4, A_4$, $D_4$, $C_4$ and the Klein four-group.  The only one of these with order $8$ is $D_4$, so $\mathrm{Gal}(K/\Q) = D_4$.
\end{proof}

\problem*{Problem 2}

Let $f$ be a holomorphic function on the unit disk $\Delta = \{ z \;|\; |z| < 1 \}$. Suppose $|f(z)| < 1$ for all $z \in \Delta$, and that
\[
f\left( \frac{1}{2} \right) = f\left( -\frac{1}{2} \right) = 0.
\]
Show that $|f(0)| \leq 1/3$.

\begin{proof}[Solution]
  We consider the function $g \cn \Delta \rarr \C$ given by
  \[
  g(z) =
  \begin{cases}
    \displaystyle\frac{f(z)}{(z - 1/2)(z + 1/2)} & \textrm{if } z \neq \pm\frac{1}{2}, \\
    f'(1/2) & \textrm{if } z = 1/2, \\
    -f'(-1/2) & \textrm{if } z = -1/2,
  \end{cases}
  \]
  which is holomorphic on $\Delta$. Next pick an arbitrary $1/2 < r < 1$, and consider the function $h(z) = (z-1/2)(z+1/2) = z^2 - 1/4$. We would like to find the minimal value of $|h|$ on the circle $C_r$ given by $|z| = r$. This is the square of the minimal value of
  \[
  |h(r e^{i\theta})|^2 =
  r^2 - \frac{r \cos 2\theta}{2} + \frac{1}{16}
  \]
  which is $r^2 - r/2 + 1/16$, hence $|h| \leq \sqrt{r^2 - r/2 + 1/16}$. Since $g = f/h$ and $|f| < 1$, it follows that $|g| \leq (r^2 - r/2 + 1/16)^{-1/2}$ on $C_r$. By the maximum modulus principle $|g| \leq (r^2 - r/2 + 1/16)^{-1/2}$ on the disk $\Delta_r$ given by $|z| \leq r$. But $|g| = |f|/|h|$, so
  \[
  |f| \leq \frac{|z^2 - 1/4|}{\sqrt{r^2 - r/2 + 1/16}},
  \]
  and
  \[
  |f(0)| \leq \frac{1}{4 \sqrt{r^2 - r/2 + 1/16}}.
  \]
  Taking the limit as $r \rarr 1$, we obtain
  \[
  |f(0)| \leq \frac{1}{3}.
  \]  
\end{proof}

\problem*{Problem 3}

Let $\C\P^n$ be complex projective $n$-space.
\begin{enumerate}[(a)]
\item Describe (without proof) the cohomology ring $H^\bullet(\C\P^n, \Z)$.
\item Let $i \cn \C\P^n \rarr \C\P^{n+1}$ be the inclusion of $\C\P^n$ as a hyperplane in $\C\P^{n+1}$. Show that there does not exist a continuous map $f \cn \C\P^{n+1} \rarr \C\P^n$ such that the composition $f \circ i$ is the identity on $\C\P^n$.
\end{enumerate}


\begin{proof}[Solution]
  \problem*{(a)}
  We have $H^\bullet(\C\P^n, \Z) \cong \Z[x]/(x^{n+1})$ where $\deg(x) = 2$.

  \problem*{(b)}
  For contradiction, assume such $f$ exists. Say $H^\bullet(\C\P^n, \Z) \cong \Z[x]/(x^{n+1})$ and $H^\bullet(\C\P^{n+1}, \Z) \cong \Z[y]/(y^{n+2})$. For degree reasons, we have $f^\ast(x) = k y$ for some $k \in \Z$. Then
  \[
  0 = f^\ast(0) = f^\ast(x^{n+1}) = f^\ast(x)^{n+1} = k^{n+1} y^{n+1},
  \]
  hence $k^{n+1} = 0$ and $k = 0$. It follows that $f^\ast = 0$, but this is impossible since
  \[
  i^\ast \circ f^\ast = (f \circ i)^\ast = \id_{\C\P^n}^\ast = \id_{H^\bullet(\C\P^n, \Z)}.
  \]  
\end{proof}

\problem*{Problem 4}

TODO

\problem*{Problem 5}

TODO

\problem*{Problem 6}

TODO

\subsection{Day 2}
\label{S:spring-2008-2}
\mbox{}

\problem*{Problem 1}

TODO

\problem*{Problem 2}

Let $V$ be an $n$-dimensional vector space over a field $K$, and $Q \cn V \x V \rarr K$ a symmetric bilinear form. By the kernel of $Q$ we mean the subspace of $V$ of vectors $v$ such that $Q(v,w) = 0$ for all $w \in V$, and by the rank of $Q$ we mean $n$ minus the dimension of the kernel of $Q$.

Let $W \subset V$ be a subspace of dimension $n-k$, and let $Q'$ be the restriction of $Q$ to $W$. Show that
\[
\rank(Q) - 2k \leq \rank(Q') \leq \rank(Q).
\]

\begin{proof}[Solution]
  Recall that for linear morphisms $f \cn U \rarr V$, $g \cn V \rarr W$, we have $\dim\Ker(g \circ f) \leq \dim\Ker f + \dim\Ker g$.

  Let $Q' \cn W \x W \rarr K$ be the symmetric bilinear form on $W$ induced by $Q$. Suppose $Q$ and $Q'$ correspond to the maps $\wtilde{Q} \cn V \rarr V^\ast$ and $\wtilde{Q'} \cn W \rarr W$. If $i \cn W \hrarr V$ is the natural inclusion, then consider the following diagram.
  \[\xymatrix{
    W \ar[r]^-{\wtilde{Q'}} \ar[d]_-{i} & W^\ast \\
    V \ar[r]^-{\wtilde{Q}} & V^\ast \ar[u]_-{i^\ast}
  }\]

  TODO

  Nasko: The second inequality is easy. I have some ideas about the latter but still haven't fully realized them.  
\end{proof}

\problem*{Problem 3}

TODO

\problem*{Problem 4}

Let $S$ be a compact orientable $2$-manifold of genus $g$, and let $S_2$ be its \emph{symmetric square}, that is, the quotient of the ordinary product $S \x S$ by the involution exchanging factors.
\begin{enumerate}
\item Show that $S_2$ is a manifold.
\item Find the Euler characteristic $\chi(S_2)$.
\item Find the Betti numbers of $S_2$.
\end{enumerate}

\begin{proof}[Solution]
  \problem*{(a)}
  It is clear that $S_2$ is Hausdorff and second countable, therefore it suffices to furnish a cover by charts. Let $q \cn S \x S \rarr S_2$ be the quotient map, and $\Delta \subset S \x S$ the diagonal. Away from $\Delta$ the map $q$ is a 2-sheeted cover, and therefore, $S_2 \setminus q(\Delta)$ is a manifold. It suffices to furnish charts around points along $q(\Delta)$. Therefore, it suffices to take $S = \R^2$ and prove $S_2$ is a manifold in that case. The key point is to note $S = \R^2 \cong \C$. Therefore, we can think of a point in $S_2$ as a pair of complex numbers, disregarding ordering. Such complex pairs $\{\alpha,\beta\}$ can always be arranged to be the root of a quadratic equation $z^2 - (\alpha+\beta)z + \alpha\beta = 0$. Note that the space of monic quadratic polynomials, properly topologized, is homeomorphic to $\C^2$ (take as coordinates the coefficients of $1$ and $z$). We have furnished a map $S_2 \rarr \C^2$. Since each monic quadratic polynomial has precisely two roots (by the fundamental theorem of algebra), it can be shown this map is a homeomorphism, and we conclude $S_2 \cong \C^2$ is a manifold.
  
  \problem*{(b)}
  TODO
  
  \problem*{(c)}
  TODO
\end{proof}

Nasko: I know how to do (a) but not (b) and (c).

\problem*{Problem 5}

TODO

\problem*{Problem 6}

TODO

\subsection{Day 3}
\label{S:spring-2008-3}
\mbox{}

\problem*{Problem 1}

For $c$ a non-zero real number, evaluate the integral
\[
\int_0^\infty \frac{\log z}{z^2 + c^2} d z.
\]

\begin{proof}[Solution]
  Note that the projected answers for $c$ and $-c$ would necessarily agree. Therefore, we may assume $c$ is positive, carry out the computation, and in the very last step replace $c$ with $|c|$. Start by choosing a branch of the logarithm well-defined on $\C \setminus (-\infty,0]i$, and define $f(z) = \log z / (z^2 + c^2)$. Make a choice of two constants $R$ and $\epsilon$ satisfying $0 < \epsilon < R$. We choose a contour which traverses the semicircle from $R$ to $-R$ in the upper half-plane, the interval from $-R$ to $-\epsilon$, followed by the semicircle from $-\epsilon$ to $\epsilon$ in the upper half-plane, and finally the interval from $\epsilon$ to $R$. It is not hard to show that letting $R \rarr \infty$ and $\epsilon \rarr 0$, the integral of $f$ along the two semicircles approaches $0$. The integral over the negative line segment is related to the integral over the positive one:
  \[
  \int_{-R}^{-\epsilon} \frac{\log z}{z^2 + c^2} d z =
  \int_\epsilon^R \frac{\log(-z)}{z^2 + c^2} d z =
  \int_\epsilon^R \frac{\log z + \pi i}{z^2 + c^2} d z =
  \int_\epsilon^R \frac{\log z}{z^2 + c^2} d z + \pi i \int_\epsilon^R \frac{\pi i}{z^2 + c^2} d z.
  \]
  The residue of $f$ at $i c$ is
  \[
  \res_{i c} f =
  \log(i c) \res_{i c} \frac{1}{z^2+c^2} =
  \frac{\log c + \pi i / 2}{2 i c}.
  \]
  Also recall the computation
  \[
  \int_0^\infty \frac{1}{z^2+c^2} d z =
  \frac{1}{c} \int_0^\infty \frac{1}{z^2 + 1} d z =
  \frac{\pi}{2 c}.
  \]
  Let $I$ denote the value of the integral we are seeking. Putting all pieces of information we listed so far, applying the Cauchy Residue Theorem, and letting $R \rarr \infty$, $\epsilon \rarr 0$, we obtain
  \[
  2 I + \frac{\pi^2 i}{2 c} = 2 \pi i \res_{i c} f = \frac{\pi(\log c + \pi i/2)}{c}.
  \]
  Simplifying the previous equation and replacing $c$ with $|c|$, we have shown that
  \[
  I = \frac{\pi \log|c|}{2|c|}. \qedhere
  \]
\end{proof}

\problem*{Problem 2}

TODO

\problem*{Problem 3}

Let $S$ be a compact orientable $2$-manifold of genus $2$ (that is, a 2-holed torus) and let $f \cn S \rarr S$ be any orientation-preserving homeomorphism of finite order.
\begin{enumerate}[(a)]
\item Show that $f$ must have a fixed point.
\item Is this statement still true if we drop the hypothesis that $f$ is orientation reversing? Prove or give a counterexample.
\item Is this statement still true if we replace $S$ by a compact orientable $2$-manifold of genus 3? Again, prove or give a counterexample.
\end{enumerate}

\begin{proof}[Solution]
  \problem*{(a)}
  TODO
  
  \problem*{(b)}
  The statement is false. If one positions symmetrically a 2-holed torus around the origin in $\R^3$, then the map $x \mapsto -x$ is an orientation-reversing homeomorphism without fixed points.
  
  \problem*{(c)}
  The statement is false if the genus is $3$. One can position a 3-holed torus symmetrically around the origin in $\R^3$, such that rotation by $\pi$ along an axis of symmetry is a orientation-preserving homeomorphism without fixed points.
\end{proof}

Nasko: Any ideas about this problem?

\problem*{Problem 4}

\begin{enumerate}[(a)]
\item State Fermat's Little Theorem on powers in the field $\F_{37}$ with $37$ elements.
\item Let $k$ be any natural number not divisible by $2$ or $3$, and let $a \in \F_{37}$ be any element. Show that there exists a unique solution to the equation
  \[
  x^k = a
  \]
  in $\F_{37}$.
\item Solve the equation
  \[
  x^5 = 2
  \]
  in $\F_{37}$.
\end{enumerate}

\begin{proof}[Solution]
  \problem*{(a)}
  Fermat's Little Theorem in this specific case would state that every $a \in \F_{37}$ satisfies $a^{37} = a$. Alternatively, we may state that every $a \in \F_{37}^\x$ satisfies $a^{36} = 1$.
  
  \problem*{(b)}
  If $a = 0$, then it is easy to see the unique solution is $x = 0$. Next, assume $a \neq 0$. Note that since $k$ is not divisible by $2$ or $3$, it is then coprime to $36 = 2^2 \cdot 3^2$, and there exist integers $s$ and $t$ satisfying $k s + 36 t = 1$. Let us start by showing uniqueness. If $x$ is some solution, then raising $x^k = a$ to the power $s$, we obtain
  \[
  a^s = x^{k s} = x^{k s + 36 t} = x^1 = x.
  \]
  For the existence claim, we note that $x = a^s$ is always a solution.
  
  \problem*{(c)}
  Note that $(-7) \cdot 5 + 1 \cdot 36 = 1$, so the desired solution is
  \[
  x = 2^{-7} = 17^{-1} = 24. \qedhere
  \]
\end{proof}

\problem*{Problem 5}

TODO

\problem*{Problem 6}

TODO

\section{Fall 2007}
\label{S:fall-2007}

\subsection{Day 1}
\label{S:fall-2007-1}
\mbox{}

\problem*{Problem 1}

Let $f(x) = x^4 - 7 \in \Q(x)$.
\begin{enumerate}[(a)]
\item Show that $f$ is irreducible in $\Q[x]$.
\item Let $K$ be the splitting field of $f$ over $\Q$.  Find the Galois group of $K/\Q$.
\item How many subfields $L \subset K$ have degree 4 over $\Q$?  How many of them are Galois over $\Q$?
\end{enumerate}

\begin{proof}[Solution]
  \problem*{(a)}
  This follows from Eisenstein's criterion with the prime $p = 7$.

  \problem*{(b)}
  By observation, $K = \Q(\sqrt[4]{7}, i)$.  This has degree 8 over $\Q$ since $[\Q(\sqrt[4]{7}) : \Q] = 4$ and $i \not in \Q(\sqrt[4]{7}$.  Hence the Galois group of $K/\Q$ has order 8.  Now, we know that the Galois group associated to the splitting field of an irreducible polynomial of degree 4 must be a transitive subgroup of $S_4$.  The transitive subgroups of $S_4$ are $S_4, A_4$, $D_4$, $C_4$ and the Klein four-group.  The only one of these with order $8$ is $D_4$, so $\mathrm{Gal}(K/\Q) = D_4$.
  
  \problem*{(c)}
  The subfields $L \subset K$ of degree 4 over $\Q$ are the subfields $\Q \subset L \subset K$ such that $[K : L] = 2$.  These are in bijective correspondence with the subgroups of $\mathrm{Gal}(K/\Q)$ of order 2.  The group $D_4$ has 5 subgroups of order 2 (if $D_4$ has the presentation $\langle x, y; x^4, y^2, xyxy\rangle$, then the subgroups are $\langle x^2 \rangle$, $\langle y \rangle$, $\langle xy \rangle$, $\langle x^2y \rangle$, $\langle x^3y \rangle$), so there are 5 such subfields $L$.  The normal subgroups correspond to Galois extensions $L/\Q$.  Only the subgroup $\langle x^2 \rangle$ is a normal subgroup of $D_4$, so only one of the subfields is Galois over $\Q$.
\end{proof}

\problem*{Problem 2}

\textbf{Lemma.} Let $f$ be a convex function on $(a, b)$.  Let $a < s < t < u < b$.  Then
\begin{equation}
  \frac{f(t) - f(s)}{t-s} \leq \frac{f(u) - f(s)}{u-s} \leq \frac{f(u) - f(t)}{u-t} .
  \tag{$\dagger$} \label{eq:convex} \end{equation}
\textbf{Proof.} Let $t = \lambda s + (1 - \lambda)u$, then $\lambda = \frac{u-t}{u-s}$.  The convexity of $f$ implies that
\[
f(t) \leq \frac{u-t}{u-s}f(s) + \frac{t-s}{u-s}f(u) .
\]
Rearranging, we obtain
\[
\frac{f(t) - f(s)}{t-s} \leq \frac{f(u) - f(s)}{u-s}.
\]
Similarly, if we let $t = \lambda u + (1 - \lambda)s$, then $\lambda = \frac{t-s}{u-s}$, and the convexity of $f$ implies that
\[
f(t) \leq \frac{t-s}{u-s}f(u) + \frac{u-t}{u-s}f(s) .
\]
Rearranging, we obtain
\[
\frac{f(u) - f(s)}{u-s} \leq \frac{f(u) - f(t)}{u-t}.
\]

Now we return to the problem.  For $x \in (a, b)$, choose $\delta > 0$ such that $(x - \delta, x + \delta) \in (a, b)$.  First suppose that $z \in (x- \delta, x)$.  Applying the second inequality in \eqref{eq:convex} to $x - \delta < z < x$, we obtain
\[
\frac{f(x) - f(x-\delta)}{\delta} \leq \frac{f(z) - f(x)}{z-x},
\]
and applying the outer inequality in \eqref{eq:convex} to $z < x < x + \delta$, we obtain
\[
\frac{f(z) - f(x)}{z-x} \leq \frac{f(x + \delta) - f(x)}{\delta}.
\]
We can obtain similar inequalities if $z \in (x, x + \delta)$.

Hence $|f(x) - f(z)| \leq C|x - z|$ for all $z \in (x - \delta, x + \delta)$.  This proves the continuity of $f$.

\problem*{Problem 3}

Let $\tau_n \cn S^n \rarr S^n$ be the antipodal map, and let $X$ be the quotient $S^n \x S^m$ by the involution $(\tau_n, \tau_m)$ -- that is,
\[
X = S^n \x S^m/(x,y) \sim (-x,-y).
\]
\begin{enumerate}[(a)]
\item What is the Euler characteristic of $X$?
\item Find the homology groups of $X$ in case $n = 1$.
\end{enumerate}

\begin{proof}[Solution]
  \problem*{(a)}
  Recall that $H_\ast(S^n) \cong \Z_{(0)} \oplus \Z_{(n)}$, so
  \[
  \chi(S^n) =
  1 + (-1)^n =
  \begin{cases}
    0 & \textrm{if $n$ is odd}, \\
    2 & \textrm{if $n$ is even}.
  \end{cases}
  \]
  K\"unneth formula implies $\chi(S^n \x S^m) = \chi(S^n) \chi(S^m)$. It is easy to see that the projection map $p \cn S^n \x S^m \rarr X$ is a $2$-fold cover, hence
  \[
  \chi(X) =
  \frac{1}{2} \chi(S^n)\chi(S^m) =
  \begin{cases}
    0 & \textrm{if $m$ or $n$ is odd}, \\
    2 & \textrm{if $m$ and $n$ are both even}.
  \end{cases}
  \]

  \problem*{(b)}
  In the case $m = 1$, a cut and paste argument enables us to identify $X$ with the 2-torus, hence $H_\bullet(X) \cong \Z_{(0)} \oplus \Z_{(1)}^2 \oplus \Z_{(2)}$. We will later see this calculation fits with the result for $m > 1$. In what follows, assume $m > 1$. Consider the following open cover of $X$:
  \begin{align*}
    A &= \exp(i(0,\pi/2) \cup i(\pi,3\pi/2))/\!\!\sim, &
    B &= \exp(i(\pi/2,\pi) \cup i(3\pi/2,2\pi))/\!\!\sim.
  \end{align*}
  It is not hard to check that $A \simeq B \simeq S^m$ and $A \cap B \simeq S^m \sqcup S^m$. The relevant parts of the associated Mayer-Vietoris sequence and the associated maps are as follows.
  \[\xymatrix{
    0 \ar[r] & H_1(X) \ar[r] \ar@{=}[d] & H_0(A \cap B) \ar[r] \ar[d]^\cong & H_0(A) \oplus H_0(B) \ar[r] \ar[d]^\cong & H_0(X) \ar[r] \ar@{=}[d] & 0 \\
    0 \ar[r] & H_1(X) \ar[r] & \Z^2 \ar[r]^{\left(\begin{smallmatrix} 1 & 1 \\ 1 & 1 \end{smallmatrix}\right)} & \Z^2 \ar[r] & H_0(X) \ar[r] & 0 \\
  }\]
  \[\xymatrix{
    0 \ar[r] & H_{m+1}(X) \ar[r] \ar@{=}[d] & H_m(A \cap B) \ar[r] \ar[d]^\cong & H_m(A) \oplus H_m(B) \ar[r] \ar[d]^\cong & H_m(X) \ar[r] \ar@{=}[d] & 0 \\
    0 \ar[r] & H_{m+1}(X) \ar[r] & \Z^2 \ar[r]^{\left(\begin{smallmatrix} 1 & \phantom{(-}1\phantom{)^{m+1}} \\ 1 & (-1)^{m+1} \end{smallmatrix}\right)} & \Z^2 \ar[r] & H_m(X) \ar[r] & 0 \\
  }\]
  We conclude
  \[
  H_\bullet(X) \cong
  \begin{cases}
    \Z_{(0)} \oplus \Z_{(1)} \Z/2_{(m)} & \textrm{if $n$ is even}, \\
    \Z_{(0)} \oplus \Z_{(1)} \oplus \Z_{(m)} \oplus \Z_{(m+1)} & \textrm{if $n$ is odd}.
  \end{cases}
  \]
  Note that this computation extends to the case $m = 1$ handled above.
\end{proof}

\problem*{Problem 4}

Construct a surjective conformal mapping from the pie wedge
\[
\Omega = \{ z = r e^{i \theta} \;|\; \theta \in (0, \pi/4), r < 1 \}
\]
to the unit disk
\[
\Delta = \{ z \;|\; |z| < 1 \}.
\]

\begin{proof}[Solution]
  Set $\Omega_1 = \Omega$. We proceed to give a sequence of conformal isomorphisms $f_i \cn \Omega_i \rarr \Omega_{i+1}$.
  \begin{align*}
    f_1(z) &= z^4 &
    \Omega_2 &= \{ z \;|\; |z| < 1 \textrm{ and } \Im z > 0 \} \\
    f_2(z) &= z+1 &
    \Omega_3 &= \{ z \;|\; |z-1| < 1 \textrm{ and } \Im z > 0 \} \\
    f_3(z) &= \frac{1}{z} &
    \Omega_4 &= \{ z \;|\; \Re z > 1/2 \textrm{ and } \Im z < 0 \} \\
    f_4(z) &= i \left( z - \frac{1}{2} \right) &
    \Omega_5 &= \{ z \;|\; Re z > 0 \textrm{ and } \Im z > 0 \} \\
    f_5(z) &= z^2 &
    \Omega_6 &= \{ z \;|\; \Im z > 0 \} \\
    f_6(z) &= \frac{i-z}{i+z} &
    \Omega_7 &= \{ z \;|\; |z| < 1 \}
  \end{align*}
  The composition $f = f_6 \circ \cdots \circ f_1$ is given by
  \[
  f(z) = -i \frac{z^8 + 2 i z^4 + 1}{z^8 - 2 i z^4 + 1}. \qedhere
  \]
\end{proof}

\problem*{Problem 5}

TODO

\problem*{Problem 6}
Compute the curvature and the torsion of the curve
	\[
	\rho(t) = (t, t^2, t^3)
	\]
in $\R^3$.

\begin{proof}[Solution]
  The curvature $\kappa$ of the curve is given by the formula
  \[
  \kappa = \frac{|\rho' \times \rho''|}{|\rho'|^3}.
  \]
  Substituting in the parametrization $\rho(t) = (t, t^2, t^3)$, we obtain
  \[
  \kappa = \frac{\sqrt{36t^4 + 36t^2 + 4}}{(1 + 4t^2 + 9t^4)^{3/2})}.
  \]
  
  The torsion of the curve is given by the formula
  \[
  \tau = \frac{(\rho' \times \rho'')\cdot \rho'''}{|\rho' \times \rho''|^2},
  \]
  and substituting in the given parametrization, we obtain
  \[
  \kappa = \frac{3}{9t^4 + 9t^2 + 1}. \qedhere
  \]
\end{proof}

\subsection{Day 2}
\label{S:fall-2007-2}
\mbox{}

\problem*{Problem 1}

Evaluate the integral
\[
\int_0^\infty \frac{x^2}{x^4 + 5 x^2 + 4} d x.
\]

\begin{proof}[Solution]
  Taking
  \[
  I =
  \int_0^\infty \frac{x^2}{x^4 + 5 x^2 + 4} d x =
  \frac{1}{2} \int_{-\infty}^\infty \frac{x^2}{(x^2+1)(x^2+4)} d x,
  \]
  we are lead to consider the meromorphic function $f(z) = z^2/((z^2+1)(z^2+4))$. It has four simple poles at $\pm i$ and $\pm 2i$ respectively. We are interested in
  \begin{align*}
    \res_i f &=
    \frac{1}{0!} \lim_{z \rarr i} (z-i) \frac{z^2}{(z^2+1)(z^2+4)} =
    -\frac{1}{6 i}, \\
    \res_{2 i} f &=
    \frac{1}{0!} \lim_{z \rarr 2i} (z-2i) \frac{z^2}{(z^2+1)(z^2+4)} =
    \frac{1}{3 i}.
  \end{align*}
  A simple estimation leads to
  \[
  I =
  \frac{1}{2} 2 \pi i \left( \res_i f + \res_{2i} f \right) =
  \frac{\pi}{6}. \qedhere
  \]
\end{proof}

\problem*{Problem 2}

TODO

\problem*{Problem 3}

TODO

\problem*{Problem 4}

Consider the following three topological spaces
\begin{align*}
  A &= \C\P^3, &
  B &= S^2 \x S^4, &
  C &= S^2 \wedge S^4 \wedge S^6,
\end{align*}
where $\C\P^3$ is the complex projective $3$-space, $S^n$ is the $n$-sphere, and $\wedge$ denotes the wedge product.
\begin{enumerate}[(a)]
\item Calculate the cohomology groups (with integer coefficients) of all three.
\item Show that $A$ and $B$ are not homotopy equivalent.
\item Show that $C$ is not homotopy equivalent to any compact manifold.
\end{enumerate}

\begin{proof}[Solution]
  \problem*{(a)}
  We know that $H^\bullet(\C\P^n) \cong \Z[\alpha]/(\alpha^{n+1})$ with $|\alpha| = 2$, so the relative cohomology groups are
  \[
  H^\bullet(\C\P^3) \cong
  \Z_{(0)} \oplus \Z_{(2)} \oplus \Z_{(4)} \oplus \Z_{(6)}.
  \]
  For the second space, we have $H^\bullet(S^2) \cong \Z_{(0)} \oplus \Z_{(2)}$ and $H^\bullet(S^4) \cong \Z_{(0)} \oplus \Z_{(4)}$. Since all groups are finitely generated and free, a simple version of the K\"unneth formula implies
  \[
  H^\bullet(S^2 \x S^4) \cong
  H^\bullet(S^2) \otimes H^\bullet(S^4) \cong
  \Z_{(0)} \oplus \Z_{(2)} \oplus \Z_{(4)} \oplus \Z_{(6)}.
  \]
  Finally, reduced cohomology is additive with respect to the wedge product. To pass from reduced to non-reduced for a connected space such as $S^2 \wedge S^4 \wedge S^6$, we only need to add a factor of $\Z$ in degree $0$. We compute
  \begin{align*}
    H^\bullet(S^2 \wedge S^4 \wedge S^6)
    &\cong
    \Z_{(0)} \oplus \wtilde{H}^\bullet(S^2 \wedge S^4 \wedge S^6) \cong
    \Z_{(0)} \oplus \wtilde{H}^\bullet(S^2) \oplus \wtilde{H}^\bullet(S^4) \oplus \wtilde{H}^\bullet(S^6) \\
    &\cong
    \Z_{(0)} \oplus \Z_{(2)} \oplus \Z_{(4)} \oplus \Z_{(6)}.
  \end{align*}
  
  \problem*{(b)}
  Note that $\alpha \wcup \alpha = \alpha^2 \neq 0$ in $H^\bullet(\C\P^3)$. On the other hand, since $H^\bullet(S^2 \x S^4) \cong H^\bullet(S^2) \otimes H^\bullet(S^4)$ as rings, the product of any two degree 2 classes is trivial there. We conclude that the cohomology rings of $\C\P^3$ and $S^2 \x S^4$ are not isomorphic, hence the spaces are not homotopy equivalent.
  
  \problem*{(c)}
  For contradiction, assume $C$ is homotopy equivalent to a compact manifold $M$. Since $C$ is connected, so is $M$. Van Kampen's Theorem implies
  \[
  \pi_1(M) \cong \pi_1(C) \cong \pi_1(S^2) \ast \pi_1(S^4) \ast \pi_1(S^6) \cong 1 \ast 1 \ast 1 \cong 1,
  \]
  hence $M$ is simply-connected. If $M$ were non-orientable, it would admit a connected 2-sheeted cover. This is however impossible in view of $\pi_1(M) = 1$, hence $M$ is orientable. A simple computation with cohomology rings reveals that the ring $H^\bullet(M) \cong H^\bullet(C)$ is trivial. In other words $\alpha \wcup \beta = 0$ if one of the degrees $|\alpha|$ or $|\beta|$ is positive for $\alpha, \beta \in H^\bullet(M)$. If $\alpha$ denotes the generator of $H^2(M)$, then Poincar\'e duality implies there exists $\beta \in H^4(M)$ such that $(\alpha \wcup \beta)[M] = 1$, in particular $\alpha \wcup \beta \neq 0$. This yields the necessary contradiction.
\end{proof}

\problem*{Problem 5}

TODO

\problem*{Problem 6}

TODO

\subsection{Day 3}
\label{S:fall-2007-3}
\mbox{}

\problem*{Problem 1}

TODO

\problem*{Problem 2}

TODO

\problem*{Problem 3}

\begin{enumerate}
\item Show that any continuous map from the $2$-sphere $S^2$ to a compact orientable manifold of genus $g \geq 1$ is homotopic to a constant map.
\item Recall that if $f \cn X \rarr Y$ is a map between compact, oriented $n$-manifolds, the induced map $f_\ast \cn H_n(X) \rarr H_n(Y)$ is a multiplication by some integer $d$, called the \emph{degree} of the map $f$. Now let $S$ and $T$ be compact oriented $2$-manifolds of genus $g$ and $h$ respectively, and $f \cn S \rarr T$ a continuous map. Show that if $g < h$, then the degree of $f$ is zero.
\end{enumerate}

\begin{proof}[Solution]
  \problem*{(a)}
  Let $f \cn S^2 \rarr \Sigma_g$ be a continuous map. Since $g \geq 1$, the universal cover of $\Sigma_g$ is $\R^2$, say via $p \cn \R^2 \rarr \Sigma_g$. Note that $\pi_1(S^2) = 1 \subset p_\ast(\pi_1(\R^2)) = p_\ast(1) = 1$, and $S^2$ is path-connected and locally path-connected, hence the lifting criterion furnished a map $\wtilde{f} \cn S^2 \rarr \R^2$ lifting $f$. Since $\R^2$ is contractible, the map $\wtilde{f}$ is nullhomotopic. Composing this homotopy with $p$, we conclude that $f$ is also nullhomotopic.

  \problem*{(b)}
  Let us start by describing the cohomology ring of $\Sigma_g$, a compact orientable surface of genus $g$. We have $H^0(\Sigma_g) \cong \Z$ and $H^2(\Sigma_g) \cong \Z$ which are generated respectively by the unit $1 \in H^0(\Sigma_g)$ and the fundamental class $[\Sigma_g] \in H^2(\Sigma_g)$. The degree 1 part $H^1(\Sigma_g)$ is generated freely by $2g$ elements $\alpha_1, \beta_1, \dots, \alpha_g, \beta_g$ which satisfy $\alpha_i \wcup \beta_i = [\Sigma_g]$ and all other products are trivial. The ring $H^\bullet(\Sigma_g;\Q)$ is analogously presented but with coefficients in $\Q$.

  For convenience set $S = \Sigma_g$ and $T = \Sigma_h$. Consider a map $f \cn \Sigma_g \rarr \Sigma_h$ for $g < h$ of degree $d = \deg f$. A dimension count implies the map $f^\ast \cn H^1(\Sigma_h;\Q) \rarr H^1(\Sigma_g;\Q)$ has nontrivial kernel. Pick a non-trivial element $\gamma$ in the kernel of this map. Up to scaling and reordering of the generators, we may write $\gamma = \alpha_1 + \gamma'$ where $\gamma'$ is a linear combination of all generators except $\alpha_1$. Note that
  \[
  \gamma \wcup \beta_1 =
  (\alpha_1 + \gamma') \wcup \beta_1 =
  [\Sigma_h].
  \]
  We have
  \[
  d [\Sigma_g] =
  f^\ast[\Sigma_h] =
  f^\ast(\gamma \wcup \beta_1) =
  f^\ast(\gamma) \wcup f^\ast(\beta_1) =
  0 \wcup f^\ast(\beta_1) = 0,
  \]
  hence $d = 0$ as desired.
\end{proof}

\problem*{Problem 4}

TODO

\problem*{Problem 5}

TODO

\problem*{Problem 6}

TODO

%%% Local Variables: 
%%% mode: latex
%%% TeX-master: "Study_Guide"
%%% End: 


% \bibliographystyle{amsplain.bst}
% \bibliography{template}

\end{document}