\section{Real Analysis}
\label{S:real-analysis}

Syllabus
\begin{description}
\item[Undergraduate] Royden, \emph{Real Analysis} (3rd ed), chapters 1-10. (math 114).
\item[Graduate] Rudin, \emph{Real and Complex Analysis}, chapters 1-9. (math 212a).
\item[Additional] Stein and Shakarchi, \emph{Real Analysis: Measure Theory, Integration and Hilbert Spaces} may also be a good source for some of this material.
\end{description}

TODO: Royden

\subsection{Rudin, Prologue}

We define the complex function
\[
\exp(z) = \sum_{n=0}^\infty \frac{z^n}{n!},
\]
and often write $e^z = \exp(z)$. The given series converges absolutely for every $z$, and uniformly on every bounded set in the complex plane. It satisfies $\exp(a+b) = \exp(a)\exp(b)$ for all $a,b \in \C$.

\begin{theorem}
  \mbox{}
  \begin{enumerate}[(a)]
  \item For every $z \in \C$, we have $e^z \neq 0$.
  \item $\exp$ is its own derivative.
  \item The restriction of $\exp$ to the real axis is a monotonically increasing positive function, and
    \begin{align*}
      \lim_{x \rarr \infty} e^x &= \infty, &
      \lim_{x \rarr -\infty} e^x &= 0.
    \end{align*}
  \item There exists a positive number $\pi$ such that $e^{\pi i/2} = i$, and such that $e^z = 1$ if and only if $z/(2\pi i)$ is an integer.
  \item $\exp$ is a periodic function with period $2 \pi i$.
  \item The mapping $t \mapsto e^{i t}$ maps the real axis onto the unit circle.
  \item If $w$ is a nonzero complex number, then $w = e^z$ for some $z \in \C$.
  \end{enumerate}
\end{theorem}

\subsection{Rudin, Chapter 1: Abstract Integration}

\begin{definition}
  A collection $\Tc$ of subsets of a set $X$ is said to be a \emph{topology} on $X$ if $\Tc$ has the following three properties:
  \begin{enumerate}[(i)]
  \item $\emptyset \in \Tc$ and $X \in \Tc$;
  \item if $V_i \in \Tc$ for $i = 1, \dots, n$, then $V_1 \cap \cdots \cap V_n \in \Tc$;
  \item if $\{V_\alpha\}$ is an arbitrary collection of members of $\Tc$, then $\bigcup_\alpha V_\alpha \in \Tc$.
  \end{enumerate}
  The elements of $\Tc$ are called \emph{open sets} in $X$, and $X$ is called a \emph{topological space}.
\end{definition}

\begin{definition}
  A map $f \cn X \rarr Y$ between two topological spaces is called \emph{continuous} if $f^{-1}(U)$ is open in $X$ for all open $U \subset Y$.

  A map $f \cn X \rarr Y$ between two topological spaces is called \emph{continuous at $x \in X$} if for every open $V \subset Y$ around $f(x)$, there exists an open $U \subset X$ such that $f(U) \subset V$.
\end{definition}

\begin{proposition}
  A map $f \cn X \rarr Y$ between two topological spaces is continuous if and only if it is continuous around all $x \in X$.
\end{proposition}

\begin{definition}
  A collection $\Mc$ of subsets of a set $X$ is said to be a $\sigma$-algebra on $X$ if $\Mc$ has the following three properties:
  \begin{enumerate}[(i)]
  \item $X \in \Mc$;
  \item if $A \in \Mc$, then $A^c = X \setminus A \in \Mc$;
  \item if $A = \bigcup_{n=1}^\infty A_n$ for some $A_n \in \Mc$, then $A \in \Mc$.
  \end{enumerate}
  We call the elements of $\Mc$ \emph{measurable sets} in $X$, and $X$ a \emph{measurable space}.
\end{definition}

\begin{definition}
  Let $X$ be a measurable space, and $Y$ a topological space. A map $f \cn X \rarr Y$ is called \emph{measurable} if $f^{-1}(U)$ is measurable in $X$ for every open $U \subset Y$. 
\end{definition}

\begin{remark}
  Rudin could have alternatively defined a measurable map between two measurable spaces. What is the motivation behind the more restrictive definition above?
\end{remark}

\begin{proposition}[Further properties of metric spaces]
  Let $X$ be a measurable space with $\sigma$-algebra $\Mc$.
  \begin{enumerate}[(a)]
  \item $\emptyset \in \Mc$.
  \item If $A_i \in \Mc$ for $1 \leq i \leq n$, then $\bigcup_{i=1}^n A_n \in \Mc$.
  \item If $A_i \in \Mc$ for $i \geq 1$, then $\bigcap_{n=1}^\infty A_n \in \Mc$.
  \item If $A, B \in \Mc$, then $A \setminus B \in \Mc$.
  \end{enumerate}  
\end{proposition}

\begin{theorem}
  Let $g \cn Y \rarr Z$ be a continuous function between topological spaces.
  \begin{enumerate}[(a)]
  \item If $f \cn X \rarr Y$ is a continuous function for a topological space $X$, then $h = g \circ f \cn X \rarr Z$ is continuous.
  \item Let $f \cn X \rarr Y$ is a measurable function for a measurable space $X$, then $h = g \circ f \cn X \rarr Z$ is measurable.
  \end{enumerate}
\end{theorem}

\begin{theorem}
  Let $u$ and $v$ be real measurable functions on a measurable space $X$, and $\Phi \cn \R^2 \rarr Y$ a continuous map for a topological space $Y$. Then $h \cn X \rarr Y$ defined by $h(x) = \Phi(u(x),v(x))$ is measurable.
\end{theorem}

\begin{corollary}
  Let $X$ be a measurable space.
  \begin{enumerate}[(a)]
  \item If $f = u + i v$, where $u$ and $v$ are real measurable functions on $X$, then $f$ is a complex measurable function on $X$.
  \item If $f = u + i v$ is a complex measurable function on $X$, then $u$, $v$, and $|f|$ are real measurable functions on $X$.
  \item If $f$ and $g$ are complex measurable functions on $X$, then so are $f + g$ and $f g$.
  \item If $E \subset X$ is a measurable set, then the function $\chi_E \cn X \rarr \C$ given by
    \[
    \chi_E(x) =
    \begin{cases}
      1 & \textrm{if } x \in E, \\
      0 & \textrm{if } x \notin E
    \end{cases}
    \]
    is measurable.
  \item If $f$ is a complex measurable function on $X$, there is a complex measurable function $\alpha$ on $X$ such that $|\alpha| = 1$ and $f = \alpha |f|$.
  \end{enumerate}
\end{corollary}

\begin{theorem}
  If $\Fc$ is any collection of subsets of $X$, then there exists a smallest $\sigma$-algebra $\Mc$ in $X$ such that $\Fc \subset \Mc$. We say $\Mc$ is \emph{generated by $\Fc$}.
\end{theorem}

\begin{definition}
  Let $X$ be a topological space. The \emph{Borel measure} $\Bc$ on $X$ is generated by the opens in $X$. The elements of $\Bc$ are called \emph{Borel sets}.
\end{definition}

\begin{proposition}
  Every continuous map is \emph{Borel measurable}, that is, measurable when the domain is endowed with the Borel measure.
\end{proposition}

\begin{theorem}
  Let $X$ be a measurable space with $\sigma$-algebra $\Mc$, $Y$ a topological space, and $f \cn X \rarr Y$ a map.
  \begin{enumerate}[(a)]
  \item If $\Omega$ is the collection of all sets $E \subset Y$ such that $f^{-1}(E) \in \Mc$, then $\Omega$ is a $\sigma$-algebra on $Y$. In short, pushforwards of $\sigma$-algebras are $\sigma$-algebras.
  \item If $f$ is measurable and $E$ is a Borel set in $Y$, then $f^{-1}(E) \in \Mc$.
  \item If $Y = [-\infty,\infty]$ and $f^{-1}((\alpha,\infty]) \in \Mc$ for all $\alpha \in \R$, then $f$ is measurable.
  \end{enumerate}
\end{theorem}

\begin{definition}
  Let $\{a_n\}$ be a sequence in $[-\infty,\infty]$, and put
  \begin{align*}
    \limsup_{n \rarr \infty} a_n &= \inf\{ \sup\{ a_k, a_{k+1}, \dots \} \;|\; k \geq 1 \}, \\
    \liminf_{n \rarr \infty} a_n &= \sup\{ \inf\{ a_k, a_{k+1}, \dots \} \;|\; k \geq 1 \}.
  \end{align*}
\end{definition}

Then
\[
\liminf_n a_n = -\limsup_n(-a_n).
\]
If $\{a_n\}$ converges, then
\[
\lim_n a_n = \limsup_n a_n = \liminf_n a_n.
\]

Suppose $\{f_n\}$ is a sequence of extended-real function on a set $X$. Then $\sup_n f_n$ and $\limsup_n f_n$ are functions defined on $X$ by
\begin{align*}
  \left( \sup_n f_n \right)(x) &= \sup_n(f_n(x)), \\
  \left( \limsup_n f_n \right)(x) &= \limsup_n(f_n(x)).
\end{align*}
Similarly for $\inf_n f_n$ and $\liminf_n f_n$. If $f(x) = \lim_n f_n(x)$ is well-defined for all $x \in X$, then we call $f$ the \emph{pointwise limit} of the sequence $\{f_n\}$.

\begin{theorem}
  If $f_n \cn X \rarr [-\infty,\infty]$ is measurable for $n \geq 1$, then so are $\sup_n f_n$ and $\limsup_n f_n$.
\end{theorem}

\begin{corollary}
  \mbox{}
  \begin{enumerate}[(a)]
  \item The limit of every pointwise convergent sequence of complex measurable functions is measurable.
  \item If $f$ and $g$ are measurable (with range $[-\infty,\infty]$), then so are $\max\{f,g\}$ and $\min\{f,g\}$. In particular, this is true of the functions $f^+ = \max\{f,0\}$ and $f^- = -\min\{f,0\}$. These are respectively called the \emph{positive} and \emph{negative parts of $f$}. We have $f = f^+ - f^-$ and $|f| = f^+ + f^-$.
  \end{enumerate}
\end{corollary}

\begin{proposition}
  If $f = g-h$, $g \geq 0$, and $f \geq 0$, then $f^+ \leq g$ and $f^- \leq h$.
\end{proposition}

\begin{definition}
  A function $s$ on a measurable space $X$ whose range consists of only finitely many points in $[0,\infty)$ is called \emph{simple}.
\end{definition}
If $\alpha_1, \dots, \alpha_n$ are the distinct values assumed by $s$, and $A_i = f^{-1}(\alpha_i)$, then $s = \sum_{i=1}^n \alpha_i \chi_{A_i}$. The function $s$ is measurable if and only if all $A_i$ are measurable.

\begin{theorem}
  Let $f \cn X \rarr [0,\infty]$ be a measurable function. There exists simple measurable functions $s_n$ on $X$ such that:
  \begin{enumerate}[(a)]
  \item $0 \leq s_1 \leq s_2 \leq \cdots \leq f$;
  \item $s_n \rarr f$ pointwise.
  \end{enumerate}
\end{theorem}

\begin{remark}
  It can be shown using the construction in the proof of the theorem above, that if $f$ is bounded, then $s_n$ is a uniformly convergent sequence.
\end{remark}

\begin{definition}
  \mbox{}
  \begin{enumerate}[(a)]
  \item A \emph{(positive) measure} is a function $\mu$, defined on a $\sigma$-algebra $\Mc$, whose range is in $[0,\infty]$ and which is \emph{countably additive}. This means that is $\{A_i\}$ is a disjoint countable collection of elements of $\Mc$, then
    \[
    \mu\left( \bigcup_{i=1}^\infty A_i \right) = \sum_{i=1}^\infty \mu(A_i).
    \]
    To avoid trivialities, we shall also assume that $\mu(A) < \infty$ for at least one $A \in \Mc$.
  \item A \emph{measure space} is a measurable space which has a positive measure defined on the $\sigma$-algebra of its measurable sets.
  \item A \emph{complex measure} is a complex-valued countable additive function defined on a $\sigma$-algebra.
  \end{enumerate}
\end{definition}

\begin{theorem}
  For any positive measure $\mu$ on a $\sigma$-algebra $\Mc$, the following holds:
  \begin{enumerate}[(a)]
  \item $\mu(\emptyset) = 0$;
  \item if $A_1, \dots, A_n$ are pairwise disjoint elements of $\Mc$, then $\mu(A_1 \cup \cdots \cup A_n) = \mu(A_1) + \cdots + \mu(A_n)$;
  \item if $A, B \in \Mc$ and $A \subset B$, then $\mu(A) \leq \mu(B)$;
  \item if $A = \bigcup_{n=1}^\infty A_n$ for $A_1 \subset A_2 \subset \cdots$ where all $A_n$ are elements of $\Mc$, then $\mu(A_n) \rarr \mu(A)$ as $n \rarr \infty$;
  \item if $A = \bigcap_{n=1}^\infty A_n$ for $A_1 \supset A_2 \supset \cdots$ where all $A_n$ are elements of $\Mc$, and $\mu(A_1)$ is finite, then $\mu(A_n) \rarr \mu(A)$ as $n \rarr \infty$.
  \end{enumerate}
\end{theorem}

\begin{example}
  Let $X$ be any set.
  \begin{enumerate}[(a)]
  \item For any $E \subset X$ define $\mu(E) = \infty$ if $E$ is infinite, and $\mu(E) = |E|$ if $E$ is finite. This $\mu$ is called the \emph{counting measure on $X$}.
  \item For $x_0 \in X$, define $\mu(E) = 1$ is $x_0 \in E$, and $\mu(E) = 0$ otherwise. This $\mu$ is called the \emph{unit mass concentrated at $x_0 \in X$}.
  \end{enumerate}
\end{example}

To make sense of arithmetic on $[0,\infty]$, we define additionally $a + \infty = \infty + a = \infty$ if $0 \leq a \leq \infty$, and
\[
a \cdot \infty = \infty \cdot a =
\begin{cases}
  \infty & \textrm{if } 0 < a \leq \infty, \\
  0 & \textrm{if } a = 0.
\end{cases}
\]
This ensures the commutative, associative, and distributive laws hold in $[0,\infty]$ without any restriction. It is important to note that cancellation laws have to be treated with some care: $a + b = a + c$ implies $b=c$ only when $a < \infty$, and $ac = ac$ implies $b=c$ only when $0 < a < \infty$.

\begin{proposition}
  If $0 \leq a_1 \leq a_2 \leq \cdots$, $0 \leq b_1 \leq b_2 \leq \cdots$, $a_n \rarr a$, and $b_n \rarr b$, then $a_n b_n \rarr a b$.
\end{proposition}

\begin{corollary}
  Sums and products of measurable functions into $[0,\infty]$ are measurable.
\end{corollary}

\begin{definition}
  If $s \cn X \rarr [0,\infty)$ is a measurable simple function of the form $s = \sum_{i=1}^n \alpha_i \chi_{A_i}$, where $\alpha_1, \dots, \alpha_n$ are distinct values of $s$, and if $E \in \Mc$, we define
  \[
  \int_E s d\mu = \sum_{i=1}^n \alpha_i \mu(A_i \cap E).
  \]
  If $f \cn X \rarr [0, \infty]$ is measurable, and $E \in \Mc$, we define
  \[
  \int_E f d\mu = \sup \int_E s d\mu,
  \]
  where the supremum is taken over all simple measurable functions $s$ such that $0 \leq s \leq f$.
\end{definition}
This is called the \emph{Lebesgue integral of $f$ over $E$, with respect to the measure $\mu$}.

\begin{proposition}
  \mbox{}
  \begin{enumerate}[(a)]
  \item If $0 \leq f \leq g$, then $\int_E f d\mu \leq \int_E g d\mu$.
  \item If $A \subset B$ and $f \geq 0$, then $\int_A f d\mu \leq \int_B f d\mu$.
  \item If $f \geq 0$ and $c \in [0,\infty)$ is a constant, then $\int_E c f d\mu = c \int_E f d\mu$.
  \item If $f(x) = 0$ for all $x \in E$, then $\int_E f d\mu = 0$, even if $\mu(E) = \infty$.
  \item If $\mu(E) = 0$, then $\int_E f d\mu = 0$, even if $f(x) = \infty$ for all $x \in E$.
  \item If $f \geq 0$, then $\int_E f d\mu = \int_X \chi_E f d\mu$.
  \end{enumerate}
\end{proposition}

\begin{proposition}
  Let $s$ and $t$ be nonnegative measurable simple functions on $X$. For $E \in \Mc$, define $\phi(E) = \int_E s d\mu$. Then $\phi$ is a measure on $\Mc$. Also $\int_X (s+t) d\mu = \int_X s d\mu + \int_E t d\mu$.
\end{proposition}

\begin{theorem}[Lebesgue's Monotone Convergence Theorem]
  Let $\{f_n\}$ be a sequence of measurable functions on $X$, and suppose that
  \begin{enumerate}[(a)]
  \item $0 \leq f_1 \leq f_2 \leq \cdots$,
  \item $f_n \rarr f$ pointwise on $X$.
  \end{enumerate}
  Then $f$ is measurable, and
  \[
  \lim_n \int_X f_n d\mu = \int_X f d\mu.
  \]
\end{theorem}

\begin{theorem}
  If $f_n \cn X \rarr [0,\infty]$ is measurable for $n \geq 1$, and $f = \sum_{n=1}^\infty f_n$, then $\int_X f d\mu = \sum_{n=1}^\infty \int_X f_n d\mu$.
\end{theorem}

The following follows by employing the counting measure on $\Z^+$.

\begin{corollary}
  If $a_{i j} \geq 0$ for all $i,j \geq 0$, then
  \[
  \sum_{i=1}^\infty \sum_{j=1}^\infty a_{i j} = \sum_{j=1}^\infty \sum_{i=1}^\infty a_{i j}.
  \]
\end{corollary}

\begin{proposition}[Fatou's Lemma]
  If $f_n \cn X \rarr [0,\infty]$ is measurable for all $n \geq 1$, then
  \[
  \int_X \left( \liminf_n f_n \right) d\mu \leq \liminf_n \int_X f_n d\mu.
  \]
\end{proposition}

\begin{theorem}
  Suppose $f \cn X \rarr [0,\infty]$ is measurable, and $\phi(E) = \int_E f d\mu$ for $E \in \Mc$. Then $\phi$ is a measure on $\Mc$, and
  \[
  \int_X f d\phi = \int_X d f d\mu.
  \]
\end{theorem}

\begin{definition}
  Let $L^1(\mu)$ be the collection of all complex measurable functions $f$ on $X$ for which $\int_X |f| d\mu < \infty$. The elements of $L^1(\mu)$ are called \emph{Lebesgue measurable functions with respect to $\mu$}.
\end{definition}

\begin{definition}
  If $f = u + i v$ for $u$ and $v$ real measurable functions on $X$, and if $f \in L^1(\mu)$, then we define
  \[
  \int_E f d\mu = \int_E u^+ d\mu - \int_E u^- d\mu + i \int_E v^+ d\mu - i \int_E v^- d\mu
  \]
  for every $E \in \Mc$. In some circumstances it is desirable to extend this to the case when $f$ is a measurable function with range in $[-\infty,\infty]$. Then, we set $\int_E f d\mu = \int_E f^+ d\mu - \int_E f^- d\mu$, provided that at least one of the integrals on the right is finite.
\end{definition}

\begin{theorem}
  If $f, g \in L^1(\mu)$ and $\alpha,\beta \in \C$, then $\alpha f + \beta g \in L^1(\mu)$, and
  \[
  \int_X (\alpha f + \beta g) d\mu = \alpha \int_X f d\mu + \beta \int_X f d\mu.
  \]
\end{theorem}

\begin{theorem}
  If $f \in L^1(\mu)$, then $\left| \int_X f d\mu \right| \leq \int_X |f| d\mu$.
\end{theorem}

\begin{theorem}[Lebesgue's Dominated Convergence Theorem]
  Suppose $\{f_n\}$ is a sequence of complex measurable functions on $X$ such that $f = \lim_n f_n$ is defined pointwise on all $X$. If there is a function $g \in L^1(\mu)$ such that $|f_n| \leq g$ for all $n \geq 1$, then $f \in L^1(\mu)$,
  \[
  \lim_n \int_X |f_n - f| d\mu = 0,
  \]
  and
  \[
  \lim_n \int_X f_n d\mu = \int_X f d\mu.
  \]
\end{theorem}

\begin{definition}
  Let $P$ be a property which a point $x \in X$ may or may not have. If $\mu$ is a measure on a $\sigma$-algebra $\Mc$ and if $E \in \Mc$, the statement ``$P$ holds almost everywhere on $E$'' (abbreviated ``$P$ holds a.e.\ on $E$'') means that there exists an $N \in \Mc$ such that $\mu(N) = 0$, $N \subset E$, and $P$ holds at every point of $E \setminus N$. Note that the concept depends strongly on the used measure $\mu$.
\end{definition}

\begin{definition}
  If $f$ and $g$ are measurable functions and if $\mu(\{ x \;|\; f(x) \neq g(x) \}) = 0$, we say that $f = g$ a.e.\ on $X$, and write $f \sim g$. Note that $\sim$ is an equivalence relation.
\end{definition}

\begin{proposition}
  If $f \sim g$, then, for every $E \in \Mc$, we have $\int_E f d\mu = \int_E g d\mu$.
\end{proposition}

\begin{theorem}
  Let $(X, \Mc, \mu)$ be a measure space, and $\Mc^\ast$ be the collection of all $E \subset X$ for which there exist $A, B \in \Mc$ with $A \subset E \subset B$ and $\mu(B \setminus A) = 0$, and define $\mu(E) = \mu(A)$ in this situation. Then $\Mc^\ast$ is a $\sigma$-algebra, and $\mu$ is a measure on $\Mc^\ast$.
\end{theorem}

\begin{definition}
  The extended measure $\mu$ is called \emph{complete}, since all subsets of sets of measure $0$ are now measurable; the $\sigma$-algebra $\Mc^\ast$ is called the $\mu$-completion of $\Mc$.
\end{definition}

In light of the previous theorem and definition, we may only require functions to be well-defined and measurable on the complement of a zero measure set. This is sometimes a very useful trick.

\begin{theorem}
  Suppose $\{f_n\}$ is a sequence of complex measurable functions defined a.e.\ on $X$ such that
  \[
  \sum_{n=1}^\infty \int_X |f_n| d\mu < \infty.
  \]
  Then the series $f(x) = \sum_{n=1}^\infty f_n(x)$ converges for almost all $x$, $f \in L^1(\mu)$, and
  \[
  \int_X f d\mu = \sum_{n=1}^\infty \int_X f_n d\mu.
  \]
\end{theorem}

\begin{theorem}
  \mbox{}
  \begin{enumerate}[(a)]
  \item If $f \cn X \rarr [0,\infty]$ is measurable, $E \in \Mc$, and $\int_E f d\mu = 0$, then $f = 0$ a.e.\ on $E$.
  \item If $f \in L^1(\mu)$ and $\int_E f d\mu = 0$ for every $E \in \Mc$, then $f = 0$ a.e.\ on $X$.
  \item If $f \in L^1(\mu)$ and $\left| \int_X f d\mu \right| = \int_X |f| d\mu$, then there is a constant $\alpha$ such that $\alpha f = |f|$ a.e.\ on $X$.
  \end{enumerate}
\end{theorem}

\begin{theorem}
  Suppose $\mu(X) < \infty$, $f \in L^1(\mu)$, $S$ is a closed set in the complex plane, and the averages
  \[
  A_E(f) = \frac{1}{\mu(E)} \int_E f d\mu
  \]
  lie in $S$ for all $E \in \Mc$ with $\mu(E) > 0$. Then $f(x) \in S$ for almost all $x \in X$.
\end{theorem}

\begin{theorem}
  Let $\{E_k\}$ be a sequence of measurable sets in $X$, such that $\sum_{k=1}^\infty \mu(E_k) < \infty$. Then almost all $x \in X$ lie in at most finitely many of the sets $E_k$.
\end{theorem}

\subsection{Rudin, Chapter 2: Positive Borel Measures}

\begin{definition}
  Let $X$ be a topological space.
  \begin{enumerate}[(a)]
  \item A set $E \subset X$ is called \emph{closed} if its complement $E^c$ is open.
  \item The \emph{closure} $\overline{E}$ of $E \subset X$ is the smallest closed set in $X$ which contains $E$.
  \item A topological space is called \emph{compact} if every open cover has a finite subcover.
  \item A neighbourhood of a point $p \in X$ is an open set which contains $p$.
  \item $X$ is \emph{Hausdorff} if for every two points can be separated by opens.
  \item $X$ is \emph{locally compact} if every point of $X$ has a neighbourhood whose closure is compact (alternatively, for every $p \in X$, there is a neighbourhood $U$ of $p$, and compact $K \subset X$ satisfying $U \subset K$).
  \end{enumerate}
\end{definition}

\begin{theorem}[Heine-Borel Theorem]
  A subset of an Euclidean space is compact if and only it is closed and bounded.
\end{theorem}

\begin{proposition}
  Every metric space is Hausdorff.
\end{proposition}

\begin{theorem}
  A closed subset of a compact space is also compact.
\end{theorem}

\begin{corollary}
  If $A \subset B \subset X$ and $B$ has a compact closure, then so does $A$.
\end{corollary}

\begin{theorem}
  Suppose $X$ is Hausdorff, $K \subset X$ is compact, and $p \in K^c$. Then there exist open sets $U$ and $V$ such that $p \in U$, $K \subset V$, and $U \cap V = \emptyset$.
\end{theorem}

\begin{corollary}
  \mbox{}
  \begin{enumerate}[(a)]
  \item Compact subsets of Hausdorff spaces are closed.
  \item In a Hausdorff space, the intersections of a closed and a compact subset is compact.
  \end{enumerate}
\end{corollary}

\begin{theorem}
  If $\{K_\alpha\}$ is a collection of compact subsets of a Hausdorff space and their intersection is empty, then there exists a finite subcollection with empty intersection.
\end{theorem}

\begin{theorem}
  Suppose $X$ is a locally compact Hausdorff space, $U \subset X$, and $K \subset X$ is compact. Then there exists an open $V \subset X$ with compact closure which satisfies
  \[
  K \subset V \subset \overline{V} \subset U.
  \]
\end{theorem}

\begin{definition}
  Let $X$ be a topological space, and consider a function $f \cn X \rarr \R$.
  \begin{enumerate}[(i)]
  \item We call $f$ \emph{lower semicontinuous} if $f^{-1}((\alpha,\infty))$ is open for all $\alpha \in \R$.
  \item We call $f$ \emph{upper semicontinuous} if $f^{-1}((-\infty,\alpha))$ is open for all $\alpha \in \R$.
  \end{enumerate}
  An alternative definition could be made by replacing $\R$ with $[-\infty,\infty]$.
\end{definition}

\begin{proposition}
  Let $X$ be a topological space. A function $f \cn X \rarr \R$ is continuous if and only if it is both upper and lower semicontinuous.
\end{proposition}

\begin{example}
  \mbox{}
  \begin{enumerate}[(a)]
  \item Characteristic functions of open sets are lower semicontinuous.
  \item Characteristic functions of closed sets are upper semicontinuous.
  \item The supremum of any collection of lower semicontinuous function is lower semicontinuous.
  \item The infimum of any collection of upper semicontinuous function is upper semicontinuous.
  \end{enumerate}
\end{example}

\begin{definition}
  Let $X$ be a topological space, and $f \cn X \rarr \C$ a function. The \emph{support of $f$} is
  \[
  \Supp f = \overline{f^{-1}(\C \setminus \{0\})}.
  \]
  The collection of all continuous complex functions on $X$ with compact support is denoted $\Cc_c(X)$.
\end{definition}

Note that $\Cc_c(X)$ is a vector space over $\C$. We could have also replaced ``continuous'' above with ``measurable'' to yield another interesting object of study.

\begin{theorem}
  The image of a compact set under a continuous map is compact.
\end{theorem}

\begin{corollary}
  The image of any $f \in \Cc_c(X)$ is compact in $\C$.
\end{corollary}

\begin{definition}
  For any compact $K \subset X$, the notation
  \[
  K \prec f
  \]
  will mean that $f \in \Cc_c(X)$, $0 \leq f \leq 1$, and $f|_K = 1$. Similarly, for any open $U \subset X$, the notation
  \[
  f \prec U
  \]
  will mean that $f \in \Cc_c(X)$, $0 \leq f \leq 1$, and $\Supp f \subset U$. Then, $K \prec f \prec U$ will simultaneously mean $K \prec f$ and $f \prec U$.
\end{definition}

\begin{proposition}[Urysohn's Lemma]
  Let $X$ be a locally compact Hausdorff space, $U \subset X$ an open, and $K \subset X$ a compact. Then there exists $f \in \Cc_c(X)$ such that $K \prec f \prec U$.
\end{proposition}

\begin{theorem}
  Compact Hausdorff spaces admit \emph{partitions of unity}. That is, for any compact space $K$, and a (finite) open cover $U_1, \dots, U_n$, there exist $f_i \in \Cc(X)$ satisfying $f_i \prec U_i$, and $\sum_i f_i = 1$. The statement also holds true for compact subspaces of locally compact Hausdorff spaces.
\end{theorem}

\begin{theorem}[Riesz Representation Theorem]
  Let $X$ be a locally compact Hausdorff space, and $\Lambda \cn \Cc_c(X) \rarr \C$ be a \emph{positive linear functional}, that is, $\Lambda$ is a morphism of vector spaces over $\C$, and for every $f \geq 0$, we have $\Lambda(f) \geq 0$. Then there exists a $\sigma$-algebra $\Mc$ on $X$ containing all Borel sets of $X$, and a unique positive measure $\mu$ on $\Mc$ which represents $\Lambda$ in the sense that:
  \begin{enumerate}[(a)]
  \item $\Lambda(f) = \int_X f d\mu$ for every $f \in \Cc_c(X)$;
  \item $\mu(K) < \infty$ for every compact $K \subset X$;
  \item for every $E \in \Mc$, we have
    \[
    \mu(E) = \inf\{ \mu(U) \;|\; E \subset U, U \textrm{ is open} \};
    \]
  \item the relation
    \[
    \mu(E) = \sup\{ \mu(K) \;|\; K \subset E, K \textrm{ is compact} \}
    \]
    holds for every open $E \subset X$, and for every $E \in \Mc$ with $\mu(E) < \infty$;
  \item $\Mc$ is complete, i.e., if $E \in \Mc$, $A \subset E$, and $\mu(E) = 0$, then $A \in \Mc$.
  \end{enumerate}
\end{theorem}

\begin{definition}
  Let $X$ be a locally compact Hausdorff space $X$. A measure $\mu$ defined on the $\sigma$-algebra of Borel sets is called a \emph{Borel measure}.
\end{definition}

\begin{definition}
  Assume $\mu$ is a positive Borel measure on $X$. We call it \emph{outer (resp.\ inner) regular} if every Borel set $E \subset X$ has property (c) (resp.\ property (d)) from the Riesz Representation Theorem. If a measure is both inner and outer regular, then we call it \emph{regular}.
\end{definition}

\begin{definition}
  A topological space is called \emph{$\sigma$-compact} if it is the union of finitely many compacts, or alternatively, every open cover has a countable subcover. We call a topological space \emph{$\sigma$-locally compact} if it is both $\sigma$- and locally compact.

  A set $E$ in a measure space is said to have \emph{$\sigma$-finite measure} if $R$ is a countable union of measurable sets $E_i$ with $\mu(E_i) < \infty$.
\end{definition}

\begin{theorem}
  Let $X$ be a $\sigma$-locally compact Hausdorff space. If $\Mc$ and $\mu$ as in the Riesz Representation Theorem, then they have the following properties:
  \begin{enumerate}[(a)]
  \item for any $E \in \Mc$ and $\epsilon > 0$, there is a closed $F \subset X$ and an open $U \subset X$ satisfying $F \subset E \subset U$ and $\mu(U \setminus F) < \epsilon$;
  \item $\mu$ is a regular Borel measure on $X$;
  \item for any $E \in \Mc$, there are sets $A$ and $B$ such that $A$ is a countable union of closed sets, $B$ is a countable intersection of open sets, $A \subset E \subset B$, and $\mu(B \setminus A) = 0$.
  \end{enumerate}
\end{theorem}

\begin{theorem}
  Let $X$ be a locally compact Hausdorff space in which every open is $\sigma$-compact. If $\mu$ is a positive Borel measure on $X$ such that $\mu(K) < \infty$ for all compact $K \subset X$, then $\mu$ is regular.
\end{theorem}

\begin{definition}
  Let $W = \prod_{i=1}^k I_i \subset \R^k$ where $I_i \subset \R$ are intervals (open, closed, or any combination thereof) with endpoints $\alpha_i$ and $\beta_i$. We call such a set a \emph{$k$-cell}. The \emph{volume of $W$} is defined to be $\vol(W) = \prod_{i=1}^k (\beta_i - \alpha_i)$.

  If $a = (a_1, \dots, a_k) \in \R^k$ and $\delta > 0$, we call the set
  \[
  Q(a, \delta) = \prod_{i=1}^k [a_i, a_i+\delta)
  \]
  a \emph{$\delta$-box with corner at $a$}.

  For an integer $n \geq 1$, let $\Omega_n$ denote the set of $2^{-n}$-boxes with corners at points of $P_n = 2^{-n} \Z^k$.
\end{definition}

\begin{proposition}
  \mbox{}
  \begin{enumerate}[(a)]
  \item For any fixed $n \geq 1$, the set $\Omega_n$ is a partition of $\R^k$.
  \item If $Q \in \Omega_n$, $Q' \in \Omega_r$, and $r < n$, then either $Q \subset Q'$ or $Q \cap Q' = \emptyset$.
  \item If $Q \in \Omega_r$, then $\vol(Q) = 2^{-r k}$; and if $r < n$, then $|P_n \cap Q| = 2^{(n-r)k}$.
  \item Every nonempty open in $\R^k$ is a countable union of disjoint boxes in $\bigcup_{n \geq 1} \Omega_n$.
  \end{enumerate}
\end{proposition}

\begin{theorem}
  There exists a positive complete measure $m$ defined on a $\sigma$-algebra $\Mc$ on $\R^k$ with the following properties:
  \begin{enumerate}[(a)]
  \item $m(W) = \vol(W)$ for every $k$-cell $W$;
  \item $\Mc$ contains all Borel sets of $\R^k$; more precisely, $E \in \Mc$ if and only if there exists $A,B \subset \R^k$ such that $A \subset E \subset B$, $A$ is a countable union of closed sets, $B$ is a countable union of opens, and $m(B \setminus A) = 0$; also, $m$ is regular;
  \item $m$ is translation-invariant, i.e., for every $E \in \Mc$ and $x \in \R^k$, we have $m(E+x) = m(E)$;
  \item if $\mu$ is any positive translation-invariant Borel measure on $\R^k$ such that $\mu(K) < \infty$ for every compact $K \subset \R^k$, then there is a constant $c > 0$ such that $\mu(E) = c \, m(E)$ for all Borel sets $E \subset \R^k$;
  \item for every linear transformation $T \cn \R^k \rarr \R^k$, we have $m(T(E)) = \det T \, m(E)$ for all $E \in \Mc$; in particular, $m(T(E)) = m(E)$ when $T$ is a rotation.
  \end{enumerate}
\end{theorem}

\begin{definition}
  The members of $\Mc$ are called \emph{Lebesgue measurable sets in $\R^k$}; $m$ is the \emph{Lebesgue measure on $\R^k$}. When clarity requires it, we shall write $m_k$ in place of $m$.

  For any $E \subset \R^k$, the measure $m$ induces a restricted measure $m_E$ on $E$ (and also a $\sigma$-algebra $\Mc_E$). We will write $L^1(E) = L^1(m_E)$, and, in particular, $L^1(\R^k) = L^1(m)$.

  If $k = 1$, $I$ is an interval (open, closed, or any combination thereof) with endpoints $a$ and $b$, and $f \in L^1(I)$, it is customary to denote
  \[
  \int_a^b f(x) d x = \int_I f d m.
  \]
\end{definition}

\begin{proposition}
  For any continuous complex function $f$ on $[a,b]$, the Riemann integral of $f$ and the Lebesgue integral of $f$ over $[a,b]$ agree.
\end{proposition}

\begin{remark}
  A natural question to ask is whether all elements of $\Mc$ are Borel. Another such is whether all subsets of $\R^k$ are measurable. The answer of both of these is negative (for the latter one see the following result).
\end{remark}

\begin{theorem}
  If $A \subset \R$ and every subset of $A$ is Lebesgue measurable, then $m(A) = 0$.
\end{theorem}

\begin{corollary}
  Every set of positive measure has a nonmeasurable subset.
\end{corollary}

\begin{theorem}[Lusin's Theorem]
  Suppose $f$ is a complex measurable function on $X$, $\mu(A) < \infty$, $f|_{X \setminus A} = 0$, and $\epsilon > 0$. Then there exists a $f \in \Cc_c(X)$ such that
  \[
  \mu(\{ x \;|\; f(x) \neq g(x)\}) < \epsilon.
  \]
  Furthermore, we may arrange so that
  \[
  \sup_{x \in X} |g(x)| \leq \sup_{x \in X} |f(x)|.
  \]
\end{theorem}

\begin{corollary}
  Assume the hypotheses of Lusin's Theorem are satisfied and that $|f| \leq 1$. Then there is a sequence $\{g_n\}$ such that $g_n \in \Cc_c(X)$, $|g_n| \leq 1$, and
  \[
  f(x) = \lim_n g_n(x) \qquad a.e.
  \]
\end{corollary}

\begin{theorem}[Vitali-Carath\'eodory Theorem]
  Suppose $f \in L^1(\mu)$, $f$ is real-valued, and $\epsilon > 0$. Then there exist functions $u, v \cn X \rarr \R$ such that $u \leq f \leq v$, $u$ is upper semicontinuous and bounded above, $v$ is lower semicontinuous and bounded below, and $\int_X (v - u) d\mu < \epsilon$.
\end{theorem}

\subsection{Rudin, Chapter 3: $L^p$-spaces}

\begin{definition}
  Consider an interval $(a,b) \subset [-\infty,\infty]$. A function $\phi \cn (a,b) \rarr \R$ is called \emph{convex} if the inequality
  \[
  \phi((1-\lambda) x + \lambda y) \leq (1-\lambda) \phi(x) + \lambda \phi(y)
  \]
  holds for all $a < x,y < b$, and $0 \leq \lambda \leq 1$.

  Graphically, the above means for any $x,y \in (a,b)$ the graph of $\phi$ between these two points is below the straight line connecting $(x,\phi(x))$ and $(y,\phi(y))$. Alternatively, convexity is equivalent to the inequality
  \[
  \frac{\phi(t)-\phi(s)}{t-s} \leq \frac{\phi(u)-\phi(t)}{u-t}
  \]
  for all $a < s < t < u < b$.
\end{definition}

The following is a consequence of the mean value theorem for differentiation and the alternative formulation of convexity above.

\begin{proposition}
  A differentiable $\phi \cn (a,b) \rarr \R$ is convex if and only if $\phi'$ is monotonically increasing.
\end{proposition}

\begin{theorem}
  If $\phi \cn (a,b) \rarr \R$ is convex, then it is also continuous.
\end{theorem}

\begin{remark}
  The statement above would not hold had the domain not been an open interval.
\end{remark}

\begin{theorem}[Jensen's Inequality]
  Let $\mu$ be a positive measure on a $\sigma$-algebra $\Mc$ on a set $X$ with $\mu(X) = 1$. If $f \in L^1(\mu)$, $a < f < b$, and $\phi$ is convex on $(a,b)$, then
  \[
  \phi\left( \int_X f d\mu \right) \leq \int_X (\phi \circ f) d\mu.
  \]
\end{theorem}

\begin{remark}
  The cases $a = -\infty$ and $b = \infty$ are not excluded. In either of these, it may occur that $\phi \circ f$ is not an element of $L^1(\mu)$. Then its integral exists in the extended sense and equals $+\infty$, in other words, the equality is vacuous.
\end{remark}

\begin{example}
  \mbox{}
  \begin{enumerate}[(a)]
  \item Taking $\phi(x) = e^x$, we obtain
    \[
    \exp\left( \int_X f d\mu \right) \leq \int_X e^f d\mu.
    \]
  \item Taking $X = \{p_1, \dots, p_n\}$, $\mu(\{p_i\}) = 1/n$ for all $1 \leq i \leq n$, and $f(p_i) = x_i$, we obtain
    \[
    \exp\frac{x_1 + \cdots + x_n}{n} \leq \frac{e^{x_1} + \cdots + e^{x_n}}{n}.
    \]
    Taking $y_i = e^{x_i}$, this turns into the AM-GM inequality
    \[
    (y_1 \cdots y_n)^{1/n} \leq \frac{y_1 + \cdots + y_n}{n}.
    \]
  \item If we modify the previous example only by setting $\mu(\{p_i\}) = \alpha_i > 0$ with $\sum_{i=1}^n \alpha_i = 1$, we obtain
    \[
    y_1^{\alpha_1} \cdots y_n^{\alpha_n} \leq \alpha_1 y_1 + \cdots + \alpha_n y_n.
    \]
  \end{enumerate}
\end{example}

\begin{definition}
  If two positive real numbers $p$ and $q$ satisfy $p + q = p q$, or alternatively $1/p + 1/q = 1$, we say they are \emph{conjugate exponents}.
\end{definition}

\begin{remark}
  Note that $1/p + 1/q = 1$ implies $1 < p,q < \infty$.
\end{remark}

\begin{theorem}
  Let $p$ and $q$ be conjugate exponents, $X$ a measure space with measure $\mu$, and $f,g \cn X \rarr [0,\infty]$ measurable functions. Then
  \[
  \tag{H\"older's inequality}
  \int_X f g d\mu \leq \left( \int_X f d\mu \right)^{1/p} \left( \int_X g d\mu \right)^{1/q},
  \]
  and
  \[
  \tag{Minkowski's inequality}
  \left( \int_X (f+g)^p d\mu \right)^{1/p} \leq \left( \int_X f^p d\mu \right)^{1/p} + \left( \int_X g^p d\mu \right)^{1/p}.
  \]
\end{theorem}

\begin{remark}
  When $p = q = 2$, H\"older's inequality becomes the well-known (Cauchy-)Schwartz inequality.
\end{remark}

\begin{remark}
  Assuming $\int_X f^p d\mu < \infty$ and $\int_X g^q d\mu < \infty$, equality holds in H\"older's inequality if and only if there exists constants $\alpha$ and $\beta$, not both $0$, such that $\alpha f^p = \beta g^q$ a.e. An analogous statement could be made regarding Minkowski's inequality.
\end{remark}

\begin{definition}
  Let $X$ be a measure space with a positive measure $\mu$. For any $0 < p < \infty$ and $f \cn X \rarr \C$ measurable, define its $L^p$-norm
  \[
  \|f\|_p = \left( \int_X |f|^p d\mu \right)^{1/p}.
  \]
  Let $L^p(\mu)$ consist of all such functions with finite $L^p$-norm.

  As before, we write $L^p(\R^k) = L^p(\mu)$ if $\mu$ is the Lebesgue measure on $\R^k$, and similarly for measurable subsets of $\R^k$.

  If $\mu$ is the counting measure on a set $A$, then we denote $\ell^p(A) = L^p(A)$. If $A$ is countably infinite, we may even write $\ell^p$. In this case, we may imagine the elements of $\ell^p$ as complex sequences $x = \{x_n\}$ satisfying
  \[
  \|x\|_p = \left( \sum_{n=1}^\infty |x_n|^p \right)^{1/p} < \infty.
  \]
\end{definition}

\begin{definition}
  Suppose $f \cn X \rarr [0,\infty]$ is measurable. Let $S$ be the set of all $\alpha \in \R$ satisfying
  \[
  \mu(f^{-1}((\alpha,\infty])) = 0.
  \]
  If $S = \emptyset$, then put $\beta = 0$. If $S \neq \emptyset$, then put $\beta = \inf S$. Since
  \[
  g^{-1}((\beta,\infty]) = \bigcup_{n=1}^n g^{-1}\left(\left(\beta+\frac{1}{n}, \infty \right]\right),
  \]
  and since the union of a countable collection of measure $0$ sets has measure $0$, it follows that $\beta \in S$. We call $\beta$ the \emph{essential supremum} of $f$.

  For any measurable $f \cn X \rarr \C$, we define its $L^\infty$-norm $\|f\|_\infty$ to be the essential supremum of $|f|$. Let $L^\infty(\mu)$ consist of all such functions with finite $L^\infty$-norm. We call these functions \emph{essentially bounded on $X$}.
\end{definition}

\begin{remark}
  It follows from the definition that $|f| \leq \lambda$ a.e.\ if and only if $\|f\|_\infty \leq \lambda$.
\end{remark}

\begin{definition}
  As before, we will write $L^\infty(\R^k) = L^\infty(\mu)$ if $\mu$ is the Lebesgue measure on $\R^k$. Similarly, we write $\ell^\infty(A) = L^\infty(\mu)$ if $\mu$ is the counting measure on a set $A$. We may write $\ell^\infty$ alone if $A$ is countable.
\end{definition}

\begin{theorem}
  Let $p$ and $q$ be conjugate exponents. If $f \in L^p(\mu)$, $g \in L^q(\mu)$, then
  \[
  \|f g\|_1 \leq \|f\|_p \|g\|_q,
  \]
  hence $f g \in L^1(\mu)$.
\end{theorem}

\begin{theorem}
  Let $1 \leq p \leq \infty$. If $f, g \in L^p(\mu)$, then
  \[
  \|f+g\|_p \leq \|f\|_p + \|g\|_p,
  \]
  hence $f+g \in L^p(\mu)$.
\end{theorem}

\begin{remark}
  If $f \in L^p(\mu)$, and $\alpha \in \C$, then $\|\alpha f\|_p = |\alpha| \, \|f\|_p$. This remark combined with the previous result implies  that $L^p(\mu)$ is a vector space over $\C$.
\end{remark}

\begin{remark}
  Let $d \cn L^p(\mu) \x L^p(\mu) \rarr [0,\infty)$ be given by $d(f,g) = \|f-g\|_p$. It is not hard to check that $d(f,g) = 0$ if and only if $f \sim g$, that is, $f = g$ a.e. In other words, $d$ descends to a metric on $L^p(\mu)/\!\!\sim$. It can also be checked that $\sim$ respects linearity, hence $L^p(\mu)/\!\!\sim$ is a vector space over $\C$.
\end{remark}

\begin{remark}
  To avoid cumbersome details, we may sometimes treat $L^p(\mu)$ as a metric space and actually refer to $L^p(\mu)/\!\!\sim$. In this case the elements of $L^p(\mu)$ are not functions, but equivalence classes of functions.
\end{remark}

\begin{remark}
  It is elementary though essential to define \emph{convergence} and being \emph{Cauchy} in $L^p(\mu)$.
\end{remark}

\begin{theorem}
  For every $1 \leq p \leq \infty$, and every positive measure $\mu$, the metric space $L^p(\mu)$ is complete.
\end{theorem}

\begin{theorem}
  If $1 \leq p \leq \infty$ and $\{f_n\}$ is a Cauchy sequence in $L^p(\mu)$ with limit $f$, then $\{f_n\}$ has a subsequence which converges pointwise a.e.\ to $f$.
\end{theorem}

\begin{theorem}
  Let $S$ be the class of all complex, measurable, simple functions on $X$ such that $\mu(s^{-1}(\C \setminus \{0\})) < \infty$. If $1 \leq p < \infty$, then $S$ is dense in $L^p(\mu)$.
\end{theorem}

\begin{theorem}
  For $1 \leq p < \infty$, $\Cc_c(X)$ is dense in $L^p(\mu)$.
\end{theorem}

\begin{remark}
  Fix $k \geq 1$, and consider the spaces $\Cc_c(\R^k) \subset L^p(\R^k)$. For every $1 \leq p \leq \infty$, the $L^p$-norm induces a genuine metric on $\Cc_c(\R^k)$ (unlike in $L^p(\R^k)$, we do not need equivalence classes). If $1 \leq p < \infty$, then $\Cc_c(\R^k)$ is dense in $L_p(\R^k)$ and the latter is complete, hence we may treat $L^p(\R^k)$ as the completion of $\Cc_c(\R^k)$ with respect to the metric induced by the $L^p$-norm. In the case $p = \infty$, the  completion of $\Cc_c(\R^k)$ is not $L^p(\R^k)$ but $\Cc_0(\R^k)$, the space of all continuous functions on $\R^k$ which ``vanish at infinity'' (see below). It is also interesting to note that for $f \in \Cc^c(\R^k)$, we have
  \[
  \|f\|_\infty = \sup_{x \in \R^k} |f(x)|.
  \]
\end{remark}

\begin{definition}
  Let $X$ be a locally compact Hausdorff space. A continuous function $f \cn X \rarr \C$ is said to \emph{vanish at infinity} if for every $\epsilon > 0$ there exists a compact $K \subset X$ such that $|f|_{X \setminus K}| < \epsilon$. The set of all functions is denoted $\Cc_0(X)$.
\end{definition}

\begin{remark}
  It is clear that $\Cc_c(X) \subset \Cc_0(X)$, and equality is attained if $X$ is compact. In that case, we may write $\Cc(X)$ for either of them.
\end{remark}

\begin{theorem}
  Let $X$ be a locally compact Hausdorff space. The completion of $\Cc_c(X)$ with respect to the supremum norm
  \[
  \|f\| = \sup_{x \in X} |f(x)|
  \]
  is $\Cc_0(X)$.
\end{theorem}

\subsection{Rudin, Chapter 4: Elementary Hilbert Space Theory}

\begin{definition}
  An \emph{inner product} for a complex vector space $H$ is a map $(-,-) \cn H \x H \rarr \C$ which satisfies the following properties:
  \begin{enumerate}[(a)]
  \item $(y,x) = \overline{(x,y)}$ for all $x,y \in H$;
  \item $(x+y,z) = (x,z) + (y,z)$ for all $x,y,z \in H$;
  \item $(\alpha x,y) = \alpha(x,y)$ for all $\alpha \in \C$, $x,y \in H$;
  \item $(x,x) \geq 0$ for all $x \in H$;
  \item $(x,x) = 0$ only if $x = 0$.
  \end{enumerate}
  A complex vector space endowed with an inner product is called an \emph{inner product space}.
\end{definition}

\begin{definition}
  The \emph{norm} of $x \in H$ is $\|x\| = \sqrt{(x,x)}$, where $\sqrt{-}$ denotes the non-negative square root.
\end{definition}

The following are immediate consequences of the given properties:
\begin{enumerate}[(a)]
\item $(0,x) = 0$ for all $x \in H$;
\item for every $y \in H$, the map $H \rarr \C$ given by $x \mapsto (x,y)$ is linear;
\item $(x,\alpha y) = \overline{\alpha}(x,y)$ for all $\alpha \in \C$, $x,y \in H$;
\item $\|x\|^2 = (x,x)$ for all $x \in H$.
\end{enumerate}

\begin{proposition}[(Cauchy-)Schwartz Inequality]
  For all $x,y \in H$, we have
  \[
  |(x,y)| \leq \|x\| \, \|y\|.
  \]
\end{proposition}

\begin{proposition}[Triangle Inequality]
  For all $x,y \in H$, we have
  \[
  \|x+y\| \leq \|x\| + \|y\|.
  \]
\end{proposition}

\begin{remark}
  Any inner product space $H$ possesses a natural metric structure in which the distance between $x,y \in H$ is given by $\|x-y\|$. This can be verified using the properties and inequalities listed above.
\end{remark}

\begin{definition}
  A \emph{Hilbert space} is an inner product space which is complete in its natural metric structure.
\end{definition}

Let $H$ denote a Hilbert space for the remainder of this chapter.

\begin{example}
  \mbox{}
  \begin{enumerate}[(a)]
  \item The vector space $\C^n$ endowed with $(x,y) = \sum_{i=1}^n x_i \overline{y_i}$ is a Hilbert space.
  \item If $\mu$ is any positive measure on a set $X$, then $L^2(\mu)$ is a Hilbert space with inner product
    \[
    (f,g) = \int_X f \overline{g} d\mu.
    \]
    Note that in this case the induced norm agrees with the $L^2$-norm.
  \item The vector space of all continuous functions $[0,1] \rarr \C$ is an inner product space when endowed with
    \[
    (f,g) = \int_0^1 f(t) \overline{g(t)} d t.
    \]
    This however is not a Hilbert space since completeness fails to hold.
  \end{enumerate}
\end{example}

\begin{theorem}
  For any fixed $y \in H$, the maps $H \rarr \C$, $H \rarr \C$, and $H \rarr \R$ respectively given by
  \[
  x \mapsto (x,y), \qquad
  x \mapsto (y,x), \qquad
  x \mapsto \|x\|
  \]
  are all continuous.
\end{theorem}

\begin{definition}
  A \emph{closed subspace} of a Hilbert space is a subspace closed in the natural metric structure.
\end{definition}

\begin{remark}
  If $M \subset H$ is a subspace, then so is its closure $\overline{M}$.
\end{remark}

\begin{definition}
  Let $V$ be a vector space. A subset $E \subset V$ is called \emph{convex} if the straight segment joining any two points of $E$ is also contained in $E$. Formally this means that if $x,y \in E$ and $0 < t < 1$, then $(1-t) x + t y \in E$.
\end{definition}

\begin{remark}
  The following properties follow immediately from the definition of convexity:
  \begin{enumerate}[(a)]
  \item every subspace of a vector space is convex;
  \item any translate of a convex set is also convex.
  \end{enumerate}
\end{remark}

\begin{definition}
  Let $V$ be an inner product space. We call two vectors $x,y \in V$ \emph{orthogonal}, and write $x \perp y$, if $(x,y) = 0$. For $x \in V$, let $x^\perp$ denote the set all vectors in $V$ perpendicular to it. Similarly, for any subspace $M \subset V$, let $M^\perp$ denote the set of all vectors in $V$ perpendicular to everything in $M$.
\end{definition}

\begin{remark}
  \mbox{}
  \begin{enumerate}[(a)]
  \item The relation $\perp$ on $V$ is symmetric.
  \item Both $x^\perp$ and $M^\perp$ are subspaces.
  \item For any $x \in H$, the space $x^\perp \subset H$ is closed. Similarly, so is $M^\perp = \bigcap_{x \in M} x^\perp$.
  \end{enumerate}
\end{remark}

\begin{theorem}
  Every nonempty, closed, convex set in a Hilbert space contains a unique element of smallest norm.
\end{theorem}

\begin{theorem}
  Let $M$ be a closed subspace of a Hilbert space $H$.
  \begin{enumerate}[(a)]
  \item Every $x \in H$ has a unique decomposition $x = P(x) + Q(x)$, where $P(x) \in M$ and $Q(x) \in M^\perp$.
  \item $P(x)$ and $Q(x)$ are the nearest points to $x$ in $M$ and $M^\perp$ respectively.
  \item The maps $P \cn H \rarr M$ and $Q \cn H \rarr M^\perp$ are both linear.
  \item Every $x \in H$ satisfies $\|x\|^2 = \|P(x)\|^2 + \|Q(x)\|^2$.
  \end{enumerate}
  The maps $P$ and $Q$ are called the \emph{orthogonal projections} of $H$ onto $M$ and $M^\perp$.
\end{theorem}

\begin{corollary}
  If $M \neq H$, then there exists $y \in H \setminus \{0\}$ such that $y \perp M$.
\end{corollary}

\begin{theorem}
  If $L \cn H \rarr \C$ is continuous and linear, then there exists a unique $y \in H$ such that $L(x) = (x,y)$ for all $x \in H$.
\end{theorem}

\begin{definition}
  Let $V$ be a vector space. For any subset $S \subset V$, let $\langle S \rangle$ denote the set of finite linear combinations of elements of $S$, also called its \emph{span}.
\end{definition}

\begin{definition}
  A collection of vectors $\{ u_\alpha \;|\; \alpha \in A \}$ in a Hilbert space $H$ is called \emph{orthonormal} if $(u_\alpha,u_\beta) = \delta_{\alpha\beta}$ for all $\alpha,\beta \in A$. To each $x \in H$, we associate a function $\what{x} \cn H \rarr \C$ given by $\alpha \mapsto (x,u_\alpha)$. The numbers $\what{x}(\alpha)$ are called the \emph{Fourier coefficients} of $x$ relative to $\{u_\alpha\}$.
\end{definition}

\begin{theorem}
  Suppose $\{ u_\alpha \;|\; \alpha \in A \}$ is an orthonormal set in a Hilbert space $H$ and $F \subset A$ is a finite subset. Let $M_F$ be the span of $\{ u_\alpha \;|\; \alpha \in F \}$.
  \begin{enumerate}[(a)]
  \item For any $\phi \cn A \rarr \C$ satisfying $\phi|_{A \setminus F} = 0$, the vector
    \[
    y = \sum_{\alpha \in F} \phi(\alpha) u_\alpha
    \]
    satisfies $\what{y}(\alpha) = \phi(\alpha)$ for all $\alpha \in A$, and
    \[
    \|y\|^2 = \sum_{\alpha \in F} |\phi(\alpha)|^2.
    \]
  \item Define $s_F \cn H \rarr H$ by
    \[
    s_F(x) = \sum_{\alpha \in F} \what{x}(\alpha) u_\alpha.
    \]
    Then for all $x \in H$, we have
    \[
    \| x - s_F(x) \| < \| x - s \|
    \]
    for all $s \ M_F$, except for $s = s_F(x)$, and
    \[
    \sum_{\alpha \in F} |\what{x}(\alpha)|^2 \leq \|x\|^2.
    \]
  \end{enumerate}
\end{theorem}


Before delving into the next result, let us clarify the meaning of $\sum_{\alpha \in A} \phi(\alpha)$ for infinite $A$. For this, we assume $0 \leq \phi \leq \infty$.  Then $\sum_{\alpha \in A} \phi(\alpha)$ will denote the supremum of finite sums $\phi(\alpha_1) + \cdots + \phi(\alpha_n)$, where $\alpha_1, \dots, \alpha_n \in A$ are distinct. Note that this agrees with $\int_A \phi d\mu$ if $\mu$ is the counting measure on $A$. We can then refer to $\ell^2(A)$ which is a Hilbert space with inner product
\[
(\phi,\psi) = \sum_{\alpha \in A} \phi(\alpha) \overline{\psi(\alpha)}.
\]
We know that the functions $A \rarr \C$ which vanish at all but finitely many points form a dense subset of $\ell^2(A)$. Furthermore, for all $\phi \in \ell^2(A)$, the set $\phi^{-1}(\C \setminus \{0\})$ is at most countable.

\begin{lemma}
  Assume the following:
  \begin{enumerate}[(a)]
  \item $X$ and $Y$ are metric spaces, and $X$ is complete;
  \item $f \cn X \rarr Y$ is continuous;
  \item $X$ has a dense subset $X_0$ on which $f$ is an isometry;
  \item $f(X_0)$ is dense in $Y$.
  \end{enumerate}
  Then $f \cn X \rarr Y$ is an isometry, and in particular, a bijection.
\end{lemma}

\begin{theorem}[Riesz-Fischer Theorem]
  Let $\{ u_\alpha \;|\; \alpha \in A \}$ be an orthonormal set in the Hilbert space $H$, and $P$ the span of $\{u_\alpha\}$. The inequality
  \[
  \sum_{\alpha \in A} |\what{x}(\alpha)|^2 \leq \|x\|^2
  \]
  holds for all $x \in H$, and $x \mapsto \what{x}$ is a continuous map $H \rarr \ell^2(A)$ whose restriction to $\overline{P}$ is an isometry.
\end{theorem}

\begin{theorem}
  Let $\{ u_\alpha \;|\; \alpha \in A \}$ be an orthonormal set in the Hilbert space $H$, and $P$ the span of $\{u_\alpha\}$. The following are equivalent:
  \begin{enumerate}[(a)]
  \item $\{u_\alpha\}$ is a maximal orthonormal set in $H$;
  \item the span of $\{u_\alpha\}$ is dense in $H$;
  \item the equality
    \[
    \sum_{\alpha \in A} |\what{x}(\alpha)|^2 = \|x\|^2
    \]
    holds for all $x \in H$;
  \item the equality
    \[
    \tag{Parseval's identity}
    \sum_{\alpha \in A} \what{x}(\alpha) \overline{\what{y}(\alpha)} = (x,y)
    \]
    holds for all $x,y \in H$.
  \end{enumerate}
\end{theorem}

\begin{definition}
  Maximal orthonormal sets are called \emph{complete orthonormal set} or \emph{orthonormal base}.
\end{definition}

\begin{remark}
  Note that an orthonormal base in a Hilbert space need not be a basis for the underlying vector space. Actually, it will only be one in the finite-dimensional case.
\end{remark}

\begin{corollary}
  If $\{ u_\alpha \;|\; \alpha \in A \}$ is a maximal orthogonal set in a Hilbert space $H$, then the map $H \rarr \ell^2(A)$ given by $x \mapsto \what{x}$ is an isomorphism of Hilbert spaces.
\end{corollary}

\begin{theorem}[Hausdorff Maximality]
  Every nonempty partially ordered set contains a maximal totally ordered subset.  
\end{theorem}

\begin{theorem}
  Every orthonormal set $B$ in a Hilbert space $H$ is contained in a maximal orthonormal set in $H$.
\end{theorem}

\begin{corollary}
  Every Hilbert space admits a maximal orthonormal set.
\end{corollary}

In what follows, we will identify maps $S^1 \rarr X$ with periodic maps $\R \rarr X$ with period $2\pi$. More concretely, we shall sometimes write $f(t)$ rather than $f(e^{i t})$, even if we think of $f$ as defined on $S^1$.

Consider the class of all Lebesgue measurable, $2\pi$-periodic functions on $\R$ for which the norm
\[
\|f\|_p = \left( \frac{1}{2\pi} \int_{-\pi}^\pi |f(t)|^p d t \right)^{1/p}
\]
is finite. Under the identification above, the set of these functions is $L^p(S^1)$. Note the normalizing factor $1/2\pi$ inserted for the sake of convenience. Similarly for $L^\infty(S^1)$ and $\Cc(S^1)$.

\begin{definition}
  A \emph{trigonometric polynomial} is a finite sum of the form
  \[
  f(t) = a_0 + \sum_{n=1}^N (a_n \cos n t + b_n \sin n t)
  \]
  for defined for $t \in \R$ where $a_i, b_j \in \C$. These coincide with finite sums of the form
  \[
  f(t)= \sum_{n=-N}^N c_n e^{i n t}
  \]
  where $c_n \in \C$ which are often more convenient.
\end{definition}

It is clear that all trigonometric polynomials have period $2\pi$. For $n \in \Z$, we set $u_n(t) = e^{i n t}$. If we define the inner product in $L^2(S^1)$ as $(f,g) = 1/2\pi \int_{-\pi}^\pi f(t)\overline{g(t)} d t$, then $(u_m,u_n) = \delta_{m n}$. The orthonormal set $\{ u_n \;|\; n \in \Z \}$ in $L^2(S^1)$ is called the \emph{trigonometric system}.

\begin{theorem}
  For any $f \in \Cc(S^1)$ and $\epsilon > 0$, there exists a trigonometric polynomial $P$ such that $|f-P| < \epsilon$.
\end{theorem}

Combining the above with the fact $\Cc(S^1)$ is dense in $L^2(S^1)$, we obtain the following.

\begin{corollary}
  The trigonometric system is a maximal orthonormal set in $L^2(S^1)$.
\end{corollary}

\begin{definition}
  For any $f \in L^1(S^1)$, we define the \emph{Fourier coefficients of $f$} by the formula
  \[
  \what{f}(n) = (f, u_n) = \frac{1}{2\pi} \int_{-\pi}^{\pi} f(t) e^{-i n t} d t.
  \]
  The \emph{Fourier series} of $f$ is $\sum_{n=-\infty}^{\infty} \what{f}(n) e^{i n t}$, and its partial sums are $s_N(t) = \sum_{n=-N}^N \what{f}(n) e^{i n t}$.
\end{definition}

The following are consequences of previous results for $L^p$ spaces.

\begin{corollary}
  \mbox{}
  \begin{enumerate}[(a)]
  \item For any $f,g \in L^2(S^1)$, we have
    \[
    \sum_{n = -\infty}^\infty \what{f}(n) \overline{\what{g}(n)} = \frac{1}{2\pi} \int_{-\pi}^{\pi} f(t)\overline{g(t)} d t,
    \]
    and the series on the left converges absolutely.
  \item For any $f \in L^2(S^1)$, we have
    \[
    \lim_{N \rarr \infty} \|f-s_N\|_2 = 0,
    \]
    since a special case of (a) implies
\[
\|f-s_N\|^2_2 = \sum_{|n| > N} \left| \what{f}(n) \right|^2.
\]
  \end{enumerate}
\end{corollary}

\subsection{Rudin, Chapter 5: Examples of Banach Space Techniques}

\begin{definition}
  A vector space $X$ is called \emph{normed} if it is endowed with a \emph{norm}, that is, a map $\|-\| \cn X \rarr [0,\infty)$ satisfying the following conditions:
  \begin{enumerate}[(a)]
  \item $\|x+y\| \leq \|x\| + \|y\|$ for all $x,y \in X$;
  \item $\|\alpha x\| = |\alpha| \, \|x\|$ for all $\alpha \in \C$, $x \in X$;
  \item $\|x\| = 0$ implies $x = 0$ for all $x \in X$.
  \end{enumerate}
\end{definition}

\begin{remark}
  Any normed vector space $X$ possesses a natural metric structure in which the distance between $x,y \in X$ is given by $\|x-y\|$. This can be verified using the properties listed above.
\end{remark}

\begin{definition}
  A \emph{Banach space} is a normed vector space which is complete with respect to the metric induced by its norm.
\end{definition}

\begin{remark}
  Every Hilbert space is a Banach space in a natural way.
\end{remark}

\begin{definition}
  Let $X$ and $Y$ be normed vector spaces. The \emph{norm} of a linear map $\Lambda \cn X \rarr Y$ is given by
  \[
  \|\Lambda\| = \sup\{ \|\Lambda(x)\| \;|\; x \in X, \|x\| \leq 1 \}.
  \]
Geometrically, $\Lambda$ maps the closed unit ball in $X$ to the closed ball of radius $\|\Lambda\|$ around $0$ in $Y$.
\end{definition}

\begin{definition}
  If $\|\Lambda\| < \infty$, then $\Lambda$ is called \emph{bounded}.
\end{definition}

\begin{theorem}
  Let $X$ and $Y$ be normed vector spaces. For a linear transformation $\Lambda \cn X \rarr Y$, the following are equivalent:
  \begin{enumerate}[(a)]
  \item $\Lambda$ is bounded;
  \item $\Lambda$ is continuous;
  \item $\Lambda$ is continuous at $0 \in X$.
  \end{enumerate}
\end{theorem}

\begin{theorem}[Baire's Theorem]
  In a complete metric space $X$, the intersection of every countable collection of dense opene subsets is dense. In particular (except in the trivial case $X = \emptyset$), the intersection is not empty.
\end{theorem}

\begin{corollary}
  In a complete metric space, the intersection of any countable collection of dense countable intersections of opens is again a dense countable intersection of opens.
\end{corollary}

The following describes why Baire's Theorem is sometimes called the Category Theorem.

\begin{definition}
  Let $X$ be a metric space. We call a set $E \subset X$ \emph{nowhere dense} if $\overline{E}$ contains no nonempty opens.
\end{definition}

\begin{definition}[Baire's terminology]
  Any countable union of nowhere dense sets is called to be of the \emph{first category}. All other subsets are called to be of the \emph{second categoty}.
\end{definition}

Using this language Baire's Theorem becomes.

\begin{corollary}
  No complete metric space is of the first category.
\end{corollary}

\begin{theorem}[Banach-Steinhauss Theorem]
  Let $X$ be a Banach space, $Y$ a normed vector space, and $\{ \Lambda_\alpha \cn X \rarr Y \;|\; \alpha \in A \}$ a collection of bounded linear transformations. Then there exists an $M < \infty$ such that
  \[
  \|\Lambda_\alpha\| \leq M
  \]
  for all $\alpha \in A$, or
  \[
  \sup_{\alpha \in A} \|\Lambda_\alpha(x)\| = \infty
  \]
  for all $x$ belonging to some countable intersection of opens in $X$ which is also dense.
\end{theorem}

\begin{remark}
  The Banach-Steinhauss Theorem is sometimes referred to as the \emph{uniform boundedness principle}.
\end{remark}

\begin{theorem}[Open Mapping Theorem]
  Let $Y$ and $V$ be the open units of two Banach spaces $X$ and $Y$ respectively. To every surjective bounded linear transformation $\Lambda \cn X \rarr Y$ there corresponds a $\delta > 0$ satisfying $\Lambda(U) \supset \delta V$.
\end{theorem}

\begin{theorem}
  If $X$ and $Y$ are Banach spaces and $\Lambda \cn X \rarr Y$ is a bijective bounded linear transformation, then there is a $\delta > 0$ such that $\|\Lambda(x)\| \geq \delta\|x\|$. In other words, $\Lambda^{-1} \cn Y \rarr X$ is a bounded linear transformation too.
\end{theorem}

TODO here

\subsection{Rudin, Chapter 6: Complex Measures}

TODO

\subsection{Rudin, Chapter 7: Differentiation}

TODO

\subsection{Rudin, Chapter 8: Integration on Product Spaces}

TODO

\subsection{Rudin, Chapter 9: Fourier Transforms}

TODO

%%% Local Variables: 
%%% mode: latex
%%% TeX-master: "Study_Guide"
%%% End: 
