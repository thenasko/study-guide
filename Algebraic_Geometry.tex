\section{Algebraic Geometry}
\label{S:algebraic-geometry}

Syllabus
\begin{description}
\item[Graduate] Harris, \emph{Algebraic geometry, a first course}, lectures 1-7, 11, 13, 14, 18. (math 137 and math 232a)
\end{description}

\subsection{Harris, Lecture 1: Affine and Projective Varieties}

\begin{itemize}
\item
  An inclusion-exclusion type formula holds for dimensions of linear spaces in projective space. Namely, if $\Lambda$ and $\Lambda'$ are linear spaces in a projective space $\P^n$ and $\overline{\Lambda,\Lambda'}$ is their span, then
\[
\dim(\overline{\Lambda,\Lambda'}) = \dim(\Lambda) + \dim(\Lambda') - \dim(\Lambda \cap \Lambda').
\]
Note that we are taking the dimension of the empty set as a linear space to be $-1$. A similar formula holds for linear (not affine) subspaces of affine space.

\item
  A set $\Gamma \subset \P^n$ of $d$ points may be described as the vanishing set of polynomials of degree $d$ and less. We say $\Gamma$ has \emph{degree} $d$. If $\Gamma$ is not contained in a line, then we may describe it by polynomials of degree $d-1$ and less.

\item
  We say that a set of points $p_i = [v_i] \in \P^n$ are \emph{independent} if so are the corresponding vectors $v_i$. The space $\P^n$ can contain at most $n+1$ independent points.

\item
  We say that a set of points $\Gamma \subset \P^n$ are in \emph{general position} if no $n+1$ or fewer of them are linearly dependent.

  \begin{theorem}
    For $k \geq 2$, any collection $\Gamma \subset \P^n$ of $d \leq k n$ points in general position may be described by polynomials of degree $k$ or less. This bound is sharp.
  \end{theorem}
  
  \begin{theorem}
    Any two ordered sets of $n+2$ points in general position in $\P^n$ are projectively equivalent.
  \end{theorem}

\item
  A hypersurface $X$ is a subvariety of $\P^n$ described as the zero locus of a single homogeneous polynomial $F$. We may always choose $F$ without repeated prime factors, and in this case, $F$ is unique up to multiplication by scalars (this requires the Nullstellensatz and holds only over algebraically closed fields). When this is done the degree of $F$ is called the \emph{degree} of the hypersurface $X$.

\item
  A \emph{complex analytic} variety is one which is the locus of (homogeneous) holomorphic functions on $\A_\C^n$ or $\P_\C^n$. All varieties in $\A_\C^n$ and $\P_\C^n$ are complex analytic since polynomials are holomorphic functions. The converse holds in projective but not in affine complex space.

  \begin{theorem}[Chow's Theorem]
    Any complex analytic subvariety $X \subset \P^n_\C$ is an algebraic variety.
  \end{theorem}
\end{itemize}

\subsection{Harris, Lecture 2: Regular Functions and Maps}

\begin{itemize}
\item
  For an affine variety $X \subset \A^n$, we define the ideal $I(X) \subset K[z_1, \dots, z_n]$ of polynomials which vanish at $X$. The \emph{coordinate ring} of $X$ is the quotient $A(X) = K[z_1,\dots,z_n]/I(X)$.
\item
  Consider an open $U \subset X$ of an affine variety, and $p \in U$ a point. We say a function $f$ on $U$ is regular at $p$ if in some neighbourhood $V$ of $p$ it is expressible as a quotient $g/h$ where $g,h \in K[z_1, \dots, z_n]$ are polynomials and $h(p) \neq 0$. We say $f$ is regular on $U$ if it is regular at all $p \in U$.

  \begin{lemma}
    The ring of regular functions regular at every point of an affine variety $X$ is the coordinate ring $A(X)$. More generally, if $U = U_f$ is a distinguished open, then the ring of regular function on $U$ is the localization $A(X)_f = A(X)[1/f]$.
  \end{lemma}

\item
  Similarly, for a projective variety $X \subset \P^n$, one can define an ideal $I(X)$ generated by all homogeneous polynomials vanishing at $X$. The \emph{homogeneous coordinate ring} of $X$ is defined by $S(X) = K[z_0,\dots,z_n]/I(X)$. Note that (the isomorphism class of) $S(X)$ is invariant under projective equivalence but not under general isomorphism of projective varieties.

\item
  If $G$ is a homogeneous polynomial, then the ring of regular functions on complement of the zero locus of $G$ is given by the $0$-th graded piece of the localization $S(X)_G = S(X)[1/G]$.

\item
  For any two positive integers $n$ and $d$, we define the \emph{Veronese map (embedding)} of degree $d$ as
  \begin{align*}
    \nu_d \cn \P^n &\rarr \P^N \\
    [X_0, \dots, X_n] &\mapsto [ \dots, X^I, \dots ]
  \end{align*}
  where $X^I$ ranges over all monomials of degree $d$ in $X_0, \dots, X_n$. It possesses the property that all hypersurfaces of degree $d$ in $\P^n$ are exactly the hyperplane sections of $\nu_d(\P^n) \subset \P^N$. The Veronese map carries any subvariety $X \subset \P^n$ isomorphically to its image $\nu_d(X)$. The dimension of the codomain can be deduced to be $N = \binom{n+d}{d}-1$. The image of a Veronese map is called a \emph{Veronese variety}.

\item
  For any positive integers $m$ and $n$, we define the associated \emph{Segre map}
  \begin{align*}
    \sigma \cn \P^n \x \P^m &\rarr \P^{(n+1)(m+1)-1} \\
    ([X_0,\dots,X_m],[Y_0,\dots,Y_n]) &\mapsto [\dots,X_i Y_j,\dots],
  \end{align*}
  where the coordinates in the target space range over all pairwise products of coordinates $X_i$ and $Y_j$. The image of $\sigma$ is called a \emph{Segre variety} and is denoted $\Sigma_{m,n}$.
\end{itemize}

\subsection{Harris, Lecture 3: Cones, Projections, and More About Products}

\begin{itemize}
\item
  Every quadric hypersurface $Q \subset \P^n$ can be put in the form $X_0^2 + \cdots + X_k^2 = 0$ for some integer $k \geq 0$. We call $k+1$ the \emph{rank} of $Q$.
  
\item
  Consider two polynomials $f$ and $g$ of degrees $m$ and $n$ respectively with coefficients in a field $K$. We would like to investigate when $f$ and $g$ have a common factor. This happens if and only if there exists a polynomial $h$ of degree $m+n-1$ divisible by both. Equivalently, this means that the polynomials $f$, $z \cdot f$, \dots, $z^{n-1} \cdot f$, $g$, $z \cdot g$, \dots, $z^{m-1} \cdot g$ fail to be linearly independent. In turn this is equivalent to the statement that the determinant
  \[
  R(f,g) =
  \left|
    \begin{array}{ccccccccccc}
      a_0 & a_1 & \cdot & \cdot & a_m & 0   & 0 & \cdot & \cdot & \cdot & 0 \\
      0   & a_0 & a_1 & \cdot & \cdot & a_m & 0 & \cdot & \cdot & \cdot & 0 \\
      \vdots \\
      0   & \cdot & \cdot & 0 & a_0 & a_1 & \cdot & \cdot & \cdot & \cdot & a_m \\
      b_0 & b_1 & \cdot & \cdot & \cdot & b_n & 0 & \cdot & \cdot & \cdot & 0 \\
      0 & b_0 & b_1 & \cdot & \cdot & \cdot & b_n & 0 & \cdot & \cdot & 0 \\
      \vdots \\
      0 & \cdot & \cdot & 0 & b_0 & b_1 & \cdot & \cdot & \cdot & \cdot & b_n
    \end{array}
  \right|
  \]
  is zero. This determinant $R(f,g)$ is often called the \emph{resultant} of $f$ and $g$. In fact, the definition above makes sense for polynomials with coefficients in any ring.

  \begin{theorem}
    Two polynomials $f$ and $g$ in one variable over a field $K$ will have a common factor if and only if $R(f,g) = 0$.
  \end{theorem}

  The resultant is useful in the proof of the following result.
  
  \begin{theorem}
    The projection $\overline{X}$ of $X \subset \P^n$ from $p \notin X$ to $\P^{n-1}$ is a projective variety.
  \end{theorem}

\item
  The image of a closed set is not always closed. In special cases, this could however be derived.
  
  \begin{theorem}
    Let $Y$ be any variety and $\pi \cn Y \x \P^n \rarr Y$ be the projection on the first factor. Then the image $\pi(X)$ of any closed subset $X \subset Y \x \P^n$ is a closed subset of $Y$.
  \end{theorem}

  This leads to the following.

  \begin{theorem}
    If $X \subset \P^n$ is any projective variety and $\phi \cn X \rarr \P^m$ any regular map, then the image of $\phi$ is a projective subvariety of $\P^m$.
  \end{theorem}

  \begin{corollary}
    If $X \subset \P^n$ is any connected variety and $f$ any regular function on $X$, then $f$ is constant.
  \end{corollary}

  \begin{corollary}
    If $X \subset \P^n$ is any connected variety other than a point and $Y \subset \P^n$ is any hypersurface then $X \cap Y = \emptyset$.
  \end{corollary}

\item
  We call a set \emph{constructible} if it is expressible as the union of locally closed subsets. Equivalently, a subset $Z \subset \P^n$ is constructible if there exists a nested sequence $X_1 \supset X_2 \supset \cdots \supset X_n$ of closed subsets of $\P^n$ such that
  \[
  Z = X_1 \setminus (X_2 \setminus (X_3 \setminus \cdots \setminus X_n)).
  \]

  \begin{theorem}[Chevalley]
    Let $X \subset \P^m$ be a quasi-projective variety, $f \cn X \rarr \P^n$ a regular map, and $U \subset X$ any constructible set. Then $f(U)$ is a constructible subset of $\P^n$.
  \end{theorem}
\end{itemize}

\subsection{Harris, Lecture 4: Families and Parameter Spaces}

\begin{itemize}
\item
  A \emph{family of projective varieties} in $\P^n$ with base $B$ is simply a closed subvariety $\Vc \subset B \x P^n$. The fibers $V_b = (\pi_1)^{-1}(b)$ of $\Vc$ over points of $b$ are referred to as \emph{members of the family}.

\item
  Let $X \subset \P^n$ be any projective variety and $\{ V_b \}$ any family of projective varieties in $\P^n$ with base $B$. Then the set
  \[
  \{ b \in B \;|\; X \cap V_b \neq \emptyset \}
  \]
  is closed in $B$. More generally, if $\{ W_b \}$ is another family of projective varieties in $\P^n$ with base $B$, then
  \[
  \{ b \in B \;|\; V_b \cap W_b \neq \emptyset \}
  \]
  is a closed subvariety of $B$.

\item
  Even more generally, if $\{ W_c \}$ is a family of projective varieties in $\P^n$ with a possibly different base $C$, then the set
  \[
  \{ (b,c) \in B \x C \;|\; V_b \cap W_c \neq \emptyset \}
  \]
  is a closed subvariety of $B \x C$.

\item
  In a similar vein, for any $X \subset \P^n$ and family $\{ V_b \}$ the set
  \[
  \{ b \in B \;|\; V_b \subset X \}
  \]
  is constructible and the set
  \[
  \{ b \in B \;|\; X \subset V_b \}
  \]
  is closed in $B$.

\item
  Recall that points in ${\P^n}^\ast$ can be thought of as hyperplanes in $\P^n$. Under this identification, the set
  \[
  \Gamma = \{ (H, p) \in {\P^n}^\ast \x \P^n \;|\; p \in H \}
  \]
  is called the \emph{universal hyperplane}. The fiber over $H \in {\P^n}^\ast$ is precisely the hyperplane $H \subset \P^n$. We call it universal since for any flat family of hyperplanes $\Vc \subset B \x \P^n$ there exists a unique regular map $B \rarr {\P^n}^\ast$ such that the diagram
  \[\xymatrix{
    \Vc \ar[r] \ar[d] & \Gamma \ar[d] \\
    B \ar[r] & {\P^n}^\ast
  }\]
  is cartesian.

\item
  For any $X \subset \P^n$, the \emph{universal hyperplane section} is
  \[
  \Omega_X = \{ (H, p) \in {\P^n}^\ast \x \P^n \;|\; p \in H \cap X \} = (\pi_2)^{-1}(X),
  \]
  where $\pi_2 \cn \Gamma \rarr \P^n$ is the projection on the second factor of t he universal hyperplane $\Gamma$.

\item
  A \emph{section} of a family of projective varieties $\Vc \subset B \x \P^n$ is a map $\sigma \cn B \rarr \Vc$ such that $\pi_1 \circ \sigma = \id_B$. A \emph{rational section} is a section defined on some non-empty open subset $U \subset B$.
\end{itemize}

\subsection{Harris, Lecture 5: Ideals of Varieties, Irreducible Decomposition, and the Nullstellensatz}

\begin{itemize}
\item
  The following result enables us to establish a correspondence between geometry and algebra. Furthermore, it implies several very useful corollaries.

  \begin{theorem}[Nullstellensatz]
    For any ideal $I \subset K[z_1, \dots, z_n]$, the ideal of functions vanishing on the common zero locus of $I$ is the radical of $I$, i.e.,
    \[
    I(V(I)) = \rad(I).
    \]
    Thus, there is a bijective correspondence between subvarieties $X \subset \A^n$ and radical ideals $I \subset K[z_1, \dots, z_n]$.
  \end{theorem}

  \begin{theorem}[Weak Nullstellensatz]
    Any ideal $I \subset K[x_1, \dots, x_n]$ with no common zeros is the unit ideal.
  \end{theorem}

  \begin{theorem}
    Every prime ideal in $K[x_1, \dots, x_n]$ is the intersection of the ideals of the form $(x_1 - a_1, \dots, x_n - a_n)$ containing it.
  \end{theorem}

\item
  We say an ideal $I$ cuts out $X \subset \A^n$ \emph{set-theoretically} if $V(I) = X$; we say $I$ cuts out $X$ \emph{scheme-theoretically} if $I = I(X)$.

\item
  Consider the case of projective varieties. There is almost a bijection between closed varieties $X \subset \P^n$ and radical homogeneous ideals $I \subset K[Z_0, \dots, Z_n]$. Define the \emph{saturation} of $I$ to be
  \[
  \overline{I} = \{ F \in K[Z_0, \dots, Z_n] \;|\; (Z_0, \dots, Z_n)^k \cdot F \subset I \textrm{ fo rsome } k \}.
  \]

  \begin{proposition}
    The following conditions for a pair of homogeneous ideals $I, J \subset K[Z_0, \dots, Z_n]$ are equivalent:
    \begin{enumerate}[(i)]
    \item $I$ and $J$ have the same saturation;
    \item $I_m = J_m$ for all $m \gg 0$;
    \item $I$ and $J$ agree locally, that is, they generate the same ideal in each localization $K[Z_0, \dots, Z_n, Z_i^{-1}]$ of $K[Z_0, \dots, Z_n]$.
    \end{enumerate}
  \end{proposition}

  We say that an ideal $I$ cuts out $X \subset \P^n$ \emph{scheme-theoretically} if $\overline{I} = I(X)$.

\item
  A variety is called \emph{irreducible} if for any pair of closed subvarieties $Y, Z \subset X$ such that $Y \cup Z = X$, either $Y = X$ or $Z = X$.

\item
  An affine variety $X \subset \A^n$ is irreducible if and only if $I(X)$ is prime.

\item
  If a projective variety $X \subset \P^n$ is irreducible, then so is every non-empty open affine $U = X \cap \A^n$. The converse is also true if we consider all hyperplane compliments of $X$, but not if we restrict our attention only to the standard opens.

\item
  A variety is irreducible if and only if every Zariski open subset is dense, i.e., every two Zariski non-empty open subsets intersect.

\item
  The image of an irreducible variety under a regular map is irreducible.

\item
  The following are two potentially useful criteria for irreducibility.

  \begin{theorem}
    Let $X \subset \P^n$ be an irreducible variety, and let $\Omega_X \subset {\P^n}^\ast \x X$ be its universal hyperplane section. Then $\Omega_X$ is irreducible.
  \end{theorem}

  \begin{theorem}
    Let $f \cn Z \rarr Y$ is a regular map. Assume that
    \begin{enumerate}[(i)]
    \item $f$ is open;
    \item $Y$ is irreducible;
    \item for a dense open set of points $p \in Y$, the fiber $f^{-1}(p)$ is irreducible.
    \end{enumerate}
    Then $Z$ is irreducible.
  \end{theorem}

\item
  The following is useful in discussing fibers.

  \begin{proposition}
    Let $\pi \cn X \rarr Y$ be any regular map with $Y$ irreducible, and let $Z \subset X$ be any locally closed subset. Then for a general point $p \in Y$ the closure of the fiber $Z_p = Z \cap \pi^{-1}(p)$ is the intersection of the closure $\overline{Z}$ of $Z$ with the fiber $X_p = \pi_{-1}(p)$.
  \end{proposition}
\end{itemize}

\subsection{Harris, Lecture 6: Grassmannians and Related Varieties}

\begin{itemize}
\item
  TODO
\end{itemize}

TODO

\subsection{Harris, Lecture 7: Rational Functions and Rational Maps}

TODO

\subsection{Harris, Lecture 11: Definitions of Dimension and Elementary Examples}

TODO

\subsection{Harris, Lecture 13: Hilbert Polynomials}

TODO

\subsection{Harris, Lecture 14: Smoothness and Tangent Spaces}

TODO

\subsection{Harris, Lecture 18: Degree}

TODO

%%% Local Variables: 
%%% mode: latex
%%% TeX-master: "Study_Guide"
%%% End: 
