\section{Algebra}
\label{S:algebra}

Syllabus
\begin{description}
\item[Undergraduate] Dummit \& Foote, \emph{Abstract Algebra}, except chapters 15, 16 and 17. (math 122, 123)
\end{description}

\subsection{Dummut \& Foote, Preliminaries}

\begin{itemize}
\item Basics
  \begin{itemize}
  \item Sets
  \item Functions/maps and related properties -- bijective, surjective, left/right inverse, image/preimage, permutation, restriction/extension
  \item Relations -- reflexive, symmetric, transitive, equivalence relations, partitions, the equivalent of the last two
  \end{itemize}
\item Integers
  \begin{itemize}
  \item Ring structure
  \item Well ordering
  \item GCD \& LCM
  \item Division algorithm
  \item Euclidean algirithm
  \end{itemize}
\item Integers modulo $n > 1$
  \begin{itemize}
  \item Quotient of $\Z$
  \item Ring structure
  \end{itemize}
\end{itemize}

\subsection{Dummut \& Foote, Chapter 1: Introduction to Groups}

\begin{itemize}
\item Definition of a group, properties, and examples -- associativity, commutativity, inverses, order of an element (denoted by $|-|$)
\item Dihedral groups
  \begin{itemize}
  \item Notation $D_{2n}$, and $|D_{2n}| = 2n$
  \item $D_{2n} = \langle r,s \;|\; r^n = s^2 = 1, r s = s r^{-1} \rangle$
  \end{itemize}
\item Symmetric group -- presented as bijections, cycles, order as LCM, conjugacy classes and partitions
\item Matrix groups \\
  If $|F| = q < \infty$, then $|GL(n, F)| = (q^n-1)(q^n-q)\cdots(q^n-q^{n-1})$.
\item The Quaternion group \\
  $Q_8 = \{\pm 1, \pm i, \pm j, \pm k\}$ \\
  $i^2 = j^2 = k^2 = -1$ \\
  $i \cdot j = k$, $j \cdot k = i$, $k \cdot i = j$, and exchange introduces a sign
\item Homomorphisms and isomorphisms -- definition
\item Group actions -- definition
\end{itemize}

\subsection{Dummut \& Foote, Chapter 2: Subgroups}

\begin{itemize}
\item Subgroup criterion -- $H \subset G$ is a subgroup if $H \neq \emptyset$ and for all $x,y \in H$, we have $x y^{-1} \in H$
\item Centralizer -- for any $A \subset G$, its centralizer is $C_G(A) = \{ g \in G \;|\; g a = a g \textrm{ for all } a \in A \}$
\item Center -- $Z(G) = C_G(G)$
\item Normalizer -- for any $A \subset G$, its normalizer is $N_G(A) = \{ g \in G \;|\; g A g^{-1} = A \}$
  \begin{itemize}
  \item $N_G(A)$ is always a normal subgroup
  \item If $A$ is a subgroup, then $N_G(A)$ is the largest subgroup such that $A \tleq N_G(A)$.
  \item $C_G(A) \tleq N_G(A)$
  \end{itemize}
\item Cyclic groups and cyclic subgroups
  \begin{itemize}
  \item Any subgroup of a cyclic subgroup is cyclic.
  \item Consider a cyclic group of size $n$. For each $m|n$, there is a unique subgroup of size $m$.
  \item The number of generators of a cyclic group of size $n$ is $\phi(n)$.
  \end{itemize}
\item Subgroups generated by subsets of a group -- definition, well-defined
\item The lattice of a subgroups of a group
\end{itemize}

\subsection{Dummut \& Foote, Chapter 3: Quotient Groups and Homomorphisms}

\begin{itemize}
\item Lagrange's Theorem
\item First Isomorphism Theorem
\item Kernels are always normal, and each normal subgroup occurs as the kernel of a homomorphism (e.g.\ the quotient one).
\item Normality is not transitive.
\item The only group of prime order $p$ is $\Z/p\Z$.
\item Every subgroup of index 2 is normal.
\item If $H, K \leq G$ are finite, then $|H K| = |H| \, |K|/|H \cap K|$.
\item The set $H K$ is a group if and only if $H K = K H$.
\item If $H \leq N_G(K)$, then $H K$ is a subgroup of $G$. In particular, if $K \tleq G$, then $H K$ is a group for all $H \leq G$.
\item $|G/H| = |G|/|H|$
\item Second (Diamond) Isomorphism Theorem: Let $A, B \leq G$ and $A \leq N_G(B)$. Then $A \cap B \tleq A$ and
  \[
  A B/B \cong A/(A \cap B).
  \]
  In particular, the conclusion holds true if $B \tleq G$.
\item Third Isomorphism Theorem: If $H, K \tleq G$ and $H \leq K$, then $K/H \tleq G/H$ and
  \[
  (G/H)/(K/H) \cong G/K.
  \]
\item Fourth Isomorphism Theorem: for $N \tleq G$ there is a correspondence between the subgroups of $G/N$ and those of $G$ containing $N$ given by taking preimages under the quotient map $G \rarr G/N$; the correspondence presernes normality and inclusions, has numerous other expected properties.
\item A group $G$ is called \emph{simple} if it is nontrivial, and its only normal subgroups are $1$ and $G$.
\item A sequence of groups
  \[
  1 = N_0 \leq N_1 \leq \cdots \leq N_{k-1} \leq N_k = G
  \]
  is called a \emph{composition series} for $G$ if $N_{i-1} \tleq N_i$ and $N_i/N_{i-1}$ is simple for all $1 \leq i \leq k$.
\item Jordan-H\"older: Every finite group has a composition series, and its factors are unique up to permutation.
\item H\"older program: (1) classify finite simple groups, and (2) the ways of ``putting them together''
\item Examples of simple groups: $\Z/p\Z$, $SL(n, F)/Z(SL(n,F))$ for any finite field $F$ and $n \geq 2$ (except $SL(2,\F_2)$ and $SL(2,\F_3)$).
\item Every element of $S_n$ may be written as a product of transpositions.
\item The alternating group $A_n$ as the kernel of the homomorphism $\sign \cn S_n \rarr \{\pm 1\}$.
\end{itemize}

\subsection{Dummut \& Foote, Chapter 4: Group Actions}

\begin{itemize}
\item $|G| = |G_x| \, |G x|$
\item Let $H$ be a subgroup of the finite group $G$, and $G$ acts on the set of left cosets $A = G/H$. Then:
  \begin{enumerate}[(a)]
  \item $G$ acts transitively on $A$;
  \item the stabilizer of the point $1 H \in A$ is $H$;
    \item the kernel of this representation is $\bigcap_{x \in G} x H x^{-1}$ which is the largest normal subgroup of $G$ contained in $H$.
  \end{enumerate}
\item Cayley's Theorem: Every group $G$ of order $n$ is isomorphic to a subgroup of $S_n$.
\item If $G$ is a finite subgroup of order $n$ and $p$ is the smallest prime dividing $|G|$, then any subgroup of index $p$ is normal.
\item The number of conjugates of a subset $S \subset G$ is $[G:N_G(S)]$. In particular, for $s \in G$ we have $N_G(s) = C_G(s)$, so the number of conjugates of $s$ is $[G:N_G(s)]$.
\item The Class Equation: Let $G$ be a finite group and $g_1, \dots, g_r$ be representatives of the distinct conjugacy classes of $G$ not contained in the center $Z(G)$. Then
  \[
  |G| = |Z(G)| + \sum_{i=1}^r [G:C_G(g_i)].
  \]
\item Any group of prime power order has a nontrivial center.
\item If $G/Z(G)$ is cyclic, then $G$ is abelian.
\item Any group of order $p^2$ is abelian, hence isomorphic to either $\Z/p^2\Z$ or $(\Z/p\Z)^2$.
\item 
\end{itemize}

TODO

\subsection{Dummut \& Foote, Chapter 5: Direct and Semidirect Products of Abelian Groups}

TODO

\subsection{Dummut \& Foote, Chapter 6: Further Topics in Group Theory}

TODO

\subsection{Dummut \& Foote, Chapter 7: Introduction to Rings}

TODO

\subsection{Dummut \& Foote, Chapter 8: Euclidean Domains, Principal Ideal Domains and Unique Factorization Domains}

TODO

\subsection{Dummut \& Foote, Chapter 9: Polynomial Rings}

TODO

\subsection{Dummut \& Foote, Chapter 10: Introduction to Module Theory}

TODO

\subsection{Dummut \& Foote, Chapter 11: Vector Spaces}

TODO

\subsection{Dummut \& Foote, Chapter 12: Modules over Principal Ideal Domains}

TODO

\subsection{Dummut \& Foote, Chapter 13: Field Theory}

TODO

\subsection{Dummut \& Foote, Chapter 14: Galois Theory}

TODO

\subsection{Dummut \& Foote, Chapter 18: Representation Theory and Character Theory}

TODO

\subsection{Dummut \& Foote, Chapter 19: Examples and Applications of Character Theory}

TODO

%%% Local Variables: 
%%% mode: latex
%%% TeX-master: "Study_Guide"
%%% End: 
